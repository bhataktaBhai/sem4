\lecture{01}{Mon 08 Jan '24}{}

\setcounter{section}{-1}
\section{The Course} \label{sec:course}
\textbf{Course website:} \href{https://www.csa.iisc.ac.in/~arindamkhan/courses/ApproxAlgo22/ApproxAlgo.html}{here} \\
\textbf{MS Teams:} 091kg9h \\
\textbf{Instructors:} Prof.~Arindam~Khan and Prof.~Anand~Louis \\
\textbf{TAs:} Aditya~Subramaniam (January) \\
\textbf{Lecture Hours:} MW 1400--1530 at CSA 112 \\
\textbf{Office Hours:} Just email.

\subsection{Grading} \label{sec:grading}
\begin{itemize}
    \item[(30\%)] Homework
    \item[(20\%)] Project. Research papers from top conferences
        (STOC, FOCS, SODA, ICALP, SoCG).
        Around 10\% on a report.
        10\% on a presentation (half an hour).
    \item[(20\%)] Midterm
    \item[(30\%)] Final
\end{itemize}

\subsection{Texts} \label{sec:texts}
Primarily \textit{The Design of Approximation Algorithms} by David Williamson
and David Shmoys, Cambridge University Press, 2011.
Available online for free. \\
Alternatively, \textit{Approximation Algorithms} by Vijay Vazirani, Springer-Verlag, 2001. \\
For specific topics, Hochbaum, Sariel, etc.

\section{Optimization Problems} \label{sec:optimization_problems}
Find the \emph{best} solution from a set of \emph{feasible} solutions.
Richard Karp introduced the concept of \NP-complete problems.
Unless $\P = \NP$, the optimization version of the problems admit no algorithms
that simultaneously (1) find optimal solution (2) in polynomial time (3) for
all instances.

\subsection{Optimization Version of Common \NP-Complete Problems} \label{sec:optimization_np}
\begin{center}
    \begin{tabular}{p{0.4\textwidth}p{0.55\textwidth}}
        \toprule
        Exact Decision Problem & Optimization Version \\
        \midrule
        \textbf{3-SAT} \newline 
        Is a 3-CNF formula satifiable?
            & \textbf{Max 3-SAT} \newline
            Find an assignment that satisfies as many clauses as possible. \\
        \midrule
        \textbf{3Col} \newline
        Is there a legal 3-coloring (all edges bichromatic) of a graph?
            & There are 2 natural corresponding optimization problems.
            \begin{itemize}
                \item \textbf{Min-Coloring} \newline
                    Color legally with as few colors as possible.
                \item \textbf{Max-3Cut} \newline
                    Color with 3 colors, make the coloring as legal as possible.
                    Can be thought of as partitioning vertices into three sets,
                    and maximizing the number of edges between the sets.
                    \newline
                    (note: the usual \textbf{Max-Cut} is \textbf{Max-2Cut}) 
            \end{itemize} \\
        \midrule
        \textbf{Vertex Cover} \newline
        Input is a graph and an integer $k$.
        Is there a vertex cover (a subset of vertices such that every edge
        includes one of the vertices) of size less than $k$?
            & \textbf{Min-Vertex-Cover} \newline
            Input is a graph.
            Output is a vertex cover.
            Value is the fraction of vertices in the cover. \\
        \bottomrule
    \end{tabular}
\end{center}

\subsection{Approximation Algorithms} \label{sec:approx}
\textbf{Task:} Solve \NP-hard optimization problems $A$, but no efficient
algorithm exists (unless $\P = \NP$).
\begin{definition}
    Let $\Pi$ be an optimization problem and let $I$ be an instance of $\Pi$.
    Then $\OPT_\Pi(I)$ is the value of the optimal solution.
\end{definition}
There might be several optimal solutions, but they all have the same value.
\begin{definition}
    Let $\alpha \ge 1$.
    $A$ is an $\alpha$-approximation algorithm for a minimization problem $\Pi$
    if for every instance $I$ of $\Pi$, \[
        A(I) \le \alpha \OPT_\Pi(I)
    \] where $A(I)$ is the value of the solution that $A$ returns for $I$.
\end{definition}
Typically, $\alpha$ takes values like $1.5$, $2$, $O(1)$, $O(\log n)$, etc. \\
Usually we omit $\Pi$ and $I$ in $\OPT_\Pi(I)$.

For a maximization problem, $A(I) \ge \frac1\alpha \OPT_\Pi(I)$.
(Sometimes in literature, $\alpha \le 1$ is used for maximization problems.)

This is also called \emph{absolute approximation}.

\NP-hard problems are very similar to each other in terms of decidability, but
can be very different in terms of approximability.
For some problems, it is \NP-hard to obtain any approximation (TSP) but for some
(Knapsack) we can get $(1+\varepsilon)$-approximation in
polynomial$(n, \frac1\varepsilon)$ time.

\begin{definition}
    $A_\varepsilon$ is a polynomial time approximation scheme (PTAS) for a
    minimization problem $\Pi$ if \[
        A_\varepsilon(I) \le (1+\varepsilon) \OPT(I)
    \]
    and for every fixed $\varepsilon > 0$, the running time of $A_\varepsilon$
    is polynomial in $n$.
\end{definition}
\begin{definition}[EPTAS] \label{def:approx:eptas}
    Efficient PTAS.
    $(1+\varepsilon)$-approximation in runtime $f(1 / \varepsilon) n^{O(1)}$,
    where the exponent of $n$ is independent of $\varepsilon$.
\end{definition}
\begin{definition}[FPTAS] \label{def:approx:fptas}
    Fully polynomial time approximation scheme.
    $(1+\varepsilon)$-approximation in runtime polynomial in both $n$ and
    $\frac1\varepsilon$.
\end{definition}
\begin{definition}[QPTAS] \label{def:approx:qptas}
    Quasi-polynomial time approximation scheme.
    $(1+\varepsilon)$-approximation in quasi-polynomial time in $n$.

    A \emph{quasi-polynomial} is between a polynomial and an exponential.
\end{definition}
\begin{definition}[PPTAS] \label{def:approx:pptas}
    Pseudo-polynomial time approximation scheme.
    $(1+\varepsilon)$-approximation in pseudo-polynomial time in $n$.

    For example, $(nB)^{O(1)}$, where $B$ is the biggest numeric data.
\end{definition}

\subsection{Asymptotic Approximation} \label{sec:asymptotic_approximation}
\begin{definition}
    The asymptotic approximation ratio $\rho_A^\infty$ of an algorithm $A$ is \[
        \lim_{n \to \infty} \rho_A^n, \quad \text{where} \quad
        \rho_A^n = \sup_{I \in \mathcal{I}} \set{\frac{A(I)}{\OPT(I)} \large\mid
        \OPT(I) = n}
    \]
\end{definition}


