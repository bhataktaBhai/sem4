% \let\emph\textsl
\let\emph\textsl
\let\emph\textsl
\let\emph\textsl
\input{~/IISc/preamble}
\usepackage{pgfplots}
\pgfplotsset{compat=1.18}

\newcounter{assignment}
\theoremstyle{definition}
\newmdtheoremenv[nobreak=true, outerlinewidth=0.7]{problem}{Problem}[assignment]
\newmdtheoremenv[nobreak=true]{exercise}[theorem]{Exercise}

\DeclareMathOperator{\sgn}{sgn}
\usepackage{bbm}
\newcommand{\ind}[1]{\mathbbm{1}_{#1}}
\newcommand{\given}{\mid}

\usepackage{pgfplots}
\pgfplotsset{compat=1.18}

\newcounter{assignment}
\theoremstyle{definition}
\newmdtheoremenv[nobreak=true, outerlinewidth=0.7]{problem}{Problem}[assignment]
\newmdtheoremenv[nobreak=true]{exercise}[theorem]{Exercise}

\DeclareMathOperator{\sgn}{sgn}
\usepackage{bbm}
\newcommand{\ind}[1]{\mathbbm{1}_{#1}}
\newcommand{\given}{\mid}

\usepackage{pgfplots}
\pgfplotsset{compat=1.18}

\newcounter{assignment}
\theoremstyle{definition}
\newmdtheoremenv[nobreak=true, outerlinewidth=0.7]{problem}{Problem}[assignment]
\newmdtheoremenv[nobreak=true]{exercise}[theorem]{Exercise}

\DeclareMathOperator{\sgn}{sgn}
\usepackage{bbm}
\newcommand{\ind}[1]{\mathbbm{1}_{#1}}
\newcommand{\given}{\mid}

\newcommand\size[1]{\##1}
\newcommand{\pp}{_{+\!+}}
% long minus
\newcommand{\lm}{\setminus}
\newcommand{\doubleslash}{/\!/}
\newcommand\multichoose[2]{\ensuremath{\left(\kern-.3em\left(\genfrac{}{}{0pt}{}{#1}{#2}\right)\kern-.3em\right)}}
\newcommand\stirling[2]{\genfrac{\{}{\}}{0pt}{}{#1}{#2}}
\newcommand\ftirling[2]{\genfrac{[}{]}{0pt}{}{#1}{#2}}
\newcommand\cycle[2]{\genfrac{\langle}{\rangle}{0pt}{}{#1}{#2}}
\newcommand\1[1]{\bm{1}_{#1}}
\newcommand\id{\mathrm{id}}
\DeclareMathOperator\type{type}

\DeclareMathOperator\degree{deg}
\DeclareMathOperator\diag{diag}

\usepackage{booktabs}
\usepackage{diagbox}
\usetikzlibrary{graphs,quotes}

\usepackage[usestackEOL]{stackengine}[2013-10-15]

\newcommand\young[2][1]{\begin{tikzpicture}[scale=#1]
    \foreach[count=\i] \x in #2 {%
        \draw (0, -\i) grid (\x, -\i-1);
    }
\end{tikzpicture}}
\DeclareMathOperator\ord{ord}
\usepackage{bbm}
\usepackage{subcaption}
