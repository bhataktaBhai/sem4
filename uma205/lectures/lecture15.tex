\lecture{2024-02-05}{}
\textbf{Recall:}
Let $C_n$ be the number of valid words with $n$ pairs of parentheses.
We had derived the ``recurrence'' \[
    C_{n+1} = \sum_{k=0}^{n} C_k C_{n-k}
\] Let $C(x)$ be the ogf of $(C_n)_{n \in \N}$.
Then the RHS has ogf $C(x)^2$ and the LHS has the ogf $\frac{C(x) - 1}{x}$
(exercise).
\begin{solution}
    The RHS is by the product formula.
    For the LHS, let $C_{+1}(x)$ be the ogf of $(C_{n+1})_n$.
    That is, \begin{align*}
        C_{+1}(x) &= \sum_{n = 0}^{\infty} C_{n+1} x^n \\
        &= \frac1x \sum_{n=0}^{\infty} C_{n+1} x^{n+1} \\
        &= \frac1x \sum_{n=1}^{\infty} C_n x^n \\
        &= \frac1x (C(x) - 1)
    \end{align*} since $C_0 = 1$ (the empty word).
\end{solution}

So \begin{align*}
    C(x) - 1 = x C(x)^2 \\
    C(x) = \frac{1 \pm \sqrt{1 - 4x}}{2x}
\end{align*}
The positive root is not a formal power series,
so $C(x) = \frac{1 - \sqrt{1 - 4x}}{2x}$.

\begin{exercise*}
    Use the binomial theorem to get \[
        C_n = \frac{1}{n+1} \binom{2n}{n}.
    \] These are called the \emph{Catalan numbers}, since they were first
    studied by Euler.
\end{exercise*}
\begin{solution}
    \begin{align*}
        C(x)
        &= \frac{1 - (1 + \sum_{n=1}^{\infty} \binom{1/2}{n} (-4x)^n}{2x} \\
        &= \frac12 \sum_{n=0}^{\infty} \binom{1/2}{n+1} (-1)^n 4^{n+1} x^n \\
        \intertext{so by \cref{thm:half_choose},}
        &= \frac12 \sum_{n=0}^{\infty} \frac{(-1)^n}{(n+1)2^{2n+1}} \binom{2n}{n}
            (-1)^n 4^{n+1} x^n \\
        &= \sum_{n=0}^{\infty} \frac{1}{n+1} \binom{2n}{n} x^n.
    \end{align*}
    which gives the desired result.
\end{solution}

\begin{definition*}[Exponential generating function] \label{def:egf}
    Given a sequence $(a_n)_{n \in \N}$, the formal power series
    $A(x) = \sum_{n \in \N} a_n \frac{x^n}{n!}$ is called the
    \emph{exponential generating function} (egf) of $(a_n)_{n \in \N}$.
\end{definition*}
\begin{examples}
    \item The egf of $(1)_{n \in \N}$ is $e^x$.
    \item Let $a_0 = 1$, $a_{n+1} = (n + 1)(a_n - n + 1)$ for $n \ge 0$.
    Then its egf is \begin{align*}
        A(x) &= \frac{1}{1 - x} + x e^{x} \\
        \implies a_n &= n! + n
    \end{align*}
\end{examples}

\begin{exercise*}[Product formula] \label{thm:egf:product}
    Let $A$ and $B$ be egfs for $(a_n)_\N$ and $(b_n)_\N$, respectively.
    Then the sequence $(c_n)_\N$ which has the egf $AB$ is given by \[
        c_n = \sum_{k=0}^{n} \binom{n}{k} a_k b_{n-k}.
    \]
\end{exercise*}

\begin{theorem*}[Exponential formula] \label{thm:egf:formula}
    Let $a_n$ be the number of ways to build some structure on $[n]$ with
    $a_0 = 0$ and $h_n$ be the number of ways to form a set partition of
    $[n]$ and then build the same structure on its blocks, with $h_0 = 1$.
    If $A$ and $H$ are the egfs of $(a_n)_n$ and $(h_n)_n$ respectively,
    then \[
        H(x) = e^{A(x)}.
    \]
\end{theorem*}
\begin{proof}
    The number of ways of partitioning $[n]$ into $k$ blocks and building
    the same structure has egf $\frac{A(x)^k}{k!}$ by the product formula
    (since order doesn't matter).
    Thus \[
        H(x) = 1 + \sum_{k=1}^{\infty}\frac{A(x)^k}{k!} = e^{A(x)}. \qedhere
    \]
\end{proof}
\begin{examples}
    \item Let $a_n$ be the number of sets of $[n]$.
    That is, $a_n = 1$ for all $n$ except $a_0 = 0$.
    Then $A(x) = e^x - 1$.
    This gives $H(x) = e^{e^{x} - 1}$, which is the egf of $B_n$.
    \item Let $i_n$ be the number of involutions in $S_n$.
    That is, \[
        i_n = \#\set{\sigma \in S_n \mid \sigma = \sigma^{-1}}.
    \]
    We had shown earlier that an involution has cycle type
    $\angled{1^{a_1}, 2^{a_2}}$.

    The structure here is to have only one- and two-cycles.
    So \[
        a_n = \begin{cases}
            1 & \text{if } n = 1, \\
            1 & \text{if } n = 2, \\
            0 & \text{otherwise}.
        \end{cases}
    \] This gives $A(x) = x + \frac{x^2}{2}$.
    So by the exponential formula, the number of involutions has egf \[
        H(x) = \sum_{n \in \N} i_n \frac{x^n}{n!} = e^{x + \frac{x^2}{2}}.
    \]
\end{examples}

\chapter{Graph Theory}
$\langle$Insert oft-repeated story about Königsberg bridges.$\rangle$
