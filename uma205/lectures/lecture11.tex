\lecture{11}{Mon 29 Jan '24}{}
We define, by convention, $\stirling{n}{0} = \delta_{n, 0}$ and
$\stirling{n}{k} = 0$ for $k > n$.

We immediately have that $\stirling n1 = \stirling nn = 1$ for $n \ne 0$.
We enumerate some more values in \cref{tab:stirling}.
\begin{table}[ht]
    \centering
    \begin{tabular}{c|cccc}
        \diagbox[width=2.5em,height=2em]{$n$}{$k$} & 1 & 2 & 3 & 4 \\
        \hline
        1 & 1 \\
        2 & 1 & 1 \\
        3 & 1 & 3 & 1 \\
        4 & 1 & 6 & 7 & 1 \\
    \end{tabular}
    \caption{Stirling numbers of the second kind}
    \label{tab:stirling}
\end{table}

\begin{proposition*} \label{thm:partition:recurrence}
    For $1 \le k \le n$, $\stirling nk = \stirling{n-1}{k-1} +
    k \stirling{n-1}{k}$
\end{proposition*}
\begin{proof}
    We split the partitions into two cases:
    \begin{itemize}
        \item The partition contains $\set{n}$ as a singleton.
            There are $\stirling{n-1}{k-1}$ such partitions.
        \item $n$ belongs to some other block.
            There are $\stirling{n-1}{k}$ ways to partition the remaining
            elements, and $k$ ways to choose which block $n$ belongs to.
            \qedhere
    \end{itemize}
\end{proof}

\begin{proposition}
    The number of surjections from $[n]$ to $[k]$ is $k! \stirling nk$.
\end{proposition}
\begin{proof}
    Any surjection is determined by a sequence $(p_1, p_2, \dots, p_k)$ of
    preimages of $1, 2, \dots, k$ respectively.
    These are simply permutations of $k$ blocks of $[n]$.
\end{proof}

\begin{corollary*} \label{thm:stirling:sum}
    For all $n \in \N$, \[
        \sum_{j=0}^{n} \stirling nj x^{\underline{j}} = x^n
    \]
\end{corollary*}
\begin{proof}
    For $x \in \N$, the RHS counts functions from $[n]$ to $[x]$.
    We split these functions by the size of the image.

    For functions of image size $j$, there are $\binom{x}{j}$ ways to choose
    the image, and $j! \stirling nj$ ways to choose the preimage.
    But $\binom{x}{j} j!$ is precisely $n^{\underline{j}}$.

    Thus both sides agree at infinitely many points, and so they are equal.
\end{proof}

\begin{definition*}[Bell numbers] \label{def:bell}
    The number of set partitions of $[n]$ is called the $n$th \emph{Bell
    number}, denoted $B_n \coloneq \sum_{k=0}^{n} \stirling nk$.
\end{definition*}

\begin{exercise*}
    Prove that $B_{n+1} = \sum_{k=0}^{n} \binom{n}{k} B_k$.
\end{exercise*}
\begin{solution}
    Let $b_k$ be the number of partitions of $[n+1]$ with $n+1$ in a block
    of size $k+1$.
    Then $b_k = \binom{n}{k} B_{n-k}$.
    This gives the desired result by the re-indexing $k \mapsto n-k$.
\end{solution}
Though this seems like a recurrence, it is not (for this course).
By ``recurrence'' we will mean a recurrence relation dependent upon at most
$M$ previous terms, for some fixed $M$.

\section{Permutations as Cycles} \label{sec:perm:cycles}
Let $S_n$ be the set of all permutations of $[n]$.
Recall that any permutation $\pi \in S_n$ can be written as a product of
cycles.
A useful convention is to skip cycles of length $1$.
Thus we write $\sigma = 6754132$ as $(1635)(27)$.
This allows us to consider $\pi$ as just a product (under composition) of
permutations which are cyclic on some subset of $[n]$.
\textit{E.g.} $\pi = (1635) \circ (27)$, where $(27)$ for example is the
permutation which swaps $2$ and $7$ and fixes everything else.

\begin{lemma}
    Let $\sigma \in S_n$ and $j \in [n]$.
    Then there exists an $i \in \N^*$ such that $\sigma^i(j) = j$.
\end{lemma}
\begin{proof}
    Consider the sequence $(\sigma(j), \dots, \sigma^n(j))$.
    If any of these are equal to $j$, we are done.
    Otherwise, by the pigeonhole principle, there exist $k < l$ such that
    $\sigma^k(j) = \sigma^l(j)$.
    Then $\sigma^{l-k}(j) = j$ (since $\sigma$ is a bijection).
\end{proof}
\begin{corollary}
    Let $\sigma \in S_n$.
    Then $\sigma^{n!} = \id$.
\end{corollary}
\begin{proof}
    By the lemma, for each $j \in [n]$, there exists an $i_j \in [n]$ such
    that $\sigma^{i_j}(j) = j$.
    Since $i_j \mid n!$ for all $j$, we have $\sigma^{n!}(j) = j$ for all
    $j$.
\end{proof}

\begin{notation}
    We will write cyclic decompositions of permutations as follows:
    \begin{itemize}
        \item Each cycle has its smallest element first.
        \item Cycles are written in increasing order of their smallest
            elements.
    \end{itemize}
\end{notation}

\begin{definition*}
    The \emph{cycle type} of a permutation $\sigma$, denoted
    $\type(\sigma)$, is the partition formed by arranging its cycle lengths
    in weakly decreasing order.
\end{definition*}
