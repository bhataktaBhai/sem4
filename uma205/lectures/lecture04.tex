\lecture{2024-01-10}{}

\begin{definition}[Cartesian product] \label{def:cartesian_product}
    Let $A$ and $B$ be sets.
    Then \[
        A \times B = \set{(a, b) \mid a \in A, b \in B}
    \] is called the \emph{Cartesian product} of $A$ and $B$.
\end{definition}
This exists by virtue of the axiom of powers (\ref{def:zfc:powers}).

We recall the definition of a relation and some properties.
\begin{definition}[Relation] \label{def:relation}
    Let $A$ and $B$ be sets.
    Then a subset $R$ of $A \times B$ is called a (binary) \emph{relation}
    from $A$ to $B$.
    If $B = A$, we say that $R$ is a relation on $A$.
\end{definition}
\begin{definition*}
    Let $R$ be a relation on a set $A$.
    We say that $R$ is
    \begin{enumerate}
        \item \textbf{reflexive} if $(a, a) \in R$ for all $a \in A$,
        \item \textbf{symmetric} if $(a, b) \in R \implies (b, a) \in R$,
        \item \textbf{antisymmetric} if $(a, b) \in R \land (b, a) \in R
        \implies a = b$,
        \item \textbf{transitive} if $(a, b) \in R \land (b, c) \in R
        \implies (a, c) \in R$.
    \end{enumerate}
    If $R$ satisfies (i), (ii) and (iv), it is said to be an
    \emph{equivalence relation}.
    We write $a \sim_R b$ for $(a, b) \in R$.

    If $R$ satisfies (i), (iii) and (iv), it is a partial order.
    We write $a \le_R b$ or $a \ge_R b$ for $(a, b) \in R$.
\end{definition*}

\begin{definition}[Equivalence class] \label{def:equivalence_class}
    Let $X$ be a set and $\sim_R$ an equivalence relation on $X$.
    The equivalence class associated with $x \in X$ is \[
        [x] = \set{y \in X \mid y \sim_R x}.
    \]
\end{definition}

\begin{definition}[Partition] \label{def:partition}
    A (set) \emph{partition} of a set $X$ is a family $\set{X_\alpha \mid
    \alpha \in I}$, where $I$ is some indexing set, such that,
    \begin{enumerate}
        \item $X_\alpha \cap X_\beta = \O$ for all $\alpha \ne \beta \in I$,
        \item $\bigcup_{\alpha \in I} X_\alpha = X$.
    \end{enumerate}
    This is also written as simply \[
        \bigsqcup_{\alpha \in I} X_\alpha = X.
    \]
\end{definition}

\begin{proposition*}[Fundamental theorem of equivalence relations]
\label{thm:fter}
    Let $X$ be a set and $\sim_R$ an equivalence relation on $X$.
    Then the family of equivalence classes $\set{[x] \mid x \in X}$ forms
    a partition of $X$.
    Conversely, every partition arises from an equivalence relation.
\end{proposition*}
\begin{proof}
    Exercise for the reader.
    % TODO
\end{proof}

\begin{definition}
    Let $X$ be a set and $\sim_R$ an equivalence relation on $X$.
    Then the set $X / \sim_R = \set{[x] \mid x \in X}$ is called the
    \emph{quotient set} of $X$ by $R$.
\end{definition}
\begin{examples}
    \item Consider $\N$ with the relation $a \sim_R b \iff a \equiv b \pmod{3}$.
    The quotient set $\N / R$ is $\set{[0], [1], [2]}$, which is morally the
    same as $\set{0, 1, 2}$.
    \item For any set $A$ with the equality relation $=$, the quotient set
    $A /{=}$ is (morally) the same as $A$.
    \item Consider $\R^2$ with $(x, y) \sim (z, w)$ if $x^2 + y^2 = z^2 + w^2$.
    Then $\R^2 /{\sim} = \set{[(r, 0)] \mid r \in \R}$ which is morally just
    the set of non-negative reals.
\end{examples}

\begin{definition*}[Function] \label{def:fn}
    Let $A$ and $B$ be sets.
    A relation $f$ from $A$ to $B$ is said to be a \emph{function} if for
    all $a \in A$, there exists a unique $b \in B$ such that $(a, b) \in f$.

    $A$ is said to be the \emph{domain}, $B$ is said to be the \emph{range}
    or \emph{codomain} of $f$.
    For a subset $C \subseteq A$, the image of $C$ under $f$ is
    $f(C) = \set{f(a) \mid a \in C}$.

    For a subset $D \subseteq B$, the \emph{preimage} or \emph{inverse image} of
    $D$ under $f$ is $f^{-1}(D) = \set{a \in A \mid f(a) \in D}$.
\end{definition*}
Note that $f(C)$ exists by the axiom of replacement.
\begin{examples}
    \item $A = B = \N$, $f(a) = a\pp$.
    Then $f(\N) = \N \setminus \set{0}$. \[
        f^{-1}(\set{a}) = \begin{cases}
            \set{a - 1} & \text{if } a > 0 \\
            \O & \text{if } a = 0
        \end{cases}
    \]
\end{examples}

\begin{definition} \label{def:fn:equality}
    Two functions $f$ and $g$ with the same domain $X$ and range $Y$ are
    equal if $f(x) = g(x)$ for all $x \in X$.
\end{definition}

\begin{definition}[Composition] \label{def:fn:composition}
    If $f : X \to Y$ and $g : Y \to Z$, then the \emph{composition} $g \circ f$
    is a function $g \circ f \colon X \to Z$ given by \[
        (g \circ f)(x) = g(f(x)).
    \]
\end{definition}
