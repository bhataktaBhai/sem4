\lecture{28}{Mon 18 Mar '24}{}
\begin{definition}[Primitive root] \label{def:nt:primitive_root}
    An integer $a \in \Z/n\Z$ is called a \emph{primitive root modulo $n$}
    if it generates $U(\Z/n\Z)$.
\end{definition}
\begin{examples}
    \item 2 is a primitive root modulo 5, but not modulo 7.
    \item There are no primitive roots modulo 12.
\end{examples}

\begin{fact} \label{thm:nt:facts} \leavevmode
    \begin{enumerate}
        \item If $p$ is an odd prime, then $U(\Z/p^l\Z)$ is cyclic.
        \item $U(\Z/2\Z)$ and $U(\Z/4\Z)$ are cyclic, but for $l \ge 3$,
        $U(\Z/2^l\Z)$ is a product of cyclic groups of order
        $2$ and $2^{l-2}$.
    \end{enumerate}
\end{fact}

\begin{theorem}
    Let $n = 2^a p_1^{a_1} \dots p_l^{a_l}$ be the prime decomposition
    of $n$.
    Then \[
        U(\Z/n\Z) \cong U(\Z/2^a\Z) \times \underbrace{U(\Z/p_1^{a_1}\Z) \times \dots \times U(\Z/p_l^{a_l}\Z)}_{\text{cyclic}}
    \]
\end{theorem}
\begin{corollary}
    An integer possesses primitive roots iff $n$ is of the form $2$, $4$,
    $p^a$, or $2p^a$ for some odd prime $p$.
\end{corollary}

\chapter{Quadratic Residues} \label{chp:quad_res}
\begin{definition}[Quadratic residue] \label{def:quad_res}
    Let $(a, m) = 1$.
    Then $a$ is called a \emph{(quadratic) residue modulo $m$} if there
    is an $x \in \Z/m\Z$ such that $x^2 \equiv a \pmod{m}$.

    If not, we say that $a$ is a \emph{(quadratic) non-residue modulo $m$}.
\end{definition}
\begin{example}
    For $m = 7$, \[
        1^2, 2^2, \dots, 6^2 \equiv 1, 4, 2, 2, 4, 1 \pmod{7}.
    \] So the quadratic residues modulo 7 are precisely $1$, $2$, and $4$.
\end{example}

\begin{proposition} \label{thm:quad_res:condition}
    Let $m = 2^e p_1^{e_1} \dots p_l^{e_l}$ and let $(a, m) = 1$.
    Then $x^2 \equiv a \pmod{m}$ has a solution iff
    \begin{itemize}
        \item if $e = 2$, then $a \equiv 1 \pmod 4$ and if $e \ge 3$, then
        $a \equiv 1 \pmod 8$; and
        \item for each $i$, $a^{(p_i - 1)/2} \equiv 1 \pmod{p_i}$.
    \end{itemize}
\end{proposition}
\begin{proof}
    By the proof of the Chinese Remainder Theorem, $x^2 \equiv a \pmod m$
    has a solution iff \[
        x^2 \equiv a (2^e), \quad x^2 \equiv a (p_1^{e_1}),
            \quad \dots, \quad x^2 \equiv a (p_l^{e_l}) % TODO: check
    \]

    We restrict our attention to $x^2 \equiv a \pmod{2^e}$ to get the first
    condition.
    First note that $1$ is the only quadratic residue modulo $4$ and
    modulo $8$.

    \begin{lemma}
        If $x^2 \equiv a \pmod 8$, where $a$ is odd, has a solution, then
        so does $x^2 \equiv a \pmod{2^e}$ for all $e \ge 3$.
    \end{lemma}
    \begin{proof}
        Induction.
        Suppose $x_0^2 \equiv a \pmod{2^e}$.
        Note that since $a$ is odd, so is $x_0$.
        Take $x_0^2 = s 2^e + a$.

        Let $x_1 = x_0 - s2^{e-1}$.
        Then modulo $2^{e+1}$, \begin{align*}
            x_1^2 &\equiv (x_0 - s2^{e-1})^2 \\
                &\equiv x_0^2 + sx_0 2^e + s^2 2^{2e-2} \\
                &\equiv a + s2^e + sx_0 2^e + s^2 2^{e-3} 2^{e+1} \\
                &\equiv a + s(1 + x_0) 2^e + 0 \\
                &\equiv a
        \end{align*} since $x_0$ is odd.
        Winduction.
    \end{proof}

    \begin{lemma}
        Let $p$ be an odd prime, $p \nmid a$ and $p \nmid n$.
        Then if $x^n \equiv a \pmod p$ has a solution,
        so does $x^n \equiv a \pmod{p^e}$ for all $e \ge 1$.
    \end{lemma}
    \begin{proof}
        Induction.
        If $x_0$ is a solution of $x^n \equiv a \pmod {p^e}$, then
        let $x_0^n = s p + a$ and choose $x_1 = x_0 + b p^{e-1}$, where
        $b$ is chosen so that $b \equiv -sx_0 \pmod p$.
        (This exists since $b$ is non-zero modulo $p$.)
        This gives a solution modulo $p^{e+1}$.
        Winduction.
    \end{proof}

    Thus, it suffices to look at $x^2 \equiv a \pmod{p_i}$.

    Let $g$ be a primitive root modulo $p = p_i$ and let $x = g^y$.
    We wish to solve $g^{2y} = g^b \pmod p$.
    It is enough to solve for $2y \equiv b\pmod{p - 1}$.
    This has a solution if $(2, p - 1) \mid b$.

    If $2 \mid b$, then $a^{\frac{p - 1}{2}} = g^{\frac{b}{2} (p - 1)} \equiv 1 \pmod p$ by Fermat's little theorem.

    Conversely, if $a^{\frac{p - 1}{2}} \equiv 1 \pmod p$,
    then $g^{\frac{b}{2} (p - 1)} \equiv 1 \pmod p$ and so
    $p - 1 \mid \frac{b}{2} (p - 1)$ which implies $2 \mid b$.
\end{proof}

\begin{definition}[Legendre symbol] \label{def:legendre_symbol}
    Let $p$ be an odd prime.
    The \emph{Legendre symbol} is defined as \[
        \lgnd{a}{p} = \begin{cases}
            \hphantom{-}1 & \text{if } a \text{ is a residue modulo } p \\
            -1 & \text{if } a \text{ is a non-residue modulo } p \\
            \hphantom{-}0 & \text{if } p \mid a
        \end{cases}
    \]
\end{definition}

\begin{proposition} \label{thm:legendre} \leavevmode
    \begin{enumerate}
        \item $a^{(p - 1)/2} \equiv \lgnd{a}{p} \pmod p$.
            \label{thm:legendre:p-1}
        \item $\lgnd{ab}{p} = \lgnd{a}{p} \lgnd{b}{p}$.
            \label{thm:legendre:product}
        \item If $a \equiv b \pmod p$, then $\lgnd{a}{p} = \lgnd{b}{p}$.
            \label{thm:legendre:congruence}
    \end{enumerate}
\end{proposition}
\begin{proof}
    If $p \mid a$ or $p \mid b$, then all of these are trivial.
    Thus assume that $p \nmid ab$.
    For (i), we have by Fermat's little theorem that \begin{align*}
        a^{p - 1} &\equiv 1 \pmod p \\
        (a^{(p - 1)/2} - 1)(a^{(p - 1)/2} + 1) &\equiv 0 \pmod p
    \end{align*} but since $\Z/p\Z$ is a field,
    one of these is zero.
    So $a^{(p - 1)/2} \equiv \pm 1 \pmod p$.
    By \cref{thm:quad_res:condition}, $a^{(p - 1)/2} \equiv 1 \pmod p$
    iff $a$ is a residue modulo $p$.

    For (ii), we have using (i) that \begin{align*}
        (ab)^{(p - 1)/2} &\equiv \lgnd{ab}{p} \pmod p \\
        \shortintertext{and}
        a^{(p - 1)/2} b^{(p - 1)/2} &\equiv \lgnd{a}{p} \lgnd{b}{p} \pmod p
    \end{align*} so $\lgnd{ab}{p} = \lgnd{a}{p} \lgnd{b}{p}$.

    (iii) is obvious from the definition.
\end{proof}
