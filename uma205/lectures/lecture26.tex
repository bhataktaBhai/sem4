\lecture{2024-03-13}{}

\begin{proposition}
    There are exactly $\phi(m)$ units in $\Z/m\Z$.
\end{proposition}
\begin{proof}
    $a \in \Z/m\Z$ is a unit iff $ax \equiv 1 \pmod{m}$ for some
    $x \in \Z/m\Z$.
    This is equivalent to $(a, m) = 1$, and there are $\phi(m)$
    such $a$ in $\set{0, 1, \dots, m-1}$.
\end{proof}

\begin{corollary}
    $\Z/p\Z$ is a field iff $p$ is prime.
\end{corollary}
\begin{proof}
    If $p$ is prime, then every element is a unit.

    Conversely, if $p = p_1 p_2$,
    then $\wbar{p_1}, \wbar{p_2} \ne \wbar{0}$,
    but $\wbar{p_1 p_2} = \wbar{0}$.
    So $\Z/p\Z$ is not a field.
\end{proof}

\begin{notation}
    We will denote by $U(\Z/m\Z)$ the set of units in $\Z/m\Z$.
\end{notation}
\begin{lemma}
    $U(\Z/m\Z)$ forms a group under multiplication.
\end{lemma}
\begin{proof}
    If $a$ and $b$ are units, then so is $ab$.

    $1$ is a unit and an identity.

    If $a$ is a unit, there exists a unique $x$ such that
    $ax \equiv 1 \pmod{m}$.
    Then $x$ is a unit and the unique inverse of $a$.
\end{proof}

\begin{theorem}[Euler] \label{thm:congruences:euler}
    If $(a, m) = 1$, then $a^{\phi(m)} \equiv 1 \pmod{m}$.
\end{theorem}
\begin{proof}
    By the previous lemma, $a \in G = U(\Z/m\Z)$ and $|G| = \phi(m)$.
    Consider the map $\psi\colon G \to G$ given by $\psi(x) = ax$.

    \textbf{Claim:} $\psi$ is a bijection. \\
    \textbf{Proof of claim:}
    Since $G$ is a group, the inverse of $a$ exists.
    Suffices to show that $\psi$ is injective (finite set).
    $\psi(x) = \psi(y) \iff ax = ay \iff x = y$.

    Using this claim, we can write \begin{align*}
        \prod_{x \in G} ax &= \prod_{x \in G} x \\
        a^{\phi(m)} \prod_{x \in G} x &= \prod_{x \in G} x \\
        a^{\phi(m)} &= 1 \qedhere
    \end{align*}
\end{proof}
\begin{corollary}[Fermat's little theorem] \label{thm:congruences:fermat}
    If $p$ is prime and $p \nmid a$, then $a^{p-1} \equiv 1 \pmod{p}$.
\end{corollary}
\begin{proof}
    $\phi(p) = p - 1$.
\end{proof}

\begin{lemma} \label{thm:cong:coprime_product}
    If $a_1$, $a_2$, \dots, $a_j$ are coprime to $m$, then so is
    $a_1 a_2 \dots a_j$.
\end{lemma}
\begin{proof}
    They are all units, so their product is a unit.
\end{proof}

\begin{lemma} \label{thm:cong:factor_product}
    If $a_1$, $a_2$, \dots, $a_j$ divide $m$ and $(a_i, a_j) = 1$ for all
    $i \ne j$, then $a_1 a_2 \dots a_j$ divides $m$.
\end{lemma}
\begin{proof}
    Induction.
    The base case $j = 1$ is obvious.

    Suppose the statement is true for $a_1$, $a_2$, \dots, $a_{j-1}$.
    Then by the previous lemma, $a_1 a_2 \dots a_{j-1}$ is coprime to $a_j$.
    So we can write $r \cdot a_1 \dots a_{j-1} + s \cdot a_j = 1$.
    Multiplying by $m$, we get \[
        r \cdot a_1 \dots a_{j-1} m + s \cdot a_j m = m.
    \] But $a_j$ divides the $m$ in the first term, and by the induction
    hypothesis, $a_1 \dots a_{j-1}$ divides the $m$ in the second term.
    So $a_1 \dots a_j$ divides $m$.
\end{proof}

\begin{theorem}[Chinese remainder theorem] \label{thm:congruences:chinese}
    Write $m = m_1 \dots m_k$ with $(m_i, m_j) = 1$ for all $i \ne j$.
    Let $b_1, \dots, b_j \in \Z$ and consider the system of congruences
    \begin{align*}
        x &\equiv b_1 \pmod{m_1} \\
        x &\equiv b_2 \pmod{m_2} \\
        &\mathrel{\makebox[\widthof{$\equiv$}]{\vdots}} \\
        x &\equiv b_k \pmod{m_k}.
    \end{align*}
    Then the system always has solutions and any two solutions differ by a
    multiple of $m$.
\end{theorem}
\begin{proof}
    Let $n_i = \frac{m}{m_i} = m_1 \dots m_{i-1} m_{i+1} \dots m_k$.
    Each $m_j$, $j \ne i$, is coprime to $m_i$, so
    by \cref{thm:cong:coprime_product}, $(m_i, n_i) = 1$.
    Thus we have $r_i$ and $s_i$ such that $r_i m_i + s_i n_i = 1$.
    Let $e_i = s_i n_i$.
    Then $e_i \equiv 1 \pmod{m_i}$.
    Since each $m_j \ne n_j$ divides $m$,
    $e_i \equiv 0 \pmod{m_j}$ for all $j \ne i$.

    This gives a solution \[
        x_0 = b_1 e_1 + b_2 e_2 + \dots + b_k e_k.
    \]

    Suppose $x_1$ is another solution.
    Then $x_1 - x_0 \equiv 0 \pmod{m_i}$ for all $i$.
    So each of $m_1$, $m_2$, \dots, $m_k$ divides $x_1 - x_0$.
    By \cref{thm:cong:factor_product}, $m$ divides $x_1 - x_0$.
\end{proof}
\begin{example}[Original example of Sunzi]
    A certain number leaves a remainder of $2$ when divided by $3$,
    a remainder of $3$ when divided by $5$, and a remainder of $2$
    when divided by $7$. What is the number?

    We have \begin{align*}
        m_1 &= 3 & m_2 &= 5 & m_3 &= 7 \\
        b_1 &= 2 & b_2 &= 3 & b_3 &= 2 \\
        \shortintertext{and compute}
        n_1 &= 35 & n_2 &= 21 & n_3 &= 15.
    \end{align*}
    We want \begin{align*}
        3 r_1 + 35 s_1 &= 1 & 5 r_2 + 21 s_2 &= 1 & 7 r_3 + 15 s_3 &= 1.
    \end{align*}
    One solution is \begin{align*}
        r_1, s_1 &= 12, -1 & r_2, s_2 &= -4, 1 & r_3, s_3 &= -2, 1.
    \end{align*}
    This gives \begin{align*}
        e_1 &= -35 & e_2 &= 21 & e_3 &= 15,
    \end{align*}
    and finally the solution \begin{align*}
        x &= 2 (-35) + 3 (21) + 2 (15) \\
        &= -70 + 63 + 30 \\
        &= 23.
    \end{align*}
\end{example}
