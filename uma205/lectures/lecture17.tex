\lecture{2024-02-09}{}
\begin{proposition}
    Let $T$ be a tree.
    \begin{enumerate}[label=(\arabic*)]
        \item Deleting any edge in $T$ disconnects it.
        \item Adding a new edge to $T$ creates a cycle.
        \item For any $v, w \in V(T)$, there is a \emph{unique} path from
        $u$ to $v$.
    \end{enumerate}
\end{proposition}
\begin{proof}
    (3) holds because if there is more than one path from $u$ to $v$,
    then we must create a cycle.

    For (2), suppose $\set{v, w} \notin E(T)$ and we add it.
    By (3), there is a unique path from $v$ to $w$ in $E(T)$, and so adding
    this edge creates a cycle.

    For (1), suppose that removing an edge $\set{v, w}$ from $T$ still left
    it connected.
    Then we would have two paths from $v$ to $w$, contradicting (3).
\end{proof}

\begin{definition}
    A vertex with degree $1$ is called a \emph{leaf} or
    \emph{pendant vertex}.
\end{definition}

\begin{lemma}
    Every tree on $n \ge 2$ vertices has at least $2$ leaves.
\end{lemma}
\begin{proof}
    Let the longest path in the tree be $(v_1, \dots, v_k)$.
    Then $v_1$ and $v_k$ must be leaves, for otherwise the path could be
    made longer.
\end{proof}

\begin{theorem}
    All trees on $n$ vertices have $n-1$ edges.
\end{theorem}
\begin{proof}
    This is clearly true for the singleton tree.
    Let $T$ be a tree with $n+1$ vertices, and let $l$ be a leaf (by the
    previous lemma).
    Removing $l$ and its incident edge gives a tree of $n$ vertices
    with $n-1$ edges.
    Thus, $T$ has $n$ edges.
    Winduction.
\end{proof}

\begin{theorem}
    Any connected graph on $n$ vertices with $n-1$ edges is a tree.
\end{theorem}
\begin{proof}
    True for $n=1$.
    %TODO
\end{proof}

\begin{definition} \label{def:forest}
    A \emph{forest} is a graph with no cycles.
\end{definition}
A tree is a connected forest.

We wish to count the number of trees on vertices labelled $[n]$.
\begin{examples}
    \item For $n = 3$, first note that there is exactly one unlabelled
    tree,
    \begin{center}
        \begin{tikzpicture}
            \graph[empty nodes, nodes={circle, draw}]{
                1 -- 2 -- 3;
            };
        \end{tikzpicture}
    \end{center}
    This gives rise to $3$ labelled trees
    \begin{center}
        \begin{tikzpicture}
            \graph[nodes={circle, draw}]{ 1 -- 2 -- 3; };
        \end{tikzpicture}
        \quad
        \begin{tikzpicture}
            \graph[nodes={circle, draw}]{ 1 -- 3 -- 2; };
        \end{tikzpicture}
        \quad
        \begin{tikzpicture}
            \graph[nodes={circle, draw}]{ 2 -- 1 -- 3; };
        \end{tikzpicture}
    \end{center}
    \item For $n = 4$, there are $2$ unlabelled trees.
    \begin{center}
        \begin{tikzpicture}
            \graph[empty nodes, nodes={circle, draw}]{
                1 -- 2 -- 3 -- 4;
            };
        \end{tikzpicture}
        \quad
        \begin{tikzpicture}
            \graph[empty nodes, nodes={circle, draw}]{
                1 -- 2 -- 3;
                2 -- 4;
            };
        \end{tikzpicture}
    \end{center}
    The first tree gives rise to $4!/2 = 12$ labelled trees, and the
    second gives rise to $4$ labelled trees.
    Total: $16$ labelled trees.
    \item For $n = 5$, there are $3$ unlabelled trees.
    \begin{center}
        \begin{tikzpicture}
            \graph[empty nodes, nodes={circle, draw}]{
                1 -- 2 -- 3 -- 4 -- 5;
            };
        \end{tikzpicture}
        \quad
        \begin{tikzpicture}
            \graph[empty nodes, nodes={circle, draw}]{
                1 -- 2 -- 3 -- 4;
                2 -- 5;
            };
        \end{tikzpicture}
        \quad
        \begin{tikzpicture}
            \graph[empty nodes, nodes={circle, draw}]{
                1 -- 2 -- 3;
                2 -- 4;
                2 -- 5;
            };
        \end{tikzpicture}
    \end{center}
    The first tree gives rise to $5!/2 = 60$ labelled trees, the second
    to $5!/2! = 60$, and the last to $5$.
    Total: $125$ labelled trees.
\end{examples}
We observe the pattern that the number of labelled trees on $n$ vertices
is $n^{n-2}$.

\begin{theorem*}[Cayley's formula] \label{thm:tree:cayley}
    The number of trees labelled $[n]$ is $n^{n-2}$.
\end{theorem*}
\begin{remark}
    The book presents a bijective proof.
\end{remark}

\begin{definition*}
    A \emph{rooted tree} $T(v)$ is a tree with a marked vertex $v$, called
    the \emph{root}.

    A \emph{branching} of a rooted tree $T(v)$ is an orientation of $T$,
    \ie, an assignment of directions to the edges of $T$, in
    which every edge is directed away from $v$.

    A \emph{rooted forest} is one where every component has a root,
    and we can constuct branchings in the same way.
\end{definition*}
% For a rooted tree $T(v)$, there is exactly one branching.

We will show that the number of branchings which is equal to the number of
rooted trees is $n^{n-1}$.
\begin{proof}[Proof of Cayley's formula]
    We start with the empty graph over $n$ vertices and add edges one at a
    time to form a branching.
    Initially, there are $n$ components.
    At the $k$\textsuperscript{th} stage, we will have $n - k$ components.
    Consider the following algorithm:

    \begin{center}
        \begin{minipage}{0.8\textwidth}
            For $1 \le k \le n - 1$, at the $k$\textsuperscript{th} stage,
            add an oriented edge $(u, v)$ from any vertex to the root of
            one of the components to which it does not belong.
        \end{minipage}
    \end{center}

    At the first stage, we have $n$ choices for $u$ and $n-1$ choices for
    $v$.
    At the second stage, we have $n$ choices for $u$ and $n-2$ choices for
    $v$, and so on.
    Thus at the $k$\textsuperscript{th} stage, we have $n(n-k)$ ways of
    forming a rooted forest.
    Continuing this way, we get that the number of branchings is
    $n^{n-1} (n-1)!$.

    But note that every branching occurs $(n-1)!$ times, because of
    different orderings of the edges.
    Thus the total number of rooted trees is $n^{n-1}$.
\end{proof}
Cayley's formula follows as a corollary.
(A factor of $n$ comes from the choice of root.)

\begin{exercise}
    The number of rooted forests on $n$ vertices is $(n+1)^{n-1}$.
\end{exercise}
\begin{proof}
    Introduce a special vertex $v_{-1}$, and consider all rooted trees
    on $n+1$ vertices with root $v_{-1}$.
    Removing $v_{-1}$ gives a rooted forest on $n$ vertices, and every
    rooted forest on $n$ vertices arises in this way.
    Thus by Cayley's formula, the number of rooted forests on $n$
    vertices is $(n+1)^{n-1}$.
\end{proof}

\begin{definition}
    Let $G = (V, E)$ and $\abs{V} = n$.
    The \emph{adjacency matrix} $A$ is the $n \times n$ matrix indexed by
    $V$ whose entries are \[
        A_{v,w} = \1{\set{v, w} \in E}.
    \]
\end{definition}

\begin{proposition*}
    Let $G$ be a graph and $A$ be its adjacency matrix.
    Then $(A^k)_{v,w}$ counts the number of walks from $v$ to $w$ of length
    $k$.
\end{proposition*}
