\chapter{Number Theory} \label{chp:nt}
\lecture{2024-02-26}{}

\section{Algebraic Structures} \label{sec:nt:algebraic_structures}
\begin{definition*}[Group] \label{def:group}
    A \emph{group} is a set $G$ together with a binary operation
    $*\colon G \times G \to G$ which satisfies the following conditions:
    \begin{enumerate}[label=\small(G\arabic*)]
        \item \textbf{associativity:}
            $(a * b) * c = a * (b * c)$ for all $a, b, c \in G$.
        \item \textbf{identity:}
            there exists an element $e \in G$, called the \emph{identity},
            such that $a * e = e * a = a$ for all $a \in G$.
        \item \textbf{inverses:}
            for all $a \in G$, there exists an element $a^{-1} \in G$
            called the \emph{inverse} of $a$, such that
            $a * a^{-1} = a^{-1} * a = e$.
    \end{enumerate}
\end{definition*}
\begin{notation}
    We write $ab$ for $a * b$.
\end{notation}
\begin{examples}
    \item $(\Z, +)$, which has $e = 0$ and $a^{-1} = -a$.
    Similarly, $(\Q, +)$, $(\R, +)$ or $(\C, +)$ are groups.
    \item $(\Q^*, *)$, which has $e = 1$
    and $a^{-1} = 1/a$.
    Similarly, $(\R^*, *)$ and $(\C^*, *)$ are groups.
    But $(\Z^*, *)$ is not a group as it lacks inverses.
    \item Let $(V, +)$ be a vector space over any field.
    Then it is also a group.
    \item Let $S_n$ be the set of all permutations on $[n]$.
    Then $(S_n, \circ)$ is a group.
    In fact, it is the only non-abelian (for $n \ge 3$) group in this list.
\end{examples}
\begin{definition}[Abelian group] \label{def:group:abelian}
    A group is said to be \emph{abelian} if the group operation is
    commutative.
\end{definition}
\begin{definition}[Subgroup] \label{def:group:sub}
    Let $(G, *)$ be a group.
    A \emph{subgroup} of $G$ is a non-empty subset $H \subseteq G$ which
    is closed under products and inverses.
\end{definition}
\begin{example}
    $(2\,\Z, +)$ is a subgroup of $(\Z, +)$.
\end{example}

\begin{definition*}[Ring] \label{def:ring_again}
    A \emph{ring} is a set $R$ together with two binary operations,
    $+\colon R \to R$ and $*\colon R \times R \to R$ called addition and
    multiplication, respectively, which satisfy the following conditions:
    \begin{enumerate}[label=\small(R\arabic*)]
        \item $(R, +)$ is an abelian group.
        \item \textbf{associativity:}
            $(a * b) * c = a * (b * c)$ for all $a, b, c \in R$.
        \item \textbf{distributivity:}
            $(a + b) * c = a * c + b * c$ and
            $a * (b + c) = a * b + a * c$ for all $a, b, c \in R$.
    \end{enumerate}
    $R$ is said to be \emph{commutative} if it saisfies
    \begin{enumerate}[label=\small(R\arabic*)]
        \setcounter{enumi}{3}
        \item \textbf{commutativity:}
            $a * b = b * a$ for all $a, b \in R$,
    \end{enumerate}
    and \emph{with identity} if
    \begin{enumerate}[label=\small(R\arabic*)]
        \setcounter{enumi}{4}
        \item \textbf{identity:}
            there exists an element $1 \in R$ such that
            $1 * a = a * 1 = a$ for all $a \in R$.
    \end{enumerate}
    A subring of $R$ is a subgroup of $R$ (under addition) which is closed
    under multiplication.
\end{definition*}
\begin{examples}
    \item $(\Z, +, *)$ is a commutative ring with identity.
    \item $(2\,\Z, +, *)$ is a commutative ring without identity.
    It sits inside $(\Z, +, *)$ as a subring.
    \item Let $M_n(F)$ be the set of all $n \times n$ matrices with entries
    in a field $F$.
    Then $(M_n(F), +, *)$ is a non-commutative ring with identity.
    \item $(\R^\R, +, *)$ is a commutative ring with identity, where
    addition and multiplication are defined pointwise.
\end{examples}

\section{Primes} \label{sec:nt:primes}
\begin{definition*}[Prime] \label{def:prime}
    An integer $p$ is said to be \emph{prime} if it has no non-trivial
    divisors (other than $\pm 1$ and $\pm p$).
    We additionally define $1$ and $-1$ to not be primes.
\end{definition*}
\begin{notation}
    We write $a \mid b$ if $b$ is divisible by $a$ and $a \nmid b$ if not.
    That is, $a \mid b \iff \exists k (ka = b)$.
    However, we leave $0 \mid 0$ undefined.
\end{notation}
We assume some standard facts about divisibility:
\begin{itemize}
    \item for all $a \ne 0$, $a \mid a$.
    \item if $a \mid b$ and $b \mid a$, then $a = \pm b$.
    \item if $a \mid b$ and $b \mid c$, then $a \mid c$.
    \item if $a \mid b$ and $a \mid c$, then $a \mid (b + c)$.
\end{itemize}
\begin{definition*}[Order and valuation] \label{def:prime_order}
    Suppose $n \in \Z$ and $p$ is a prime such that $p^\alpha \mid n$ but
    $p^{\alpha + 1} \nmid n$.
    Then we call $\alpha$ the \emph{order of $n$ at $p$} or the
    \emph{$p$-adic valuation of $n$}, written $\ord_p(n) = \alpha$.
\end{definition*}

\subsection{Unique prime factorization} \label{sec:nt:ftoa}
\begin{lemma*}
    If $a, b \in \Z$ and $b > 0$, then there exist $q, r \in \Z$ such that
    $a = qb + r$ and $0 \le r < b$.
\end{lemma*}
\begin{proof}
    Take the supremum of the set $X = \set{k \in \Z, kb \le a}$ to be $q$
    and $r = a - qb$.
    Since $q \in X$, we have $0 \le r$.
    Since $q + 1 \notin X$, we have $r < b$.
\end{proof}

\begin{definition}
    Let $a_1, \dots, a_n \in \Z$.
    Define $A = (a_1, \dots, a_n)$ to be the set of all linear combinations
    of $a_1, \dots, a_n$ over $\Z$.
\end{definition}
Note that if $a, b \in A$, then $a \pm b \in A$ and $ka \in A$ for all
$k \in \Z$.
This motivates the following generalisation:
\begin{definition*}[Ideal] \label{def:ring:ideal}
    A \emph{left} (resp. \emph{right}) \emph{ideal} of a ring $R$
    is a subring $I \subseteq R$ which is closed under multiplication
    by elements of $R$ on the left (resp. right).
    That is, for all $a \in I$ and $r \in R$, we have $ra \in I$ (resp.
    $ar \in I$).

    For a commutative ring, we simply call it an \emph{ideal}.
\end{definition*}
\begin{lemma*}
    If $a, b \in \Z$, then there exists a $d \in \Z$ such that
    $(a, b) = (d)$.
\end{lemma*}
\begin{proof}
    Assume that both $a$ and $b$ are non-zero.
    Then there exist positive elements in $(a, b)$.
    Let $d$ be the smallest positive element in $(a, b)$.
    Then $(d) \subseteq (a, b)$.

    For the reverse inclusion, suppose $c \in (a, b)$.
    Apply the previous lemma to conclude the existence of $q$ and $r$ such
    that $c = qd + r$ with $0 \le r < d$.
    Since $c, d \in (a, b)$, $r = c - qd \in (a, b)$.
    But $d$ is the smallest positive element in $(a, b)$, so $r$ cannot be
    positive.
    Thus $r = 0$ and $c = qd \in (d)$.
\end{proof}

\begin{definition}[GCD] \label{def:gcd}
    Let $(a, b) \in \Z$.
    An integer $d$ is a \emph{greatest common divisor} (gcd) of $a$ and $b$
    if it divides both $a$ and $b$, and any other common divisor also
    divides $d$.
\end{definition}
