\lecture{05}{Fri 12 Jan '24}{}

\begin{definition}
    A function $f\colon A \to B$ is said to be
    \begin{itemize}
        \item \emph{injective}, if $f(x) = f(y)$ implies $x = y$,
        \item \emph{surjective}, if $f(A) = B$,
        \item \emph{bijective}, if it is both injective and surjective.
        \item an \emph{involution}, if $f(f(x)) = x$ for all $x \in A$.
    \end{itemize}
\end{definition}
\begin{exercise}
    Let $f\colon A \to B$ be an involution.
    Show that $f$ is bijective.
\end{exercise}
\begin{solution}
    $f$ is surjective since everything is in the range.
    Injective since $f(x) = f(y) \implies f(f(x)) = f(f(y)) \implies x = y$.
\end{solution}

A function is bijective iff for any $b \in B$ there is a unique $a \in A$
such that $f(a) = b$.
\begin{definition}
    Let $f\colon A \to B$ be bijective.
    The \emph{inverse} of $f$ is the function $f^{-1}\colon B \to A$
    where $f^{-1}(b)$ is the unique $a \in A$ such that $f(a) = b$.
\end{definition}

\begin{axiom}[Powers] \label{def:zfc:powers}
    Let $X$ and $Y$ be sets.
    Then there exists a set, denoted $Y^X$, consisting of all functions
    from $X \to Y$.
\end{axiom}
\begin{exercise}
    Let $X$ be a set.
    Then $\set{Y \mid Y \subseteq X}$ is also a set.
\end{exercise}
\begin{solution}
    The property $P(F, X_F)$ given by \[
        P(F, X_F) \iff F \in 2^X \land X_F = \set{x \in X \mid F(x) = 1}
    \] is satisfied by at most one $X_F$ for any $F$.
    Thus applying the axiom of replacement on $2^S$ gives the desired set.
\end{solution}

\begin{axiom}[Unions] \label{def:zfc:unions}
    Let $A$ be a set whose elements are also sets.
    Then there exists a set, denoted $\bigcup A$, whose elements are the
    elements of the elements of $A$.
    Thus $x \in \bigcup A \iff x \in S$ for some $S \in A$.
\end{axiom}
\begin{remark}
    This axiom implies \cref{def:zfc:pairwise_union}.

    Let $I$ be a set such that $A_\alpha$ is a set for all $\alpha \in I$.
    Then $\set{A_\alpha \mid \alpha \in I}$ is a set by the axiom of
    replacement.
    Thus $\bigcup_{\alpha \in I} A_\alpha$ is a set.
\end{remark}

\begin{definition}
    Two sets $X$ and $Y$ are said to have the same \emph{cardinality} if
    there exists a bijection $f\colon X \to Y$.

    Let $n \in \N$.
    If a set $X$ has the same cardinality as $\set{0, 1, \ldots, n - 1}$,
    then $X$ is said to be \emph{finite} and have cardinality $n$.
\end{definition}

\begin{definition}
    A set $X$ is \emph{countably infinite} or \emph{countable} if it has the
    same cardinality as $\N$, is \emph{at most countable} if it is finite or
    countable, and is \emph{uncountable} otherwise.
\end{definition}
\begin{exercise}
    Let $m < n$ be naturals.
    Show that there is
    \begin{enumerate}
        \item no surjection from $[m]$ to $[n]$\footnotemark.
        \item no injection from $[n]$ to $[m]$.
        \item a bijection from $[a]$ to $[b]$ iff $a = b$.
    \end{enumerate}
\end{exercise}
\footnotetext{$[n] = \set{1, \dots, n}$}

\begin{exercise}[Properties of countable sets] \leavevmode
    \begin{enumerate}
        \item If $X$ and $Y$ are countable, then so is $X \cup Y$.
        \item The set $\set{(n, m) \in \N \times \N \mid 0 \le m \le n}$ is
            countable.
        \item $\N \times \N$ is countable.
    \end{enumerate}
\end{exercise}
\begin{theorem}
    Let $X$ be an arbitrary set.
    Then $X$ and $2^X$ cannot have the same cardinality.
\end{theorem}
\begin{proof}
    Let $f\colon X \to 2^X$.
    Consider $A = \set{x \in X \mid x \notin f(x)} \subseteq X$.
    So $A \in 2^X$.
    Since for any $x \in X$, $x \in A \iff x \notin f(x)$, we have
    $f(x) \ne A$ for all $x \in X$.
    Thus $f$ is not surjective.
\end{proof}
