\lecture{09}{Mon 23 Jan '24}{}
Recall the statement of strong induction provided in
\cref{thm:strong_induction}.
\begin{theorem*}[Strong induction]
    Let $m_0 \in \N$ and let $P(m)$ be a property for all $m \in \N$.
    Suppose for all $m \ge m_0$, we have the following:
    if $P(m')$ holds for all $m_0 \le m' < m$, then $P(m)$ holds.
    Then $P(m)$ holds for all $m \ge m_0$.
\end{theorem*}
\begin{example}
    Consider the following reccurrence of order $2$: \begin{align*}
        f(0) &= 1 \\
        f(1) &= 2 \\
        f(n+1) &= f(n-1) + 2f(n), \quad \text{for all } n \ge 1.
    \end{align*}
    Then, $f(n) \le 3^n$ for all $n \in \N$.
    \begin{proof}
        The base cases $n = 0$ and $n = 1$ are true by observation.
        Now suppose $f(m) \le 3^m$ for all $m < n$.
        Then, \begin{align*}
            f(n) &= f(n-1) + 2f(n-2) \\
            &\le 3^{n-1} + 2 \cdot 3^{n-2} \\
            &= 7 \cdot 3^{n-2} \\
            &< 3^n. \qedhere
        \end{align*}
    \end{proof}
\end{example}

\section{Permutations} \label{sec:permutations}
\begin{definition*}[Permutation] \label{def:permutation}
    The arrangement of at most countable different objects in a linear order
    such that each object occurs exactly once is called a \emph{permutation}.
\end{definition*}
\begin{example}
    Permutations of the set $\set{1, 2, 3}$ are \[
        123, 132, 213, 231, 312, 321.
    \]
\end{example}

\begin{proposition}
    The number of permutations of $n$ objects, denoted $n!$ and read
    ``$n$ factorial'', is equal to \[
        n! = n \cdot (n-1) \cdot (n-2) \cdots 2 \cdot 1.
    \] where by convention, $0! = 1$.
\end{proposition}
\begin{proof}
    To arrange $n$ objects in a linear order, we have $n$ choices for the
    first object.
    Each choice leaves $n-1$ objects to be arranged, so we have
    $n! = n \cdot (n-1)!$ which gives the result by the base case $1! = 1$.
\end{proof}

We have several notations of a permutation.
\begin{itemize}
    \item A bijection $\pi\colon [n] \to [n]$.
    \item Two line notation: \[
        \begin{pmatrix}
            1 & 2 & \dots & n \\
            \pi(1) & \pi(2) & \dots & \pi(n)
        \end{pmatrix}.
    \]
    \item One line notation: \[
        \begin{pmatrix}
            \pi(1) & \pi(2) & \dots & \pi(n)
        \end{pmatrix}
    \]
    \item Cycle notation.
    For example, $(1347)(26)(5)$ is cycle notation for the permutation \[
        \begin{pmatrix}
            1 & 2 & 3 & 4 & 5 & 6 & 7 \\
            3 & 6 & 4 & 7 & 5 & 2 & 1
        \end{pmatrix}.
    \]
\end{itemize}
How fast does $n!$ grow?
\begin{fact*}[Stirling's formula] \label{thm:permutations:stirling}
    $n! \sim \sqrt{2\pi n} \left( \frac{n}{e} \right)^n$, where
    $a_n \sim b_n$ means $\lim\limits_{n \to \infty} \frac{a_n}{b_n} = 1$.
\end{fact*}

\begin{proposition}
    The number of arrangements of length $k$ among $n$ objects where each
    of them appears at most once is $\frac{n!}{(n - k)!}$.
\end{proposition}
\begin{proof}
    We arrange the $n$ objects in a linear order.
    Then the first $k$ of them form an arrangement of length $k$ where each
    object appears at most once.

    For each $k$-arrangement, we have $(n-k)!$ ways to arrange the remaining
    $n-k$ objects.
    Thus each $k$-arrangement is counted $(n-k)!$ times.
    Dividing by $(n-k)!$ gives the result.
\end{proof}
\begin{notation}
    The \emph{Pochhammer symbol} or \emph{rising factorial} is defined as \[
        x^{\wbar{n}} = x(x+1)(x+2) \cdots (x+n-1).
    \] Thus for example, $1^{\wbar{n}} = n!$.
    Similarly, the \emph{falling factorial} is defined as \[
        x^{\underline{n}} = x(x-1)(x-2) \cdots (x-n+1).
    \]
\end{notation}

\begin{definition*}[Multiset permutation] \label{def:permutations:multiset}
    Suppose we have $a_1$ objects of type $1$, $a_2$ objects of type $2$,
    and so on, up to $a_k$ objects of type $k$.
    Then a linear order of these objects is called a \emph{multiset
    permutation}.
\end{definition*}
\begin{proposition}
    The number of multiset permutations of $a_1$ objects of type $1$,
    \dots, $a_k$ objects of type $k$, where $a_1 + a_2 + \cdots + a_k = n$,
    is given by the multinomial coefficient \[
        \binom{n}{a_1, a_2, \dots, a_k} = \frac{n!}{a_1!a_2!\cdots a_k!}.
    \] For the special case of $k = 2$, we have \[
        \binom{n}{j, n-j} \eqcolon \binom{n}{j} = \frac{n!}{j!(n-j)!}.
    \]
\end{proposition}
\begin{proof}
    We arrange the $n$ objects in a linear order.
    We have $n!$ ways to do so.

    How many times is each multiset permutation counted?
    For each $i$, we have $a_i!$ ways to arrange the $a_i$ objects of type
    $i$.
    Dividing by $\prod a_i!$ gives the result.
\end{proof}

\begin{proposition}
    The number of multisets of $k$ distinct objects with repetition allowed
    of length $n$ is $\binom{n+k-1}{n}$.
    This is also denoted as $\multichoose{n}{k}$, read
    ``$n$ multichoose $k$''.
\end{proposition}

\begin{proposition}
    The number of $j$ element subsets of $[n]$ is $\binom{n}{j}$.
\end{proposition}
\begin{corollary}
    $\sum_{j=0}^{n} \binom{n}{j} = 2^n$.
\end{corollary}

\begin{remarks}
    \item We define $\binom{n}{j} = 0$ for $n \in \N$ but
    $j \in \N \setminus \set{0, 1, \dots, n}$.
    \item If $n \in \N$, then we define $\binom{-n}{j}$ to be
    $(-1)^j \binom{n+j-1}{j}$.
    \item We leave $\binom{n}{j}$ undefined for $j \notin \N$.
    \item The binomial theorem states that \[
        (1+x)^r = \sum_{j=0}^{\infty} \binom{r}{j} x^j.
    \] This is a finite sum if $r \in \N$ by the first remark.
    If $r \in \R \setminus \N$, then sum is infinite, where \[
        \binom{r}{k} \eqcolon \frac{r(r-1)\cdots(r-k+1)}{k!}.
    \]
\end{remarks}

\begin{proposition}
    $\binom{n}{k} + \binom{n}{k+1} = \binom{n+1}{k+1}$ for all $n, k \in \N$.
\end{proposition}
The table of binomial coefficients is called Pascal's triangle.
% \begin{center}
%     \begin{tabular}{cccccccccc}
%         & & & & 1 & & & & & \\
%         & & & 1 & & 1 & & & & \\
%         & & 1 & & 2 & & 1 & & & \\
%         & 1 & & 3 & & 3 & & 1 & & \\
%         1 & & 4 & & 6 & & 4 & & 1 & \\
%     \end{tabular}
% \end{center}

% copied from https://tex.stackexchange.com/a/142138
\def\x{\hspace{3ex}}    %BETWEEN TWO 1-DIGIT NUMBERS
\def\y{\hspace{2.45ex}}  %BETWEEN 1 AND 2 DIGIT NUMBERS
\def\z{\hspace{1.9ex}}    %BETWEEN TWO 2-DIGIT NUMBERS
\stackMath
\begin{center}
    \Longstack{
        1\\
        1\x 1\\
        1\x 2\x 1\\
        1\x 3\x 3\x 1\\
        1\x 4\x 6\x 4\x 1\\
        1\x 5\y 10\z 10\y 5\x 1\\
        1\x 6\y 15\z 20\z 15\y 6\x 1
    }
\end{center}

\begin{proposition}[Identities] \label{thm:permutations:identities}
    \leavevmode
    \begin{itemize}
        \item $\sum_{k=0}^{n} (-1)^k \binom{n}{k} = 0$.
        \item $\sum_{k=0}^{n} k \binom{n}{k} = n2^{n-1}$.
        \item $\sum_{k=0}^{n} \binom{n}{k}^2 = \binom{2n}{n}$.
    \end{itemize}
\end{proposition}
\begin{proof}
    Algebraic proofs:
    \begin{itemize}
        \item Consider the expansion of $(1-1)^n$.
        \item Consider the derivative of $(1+x)^n$, evaluated at $x = 1$.
        \item Consider the expansion of $(1+x)^n(1+x)^n$.
    \end{itemize}
\end{proof}

\begin{theorem}[Zhu-Vandermonde theorem] \label{thm:permutations:zhu}
    For all $m, n, r \in \N$, we have \[
        \sum_{k=0}^{r} \binom{m}{k} \binom{n}{r-k} = \binom{m+n}{r}.
    \]
\end{theorem}
\begin{proof}
    Consider a set of $m$ boys and $n$ girls.
    Then the number of ways to choose $r$ out of these is $\binom{m+n}{r}$,
    but also the LHS, where each summand is the number of ways of doing so
    while choosing exactly $k$ boys.

    Or algebraically, compare coefficients of $x^r$ in $(1+x)^m(1+x)^n$ and
    $(1+x)^{m+n}$.
\end{proof}
