\lecture{03}{Mon 08 Jan '24}{}

\chapter{Axioms of Set Theory (ZFC)} \label{sec:zfc}
\begin{definition}[Set] \label{def:zfc:set}
    A set is a well-defined collection of objects, which we call elements.
    We will write $x \in A$ to say that $x$ is an element of $A$.
\end{definition}
Well-defined means that given any object, we can state without ambiguity whether
it is an element of the set or not.

\begin{axiom} \label{def:zfc:sets_are_objects}
    Sets are themselves objects.
    If $A$ and $B$ are sets, it is meaningful to ask whether $A$ is an element
    of $B$.
\end{axiom}

\begin{axiom}[Extensionality] \label{def:zfc:extensionality}
    Two sets $A$ and $B$ are equal, written $A = B$, if every element of $A$ is
    a member of $B$ and vice versa.
\end{axiom}

\begin{axiom}[Existence] \label{def:zfc:existence}
    There exists a set, denoted by $\O$ or $\{\}$, known as the empty set, which
    does not contain any elements, \ie, $x \notin \O$ for all objects $x$.
\end{axiom}

\begin{problem}
    $\O$ is unique.
\end{problem}
\begin{proof}
    Suppose $\O$ and $\O'$ are both empty sets.
    Then $x \in \O \iff x \in \O'$ since both are always false.
\end{proof}

\begin{lemma}[Single choice] \label{thm:zfc:single_choice}
    Let $A$ be a non-empty set.
    Then there exists an object $x$ such that $x \in A$.
\end{lemma}
\begin{proof}
    If not, then $x \notin A$ for all objects $x$ and so $A = \O$.
\end{proof}
Thus, we can choose an element of $A$ to certify its non-emptiness.

\begin{axiom}[Pairing] \label{def:zfc:pairing}
    If $a$ is an object, there exists a set, denoted $\set{a}$, whose only
    element is $a$.
    Similarly, if $a$ and $b$ are objects, there exists a set, denoted
    $\set{a, b}$, whose only elements are $a$ and $b$.
\end{axiom}
For example, we can now construct $\O$, $\set{\O}$, $\set{\set{\O}}$,
$\set{\O, \set{\O}}$, etc, all of which are distinct.

\begin{axiom}[Pairwise union] \label{def:zfc:pairwise_union}
    Given sets $A$ and $B$, there exists a set, denoted $A \cup B$, called the
    union of $A$ and $B$, which consists of exactly the elements in $A$, $B$, or
    both.
\end{axiom}

\begin{problem}
    $A \cup B = B \cup A$.
\end{problem}
\begin{proof}
    By commutativity of $\lor$.
\end{proof}
\begin{problem}
    $(A \cup B) \cup C = A \cup (B \cup C)$.
\end{problem}
\begin{proof}
    By associativity of $\lor$.
\end{proof}

\begin{definition}[Subset] \label{def:zfc:subset}
    $A$ is a subset of $B$ if every element of $A$ is alaso an element of $B$,
    denoted $A \subseteq B$.
\end{definition}

\begin{axiom}[Specification] \label{def:zfc:specification}
    (also called Separation).
    Let $A$ be a set and let $P(x)$ be a property for every $x \in A$.
    Then there exists a set $S = \set{x \in A \mid P(x)}$ where $x \in S$ iff
    $x \in A$ and $P(x)$ is true.
\end{axiom}
We can now define the intersection, $A \cap B$, and difference, $A \setminus B$,
of sets $A$ and $B$.
\begin{definition}
    Let $A$ and $B$ be sets.
    we define the intersection $A \cap B = \set{x \in A \mid x \in B}$ and the
    difference $A \setminus B = \set{x \in A \mid x \notin B}$.

    $A$ and $B$ are said to be disjoint if $A \cap B = \O$.
\end{definition}

Recall that sets form a Boolean algebra under the operations $\cup$, $\cap$, and
$\setminus$.
For example, $A \cap (B \cup C) = (A \cap B) \cup (A \cap C)$, de Morgan's laws,
etc.

\begin{axiom}[Replacement] \label{def:zfc:replacement}
    Let $A$ be a set and let $P(x, y)$ be a property for every $x \in A$ and
    every object $y$, such that for every $x \in A$ there is at most one $y$ for
    which $P(x, y)$ is true.
    Then there exists a set $S = \set{y \mid P(x, y) \text{ is true for some }
    x \in A}$.
    That is, $y \in S$ iff $P(x, y)$ is true for some $x \in A$.
\end{axiom}
\begin{examples}
    \item Let $A = \set{7, 9, 22}$ and $P(x, y) \equiv y = x\pp$.
    Then $S = \set{8, 10, 23}$.
    \item Let $A = \set{7, 9, 22}$ and $P(x, y) \equiv y = 1$.
    Then $S = \set{1}$.
\end{examples}

\begin{axiom}[Infinity] \label{def:zfc:infinity}
    There exists a set, denoted $\N$, whose objects are called natural numbers,
    \ie, an object $0 \in \N$, and $n\pp$ for every $n \in \N$, such that the
    Peano axioms hold.
\end{axiom}

\begin{axiom}[Foundation] \label{def:zfc:foundation}
    (also called Regularity).
    If $A$ is a non-empty set, then there exists at least one $x \in A$ which
    is either not a set or is disjoint from $A$.
\end{axiom}
For example, if $A = \set{\set{1, 2}, \set{1, 2, \set{1, 2}}}$, then
$\set{1, 2}$ is an element of $A$ which is disjoint from $A$.
