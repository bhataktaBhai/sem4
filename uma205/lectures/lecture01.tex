\lecture{01}{Wed 03 Jan '24}{}

\setcounter{section}{-1}
\section{The Course} \label{sec:course}
\textbf{Instructor:} Prof.~Arvind~Ayyer\\
\textbf{Office:} X-15\\
\textbf{Office hours:} TBD\\
\textbf{Lecture hours:} MWF 11:00--11:50\\
\textbf{Tutorial hours:} Tue 9:00--9:50

80\% attendance is mandatory.

\textbf{Prerequisites:} UMA101 and UMA102
\textbf{Texts:} Several
\begin{itemize}
    \item \textit{Analysis I}, Terence Tao.
\end{itemize}

\subsection*{Grading} \label{sec:grading}
\begin{itemize}
    \item[(20\%)] Quizzes on alternate Tuesdays, worst dropped.
    No makeup quizzes, but if a quiz is missed for a medical reason
    (with certificate), that quiz will be dropped.
    \item[(30\%)] Midterm
    \item[(50\%)] Final
\end{itemize}
Homeworks after every class, ungraded.
Exams are closed book and closed notes, with no electronic devices allowed.

\subsection*{Aims of the Course} \label{sec:aims}
\begin{itemize}
    \item Deal with formal mathematical structures.
    \item Learning the axiomatic method.
    \item See how more complicated structures arise from simpler ones.
\end{itemize}

\section{Peano's Axioms} \label{sec:peano}
We try to formulate two fundamental quantities: $0$ and the successor function
$n \mapsto n\pp$.
\begin{enumerate}[label=(P\arabic*)]
    \item $0$ is a natural number. \label{peano:0}
    \item If $n$ is a natural number, so is $n\pp$. \label{peano:closure}
    \item $0$ is not the successor of any natural number.
    \label{peano:no_predecessor}
    \item Different natural numbers have different successors.
    \label{peano:injection}
    \item (Principle of mathematical induction) Let $P(n)$ be any ``property''
    for a natural number $n$.
    Suppose that $P(0)$ is true, and that $P(n\pp)$ is true whenever $P(n)$ is
    true.
    Then $P$ is true for all natural numbers.
    \label{peano:induction}
\end{enumerate}
Denote \emph{the} set of natural numbers by $\N$.
(Any two sets satisfying the Peano axioms are isomorphic.)
Note that $\N$ is itself infinite, but all of its elements are finite.
\begin{proof}
    $0$ is finite.
    If $n$ is finite, then $n\pp$ is finite.
    Thus, by induction, all natural numbers are finite.
    (\textcolor{Red}{But wtf is a finite number?})
\end{proof}
\begin{remark} \leavevmode
    \begin{itemize}
        \item There exist number systems which admit infinite numbers.
        For example, cardinals, ordinals, etc.
        \item This way of thinking is \textit{axiomatic}, but not constructive.
    \end{itemize}
\end{remark}
