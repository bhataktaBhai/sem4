\lecture{06}{Mon 15 Jan '24}{}

\textbf{News:} Quiz 1 tomorrow.
Material upto and including lecture 6.

\begin{definition}
    Let $I$ be a possibly infinite indexing set and for all $\alpha \in I$
    let $X_\alpha$ be a set.
    Then its (possibly infinite) Cartesian product is defined as
    \[
        \prod_{\alpha \in I} X_\alpha = \bigg\{
            (x_\alpha)_{\alpha \in I} \in \Big(\bigcup_{\beta \in I}
            X_\beta\Big)^I \;\bigg\vert\; x_\alpha \in X_\alpha
            \text{ for all } \alpha \in I
        \bigg\}
    \]
\end{definition}

\begin{exercise}
    For any sets $I$ and $X$, $\prod_{\alpha \in I} X = X^I$.
\end{exercise}
\begin{axiom}[Choice] \label{def:zfc:choice}
    Let $I$ be a set and for all $\alpha \in I$ let $X_\alpha \ne \O$.
    Then $\prod_{\alpha \in I} X_\alpha$ is non-empty.
\end{axiom}

\begin{definition}
    A \emph{choice function} on $X$ is a function $f\colon 2^X \setminus \O
    \to X$ such that for all non-empty $S \subseteq X$, $f(S) \in S$.
\end{definition}
\begin{fact}
    The existence of a choice function for every $X$ is equivalent to the
    axiom of choice.
\end{fact}
\begin{remark}
    A variant of AoC is the \emph{axiom of countable choice}, which requires
    $I$ to be at most countable.
\end{remark}

\begin{lemma}
    Let $E$ be a bounded above non-empty subset of \R.
    Then there exists a sequence $(a_n)_{n \in \N}$ such that $a_n \in E$ for
    all $n$ and $\lim_{n \to \infty} a_n = \sup E$.
\end{lemma}
\begin{proof}
    Let $X_n = \set{x \in E \mid \sup E - \frac1n \le x \le \sup E}$.
    Each $X_n$ is non-empty.
    By AoC, there exists a sequence $(a_n)_{n \in \N}$ such that for all
    $n$, $a_n \in X_n$.
    Thus $a_n \in E$ for all $n$ and $\lim_{n \to \infty} a_n = \sup E$.
\end{proof}

\begin{definition}
    Let $(P, \le)$ be a poset.
    A subset $Y \subseteq P$ is called a \emph{chain} or \emph{totally
    ordered} if for any $y, y' \in Y$, either $y \le y'$ or $y' \le y$.
\end{definition}
\begin{definition}
    Let $(P, \le)$ be a poset and $Y \subseteq P$.
    We say that $y$ is a \emph{minimal} (resp. \emph{maximal}) element of
    $Y$ if there is no $y' \in Y$ such that $y' < y$ (resp. $y' > y$).
\end{definition}

\begin{definition}
    Let $(P, \le)$ be a poset and $Y \subseteq P$ be a chain.
    We say that $Y$ is \emph{well-ordered} if every non-empty subset of $Y$
    has a minimal element.
\end{definition}

\setcounter{axiom}{\value{axiom}-1}
\begin{axiom}[Well-ordering principle] \label{def:zfc:well-ordering}
    Given any set $X$, there exists a well-ordering on $X$.
\end{axiom}
\setcounter{axiom}{\value{axiom}-1}
\begin{axiom}[Zorn's lemma] \label{def:zfc:zorn}
    Let $(X, \le)$ be a non-empty poset such that every chain $Y$ of $X$ has
    an upper bound (there exists an $x \in X$ such that $y \le x$ for all
    $y \in Y$).
    Then $X$ has a maximal element.
\end{axiom}

\begin{fact}
    The axiom of choice, well-ordering principle, and Zorn's lemma are
    equivalent.
\end{fact}
\begin{proof}
    \begin{description}
        \item[Zorn $\implies$ AoC.]
            Let $X \ne \O$ and let $P$ be the set of ordered pairs $(Y, f)$
            where $Y \subseteq X$ and $f$ is a choice function on $Y$.
            Define $(Y, f) \le (Y', f')$ if $Y \subseteq Y'$ and
            $f' \vert_Y = f$.
            $P$ is non-empty because $\set{x} \subseteq X$ has a choice
            function for all $x \in X$.

            Let $C$ be a chain in $P$.
            Then let $\overline{Y} = \bigcup_{(Y, f) \in C} Y$ and define
            $\overline{f}$ by setting $\overline{f}(S) = f(S)$ for any $f$
            for which $f(S)$ is defined.
            Then $(\overline{Y}, \overline{f})$ is an upper bound for $C$.

            By Zorn's lemma, there exists a maximal element of $P$, say
            $(Y, f)$.
            If $x \in X \setminus Y$, we can extend $f$ to $Y \cup \set{x}$
            by defining $f(S) = x$ for any $S$ containing $x$.
            This contradicts the maximality of $(Y, f)$.
            Thus $X \setminus Y$ must be empty, and so $f$ is a choice
            function on $X$.
        \item[AoC $\implies$ Zorn.]
            Let $P$ be a poset whose every chain has an upper bound.
            Suppose $P$ has no maximal element.
            Pick $x_0 \in P$ using a choice function.
            Since $x_0$ is not maximal, there exists an $x_1$ larger than
            $x_0$, and $x_2$ larger than $x_1$, and so on.
            This gives a chain $x_0 < x_1 < x_2 < \dots$.
            But then $x_\omega$ is an upper bound for this chain.
            This gives another chain $x_\omega < x_{\omega + 1} < \dots$.
            But then $x_{2\omega}$ is an upper bound for this chain.

            Continuing in this way, we get a chain which is ``larger'' than
            $P$ itself, a contradiction.
    \end{description}
\end{proof}
