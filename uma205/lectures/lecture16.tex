\section{Graphs \& Trees} \label{sec:graphs_trees}
\lecture{16}{Wed 07 Feb '24}{}

\begin{definition}[Simple graph] \label{def:graph}
    A \emph{simple graph} is an ordered pair $G = (V, E)$, where $V$ is a
    set of \emph{vertices} and $E$ is a collection of $2$-element subsets of
    $V$, called \emph{edges}.

    A vertex $v \in V$ is said to be \emph{incident} to an edge $e \in E$ if
    $v \in e$.
    Vertices $v_1, v_2 \in V$ are said to be \emph{adjacent} if
    $\set{v_1, v_2} \in E$.
\end{definition}
In general, we could also have \emph{loops}, which are edges from one vertex
to itself, and \emph{multiple edges}, which are two or more edges between
the same pair of vertices.
One could also have \emph{weighted} edges.
We will not define these formally, since it is too many definitions.

We will also only focus on finite graphs, \ie, graphs with a finite number
of vertices (which implies a finite number of edges).

\begin{definition}[Walks, paths and trails] \label{def:graph:walk}
    \leavevmode
    \begin{itemize}
        \item A \emph{walk} is a sequence of vertices
        $(v_1, v_2, \dots, v_n)$, where $v_i$ is adjacent to $v_{i+1}$ for
        $1 \leq i < n$.
        \item A walk is a \emph{trail} if all the edges are distinct.
        \begin{itemize}
            \item A trail is said to be \emph{Eulerian} if all the edges of
            the graph are used in it.
        \end{itemize}
        \item A walk is a \emph{path} if all the vertices are distinct.
        \item A walk, trail, or path is \emph{closed} if $v_1 = v_n$.
        \item A closed path\footnotemark\ is a \emph{cycle}.
    \end{itemize}
\end{definition}
\footnotetext{How can a path be closed?
    We mean that the first and last vertices are the same,
    but no other vertex repeats.}

\begin{definition}[Connectedness] \label{def:graph:connected}
    A graph is \emph{connected} if there is a path between any two vertices.
\end{definition}

\begin{definition}[Degree] \label{def:graph:degree}
    The \emph{degree} of a vertex $v$ in a graph $G = (V, E)$ (possibly
    with loops or parallel edges) is the number of edges incident to $v$.
    An edge from $v$ to itself counts twice.
\end{definition}
\begin{theorem*}[Königsberg] \label{thm:graph:euler}
    A connected graph $G$ (possibly with loops or parallel edges) has a
    closed Eulerian trail iff all vertices of $G$ have even degree.
\end{theorem*}
\begin{proof}
    Suppose $G$ has a closed Eulerian trail.
    Then, every time a vertex is visited, it is also left by a different
    edge.
    Thus each vertex has even degree.

    Conversely, suppose all vertices have even degree.
    Choose any vertex $v$ and an edge $e_1$ incident to $v$ (this exists
    by connectedness, except in the case of the singleton graph) and
    go to $v_1$.
    Pick another edge $e_2$ incident to $v_1$ and go to $v_2$.
    Continue in this way until we return to $v$, forming a closed trail
    $C_1$.
    This must happen, since $G$ is finite, and each vertex has even degree.
    If $C_1$ contains every edge, we are done.

    Otherwise, there must exist a vertex $w$ in $C_1$, which is incident
    to an edge not yet visited, since the graph is connected.
    Continue from $w$ as before to form a disjoint closed trail $C_2$.
    We can merge this with $C_1$ to form a longer closed trail.

    Continue in this way until we have used all edges.
\end{proof}
\begin{corollary*}
    The Königsberg bridge problem has no solution.
\end{corollary*}
\begin{corollary}
    A connected graph $G$ (possibly with loops or parallel edges) has an
    Eulerian trail iff it has at most two vertices of odd degree.
\end{corollary}
\begin{proof}
    First note that it is not possible for exactly one vertex to have odd
    degree.

    Suppose $G$ has an Eulerian trail.
    If it is closed, then all vertices have even degree.
    If not, add an edge between the first and last vertices to form a cycle,
    and apply the theorem.

    Conversely, suppose $G$ has at most two vertices of odd degree.
    If there are none, then $G$ has a closed Eulerian trail.
    If there are two, then add an edge between them to form a cycle,
    and apply the theorem.
\end{proof}

\begin{definition*}[Hamiltonian path] \label{def:graph:hamiltonian}
    A \emph{Hamiltonian path} is a cycle that visits every vertex.
\end{definition*}
\begin{fact*}[Dirac's theorem] \label{thm:graph:dirac}
    If every vertex of $G$ has degree at least $n/2$, where $\abs{v} = n$,
    then $G$ has a Hamiltonian path.
\end{fact*}

\begin{definition}[Graph isomorphism] \label{def:graph:isomorphism}
    Two graphs $G_1 = (V_1, E_1)$ and $G_2 = (V_2, E_2)$ are said to be
    \emph{isomorphic} if there exists a bijection $\Phi\colon V_1 \to V_2$
    such that $\set{v_1, v_2} \in E_1$ iff
    $\set{\Phi(v_1), \Phi(v_2)} \in E_2$.
\end{definition}

\begin{definition}[Tree] \label{def:tree}
    A connected graph with no cycles is called a \emph{tree}.
\end{definition}
% \begin{example}
%     These are all the trees with exactly $6$ vertices:
%     % use the tikzlibrary for graphs
%     \begin{center}
%         \begin{tikzpicture}
%             \graph[empty nodes, nodes={circle, draw}]
%             { 1 -- 2 -- 3 -- 4 -- 5 -- 6; };
%             \vspace{1em}
%             \graph[empty nodes, nodes={circle, draw}]
%             { 1 -- 2 -- 3 -- 4 -- 5, 3 -- 6; };
%         \end{tikzpicture}
%     \end{center}
% \end{example}
