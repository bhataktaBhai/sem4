\section{Generating Functions} \label{sec:genfun}
\lecture{2024-02-02}{}
\begin{definition}[Formal power series] \label{def:genfun:formal}
    The set of \emph{formal power series} $\R[[x]]$ consists of formal sums
    of the form $\sum_{n=0}^{\infty} a_n x^n$, where $a_n \in \R$ for all
    $n \in \N$.
\end{definition}
$\R[[x]]$ is the completion of the ring of polynomials $\R[x]$.
One can define $\R[[x]]$ as the set of all sequences $(a_n)_{n=0}^{\infty}$
of real numbers, with addition define componentwise and multiplication
defined by convolution.
That is, \begin{align*}
    (a_n)_{n \in \N} + (b_n)_{n \in \N} &= (a_n + b_n)_{n \in \N} \\
    (a_n)_{n \in \N} \times (b_n)_{n \in \N} &=
        \left(\sum_{k=0}^{n} a_k b_{n-k} \right)_{n \in \N}
\end{align*}

\begin{definition*}[Generating function] \label{def:genfun}
    Given a sequence $(a_n)_{n \in \N}$, the FPS \[
        A(x) = \sum_{n=0}^{\infty} a_n x^n
    \] is called the \emph{ordinary generating function} of $(a_n)_n$.
\end{definition*}
\begin{example}
    Let $a_0 = 50$ and $a_n = 4a_{n-1} - 100$ for $n \ge 1$.
    Then \begin{align*}
        A(x) &= \sum_{n=0}^{\infty} a_n x^n \\
             &= 50 + \sum_{n=1}^{\infty} (4a_{n-1} - 100) x^n \\
             &= 50 + 4x \sum_{n=0}^{\infty} a_n x^n
                - 100x \sum_{n=0}^{\infty} x^n \\
             &= 50 + 4x A(x) - \frac{100x}{1 - x}
        \shortintertext{so}
        A(x) &= \frac{50}{1 - 4x} - \frac{100x}{(1 - x)(1 - 4x)}.
    \end{align*}
    We split the fraction into partial fractions as \begin{align*}
        \frac{x}{(1-x)(1-4x)}
            &= \frac13 \frac{(1-x) - (1-4x)}{(1-x)(1-4x)} \\
            &= \frac{1}{3(1-4x)} - \frac{1}{3(1-x)}.
    \end{align*}
    This gives \begin{align*}
        A(x) &= \frac{50}{1-4x} - \frac{100}{3(1-4x)} + \frac{100}{3(1-x)}\\
             &= \frac{50}{3(1-4x)} + \frac{100}{3(1-x)} \\
             &= \frac{50}{3} \sum_{n=0}^{\infty} 4^n x^n
                + \frac{100}{3} \sum_{n=0}^{\infty} x^n.
    \end{align*}
    Thus $a_n = \frac{50}{3} 4^n + \frac{100}{3}$.
\end{example}

\begin{theorem*}
    Let $p_n$ be the number of partitions of $n$.
    Then \[
        \sum_{n=0}^{\infty} p_n q^n = \prod_{n=1}^{\infty} \frac{1}{1-q^n}.
    \]
\end{theorem*}
\begin{proof}
    Expand the RHS as \[
        (1 + q + q^2 + \dots)(1 + q^2 + q^4 + \dots)\dots
    \] Let any term in the product be $q^{a_1} q^{2a_2} \dots$.
    This can be identified with the partition $\lambda = \angled{
        1^{a_1}, 2^{a_2}, \dots
    }$.
    Thus the coefficient of $q^n$ in the product is the number of partitions
    of $n$.
\end{proof}
We are now equipped to prove \cref{thm:partitions:euler}.
\begin{proof}
    Following the idea of the previous proof, we can see that the number
    of partitions of $n$ into odd parts has ogf \begin{align*}
        \sum_{n=0}^{\infty} p_{\text{odd}}(n) q^n
            &= \prod_{n=1}^{\infty} \frac{1}{1 - q^{2n-1}}
        \intertext{and the number of partitions into distinct parts has ogf}
        \sum_{n=0}^{\infty} p_{\text{distinct}}(n) q^n
            &= \prod_{n=1}^{\infty} 1 + q^n.
        \intertext{With some algebraic manipulation, we produce}
        \prod_{n=1}^{\infty} 1 + q^n
            &= \prod_{n=1}^{\infty} \frac{1 - q^{2n}}{1 - q^n} \\
            &= \prod_{n=1}^{\infty} \frac{1}{1 - q^{2n-1}}
    \end{align*}
    since all the even terms cancel out.
\end{proof}

\begin{exercise*}[Product formula]
    Let $A(x)$ and $B(x)$ be the ogfs of sequences $(a_n)_n$ and $(b_n)_n$
    respectively.
    Then $A(x) \cdot B(x)$ is the ogf of the sequence $(c_n)_n$ where
    $c_n = \sum_{k=0}^{n} a_k b_{n-k}$.
\end{exercise*}
\begin{proof}
    The coefficient of $x^n$ in $A(x) \cdot B(x)$ is obviously \[
        \sum_{k=0}^{n} a_k b_{n-k}.
    \]
\end{proof}

\begin{example}
    Let $C_n$ be the number of ways of forming a valid word with $n$ pairs
    of parentheses.

    Let $C(x)$ be the generating function of $(C_n)_n$.
    The key idea of the solution is to consider the matching closing
    parenthesis of the first opening parenthesis.
    Thus any valid word $w$ can be uniquely written as $w = (w_1)w_2$,
    where $w_1$ and $w_2$ are both valid.
    This gives \[
        C_{n+1} = \sum_{k=0}^n C_k C_{n-k}.
    \]
\end{example}
