\documentclass[12pt]{article}
\let\emph\textsl
\let\emph\textsl
\let\emph\textsl
\input{~/IISc/preamble}
\usepackage{pgfplots}
\pgfplotsset{compat=1.18}

\newcounter{assignment}
\theoremstyle{definition}
\newmdtheoremenv[nobreak=true, outerlinewidth=0.7]{problem}{Problem}[assignment]
\newmdtheoremenv[nobreak=true]{exercise}[theorem]{Exercise}

\DeclareMathOperator{\sgn}{sgn}
\usepackage{bbm}
\newcommand{\ind}[1]{\mathbbm{1}_{#1}}
\newcommand{\given}{\mid}

\usepackage{pgfplots}
\pgfplotsset{compat=1.18}

\newcounter{assignment}
\theoremstyle{definition}
\newmdtheoremenv[nobreak=true, outerlinewidth=0.7]{problem}{Problem}[assignment]
\newmdtheoremenv[nobreak=true]{exercise}[theorem]{Exercise}

\DeclareMathOperator{\sgn}{sgn}
\usepackage{bbm}
\newcommand{\ind}[1]{\mathbbm{1}_{#1}}
\newcommand{\given}{\mid}

\usepackage{pgfplots}
\pgfplotsset{compat=1.18}

\newcounter{assignment}
\theoremstyle{definition}
\newmdtheoremenv[nobreak=true, outerlinewidth=0.7]{problem}{Problem}[assignment]
\newmdtheoremenv[nobreak=true]{exercise}[theorem]{Exercise}

\DeclareMathOperator{\sgn}{sgn}
\usepackage{bbm}
\newcommand{\ind}[1]{\mathbbm{1}_{#1}}
\newcommand{\given}{\mid}


\title      {Assignment 7}
\setcounter{assignment}{7}
\setcounter   {section}{7}
\author{Naman Mishra}
\date{28 February, 2024}

\begin{document}
\maketitle
\begin{problem*}
    Consider the function $f\colon \R \to [0, 1]$ given by \[
        f(x) = \begin{cases}
            \frac1q & x = \frac{p}{q}, \text{ with } p \in \Z \text{ and }
                q \in \N^*, \gcd(p, q) = 1, \\
            0 & x \notin \Q.
        \end{cases}
    \] Show that $f$ is discontinuous at every rational number,
    and continuous elsewhere.
    (Here, we have used the fact that every rational number $x \in \Q$
    admits a unique representation of the form $p/q$,
    with $p$ and $q$ as described above.)
\end{problem*}
\begin{solution}
    Discontinuity at rational numbers is easy.
    Let $x$ be a rational number, so $f(x) > 0$.
    Then, for any $\delta > 0$, there exists a
    $y = x + \frac{\delta}{\sqrt 2} \in B(x; \delta)$
    such that $f(y) = 0$.

    For continuity at an irrational number $x$, let $\epsilon > 0$.
    Let $q \in \N^*$ be such that $\frac1q < \epsilon$.
    Choose \[
        \delta = \inf\set*[\Big]{\abs*[\Big]{x - \frac{p}{q!}} : p \in \Z}.
    \] This is positive since $B(x; \frac1{2q!})$ can contain at most one
    such $p/q!$, which cannot be $x$.

    Note that for $f(y) = \frac1n$, $n \le q$, $y$ must be of the form
    $m/n$, and hence of the form $p/q!$ for some $p \in \N$.

    Then $B(x; \delta)$ contains no such $y$.
    This is obvious by some light casework.

    Thus for all $y \in B(x; \delta)$, $f(y) = 0$ if $y \notin \Q$, and
    $f(y) = \frac1n < \frac1q$ for some $n > q$ if $y \in \Q$.
    In any case, $\abs{f(y) - f(x)} < \varepsilon$.
\end{solution}

\begin{problem*}
    Let $(X, d)$ be a metric space and $A \subseteq X$ be a nonempty subset.
    Define \[
        f_A(x) = \inf\set{d(x, y) \mid y \in A}, \quad x \in X.
    \]
    \begin{enumerate}[label=(\alph*)]
        \item Show that $f_A$ is uniformly continuous on $X$.
        \item Prove that $x \in \wbar{A}$ if and only if $f_A(x) = 0$.
    \end{enumerate}
\end{problem*}
\begin{solution}
    Notation: $D_x = \set{d(x, a) \mid a \in A}$.

    For uniform continuity, choose $\delta < \varepsilon$ for any given
    $\varepsilon > 0$.
    Let $x, y \in X$ be such that $d(x, y) < \delta$.
    By the triangle inequality, $d(x, a) \le d(x, y) + d(y, a)
    < d(y, a) + \varepsilon$ for every $a \in A$.

    Then $f_A(x) \le d(x, a) < d(y, a) + \delta$ for every $a \in A$.
    Thus $f_A(x) - \delta \le d(y, a)$ for every $a \in A$.
    Hence $f_A(x) - \delta$ is a lower bound of $D_y$.
    So $f_A(y) \ge f_A(x) - \delta$.

    Similarly $f_A(x) \ge f_A(x) - \delta$.
    Thus $\abs{f_A(x) - f_A(y)} \le \delta < \varepsilon$.

    For the second part, \begin{align*}
        x \in \wbar{A}
        &\iff \forall \varepsilon > 0 \;\exists a \in A (d(x, a) < \varepsilon) \\
        &\iff \forall \varepsilon > 0 \;\exists d \in D_x (d < \varepsilon) \\
        &\iff f_A(x) = 0. \qedhere
    \end{align*}
\end{solution}

\begin{problem*}
    Show that uniformly continuous functions map Cauchy sequences to
    Cauchy sequences.
    Is the converse true?
\end{problem*}
\begin{proof}
    Let $f\colon X \to Y$ be uniformly continuous.
    Let $(x_n)_{n \in \N}$ be a Cauchy sequence in $X$.

    Let $\varepsilon > 0$ be given.
    Choose $\delta > 0$ such that $d_X(x, y) < \delta$ implies
    $d_Y(f(x), f(y)) < \varepsilon$.
    Since $(x_n)_n$ is Cauchy, there exists $N \in \N$ such that
    $d_X(x_n, x_m) < \delta$ for all $n, m \ge N$.

    Then $d_Y(f(x_n), f(x_m)) < \varepsilon$ for all $n, m \ge N$.

    Is the converse true?
    Take $f\colon \R \to \R$ to be $x \mapsto x^2$.
    $f$ is continuous, so it maps Cauchy (hence convergent) sequences
    to convergent (hence Cauchy) sequences
    (sequential characterization of continuity).
    But $f$ is not uniformly continuous, since for any $\delta > 0$,
    $f(\frac1{\delta} + \delta) - f(\frac1{\delta}) = 2 + \delta^2 > 1$.

    What went wrong?
    Both the domain and codomain are unbounded.
    Which one is the problem?
\end{proof}

\begin{proposition} \label{thm:cauchy_unif_when}
    Let $X$ and $Y$ be metric spaces.
    Suppose that $X$ is compact and $Y$ is complete.
    Let $f\colon X \to Y$ be such that it maps every Cauchy sequence
    to a Cauchy sequence.
    Then $f$ is uniformly continuous.
\end{proposition}

\begin{lemma}
    Let $f$ be a function from some metric space $X$ to a
    complete metric space $Y$.
    Suppose that $f$ maps Cauchy sequences to Cauchy sequences.
    Then $f$ is continuous.
\end{lemma}
\begin{proof}
    Let $x \in X$.
    Let $(x_n)_n$ be a sequence in $X$ converging to $x$.
    Then $(x_n)_n$ is Cauchy.
    Since $f$ maps Cauchy sequences to Cauchy sequences,
    $(f(x_n))_n$ is Cauchy.
    Since $Y$ is complete, $(f(x_n))_n$ converges to some $y \in Y$.
    Thus $\lim_{n \to \infty} f(x_n) = y$ exists.

    Let $(x_n)_n$ and $(x'_n)_n$ be sequences in $X$ converging to $x$.
    Interleave the sequences to get a sequence $(\tilde{x}_n)_n$
    converging to $x$.
    Then $\lim_{n \to \infty} f(\tilde{x}_n)$ exists, so
    $\lim_{n \to \infty} f(x_n) = \lim_{n \to \infty} f(x'_n)$.

    \textbf{Sequential characterization:}
    Suppose $f$ is not continuous at $x$.
    Then there exists $\varepsilon > 0$ such that for all $\delta > 0$,
    there exist $x_\delta \in B_X(x; \delta)$ such that
    $d_Y(f(x_\delta), f(x)) \ge \varepsilon$. \\
    Consider the sequence $(x_{1/n})_n$.
    Then $(x_{1/n})_n$ converges to $x$, so $(f(x_{1/n}))_n$ converges to
    whatever $(f(x))_n$ converges to, which is $f(x)$.
    Contradiction.
\end{proof}

\begin{proof}[Proof of \cref{thm:cauchy_unif_when}]
    Let $\varepsilon > 0$.
    Let $\delta_x > 0$ be such that
    $f(B_X(x; \delta_x)) \subseteq B_Y(f(x); \varepsilon/2)$
    {\footnotesize [which exists by continuity]}.

    Let $\mcU$ be a finite subcover of $\set{B_X(x; \delta_x/2)}_{x \in X}$
    {\footnotesize [which exists by compactness]}.
    Choose the minimum of these $\delta_x/2$ to get $\delta > 0$.

    Let $x_1, x_2 \in X$ within $\delta$ of each other.
    Since $\mcU$ is a cover, there exists $x \in X$ such that
    $x_1 \in B_X(x; \delta_x/2) \in \mcU$, so
    $d(f(x_1), f(x)) < \varepsilon/2$.
    By the $\triangle$ inequality,
    $d(x_2, x) < \delta_x/2 + \delta < \delta_x$,
    so that $d(f(x_2), f(x)) < \varepsilon/2$.
    Then $d(f(x_1), f(x_2)) < \varepsilon$.
\end{proof}

\begin{proposition*} \label{thm:cauchy_unif_when_better}
    Completeness in \cref{thm:cauchy_unif_when} is unnecessary.
\end{proposition*}
\begin{proof}
    Let $x \in X$ and $(x_n)_{n \in \N} \to x$.
    Interleave this with the constant sequence $(x)_{n \in \N}$ to get
    a sequence $(\tilde{x}_n)_{n \in \N}$ converging to $x$.
    Then $(f(\tilde{x}_n))_{n \in \N}$ is Cauchy.
    But every other term is $f(x)$, so $(f(x_n))_{n \in \N} \to f(x)$.
    By sequential characterization, $f$ is continuous.

    The rest of the proof doesn't require completeness of $Y$.
\end{proof}

\begin{exercise}
    The counterexample $x \mapsto x^2$ had both the domain and codomain
    unbounded.

    Construct a function $f\colon X \to Y$ between metric spaces $X$ and $Y$
    such that $X$ is closed, $Y$ is compact,
    $f$ maps Cauchy sequences to Cauchy sequences,
    but $f$ is not uniformly continuous.
\end{exercise}
\begin{solution}
    Take $f\colon \R \to [-1, 1]$ to be $x \mapsto \sin x^2$.
    This is continuous, so it maps Cauchy sequences to Cauchy sequences.

    But $f$ is not uniformly continuous.
    Take $\varepsilon = 1$.
    For any $\delta > 0$, choose $n = \ceil{\frac1{\delta^2}}$
    and $x^2 = n \pi$, $y^2 = (n + \frac12) \pi$.
    Then $y^2 - x^2 = (y - x)(x + y) = \frac12 \pi$.
    So $y - x < \frac1{4\sqrt n} \sqrt \pi < \frac1{\sqrt n} \le \delta$.
    But $\abs{f(x) - f(y)} = 1$.
\end{solution}

\begin{exercise}
    Construct a function $f\colon X \to Y$ between metric spaces $X$ and $Y$
    such that $X$ is bounded,
    $f$ maps Cauchy sequences to Cauchy sequences,
    but $f$ is not uniformly continuous.
\end{exercise}
\begin{solution}
    Take $f\colon (-\frac{\pi}{2}, \frac{\pi}{2}) \to \R$
    to be $x \mapsto \tan^{-1} x$.
\end{solution}

\begin{exercise*}
    Construct a function $f\colon X \to Y$ between metric spaces $X$ and $Y$
    such that $X$ is bounded, $Y$ is compact,
    $f$ maps Cauchy sequences to Cauchy sequences,
    but $f$ is not uniformly continuous.
\end{exercise*}
\begin{solution}
    Take $f\colon (-\frac{\pi}{2}, \frac{\pi}{2}) \to [-1, 1]$ to be
    $x \mapsto \sin((\tan^{-1} x)^2)$.
\end{solution}

\begin{exercise*}
    Construct a function $f\colon X \to Y$ between metric spaces $X$ and $Y$
    such that $f(X)$ is bounded but incomplete,
    $f$ maps Cauchy sequences to Cauchy sequences,
    but $f$ is not uniformly continuous.
\end{exercise*}
\begin{solution}
    Take $f\colon [1, \infty) \to (0, 2]$ to be
    $x \mapsto \cos^2(\pi x^2) + \frac1{x^2}$.

    If boundedness of $X$ is desired, compose $f$ with $\tan^{-1}$ again.
\end{solution}

\begin{problem*}
    An $F_\sigma$ set is a countable union of closed sets.
    Complete the following steps to prove that the discontinuity set $D_f$
    of any function $f\colon \R \to \R$ is an $F_\sigma$ set.
    \begin{enumerate}[label=(\alph*)]
        \item Given $f\colon \R \to \R$ and $\alpha > 0$,
        $f$ is said to be $\alpha$-continuous at $x \in \R$
        if there exists a $\delta > 0$ such that for all
        $y, z \in B(x; \delta)$, $\abs{f(y) - f(z)} < \alpha$.
        Show that the set $D^\alpha = \set{x \in \R \mid f \text{ is not
        $\alpha$-continuous at } x}$ is closed, for each $\alpha > 0$.
        \item Show that $D^\alpha \subseteq D_f$ for any $\alpha > 0$.
        \item Show that \[
                D_f = \bigcup_{n \in \N} D^{1/n}.
            \]
    \end{enumerate}
\end{problem*}
\begin{solution}
    Let $x$ be a limit point of $D^\alpha$.
    Suppose $f$ is $\alpha$-continuous at $x$.
    Then there exists a $\delta$-ball around $x$ such that
    $\abs{f(y) - f(z)} < \alpha$ for all $y, z \in B(x; \delta)$.
    But $x$ is a limit point, so there exists a $y \in D^\alpha$ such that
    $y \in B(x; \delta/2)$.
    Then $B(y; \delta/2) \subseteq B(x; \delta)$,
    so $f$ is $\alpha$-continuous at $y$.
    Contradiction.

    Suppose $f$ is continuous at $x$.
    Then for any $\alpha > 0$, there exists a $\delta$-ball around $x$
    such that the output of every point in the ball is within $\alpha/2$
    of $f(x)$.
    Hence the output of any pair of points in the ball is within $\alpha$
    of each other.
    Thus $f$ is $\alpha$-continuous at $x$.
    Thus $D_f^C \subseteq (D^\alpha)^C$ for any $\alpha > 0$.

    Finally, \begin{align*}
        x \in D_f &\iff
        \exists \alpha > 0 \; \forall \delta > 0 \;\exists y \in B(x; \delta)
        \text{ such that } \abs{f(x) - f(y)} \ge \alpha \\
        &\implies \exists \alpha > 0 \; \forall \delta > 0 \; \exists z, y
        \in B(x; \delta) \text{ such that } \abs{f(y) - f(z)} \ge \alpha \\
    \intertext{by taking $z = x$}
        &\iff \exists \alpha > 0 (x \in D^\alpha) \\
        &\iff \exists n \in \N (x \in D^{1/n}) \\
    \intertext{since $D^a \subseteq D^b$ for $a < b$.}
        &\iff x \in \bigcup_{n \in \N} D^{1/n}.
    \end{align*}
    Conversely $\bigcup_{n \in \N} D^{1/n} \subseteq D_f$ by step (b).
    Thus, $D_f \in F_\sigma$.
\end{solution}

\end{document}
