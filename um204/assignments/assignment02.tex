\documentclass[12pt]{article}
\input{~/IISc/sem4/preamble}
\input{~/IISc/sem4/preamble/mvc}
\usepackage{pgfplots}
\pgfplotsset{compat=1.18}
\usepackage{tikz}
\usetikzlibrary{graphs, graphs.standard, quotes, positioning, arrows.meta}

\newcommand{\ind}[1]{\bm{1}_{#1}}
\newcommand{\given}{\mid}
\newcommand{\pms}{\{-1, 1\}}
\DeclareMathOperator*{\argmax}{argmax}
\DeclareMathOperator*{\argmin}{argmin}
\DeclareMathOperator*{\E}{\mathbb E}

\usepackage{amsmath}
\DeclareMathOperator\divv{div}
\DeclareMathOperator\hess{Hess}
% \DeclareMathOperator\Tr{Tr}

\usepackage{pgfplots}
\pgfplotsset{compat=1.18}
\usepackage{tikz}
\usetikzlibrary{graphs, graphs.standard, quotes, positioning, arrows.meta}

\newcommand{\ind}[1]{\bm{1}_{#1}}
\newcommand{\given}{\mid}
\newcommand{\pms}{\{-1, 1\}}
\DeclareMathOperator*{\argmax}{argmax}
\DeclareMathOperator*{\argmin}{argmin}
\DeclareMathOperator*{\E}{\mathbb E}

\usepackage{amsmath}
\DeclareMathOperator\divv{div}
\DeclareMathOperator\hess{Hess}
% \DeclareMathOperator\Tr{Tr}

\usepackage{pgfplots}
\pgfplotsset{compat=1.18}
\usepackage{tikz}
\usetikzlibrary{graphs, graphs.standard, quotes, positioning, arrows.meta}

\newcommand{\ind}[1]{\bm{1}_{#1}}
\newcommand{\given}{\mid}
\newcommand{\pms}{\{-1, 1\}}
\DeclareMathOperator*{\argmax}{argmax}
\DeclareMathOperator*{\argmin}{argmin}
\DeclareMathOperator*{\E}{\mathbb E}


\title{Assignment 2}
\setcounter{assignment}{2}
\author{Naman Mishra}
\date{07 January, 2024}

\begin{document}
\maketitle
\begin{problem*}
    Let $F$ and $G$ be ordered fields with the LUB property.
    In Lecture 04, we defined $h\colon F \to G$ as \[
        h(z) = \sup_G\set{w \in \Q : w \le z}.
    \] Show that $h$ is a field isomorphism, \ie,
    \begin{enumerate}[label=(\arabic*)]
        \item $h$ is a bijection between $F$ and $G$,
        \item $h(x + y) = h(x) + h(y)$ for all $x, y \in F$,
        \item $h(x \cdot y) = h(x) \cdot h(y)$ for all $x, y \in F$.
    \end{enumerate}
\end{problem*}
\begin{proof}
    \refifdef{thm:R:unique}{\Cref}{Lecture 4}.
\end{proof}

\begin{problem*}
    In this problem, you may assume the well-definedness, commutativity and
    associativty of addition of Dedekind cuts (as defined in Lecture 04).
    Let $O = \set{z \in \Q : z < 0}$.
    Verify that $O$ is a Dedekind cut, and $A + O = A$ for all Dedekind cuts
    $A$.
    Let $A$ be a Dedekind cut.
    Define a Dedekind cut $B$ such that $A + B = O$.
    You must justify your answer.
\end{problem*}
\begin{proof}
    \refifdef{thm:R:dedekind:negative}{\Cref}{Lecture 4}.
\end{proof}

\begin{problem*}
    Let $a = (a_n)_{n \in \N}$ and $b = (b_n)_{n \in \N}$ be sequences of
    rational numbers such that $b_n \ne 0$ for all $n \in \N$.
    Suppose \[
        \lim_{n \to \infty} \frac{a_n}{b_n} = 1.
    \]
    \begin{enumerate}
        \item Are $a$ and $b$ equivalent?
        \item Are $a$ and $b$ equivalent if $a$ is a \Q-bounded sequence?
    \end{enumerate}
\end{problem*}
\begin{solution} \leavevmode
    \begin{enumerate}
        \item No. $a_n = n + 1$ and $b_n = n$ gives a counterexample.
        \item Yes.
    \end{enumerate}
    Let $a$ be bounded by $M$.
    Let $n_0$ be such that for all $n \ge n_0$, $\frac12 < \frac{a_n}{b_n}$.
    Then, for all $n \ge n_0$, $\abs{b_n} < 2 \abs{a_n} \le 2M$.
    Thus $b$ is bounded.

    Let $\varepsilon > 0$.
    Let $N$ be such that for all $n \ge N$, \[
        \abs{\frac{a_n}{b_n} - 1} < \frac{\varepsilon}{2M}.
    \] Then for all $n \ge N$, \begin{align*}
        \abs{a_n - b_n} &= \abs{b_n} \abs{\frac{a_n}{b_n} - 1} \\
                  &< 2M \frac{\varepsilon}{2M} \\
                  &= \varepsilon. \qedhere
    \end{align*}
\end{solution}

\begin{problem*}
    You cannot use the existence (or the LUB property) of the ordered field
    of real numbers in this problem, so you must work ``within'' \Q.
    \begin{enumerate}[label=(\arabic*)]
        \item Show that every monotone \Q-bounded monotone sequence
        of rational numbers is \Q-Cauchy.
        \item Consider the following sequence: \[
            x_n = \begin{cases}
                2, & \text{if } n = 0, \\
                x_{n-1} - \dfrac{x_{n-1}^2 - 2}{2 x_{n-1}}
                    & \text{if } n \ge 1.
            \end{cases}
        \] Confirm that $(x_n)_{n \in \N}$ is well-defined, \ie,
        $x_n \ne 0$ for all $n \in \N$.
        Show that $(x_n)_{n \in \N}$ is \Q-Cauchy, but not convergent in \Q.
    \end{enumerate}
\end{problem*}
\begin{solution}
    \begin{enumerate}[label=(\arabic*)]
        \item Let $(x_n)_{n \in \N}$ be a monotone \Q-bounded sequence.
        WLOG assume that it is increasing.

        Let $\varepsilon > 0$.
        Let \[
            N = \inf\set{n \in \Z \mid n\varepsilon \text{ is an upper
            bound of } (x_n)_n}.
        \]
        Why is the set non-empty?
        There is an upper bound $b$ for $\set{x_n}_n$.
        Archimedeanly, there is an $n_0 : n_0\varepsilon > b$.
        Then $n_0$ belongs to the set.

        Why is it bounded below?
        Archimedeanly, there is an $n_1 : n_1\varepsilon < x_0$.
        Then $n_1$ is a lower bound of the set.

        Thus $N \varepsilon$ is an upper bound, but $(N-1)\varepsilon$
        is not.
        So there exists some $M$ such that $x_M > (N-1)\varepsilon$.
        Then for all $n \ge M$, $(N-1)\varepsilon < x_n \le N\varepsilon$.
        So for any $n \ge m \ge M$, \[
            \abs{x_n - x_m} = x_n - x_m
                      < N\varepsilon - (N-1)\varepsilon
                      = \varepsilon.
        \]
        % Suppose $(x_n)_n$ is not \Q-Cauchy.
        %
        % Then there exists $\varepsilon > 0$ such that for all $N \in \N$,
        % there exist $n > m \ge N$ such that
        % $\abs{x_n - x_m} \ge \varepsilon$.
        %
        % Let $b$ be an upper bound for $(x_n)_n$.
        % Suppose $b - \varepsilon$ is not an upper bound.
        % Then there exists $n_0 \in \N$ such that
        % $x_{n_0} > b - \varepsilon$, and hence $x_n > b - \varepsilon$ for
        % all $n \ge n_0$.
        % So for all $n \ge n_0$, $b - \varepsilon < x_n \le b$.
        % But there exist $n > m \ge n_0$ such that \[
        %     \varepsilon \le \abs{x_n - x_m}
        %       = x_n - x_m
        %       \le b - (b - \varepsilon)
        %       = \varepsilon,
        % \] a contradiction.
        %
        % Thus $b - \varepsilon$ is an upper bound for $(x_n)_n$.
        % By induction, $b - n\varepsilon$ is an upper bound for every $n$.
        % But this is eventually less than $x_0$, a contradiction.
        \item \begin{align*}
            x_n &= \frac{x_{n-1}^2 + 2}{2x_{n-1}}
                \tag{$*$} \label{eq:simp} \\
            \implies x_n^2 - 2 &= \frac{x_{n-1}^4 + 4 x_{n-1}^2 + 4 - 8x_{n-1}^2}{4x_{n-1}^2} \\
                &= \frac{(x_{n-1}^2 - 2)^2}{4x_{n-1}^2} \\
        \end{align*}
        This shows that $x_n^2 > 2$ for all $n \in \N$.
        From \eqref{eq:simp}, $x_{n-1} > 0$ implies $x_n > 0$, so
        $(x_n)_n > 0$.
        Thus \[
            x_n - x_{n-1} = -\frac{x_{n-1}^2 - 2}{2x_{n-1}} < 0
        \] and so $(x_n)_n$ is decreasing.

        From the first part, $(x_n)_n$ is \Q-Cauchy.
        But suppose it had a limit $x \in \Q$.
        Note that $x \ne 0$, since $x_n^2 > 2$.
        Then \[
            x = x - \frac{x^2 - 2}{2x} \implies x^2 = 2,
        \] which is not possible. \qedhere
    \end{enumerate}
\end{solution}

\begin{problem*}
    A \emph{digit} is any element of the set $S = \set{0, 1, \dots, 9}$.
    An \emph{admissible sequence of digits} is a sequence
    $(a_n)_{n \in \N^*} \subseteq S$ satisfying the property:
    there is no $N \ge 1$ such that $a_n = 9$ for all $n \ge N$.
    Given $x \in [0, 1)$, we say that an admissible sequence of digits
    $(d_n)_{n \in \N^*}$ is a \emph{decimal representation} of $x$ if \[
        \sup\set*[\bigg]{D_n \coloneq \sum_{j=1}^n \frac{d_j}{10^j}
            \Bigm\vert n \in \N} = x.
    \] Show that every admissible sequence of digits is the decimal
    representation of a number $x \in [0, 1)$, and conversely, every
    $x \in [0, 1)$ admits a unique decimal representation defined as above.
    \\
    \textbf{Note:} In this problem, you may freely use the standard
    properties of real numbers.
\end{problem*}
\begin{solution}
    Let $(d_n)_n$ be an admissible sequence of digits.
    Note that the given set is non-empty, and bounded above by
    $\sum_{j=1}^n \frac{9}{10^j} = 1$.
    Thus the supremum exists.
    It is obviously non-negative.

    Why is it less than $1$?
    Let $i$ be the first index where $d_i \ne 9$.
    Then for each $n \ge i$, \begin{align*}
        \sum_{j=1}^n \frac{d_j}{10^j}
            &= \sum_{j=1}^{i-1} \frac{d_j}{10^j}
                + \frac{d_i}{10^i} + \sum_{j=i+1}^n \frac{d_j}{10^j} \\
            &\le \sum_{j=1}^{i-1} \frac{d_j}{10^j}
                + \frac{d_i}{10^i} + \sum_{j=i+1}^n \frac{9}{10^j} \\
            &= \sum_{j=1}^{i-1} \frac{d_j}{10^j}
                + \frac{d_i + 1}{10^i} \\
            &\le \sum_{j=1}^{i-1} \frac{d_j}{10^j}
                + \frac{9}{10^i} < 1.
    \end{align*} Of course, for each $n < i$, the inequality still holds.
    Thus $\sum_{j=1}^i \frac{d_j}{10^j} + \frac{1}{10^i}$ is an upper bound
    less than $1$.

    For the converse, let $d_0 = 0$.
    For each $n \in \N^*$, let \[
        d_n =
            \max\set*[\Big]{d \in S \bigm| D_{n-1} + \frac{d}{10^n} \le x},
    \] where $D_{n-1} = \sum_{j=1}^{n-1} \frac{d_j}{10^j}$.
    This ensures that $D_n \le x < D_n + 10^{-n}$ for each $n$.
    How? Holds trivially for $n = 0$.
    If it holds for $n-1$, then $d_n$ is chosen such that \[
        D_n \le x < D_n + \frac{1}{10^n},
    \] if $d_n \ne 9$.
    But if $d_n = 9$, then
    $D_n \le x < D_{n-1} + 10^{-n+1} = D_n + 10^{-n}$.
    Thus it holds for all $n$.

    Then $x$ is an upper bound for $(D_n)_n$.
    Suppose there is a lesser upper bound $s$.
    Then there exists $n$ such that $10^{-n} < x - s$.
    But then $x < D_n + 10^{-n} < D_n + x - s$ so $s < D_n$.
    Thus $\sup_n D_n = x$.

    For uniqueness, let $(d_n)_n$ and $(d'_n)_n$ be two distinct admissible
    sequences of digits.
    Suppose they first differ at the index $j$, with $d_j < d'_j$,
    and that $d_k \ne 9$ with $k \ge j$.
    Then for $n \ge k$,
    \begin{align*}
        D_n  &= D'_{j-1} + \frac{d_j}{10^j}
                + \sum_{i=j+1}^n \frac{d_i}{10^i} \\
            &\le D'_{j-1} + \frac{d'_j - 1}{10^j}
                + \sum_{i=j+1}^n \frac{d_i}{10^i} \\
            &\le D'_{j-1} + \frac{d'_j - 1}{10^j}
                + \sum_{i=j+1}^n \frac{9}{10^i} - \frac1{10^k} \\
            &= D'_{j-1} + \frac{d'_j}{10^j} - \frac1{10^k} - \frac1{10^n} \\
            &< D'_j - \frac1{10^k}.
    \end{align*} Thus $\set{D_n}$ is bounded above by
    $\sup_n D'_n - 10^{-k}$.
    So $\sup_n D_n < \sup_n D'_n$.
\end{solution}

\end{document}
