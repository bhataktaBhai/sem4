\documentclass[12pt]{article}
\let\emph\textsl
\let\emph\textsl
\let\emph\textsl
\input{~/IISc/preamble}
\usepackage{pgfplots}
\pgfplotsset{compat=1.18}

\newcounter{assignment}
\theoremstyle{definition}
\newmdtheoremenv[nobreak=true, outerlinewidth=0.7]{problem}{Problem}[assignment]
\newmdtheoremenv[nobreak=true]{exercise}[theorem]{Exercise}

\DeclareMathOperator{\sgn}{sgn}
\usepackage{bbm}
\newcommand{\ind}[1]{\mathbbm{1}_{#1}}
\newcommand{\given}{\mid}

\usepackage{pgfplots}
\pgfplotsset{compat=1.18}

\newcounter{assignment}
\theoremstyle{definition}
\newmdtheoremenv[nobreak=true, outerlinewidth=0.7]{problem}{Problem}[assignment]
\newmdtheoremenv[nobreak=true]{exercise}[theorem]{Exercise}

\DeclareMathOperator{\sgn}{sgn}
\usepackage{bbm}
\newcommand{\ind}[1]{\mathbbm{1}_{#1}}
\newcommand{\given}{\mid}

\usepackage{pgfplots}
\pgfplotsset{compat=1.18}

\newcounter{assignment}
\theoremstyle{definition}
\newmdtheoremenv[nobreak=true, outerlinewidth=0.7]{problem}{Problem}[assignment]
\newmdtheoremenv[nobreak=true]{exercise}[theorem]{Exercise}

\DeclareMathOperator{\sgn}{sgn}
\usepackage{bbm}
\newcommand{\ind}[1]{\mathbbm{1}_{#1}}
\newcommand{\given}{\mid}


\title      {Assignment 3}
\setcounter{assignment}{3}
\setcounter   {section}{3}
\author{Naman Mishra}
\date{25 January, 2024}

\begin{document}
\maketitle
\begin{problem*}
\end{problem*}

\begin{solution}[Case 1]
    Let $b > 1$ and $n > 0$.
    Let $B = \set{t \in \R : t > 0, t^n < b}$.
    $B$ is non-empty since $1 \in B$.
    \\
    Since $b > 1$, $b^n > b$.
    Moreover, since $0 < x < y \implies x^n < y^n$,
    we have that $B$ is bounded above by $b$.
    Thus $b$ has a supremum.

    We now show that $(\sup B)^n = b$.
    For this we show that $B$ has no largest element.
    For any $u \in B$, let $\delta = \min\set{\frac{b - u^n}{2^n u^n}, 1}$.
    Then \begin{align*}
    (u(1 + \delta))^n
        &= u^n (1 + n \delta + \dots + \delta^n) \\
        &\le u^n \paren{1 + \delta \sum_{j=1}^{n} \binom{n}{j}} \\
        &< u^n + 2^n u^n \delta \\
        &\le u^n + b - u^n \\
        &= b
    \end{align*}
    Thus $u(1 + \delta)$ is an element of $B$ greater than $u$.

    This implies that $\sup B \notin B$.
    Now let $u = \sup B$ and suppose $u^n > b$.
    Let $\delta = \min\set{\frac{u^n - b}{2^n u^n}, 1}$.
    Then \begin{align*}
    (u(1 - \delta))^n
        &\ge u^n (1 - n \delta) \\
        &\ge u^n - 2^n u^n \delta \\
        &= b.
    \end{align*}
    Thus $u(1 - \delta)$ is an upper bound of $B$ less than $u$.
    This contradicts that $u$ is the supremum.

    Thus, $(\sup B)^n = b$.
    Using $0 < x < y \implies x^n < y^n$, we have that $\sup B$ is the
    only positive real number whose $n$-th power is $b$,
    so $t^n = b \implies t = \sup B$.
\end{solution}
\begin{solution}[Case 2]
    Let $mq = np$.
    For $b$ and $t$ positive,
    \begin{align*}
        t^n < b^m
            &\implies t^{np} < b^{mp} \\
            &\implies t^{mq} < b^{mp} \\
            &\implies (t^q)^m < (b^p)^m \\
            &\implies t^q < b^p.
    \end{align*}
    The last implication may seem hairy, but it also follows directly from
    $0 < x < y \implies x^m < y^m$.
    Similarly, \[
        t^q < b^p \implies t^n < b^m.
    \] Thus \[
        \sup\set{t \in \R : t > 0, t^n < b^m}
            = \sup\set{t \in \R : t > 0, t^q < b^p}.
    \]

    For the rest of the proof, we can give up.
\end{solution}

\begin{problem*}
    Given $x = (x_1, \dots, x_n) \in \R^n$ and $p > 0$, define \[
        \norm{x}_p = \paren*[\bigg]{\sum_{i=1}^{n} \abs{x_i}^p}^{1/p}
        \mkern-18mu.
    \]
    \begin{enumerate}[label=(\alph*)]
        \item Show that if $p, q > 1$ satisfy $\frac1p + \frac1q = 1$, then
        \[
            \sum_{i=1}^{n} \abs{x_i y_i} \le \norm{x}_p \norm{y}_q,
            \quad \forall x, y \in \R^n.
        \] You may directly use Young's inequality: if $a, b \ge 0$, then
        $ab \le \frac{a^p}{p} + \frac{b^q}{q}$.
        \item Let $d_p(x, y) = \norm{x - y}_p$, $x, y \in \R^n$.
        Show that $(\R^n, d_p)$ is a metric space if $p \ge 1$.
        \item Show that $(\R^n, d_p)$ is not a metric space if
        $p \in (0, 1)$.
    \end{enumerate}
\end{problem*}
\begin{lemma}[Young's inequality] \label{thm:young}
    Let $p, q > 1$ satisfy $\frac1p + \frac1q = 1$.
    Then for $a, b \ge 0$, $ab \le \frac{a^p}{p} + \frac{b^q}{q}$.
\end{lemma}
\begin{proof}
    
\end{proof}

\begin{solution} \leavevmode
    \begin{enumerate}[label=(\alph*)]
        \item Apply Young's inequality on $\frac{\abs{x_i}}{\norm{x}_p}$
        and $\frac{\abs{y_i}}{\norm{y}_q}$.
        Then \begin{align*}
            \sum_{i=1}^{n} \frac{\abs{x_i y_i}}{\norm{x}_p \norm{y}_q}
                &\le \sum_{i=1}^{n} \paren*[\bigg]{
                    \frac1p \frac{\abs{x_i}^p}{\norm{x}_p^p}
                    + \frac1q \frac{\abs{y_i}^q}{\norm{y}_q^q}
                } \\
                &= \frac1p \sum_{i=1}^{n} \frac{\abs{x_i}^p}{\norm{x}_p^p}
                + \frac1q \sum_{i=1}^{n} \frac{\abs{y_i}^q}{\norm{y}_q^q} \\
                &= \frac1p + \frac1q \\
                &= 1.
        \end{align*}
    \end{enumerate}
\end{solution}

\end{document}
