\documentclass[12pt]{article}
\let\emph\textsl
\let\emph\textsl
\let\emph\textsl
\input{~/IISc/preamble}
\usepackage{pgfplots}
\pgfplotsset{compat=1.18}

\newcounter{assignment}
\theoremstyle{definition}
\newmdtheoremenv[nobreak=true, outerlinewidth=0.7]{problem}{Problem}[assignment]
\newmdtheoremenv[nobreak=true]{exercise}[theorem]{Exercise}

\DeclareMathOperator{\sgn}{sgn}
\usepackage{bbm}
\newcommand{\ind}[1]{\mathbbm{1}_{#1}}
\newcommand{\given}{\mid}

\usepackage{pgfplots}
\pgfplotsset{compat=1.18}

\newcounter{assignment}
\theoremstyle{definition}
\newmdtheoremenv[nobreak=true, outerlinewidth=0.7]{problem}{Problem}[assignment]
\newmdtheoremenv[nobreak=true]{exercise}[theorem]{Exercise}

\DeclareMathOperator{\sgn}{sgn}
\usepackage{bbm}
\newcommand{\ind}[1]{\mathbbm{1}_{#1}}
\newcommand{\given}{\mid}

\usepackage{pgfplots}
\pgfplotsset{compat=1.18}

\newcounter{assignment}
\theoremstyle{definition}
\newmdtheoremenv[nobreak=true, outerlinewidth=0.7]{problem}{Problem}[assignment]
\newmdtheoremenv[nobreak=true]{exercise}[theorem]{Exercise}

\DeclareMathOperator{\sgn}{sgn}
\usepackage{bbm}
\newcommand{\ind}[1]{\mathbbm{1}_{#1}}
\newcommand{\given}{\mid}


\title{Assignment 1}
\author{Naman Mishra}
\date{07 January, 2024}

\begin{document}
\maketitle
\setcounter{assignment}{1}

\begin{problem*}
    Let $(\Z, +_\Z, \cdot_\Z, \le_\Z)$ be defined as in class.
    Recall that we identify $n \in \N$ with $[(n, 0)] \in \Z$.
    Show that any element of $\Z$ is either $m$ or $-m$ for some $m \in \N$.
\end{problem*}
\begin{proof}
    Proved in \refifdef{thm:Z:prop}{\cref}%
    {the last proposition on integers}.
\end{proof}

\begin{problem*}
    Recall the construction of $\Q$ as the set of equivalence classes
    of the relation $R$ on $\Z \times \Z \setminus \set{0}$ given by
    $(a, b) R (c, d) \iff ad = bc$.
    We say that $[(a, b)] \le [(c, d)]$ if $(bc - ad)(bd) \ge 0$.
    Using only the arithmetic and order properties of integers,
    show that the relation $\le$ is well-defined.
    Remember you are no allowed to divide yet!
\end{problem*}
\begin{proof}
    \refifdef{thm:Q:well-defined}{\Cref}%
    {Proved immediately after the definition}.
\end{proof}

\begin{problem*} \label{prb:ZQ:lub}
    Without assuming the existence of irrational numbers, show that
    \begin{enumerate}[label=(\alph*)]
        \item $(\Z, \le)$ has the least upper bound property.
        \item $(\Q, \le)$ does not have the least upper bound property.
    \end{enumerate}
    \textit{You may directly cite any theorem(s) proved in class.}
\end{problem*}
\begin{proof} \leavevmode
    \begin{enumerate}[label=(\alph*)]
        \item Let $S$ be a non-empty bounded above subset of $\Z$.
        Let $b$ be an upper bounded of $S$ and let
        $f \colon \Z \to \N$ be as $f(x) = b - x$.
        By the well-ordering principle, $f(S)$ has a least element $m$.
        Then $b - m$ is the maximum of $S$.
        \item \refifdef{thm:Q:no_lub}{\Cref}{Corollary 1.21}. \qedhere
    \end{enumerate}
\end{proof}

\begin{problem*} \label{prb:archimedean<=>dense}
    Let $F$ be an ordered field.
    Recall that $\Q \subseteq F$.
    Show that the following two statements are equivalent.
    \begin{enumerate}
        \item For every $a, b > 0$ in $F$,
            there is an $n \in \N$ such that $na > b$.
            \label{prb:archimedean<=>dense:==>}
        \item For every $a < b$ in $F$,
            there is an $r \in \Q$ such that $a < r < b$.
            \label{prb:archimedean<=>dense:<==}
    \end{enumerate}
\end{problem*}
\begin{proof}
    Suppose \cref{prb:archimedean<=>dense:==>} holds.
    Let $a < b$ in $F$.
    Then $1 / (b - a) > 0$.
    Let $n \in \N$ be such that $n > 1 / (b - a)$, \ie, $1/n < b - a$.
    % Let $S = \set{m \in \Z \mid m \cdot \frac1n \le a}$.
    We first show that there is a rational at most $a$.
    If $a \ge 0$, this is trivial.
    Otherwise, $-a > 0$ and so by \cref{prb:archimedean<=>dense:==>},
    there is an $m \in \N$ such that $m > 1 / (-a) \iff -1 / m < a$.
    Thus the set $S = \set{k \in \Z \mid k \cdot \frac1n \le a}$
    is non-empty.
    By \cref{prb:archimedean<=>dense:==>}, it is bounded above.
    By \cref{prb:ZQ:lub}(a), it has a maximum $M$.
    Then $\frac{M}{n} \le a < \frac{M + 1}{n} \le a + \frac1n < b$.
    Thus $\frac{M + 1}{n}$ is the required rational.

    Suppose \cref{prb:archimedean<=>dense:<==} holds.
    Let $0 < a, b$.
    Then there exist $p \in \N$ and $q \in \N^*$ such that
    $0 < b/a < p/q < b/a + 1$.
    Since $1 \le q$, $p/q \le p$.
    Then $b < pa$ as required.
\end{proof}

\begin{problem*}
    Let $F$ be a field.
    An absolute value of $F$ is a function $A \colon F \to \R$ satisfying
    \begin{enumerate}[label=(\arabic*)]
        \item $A(x) \ge 0$ for all $x \in F$,
        \item $A(x) = 0$ if and only if $x = 0$,
        \item $A(xy) = A(x)A(y)$ for all $x, y \in F$,
        \item $A(x + y) \le A(x) + A(y)$ for all $x, y \in F$.
    \end{enumerate}
    A subset $S \subseteq F$ is said to be $A$-bounded if there exists an
    $M > 0$ such that $A(s) \le M$ for all $s \in S$.
    This is a way to define boundedness of sets
    in the absence of an order relation.

    Let $p \in \N$ be a prime number.
    Define $\nu_p \colon \Z \to \Z \cup \set{\infty}$ by \[
        \nu_p(n) = \begin{cases}
            \max\set{k \in \N : p^k \mid n}, & \text{if } n \ne 0, \\
            \infty, & \text{if } n = 0.
        \end{cases}
    \] Extend $\nu_p$ to $\Q$ by \[
        \nu_p(a/b) = \nu_p(a) - \nu_p(b), \quad a, b \in \Z, b \ne 0.
    \] Now, define $A_p \colon \Q \to \R$ by $A_p(x) = e^{-\nu_p(x)}$ if
    $x \ne 0$, and $A_p(0) = 0$.
    \begin{enumerate}[label=(\alph*)]
        \item Show that $A_p$ is an absolute value on $\Q$.
        \item Show that \[
            A_p(x + y) \le \max\set{A_p(x), A_p(y)}, \quad x, y \in \Q.
        \]
        \item Show that $\Z$ is $A_p$-bounded.
    \end{enumerate}
    \textit{You may use basic facts about factorization without proof,
    but clearly state what you are using.}
\end{problem*}
\begin{proof}
    $A_p$ satisfies (1) and (2) by definition.

    Let $x = a/b$, $y = c/d$ in $\Q$.
    If either is zero, (3) holds trivially. \\
    Otherwise $xy = ac/bd$ with $a, b, c, d \in \Z^*$.
    Let $a = p^{\nu_p(a)} a'$, $c = p^{\nu_p(c)} c'$, where $a', c'$ are
    coprime to $p$.
    Then $ac = p^{\nu_p(a) + \nu_p(c)} (a'c')$.
    Thus $\nu_p(ac) = \nu_p(a) + \nu_p(c)$.
    Similarly, $\nu_p(bd) = \nu_p(b) + \nu_p(d)$.
    Thus $\nu_p(xy) = \nu_p(x) + \nu_p(y)$ and so
    $A_p(xy) = A_p(x) A_p(y)$.

    (4) follows from (b), which we prove now.
    If either $x$ or $y$ is zero, (b) holds trivially.
    Let \[
        x = \frac{p^\alpha a}{p^\beta b}, \quad
        y = \frac{p^\gamma c}{p^\delta d}, \quad
    \] where $a, b, c, d \in \Z^*$ are coprime to $p$.
    Thus $\nu_p(x) = \alpha - \beta$ and $\nu_p(y) = \gamma - \delta$.
    WLOG suppose that $A_p(x) \ge A_p(y) \iff \nu_p(x) \le \nu_p(y)$
    which gives $\alpha - \beta \le \gamma - \delta$.
    \begin{align*}
        x + y &= \frac{p^{\alpha + \delta} ad
                    + p^{\beta + \gamma} bc}{p^{\beta + \delta} bd} \\
        &= \frac{p^{\alpha + \delta}
                (ad + p^{\beta + \gamma - \alpha - \delta} bc)}
            {p^{\beta + \delta} bd}
    \end{align*}
    Thus $\nu_p(x + y) \ge \alpha + \delta - \beta - \delta
                   = \alpha - \beta$ and so
    $A_p(x + y) \le A_p(x) = \max\set{A_p(x), A_p(y)}$.

    (c) follows from $\nu_p(x) \ge 0$, so that $A_p(x) \le 1$
    for all $x \in \Z$.
\end{proof}

\end{document}
