\documentclass[12pt]{article}
\let\emph\textsl
\let\emph\textsl
\let\emph\textsl
\input{~/IISc/preamble}
\usepackage{pgfplots}
\pgfplotsset{compat=1.18}

\newcounter{assignment}
\theoremstyle{definition}
\newmdtheoremenv[nobreak=true, outerlinewidth=0.7]{problem}{Problem}[assignment]
\newmdtheoremenv[nobreak=true]{exercise}[theorem]{Exercise}

\DeclareMathOperator{\sgn}{sgn}
\usepackage{bbm}
\newcommand{\ind}[1]{\mathbbm{1}_{#1}}
\newcommand{\given}{\mid}

\usepackage{pgfplots}
\pgfplotsset{compat=1.18}

\newcounter{assignment}
\theoremstyle{definition}
\newmdtheoremenv[nobreak=true, outerlinewidth=0.7]{problem}{Problem}[assignment]
\newmdtheoremenv[nobreak=true]{exercise}[theorem]{Exercise}

\DeclareMathOperator{\sgn}{sgn}
\usepackage{bbm}
\newcommand{\ind}[1]{\mathbbm{1}_{#1}}
\newcommand{\given}{\mid}

\usepackage{pgfplots}
\pgfplotsset{compat=1.18}

\newcounter{assignment}
\theoremstyle{definition}
\newmdtheoremenv[nobreak=true, outerlinewidth=0.7]{problem}{Problem}[assignment]
\newmdtheoremenv[nobreak=true]{exercise}[theorem]{Exercise}

\DeclareMathOperator{\sgn}{sgn}
\usepackage{bbm}
\newcommand{\ind}[1]{\mathbbm{1}_{#1}}
\newcommand{\given}{\mid}


\title{Lecture 28 -- Questions}
\setcounter{assignment}{28}
\author{Naman Mishra}
\date{23 March, 2024}

\begin{document}
\maketitle

\begin{problem*}
    Construct a function $f\colon (-1, 1) \to \R$ such that $f$ is
    differentiable on $(-1, 1)$, $f'(0) > 0$, but $f$ is not monotone in
    any neighbourhood of $0$.
\end{problem*}
\begin{solution}
    \[
        f(x) = \begin{cases}
            \frac12 x + x^2\sin\left(\frac{1}{x}\right) & x \neq 0, \\
            0 & x = 0.
        \end{cases}
    \] Then for $x \ne 0$, \[
        f'(x) = \frac12 + 2x \sin \frac1x - \cos \frac1x,
    \] and $f'(0) = \frac12$.

    $f$ is differentiable everywhere, so if it is monotone on an interval,
    $f'$ must have the same sign on that interval.

    But for each $x_n = \frac1{2n\pi} \xrightarrow{n \to \infty} 0$,
    $f'(x_n) < 0$.
    So no neighbourhood of $0$ can contain only positive values of $f'$.
\end{solution}

\begin{problem*}
    Let $f\colon (a, b) \to \R$ be differentiable and monotonically
    increasing.
    Suppose \[
        Z = \set{x \in (a, b) : f'(x) = 0}
    \] does not have any limit points in $(a, b)$.
    Is $f$ necessarily strictly increasing on $(a, b)$?
\end{problem*}
\begin{solution}
    Obviously?
    Suppose it is only weakly increasing, that is,
    there exist $x_1 < x_2$ such that $f(x_1) = f(x_2)$.
    Since it is increasing, $f$ is constant on $[x_1, x_2]$ and
    $f'$ is $0$ on $(x_1, x_2)$.
    Then $x_1$ is a limit point of $Z$.
\end{solution}

\begin{problem*}
    Is there a differentiable function $f\colon \R \to \R$ such that
    $f'$ vanishes only on the rationals?
\end{problem*}

\begin{problem*}
    Construct three functions $f, g, h\colon \R \to \R$ such that
    \begin{enumerate}
        \item they are continuous on $[0, 1]$,
        \item they are differentiable on $(0, 1)$, and
        \item there is no $c \in (0, 1)$ for which the vectors \[
            (f'(c), g'(c), h'(c))
            \quad \text{and} \quad
            (f(1) - f(0), g(1) - g(0), h(1) - h(0))
        \] are linearly dependent in $\R^3$.
    \end{enumerate}
\end{problem*}
\begin{solution}
    Take the helix \[
        f(x) = x, \quad
        g(x) = \cos 2\pi x, \quad
        h(x) = \sin 2\pi x.
    \] Then \[
        (f', g', h')(c) = (1, -2\pi \sin 2\pi c, 2\pi \cos 2\pi c)
    \] and \[
        (f, g, h)(1) - (f, g, h)(0) = (1, 0, 0).
    \] But both $g'(c)$ and $h'(c)$ cannot be $0$ simultaneously.
\end{solution}

\end{document}
