\lecture{2024-03-20}{Taylor's theorem and L'H\^opital's rule}
\begin{proof}
    Let \[
        h(x) = (f(b) - f(a))g(x) - (g(b) - g(a))f(x).
    \] Then \begin{align*}
        h(a) &= f(b) g(a) - \cancel{f(a) g(a)}
                - g(b) f(a) + \cancel{g(a) f(a)} \\
             &= f(b) g(a) - g(b) f(a) \\
        h(b) &= \cancel{f(b) g(b)} - f(a) g(b)
                - \cancel{g(b) f(b)} + g(a) f(b) \\
             &= f(b) g(a) - g(b) f(a).
    \end{align*}
    Thus $h(a) = h(b)$.
    By Rolle's theorem, there exists $c \in (a, b)$ such that $h'(c) = 0$.
    This proves the result.
\end{proof}

\begin{exercise}
    Prove that if $f\colon (a, b) \to \R$ is differentiable on $(a, b)$
    with $f' > 0$ on the entire interval, then $f$ is strictly increasing.

    Similarly if $f' \ge 0$ on the entire interval, then $f$ is
    weakly increasing.
\end{exercise}
\begin{proof}
    Let $x, y \in (a, b)$ with $x < y$.
    Suppose $f(x) \ge f(y)$.
    Then by the mean value theorem, there exists $c \in (x, y)$ such that
    $f'(c) = \frac{f(y) - f(x)}{y - x} \le 0$.

    If $f(x) > f(y)$, then $f'(c) < 0$.
    This proves both parts of the exercise.
\end{proof}

\begin{theorem*}[Taylor's theorem] \label{thm:taylor}
    Let $n \in \N^*$.
    Suppose that
    $f\colon (a, b) \to \R$ is $n$ times differentiable on $(a, b)$.
    Further assume that $f, f', \dots, f^{(n-1)}$ extend continuously
    to $[a, b]$.
    Then there exists a point $c \in (a, b)$ such that \[
        f(b) = f(a) + f'(a)(b - a) + \dots
        + \frac{f^{(n-1)}(a)}{(n-1)!}(b - a)^{n-1} + \frac{f^{(n)}(c)}{n!}(b - a)^n.
    \]
\end{theorem*}
\begin{proof}
    Suppose the result holds in the case when \begin{equation}
        f(a) = f'(a) = \dots = f^{(n-1)}(a) = 0. \tag{$\star$}
        \label{eq:simple_taylor}
    \end{equation}
    Let $f$ be given as per the theorem.
    Define \begin{align*}
        F(x) &= f(x) - \sum_{j=0}^{n-1} \frac{f^{(j)}(a)}{j!}(x - a)^j \\
        \implies F^{(k)}(x) &= f^{(k)}(x) - f^{(k)}(a) -
            \sum_{j=k+1}^{n-1} a_j (x - a)^{j-k} \\
        \implies F^{(k)}(a) &= 0
    \end{align*} for every $k \in \set{0, 1, \dots, n-1}$.

    Then from the case (\ref{eq:simple_taylor}),
    there exists $c \in (a, b)$ such that \[
        F(b) = \frac{F^{(n)}(c)}{n!}(b - a)^n.
    \] But $F^{(n)} = f^{(n)}$, so this immediately gives \[
        f(b) = f(a) + f'(a)(b - a) + \dots
            + \frac{f^{(n-1)}(a)}{(n-1)!}(b - a)^{n-1}
            + \frac{f^{(n)}(c)}{n!}(b - a)^n.
    \]
    But how do we prove the case (\ref{eq:simple_taylor})?

    Suppose $f$ satisfies (\ref{eq:simple_taylor}).
    Define \[
        g(x) = f(x) - f(b) \frac{(x-a)^n}{(b-a)^n}.
    \] This ensures that $g(a) = g(b) = 0$, and since derivatives of $f$
    extend continuously to $[a, b]$, so do the derivatives of $g$.
    Then $g'(a) = g''(a) = \dots = g^{(n-1)}(a) = 0$.
    For the $n$th derivative, \[
        g^{(n)}(x) = f^{(n)}(x) - \frac{f(b) n!}{(b - a)^n}.
    \]

    Now apply Rolle's theorem iteratively to $g$, $g'$, \dots, $g^{(n-1)}$.
    \begin{itemize}
        \item[($1$)] From the first application, there exists $c_1 \in (a, b)$
        such that $g'(c_1) = 0$.
        \item[($2$)] But $g'(a)$ is also zero, so by a second application
        there exists $c_2 \in (a, c_1)$ such that $g''(c_2) = 0$.
        \item[$\vdots$\hphantom{)}]
        \item[($n-1$)] Continuing in this way, we find $c_{n-1}$ such that
        $g^{(n-1)}(c_{n-1}) = 0$.
        \item[($n$)] In the final application, we find $c_n \in (a, c_{n-1})$
        such that \[
            g^{(n)}(c_n) = f^{(n)}(c_n) - \frac{f(b) n!}{(b - a)^n} = 0.
        \]
    \end{itemize}
    Thus finally, we have that there exists a point $c_n \in (a, b)$
    such that $f(b) = \frac{(b - a)^n}{n!} f^{(n)}(c_n)$.
    This proves the case (\ref{eq:simple_taylor}), and hence the theorem.
\end{proof}

\begin{theorem*}[L'H\^opital's rule] \label{thm:lhopital}
    Let $a, b \in \wbar{\R}$.
    Suppose $f, g\colon (a, b) \to \R$ are differentiable on $(a, b)$,
    and that $g$ and $g'$ are never zero on $(a, b)$.
    Suppose also that there exists an $A \in \wbar{\R}$ such that \[
        \lim_{x \downarrow a} \frac{f'(x)}{g'(x)} = A.
    \] Then \[
        \lim_{x \downarrow a} \frac{f(x)}{g(x)} = A
    \] whenever any of the following conditions hold.
    \begin{enumerate}
        \item $\lim_{x \downarrow a} f(x) = \lim_{x \downarrow a} g(x) = 0$, or
        \item $\lim_{x \downarrow a} g(x) = +\infty$, or
        \item $\lim_{x \downarrow a} g(x) = -\infty$.
    \end{enumerate}
\end{theorem*}
\begin{proof}[Case 1]
    Suppose $\lim_{x \downarrow a} f(x) = \lim_{x \downarrow a} g(x) = 0$.
    We will only consider the case when $a, A \in \R$.

    Let $\varepsilon > 0$.
    Then there exists $\delta > 0$ such that
    for every $x \in (a, a + \delta)$, \[
        A - \varepsilon < \frac{f'(x)}{g'(x)} < A + \varepsilon.
    \] Fix a $y \in (a, a + \delta)$.
    Let $(z_n)_{n \in \N} \subseteq (a, y)$ be a sequence converging to
    $a$ such that $g(z_n)$ is never $g(y)$.%
    \footnote{Why does such a sequence exist?
    Since $g'$ is never zero on $(a, b)$,
    it cannot be constant on any interval.
    Thus every interval $(a, a+\frac1n) \cap (a, b)$ contains a point $z_n$
    such that $g(z_n) \ne g(y)$.}

    Then by the generalized mean value theorem, for every $n \in \N$,
    there exists a $w_n \in (z_n, y)$ such that \[
        \frac{f(y) - f(z_n)}{g(y) - g(z_n)} = \frac{f'(w_n)}{g'(w_n)}.
    \] But then \[
        A - \varepsilon
            < \frac{f(y) - f(z_n)}{g(y) - g(z_n)}
                < A + \varepsilon.
    \] Taking limits as $n \to \infty$, we find \[
        A - \varepsilon < \frac{f(y)}{g(y)} < A + \varepsilon.
    \] Since $\varepsilon > 0$ was arbitrary, we have that \[
        \lim_{y \downarrow a} \frac{f(y)}{g(y)} = A.
    \]
\end{proof}
