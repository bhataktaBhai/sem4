\lecture{2024-03-14}{Differentiability}
\begin{proof}
    WLOG let $f\colon (a, b) \to \R$ be increasing.
    Let $c \in (a, b)$.

    We will show that \[
        \lim_{x \uparrow c} f(x) = \sup_{x \in (a, c)} f(x).
    \] Let $L^- = \sup_{x \in (a, c)} f(x)$.
    For any $\varepsilon > 0$, there is some $x \in (a, c)$ such that
    $L^- - \varepsilon < f(x) \leq L^-$.
    Choose $\delta = c - x$.
    Then by monotonicity, for $x' \in (c - \delta, c)$, we have \[
        L^- - \varepsilon < f(x) \le f(x') \le L^-.
    \] Thus, $\lim_{x \uparrow c} f(x) = L^-$.
    This is sufficient to prove the theorem.

    Since $f(c)$ is an upper bound for $\set{f(x) \mid x \in (a, c)}$,
    we also have $L^- \le f(c)$.
    By symmetry, \[
        \lim_{x \uparrow c} f(x) \le f(c) \le \lim_{x \downarrow c} f(x).
    \]
\end{proof}

\begin{corollary}
    The set of discontinuities of a monotone function on an interval is
    at most countable.
\end{corollary}
\begin{proof}
    Let $f\colon (a, b) \to \R$ be increasing and let $D$ be its set of
    discontinuities.
    We will construct an injection from $D$ to $\Q$.

    For each $c \in D$, we have \[
        \lim_{x \uparrow c} f(x) < \lim_{x \downarrow c} f(x),
    \] since equality would force them to be equal to $f(c)$.
    By the density of $\Q$ in $\R$, map $c$ to some $q_c \in \Q$ between
    these two limits.

    Then for any $c < d \in D$, we have $q_c < q_d$, since \[
        q_c < \lim_{x \downarrow c} f(x) \le \lim_{x \uparrow d} f(x) < q_d.
    \] Thus the map is injective.
\end{proof}

However! Discontinuities being countable does not imply that they cannot
be dense.
\begin{example}
    Let $D \subseteq (0, 1)$ be a dense countable set.
    Let $\set{x_1, x_2, \dots}$ be an enumeration.
    Define \[
        f(x) = \sum_{n : x_n < x} \frac1{2^n}.
    \]
\end{example}
