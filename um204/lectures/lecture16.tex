\begin{proof}[Proof of $(3) \Rightarrow (2)$]
\lecture{2024-02-07}{Completeness of metric spaces}
    Let $S \subseteq E$ be an infinite set.
    Thus there exists a sequence $(x_n)_n \subseteq S$ of distinct elements.

    By $(3)$, there exists a subsequence $(x_{n_k})_k$ of $(x_n)_n$
    such that $(x_{n_k})_k$ is convergent to some $x \in E$.
    By the sequential characterization of closures, $x \in \wbar{S}$.

    Thus for all $\varepsilon > 0$, there exists a $k_\varepsilon \in \N$
    such that for all $k \ge k_\varepsilon$, we have that
    $d(x_{n_k}, x) < \varepsilon$.
    Thus $x$ is a limit point of $S$ in $E$.
\end{proof}

\section{Cauchy Sequences \& Completeness} \label{sec:cauchy}
Recall the HW2 problem to show that the sequence $(x_n)_n$ given by \[
    x_n = \begin{cases}
        2 & \text{if } n = 0 \\
        x_{n-1} - \frac{x_{n-1}^2 - 2}{2x_{n-1}} & \text{if } n \ge 1
    \end{cases}
\] is \Q-Cauchy but not convergent in \Q.
This is an application of the Newton-Raphson method.
\subsection{Newton-Raphson Method (Informal)}
\label{sec:newton-raphson_method}
Given a function $f\colon \R \to \R$, we want to find a root of $f$.
We pick some initial guess $x_0 \in \R$, and iterate via \[
    x_n = x_{n-1} - \frac{f(x_{n-1})}{f'(x_{n-1})}.
\] Under \emph{some} assumptions on $f$ and $x_0$, $(x_n)_n$ is Cauchy.
Then \[
    f(x_{n-1}) = f'(x_{n-1})(x_{n-1} - x_n) \to 0
\] If $\lim_{n \to \infty} x_n = l$, and $f$ is continuous, then \[
    f(l) = \lim_{n \to \infty} f(x_n) = 0.
\]
\begin{definition}[Cauchy sequence] \label{def:cauchy:sequence}
    Let $(x_n)_{n \in \N} \subseteq (X, d)$.
    We say that $(x_n)$ is a \emph{Cauchy sequence} if for every
    $\varepsilon > 0$, there exists an $N \in \N$ such that whenever
    $n, m \ge N$, $d(x_n, x_m) < \varepsilon$.
\end{definition}
\begin{definition*}[Completeness] \label{def:cauchy:completeness}
    $(X, d)$ is said to be a \emph{complete} metric space if every Cauchy
    sequence in $(X, d)$ is convergent.
\end{definition*}

\begin{theorem} \leavevmode
    \begin{enumerate}[label=(\alph*)]
        \item Every convergent sequence is Cauchy.
        \item Every Cauchy sequence is bounded.
    \end{enumerate}
\end{theorem}
\begin{proof}
    Trivial.
\end{proof}

\begin{theorem*}
    Every compact metric space is complete.
\end{theorem*}
\begin{proof}
    Let $(X, d)$ be compact and let $(x_n)_n$ be a Cauchy sequence in $X$.
    Since $X$ is compact, $(x_n)_n$ has a convergent subsequence
    $(x_{n_k})_k$ converging to some $x \in X$
    (by \cref{thm:compact:subsequential_limit}).

    Then $(x_n)_n$ also converges to $x$ by the triangle inequality.
    For large enough $n$, $d(x_n, x) \le d(x_n, x_{n_n}) + d(x_{n_n}, x)
    < 2\varepsilon$.
\end{proof}
\begin{theorem}
    $(\R^d, \norm{\cdot})$ is complete.
\end{theorem}
\begin{proof}
    Let $(x_n)_n$ be a Cauchy sequence in $\R^d$.
    Then it must be bounded.
    Take a closed ball $B$ centered at $x_0$ containing all elements of
    $(x_n)_n$.
    This is compact, and so the above theorem applies to give that
    $(x_n)_n$ has a limit in $B \subseteq \R^d$.
\end{proof}

\begin{exercise}
    Every increasing and bounded above sequence in \Q\ or \R\ is Cauchy.
\end{exercise}
\begin{proof}
    Suppose not.
    Then there exists an $\varepsilon > 0$ such that for all $N \in \N$,
    there exist $n(N) > m(N) \ge N$ such that
    $\abs{x_{n(N)} - x_{m(N)}} \ge \varepsilon$.

    Let $m_0 = m(0)$ and $n_0 = n(0)$.
    For $k \ge 1$, let $m_k = m(n_{k-1})$ and $n_k = n(n_{k-1})$.
    Then \begin{align*}
        x_{n_k} &\ge x_{m_k} + \varepsilon \\
               &\ge x_{n_{k-1}} + \varepsilon
    \end{align*} and so $(x_{n_k})_k$ is a subsequence with each term at
    least $\varepsilon$ more than the last.
    Thus $x_{n_k} \ge x_0 + k\varepsilon$ for all $k \in \N$, which
    contradicts boundedness.
\end{proof}
Alternatively, we could see \Q\ as a subset of \R, and use the fact that
a bounded monotone sequence in \R\ is convergent.
