\chapter{Differentiation} \label{chap:diff}
\section{Introduction} \label{sec:diff:intro}
\lecture{2024-03-18}{Differentiation and mean value theorem}
\begin{definition}[Differentiation] \label{def:diff}
    Let $f\colon (a, b) \to \R$ and $c \in (a, b)$.
    We say that $f$ is \emph{differentiable at $c$} with derivative $f'(c)$
    if \[
        \lim_{h \to 0} \frac{f(c + h) - f(c)}{h} = f'(c).
    \]
    We say that $f$ is \emph{differentiable} if it is differentiable at
    each $c \in (a, b)$, and the map $f'$ is called the \emph{derivative}
    of $f$.
\end{definition}
\begin{exercise}
    Differentiability at a point implies continuity at that point.
\end{exercise}
\begin{exercise}
    If $f, g: (a, b) \to \R$ are differentiable at $c \in (a, b)$, then
    \begin{enumerate}
        \item so is $f + g$ with $(f + g)'(c) = f'(c) + g'(c)$.
        \item so is $fg$ with $(fg)'(c) = f'(c) g(c) + f(c) g'(c)$.
        \item so is $f/g$ with
        $(f/g)'(c) = \frac{f'(c) g(c) - f(c) g'(c)}{g(c)^2}$,
        provided $g(c) \ne 0$.
    \end{enumerate}
\end{exercise}
\newpage
\begin{exercise}[Chain rule]
    Suppose $f\colon (a, b) \to \R$ is differentiable at $c \in (a, b)$.
    Suppose $f((a, b)) \subseteq (p, q)$ and $g\colon (p, q) \to \R$ is
    differentiable at $f(c)$.
    Then $g \circ f$ is differentiable at $c$ and \[
        (g \circ f)'(c) = g'(f(c)) f'(c).
    \]
\end{exercise}
\begin{proof}
    Call a point $x \ne c$ such that $f(x) = f(c)$ a `bro' of $c$.

    Let $(x_n)_{n \in \N} \subseteq (a, b) \setminus \set{c}$
    be an (eventually) bro-free sequence converging to $c$.
    Then \begin{align*}
        \lim_{n \to \infty} \frac{g(f(x_n)) - g(f(c))}{x_n - c}
            &= \lim_{n \to \infty} \frac{g(f(x_n)) - g(f(c))}{f(x_n) - f(c)}
                \frac{f(x_n) - f(c)}{x_n - c} \\
            &= g'(f(c)) f'(c).
    \end{align*}
    We have two cases for sequences
    $(x_n)_n \subseteq (a, b) \setminus \set{c}$ converging to $c$.
    \begin{itemize}
        \item If there is a neighbourhood of $c$ that contains no bros,
        then all such sequences are eventually bro-free.
        Thus by sequential characterization of limits,
        $(g \circ f)'(c) = g'(f(c)) f'(c)$.
        \item If every neighbourhood of $c$ contains a bro,
        then $f'(c) = 0$.
        A bro-free sequence will give the above limit as $g'(f(c))f'(c)=0$.
        A bro-only sequence will make the above limit $0$.
        A mix of these can be split into a bro-only sequence and a
        bro-free sequence, both of which have limit $0$.

        Thus by sequential characterization of limits, \[
            (g \circ f)'(c) = 0 = g'(f(c)) f'(c). \qedhere
        \]
    \end{itemize}
\end{proof}

Recall that derivatives are not very well-behaved!
\begin{example}
    Let \[
        f(x) = \begin{cases}
            x^2 \sin \frac1x & \text{if } x \ne 0, \\
            0 & \text{if } x = 0.
        \end{cases}
    \]
    We show that $f$ is differentiable on \R, but $f'$ is discontinuous
    at $0$.
    \begin{proof}
        When $x \ne 0$, we use the fact that polynomials and trignometric
        functions are differentiable, so that \[
            f'(x) = 2x \sin \frac1x - \cos \frac1x.
        \] At $x = 0$, we have \[
            \lim_{h \to 0} \frac{f(h) - f(0)}{h}
            = \lim_{h \to 0} h \sin \frac1h = 0.
        \] Thus the derivative is well-defined with \[
            f'(x) = \begin{cases}
                2x \sin \frac1x - \cos \frac1x & \text{if } x \ne 0, \\
                0 & \text{if } x = 0.
            \end{cases} \qedhere
        \]
    \end{proof}
\end{example}

\section{Mean Value Theorems \& Applications} \label{sec:mvt}
\begin{theorem} \label{thm:stationary}
    Let $f\colon [a, b] \to \R$ be a function.
    Suppose $f$ attains a local extremum at $c \in (a, b)$ and $f$ is
    differentiable at $c$.
    Then $f'(c) = 0$.
\end{theorem}
\begin{proof}
    Suppose WLOG that $f$ attains a local maximum at $c$.
    Then we have $f(c) \ge f(x)$ for all $x$ close to $c$.
    But then for $h > 0$, \[
        \frac{f(c + h) - f(c)}{h} \le 0
    \] and \[
        \frac{f(c - h) - f(c)}{-h} \ge 0.
    \] If the left hand and right hand limits of the difference quotient
    exist and are equal, then they must both be zero.
\end{proof}

\begin{exercise}
    Let $f\colon (a, b) \to \R$ be a function.
    Suppose $f$ is differentiable at $c \in (a, b)$ and $f'(c) > 0$.
    Then does there exist an interval around $c$ such that $f$ is
    increasing on that interval?
\end{exercise}
\begin{solution}
    No.
    Consider \[
        f(x) = \begin{cases}
            x^2 \sin \frac{1}{x} + \frac12 x & \text{if } x \ne 0, \\
            0 & \text{if } x = 0.
        \end{cases}
    \] $f'(0) = 0$, and for $x \ne 0$, \[
        f'(x) = 2x \sin \frac1x - \cos \frac1x + \frac12.
    \] This attains negative values in any neighbourhood of $0$.
    Thus $f$ is not increasing in any neighbourhood of $0$.
\end{solution}

\begin{exercise}
    Let $f\colon (a, b) \to \R$ be differentiable.
    Suppose $f'(x) \ge 0$ for all $x \in (a, b)$, and furthermore that
    \begin{itemize}
        \item $f'$ is not identically zero on any interval.
        Is $f$ strictly increasing on $(a, b)$?
        \item $f'$ is zero on a discrete set of points.
        Is $f$ strictly increasing on $(a, b)$?
    \end{itemize}
\end{exercise}

\begin{theorem}[Intermediate value property] \label{thm:diff:ivt}
    Suppose $f\colon (p, q) \to \R$ is differentiable and
    $a < b \in (p, q)$.
    Suppose $f'(a) < \lambda < f'(b)$.
    Then there exists a $c \in (a, b)$ such that $f'(c) = \lambda$.
\end{theorem}
\begin{proof}
    Let $g(x) = f(x) - \lambda x$ for $x \in (p, q) \supseteq [a, b]$.
    Then $g'(x) = f'(x) - \lambda$ so that $g'(a) < 0 < g'(b)$.
    This means that $g$ is strictly decreasing from $a$
    and strictly increasing to $b$.

    Thus $g$ attains a minimum at some $c \in (a, b)$.
    Then $g'(c) = 0$ so that $f'(c) = \lambda$.
\end{proof}

\begin{exercise}
    If $g\colon (a, b) \to \R$ has a simple discontinuity at $c \in (a, b)$,
    then $g$ does not satisfy the intermediate value property on some
    neighbourhood of $c$.
\end{exercise}
\begin{proof}
    We have two cases.
    \begin{description}
        \item[$\bm{f(c^-) = f(c^+) = L \ne f(c)}$.]
        Choose $\lambda = \frac12 (L + f(c))$.
        In some $\delta$ neighbourhood of $c$ (excluding $c$ itself),
        we have $f(x) - L < \frac12 (f(c) - L)$ so that
        $f(x) < \lambda$.
        At $c$ itself, we have $f(c) > \lambda$.
        Thus $\lambda$ is never attained.
        \item[$\bm{f(c^-) \ne f(c^+)}$.]
        Choose $\lambda$ between the two limits, but unequal to $f(c)$.
        \qedhere
    \end{description}
\end{proof}

\begin{corollary}
    Let $f\colon (p, q) \to \R$ be differentiable.
    Then $f'$ only has discontinuities of the second kind.
\end{corollary}

\begin{theorem*}[Generalised mean value theorem] \label{thm:mvt:general}
    Let $f, g\colon [a, b] \to \R$ be continuous functions that are
    differentiable on $(a, b)$.
    Then there exists a $c \in (a, b)$ such that \[
        (f(b) - f(a)) g'(c) = (g(b) - g(a)) f'(c).
    \]
\end{theorem*}
\begin{remark}
    This is also called the Cauchy mean value theorem.

    If $g(x) = x$, we recover the mean value theorem.
    If furthermore $f(a) = f(b)$, we recover Rolle's theorem.

    One interpretation is that the ratio of the average rate of change of
    two functions is achieved by the ratio of their instantaneous rates of
    change at some point.

    Another is that if $x \mapsto (f(x), g(x))$ is a path in $\R^2$, then
    the line joining the endpoints of the path is parallel to the tangent
    at some point on the path.
\end{remark}
