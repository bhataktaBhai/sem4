\lecture{2024-03-22}{}
\begin{proof}[Case 2]
    Suppose that $\lim_{x \downarrow a} g(x) = +\infty$.
    Let $\varepsilon > 0$ and again choose $\delta$ such that for every
    $x \in (a, a + \delta)$, \[
        \abs{\frac{f'(x)}{g'(x)} - A} < \varepsilon.
    \]
    Fix a $y \in (a, b)$.
    Then there exists a $\delta' \in (0, \delta)$ such that
    for every $z \in (a, a + \delta')$, \[
        g(z) > \abs{g(y)} + 1 > g(y), 0.
    \]
    \textcolor{Red}{I cannot follow Purvi's proof.}
\end{proof}

\section{Vector-valued Functions} \label{sec:diff:vector}
The definition is identical to the scalar case, the only point being how to
interpret the limit in the definition.
\begin{definition}[Differentiation] \label{def:diff:vector}
    Let $f\colon (a, b) \to \textcolor{Red}{\R^n}$ and $c \in (a, b)$.
    We say that $f$ is \emph{differentiable at $c$} with derivative $f'(c)$
    if \[
        \textcolor{Red}{\lim_{h \to 0} \norm{\frac{f(c + h) - f(c)}{h} - f'(c)} = 0}.
    \]
    We say that $f$ is \emph{differentiable} if it is differentiable at
    each $c \in (a, b)$, and the map $f'$ is called the \emph{derivative}
    of $f$.
\end{definition}
\begin{remarks}
    \item If $f(x) = (f_1(x), f_2(x), \dots, f_n(x))$, then
    $f$ is differentiable at $c$ iff each $f_i$ is differentiable at $c$,
    and \[
        f'(c) = (f_1'(c), f_2'(c), \dots, f_n'(c)).
    \]
    \item If $f, g\colon (a, b) \to \R^n$ are differentiable at some $c$,
    then \begin{align*}
        (f + g)'(c) &= f'(c) + g'(c), \\
        \innerp fg'(c) &= \innerp{f'(c)}{g(c)} + \innerp{f(c)}{g'(c)}, \\
    \end{align*}
\end{remarks}

\begin{theorem}[Mean value inequality] \label{thm:diff:vector:mvt}
    Suppose $f\colon [a, b] \to \R^n$ is continuous on $[a, b]$ and
    differentiable on $(a, b)$.
    Then there exists a $c \in (a, b)$ such that \[
        \norm{f(b) - f(a)} \leq (b - a) \norm{f'(c)}.
    \]
\end{theorem}
\begin{proof}
    Let $g(x) = \innerp{f(x)}{f(b) - f(a)}$.
    Then $g$ is continuous on $[a, b]$ and differentiable on $(a, b)$.
    By the mean value theorem, there exists a $c \in (a, b)$ such that
    \begin{align*}
        g(b) - g(a) &= g'(c) (b - a) \\
        \norm{f(b) - f(a)}^2 &= \innerp{f'(c)}{f(b) - f(a)} (b - a) \\
        &\le \norm{f'(c)} \norm{f(b) - f(a)} (b - a),
    \end{align*} so that \[
        \norm{f(b) - f(a)} \le \norm{f'(c)} (b - a).
    \] (if $f(b) = f(a)$, the inequality is trivially true).
\end{proof}
