\lecture{2024-04-01}{}
\begin{theorem} \label{thm:ssfn:sup}
    Let $(f_n)_{n \in \N}$ converge pointwise to $f$.
    Put \[
        M_n = \sup_{x \in X} d_Y(f_n(x), f(x)).
    \] Then $f_n \unifto f$ iff $M_n \to 0$.
\end{theorem}
\begin{proof}
    \begin{align*}
        f_n \unifto f
        &\iff \forall\varepsilon \,\exists N \,\forall n \ge N \,\forall x\,
            (d(f_n(x), f(x)) < \varepsilon) \\
        &\iff \forall\varepsilon \,\exists N \,\forall n \ge N\,
            (\varepsilon \text{ is an upper bound for } d(f_n(x), f(x))) \\
        &\iff \forall\varepsilon \,\exists N \,\forall n \ge N\,
            (M_n \le \varepsilon) \\
        &\iff M_n \to 0. \qedhere
    \end{align*}
\end{proof}

\section{Continuity} \label{sec:ssfn:cont}
\begin{theorem} \label{thm:ssfn:limswap}
    Let $(f_n\colon E \subseteq X \to \C)_{n \in \N}$ converge uniformly
    to $f\colon E \to \C$.
    Let $x$ be a limit point of $E$, and suppose that \[
        \lim_{z \to x} f_n(z) = A_n.
    \] Then $(A_n)_{n \in \N}$ converges, and \[
        \lim_{n \to \infty} A_n = \lim_{z \to x} f(z).
    \] In other words, \[
        \lim_{n \to \infty} \lim_{z \to x} f_n(z)
            = \lim_{z \to x} \lim_{n \to \infty} f_n(z).
    \]
\end{theorem}
\begin{proof}
    Let $\varepsilon > 0$.
    Since $f_n \unifto f$, there exists an $N$ such that \[
        d(f_m(z), f_n(z)) < \varepsilon
    \] for all $z \in E$ and $m, n \ge N$.

    Thus for close enough $z$ and large enough $m, n$, \[
        d(A_m, A_n) \le d(A_m, f_m(z)) + d(f_m(z), f_n(z)) + d(f_n(z), A_n)
            < 3\varepsilon.
    \] Thus $(A_n)_{n \in \N}$ is Cauchy and hence converges to some $A$.

    Next, for large enough $n$ and close enough $z$, \[
        d(A, f(z)) \le d(A, A_n) + d(A_n, f(z)) < 2\varepsilon,
    \] which proves that \[
        \lim_{z \to x} f(z) = A. \qedhere
    \]
\end{proof}

\begin{theorem*} \label{thm:ssfn:cont}
    Let $f_n\colon E \subseteq X \to \C$ be continuous for all $n$.
    If $f_n \unifto f$, then $f$ is continuous.
\end{theorem*}
\begin{proof}
    From the previous theorem, \begin{align*}
        \lim_{z \to x} f(z) &= \lim_{n \to \infty} \lim_{z \to x} f_n(z) \\
        &= \lim_{n \to \infty} f_n(x) \\
        &= f(x). \qedhere
    \end{align*}
\end{proof}

\begin{definition}[Supremum norm] \label{def:ssfn:supnorm}
    Let $\mcC(X)$ denote the set of all complex-valued, continuous and
    bounded functions on $X$.

    The \emph{supremum norm} on $\mcC(X)$ is defined by \[
        \norm{f}_\infty = \sup_{x \in X} \abs{f(x)}.
    \]
\end{definition}
It is easy to check that this is a metric.
\begin{remark}
    \Cref{thm:ssfn:sup} can be restated as \[
        f_n \unifto f \iff f_n \to f \text{ in } (\mcC(X), \norm{\cdot}_\infty).
    \]
\end{remark}

\begin{theorem}[Completeness] \label{thm:ssfn:cont:completeness}
    $(\mcC(X), \norm{\cdot}_\infty)$ is complete.
\end{theorem}
\begin{proof}
    Let $(f_n)_{n \in \N}$ be a Cauchy sequence in $\mcC(X)$.
    That is, for all $\varepsilon > 0$, there exists an $N$ such that \[
        \norm{f_m - f_n}_\infty < \varepsilon
    \] for all $m, n \ge N$.

    But $\sup A - \sup B \le \sup (A - B)$, so \[
        \abs{\sup_{x \in X} f_m(x) - \sup_{x \in X} f_n(x)}
            \le \sup_{x \in X} \abs{f_m(x) - f_n(x)} < \varepsilon.
    \] By \cref{thm:ssfn:cauchy}, $(f_n)_{n \in \N}$ converges uniformly
    to some $f\colon X \to \C$.

    By \cref{thm:ssfn:cont}, $f$ is continuous.
    Since $f_n$ is bounded, $f$ is bounded.
    Thus $f \in \mcC(X)$, and $f_n \to f$ in $(\mcC(X), \norm{\cdot}_\infty)$.
\end{proof}
