\section{Series} \label{sec:series}
\lecture{2024-02-26}{Series definition, examples and properties}
We have seen infinite sums already, in the form of decimal expansions,
and also in computing the ``length'' of the Cantor set.

\begin{definition}
    Let $(a_n)_{n \in \N} \subseteq \C$.
    We say that the infinite sum \[
        \sum_{n=0}^\infty a_n
    \] converges to $a$ iff the sequence of partial sums $(S_n)_{n \in \N}$
    given by \begin{align*}
        S_0 &= a_0 \\
        S_1 &= a_0 + a_1 \\
           &\mathrel{\makebox[\widthof{=}]{\vdots}} \\
        S_n &= a_0 + a_1 + \dots + a_n
    \end{align*} converges to $a$.
    In that case, $a$ is said to be the sum of the given series.
    If $(S_n)_n$ diverges, then the series is said to diverge.
\end{definition}
\begin{example}
    Let $z \in \C$.
    Let $a_n = z^n$ for $n \in \N$.
    We wish to compute $\sum_{n=0}^{\infty} z^n$.
    Note that \begin{align*}
        S_n &= z^0 + z^1 + \dots + z^n \\
        \implies (1 - z) S_n &= 1 - z^{n+1}
        \shortintertext{so if $z \neq 1$,}
        S_n &= \frac{1 - z^{n+1}}{1 - z}
    \end{align*}
    We have three cases:
    \begin{itemize} % TODO
        \item If $|z| < 1$, then $\lim_{n \to \infty} \abs{z}^{n+1} = 0$.
        So by the algebra of limits,
        $\lim_{n \to \infty} S_n = \frac{1}{1 - z}$ so that the series
        converges to $\frac{1}{1 - z}$.
        Here, we have used that for any complex sequence $(z_n)_n$,
        $z_n \to 0$ iff $\abs{z_n} \to 0$.
        \item If $|z| > 1$, then
        $\lim_{n \to \infty} z^{n+1}$ does not exist,
        so the series diverges.
        So by the algebra of limits,
        $\lim_{n \to \infty} S_n$ does not exist.
        \item For $\abs{z} = 1$, we consider a few special cases first.
        \begin{itemize}
            \item If $z = 1$, then $S_n = n+1 \to \infty$.
            \item If $z = -1$, then $S_n = \1_{\text{$n$ is even}}$ which
            is divergent.
        \end{itemize}
        Even for the general case, we can use \cref{thm:series:convergence}
        to conclude that the series diverges, since $z^n \not\to 0$.
    \end{itemize}
\end{example}
\begin{theorem} \label{thm:series:convergence}
    Let $(a_n)_{n \in \N} \subseteq \C$.
    Then
    \begin{enumerate}[label=(\arabic*)]
        \item $\sum_{n = 0}^{\infty} a_n$ converges iff for every
        $\varepsilon > 0$, there exists an $N \in \N$ such that for all
        $n > m \ge N$, we have that
        $\abs{\sum_{j=m}^{n-1} a_j} < \varepsilon$.
        \item If $\sum_{n = 0}^{\infty} a_n$ converges, then
        $\lim_{n \to \infty} a_n = 0$.
        \item Suppose $a_n \ge 0$ ($a_n$ is real).
        Then $\sum_{n = 0}^{\infty} a_n$ converges iff the sequence of
        partial sums is bounded above.
    \end{enumerate}
\end{theorem}
\begin{proof} \leavevmode
    \begin{enumerate}[label=(\arabic*)]
        \item The metric space
        $(\C, \abs{\cdot}) \cong (\R^2, \norm{\cdot})$ is complete,
        so we can use the Cauchy criterion for convergence applied to the
        sequence of partial sums.
        \item From (1) using $n = m + 1$.
        \item Monotone convergence theorem applied to the SOPS. \qedhere
    \end{enumerate}
\end{proof}
\begin{theorem*}[Cauchy condensation test] \label{thm:series:condensation}
    Let $a_1 \ge a_2 \ge \dots \ge 0$.
    Then $\sum_{n=1}^\infty a_n$ converges iff
    $\sum_{n=1}^\infty 2^n a_{2^n}$ converges.
\end{theorem*}
