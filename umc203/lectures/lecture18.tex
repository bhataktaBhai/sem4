\chapter{Maximum Likelihood Estimation} \label{chp:mle}
\lecture[18]{2024-03-26}{Maximum Likelihood Estimation}

\begin{definition} \label{def:mle}
    Let $X_1, X_2, \dots$ be i.i.d. random variables drawn from a
    distribution $\Pr_{\theta}$, where $\theta$ belongs to a parameter space
    $\Theta$.
    The \emph{likelihood function} is defined as \[
        L_n(\theta) = \prod_{i=1}^{n} \Pr_{\theta}(X_i),
    \]
    which of course motivates the
    \emph{log-likelihood function} \[
        \ell_n(\theta) = \sum_{i=1}^{n} \log \Pr_{\theta}(X_i).
    \] The \emph{maximum likelihood estimator} (MLE) is defined as \[
        \hat{\theta}_n = \argmax_{\theta \in \Theta} \ell_n(\theta).
    \]
\end{definition}

\begin{definition*}[KL divergence] \label{def:kl_div}
    The \emph{Kullback-Leibler divergence} of the distribution $P$ from the
    distribution $Q$ is defined as \[
        \kld{P}{Q} = \E_{X \sim P} \left[\log \frac{P(X)}{Q(X)} \right].
    \]
    For discrete distributions, this is \[
        \kld{P}{Q} = \sum_{x} P(x) \log \frac{P(x)}{Q(x)}.
    \]
\end{definition*}

\begin{lemma}
    For all $x \in \R_+$, \[
        \log x \le x - 1
    \]
\end{lemma}
\begin{proof}
    $\log$ is concave (convex down),
    so the tangent at $x = 1$ always lies above the curve.
\end{proof}
\begin{proposition*}
    For all distributions $P$ and $Q$, \[
        \kld{P}{Q} \ge 0.
    \]
\end{proposition*}
\begin{proof}
    We write $\E_P$ to mean $\E_{X \sim P}$.
    \begin{align*}
        -\kld{P}{Q} &= \E_P \left[\log \frac{Q(X)}{P(X)} \right] \\
        &\le \E_P \left[\frac{Q(X)}{P(X)} - 1 \right] \\
        &= \int \frac{Q(x)}{P(x)} P(x) \dd x - 1 \\
        &= 0.
    \end{align*}
    For equality to hold, $P = Q$ almost surely.
\end{proof}

\begin{exercise}
    Find the MLE of $X_1, X_2, \dots, X_n \sim \Ber(\theta)$.
\end{exercise}
\begin{solution}
    The log-likelihood function is \begin{align*}
        \ell_n(\hat\theta)
            &= \sum_{i=1}^{n} \log \Pr_{\hat\theta}(X_i) \\
            &= \sum_{i=1}^{n} X_i \log \hat\theta
                + (1 - X_i) \log{}(1 - \hat\theta) \\
            &= n \wbar{X}_n \log \hat\theta
                + n (1 - \wbar{X}_n) \log{}(1 - \hat\theta),
    \end{align*} where $\wbar{X}_n = \frac{1}{n} \sum_{i=1}^{n} X_i$.
    Then for any $\hat\theta, \theta' \in [0, 1]$, \begin{align*}
        \frac{\ell_n(\hat\theta) - \ell_n(\theta')}n
            &= \wbar{X}_n \log \frac{\hat\theta}{\theta'}
                + (1 - \wbar{X}_n) \log \frac{1 - \hat\theta}{1 - \theta'}.
    \end{align*}
    For $\hat\theta = \wbar{X}_n$, this is precisely \[
        \frac{\ell_n(\hat\theta) - \ell_n(\theta')}n
            = \kld{\Ber \hat\theta}{\Ber \theta'}
            \ge 0.
    \] Thus the MLE is $\what\theta_n = \wbar{X}_n$.
\end{solution}

\begin{definition*}[Entropy] \label{def:entropy}
    The \emph{entropy} of a distribution $P$ is defined as \[
        H(P) = -\E_{X \sim P} \log P(X).
    \]
\end{definition*}

\begin{exercise*}
    Find MLE of $X_1, X_2, \dots, X_n \sim \mcN(\mu_0, \Sigma_0)$.
\end{exercise*}
\begin{solution}
    The log-likelihood function is \[
        \ell_n(\mu, \Sigma)
            = -\frac12 \log \det \Sigma
                -\frac12 \sum_{i=1}^n \norm{X_i - \mu}^2_{\Sigma^{-1}}
                + \text{constant}.
    \] Let us first fix $\Sigma$.
    Let $\wbar{X}_n = \frac1n \sum_{i=1}^{n} X_i$.
    Then ignoring the constant terms, we have to maximize \begin{align*}
        &\mkern-10mu\sum_{i=1}^{n} \norm{X_i - \wbar{X}_n
                + \wbar{X}_n - \mu}^2_{\Sigma^{-1}} \\
        &= \sum_{i=1}^{n} \bigg(\norm{X_i - \wbar{X}_n}^2_{\Sigma^{-1}}
            + 2 \innerp{X_i - \wbar{X}_n}{\wbar{X}_n - \mu}_{\Sigma^{-1}}
            + \norm{\wbar{X}_n - \mu}^2_{\Sigma^{-1}}\bigg) \\
        &= \sum_{i=1}^{n} \norm{X_i - \wbar{X}_n}^2_{\Sigma^{-1}}
            + 2 \braket[\bigg]{\cancelto{0}{\sum_{i=1}^{n} X_i - n \wbar{X}_n}}{\Sigma^{-1}}{\wbar{X}_n - \mu}
            + n \norm{\wbar{X}_n - \mu}^2_{\Sigma^{-1}} \\
        &= \sum_{i=1}^{n} \norm{X_i - \wbar{X}_n}^2_{\Sigma^{-1}}
            + n \norm{\wbar{X}_n - \mu}^2_{\Sigma^{-1}}.
    \end{align*}
    Thus for any value of $\Sigma$, $\ell(\mu, \Sigma)$ is maximized when
    $\mu = \wbar{X}_n$.

    Now let us fix $\mu = \wbar{X}_n$.
    Let $S = \sum_{i=1}^{n} \outerp{X_i - \mu}{X_i - \mu}$.
    Then \begin{align*}
        \sum \norm{X_i - \mu}^2_{\Sigma^{-1}}
        &= \sum \braket{X_i - \mu}{\Sigma^{-1}}{X_i - \mu} \\
        &= \sum \Tr(\Sigma^{-1} \outerp{X_i - \mu}{X_i - \mu}) \\
        &= \Tr(\Sigma^{-1} S).
    \end{align*}

    Then \begin{align*}
        \ell_n(\mu, \Sigma) - \ell_n(\mu, S)
            &\propto -\log \det \Sigma - \Tr(\Sigma^{-1} S)
                +\log \det S + \Tr(S^{-1} S) \\
            &= \log \det (\Sigma^{-1} S) - \Tr(\Sigma^{-1} S) + n \\
            &= \sum \log \lambda_i - \sum \lambda_i + n,
        \intertext{where $\lambda$s are the eigenvalues of $\Sigma^{-1} S$.}
            &= \sum (\log \lambda_i - (1 - \lambda_i)) \\
            &\le 0
    \end{align*} with equality iff each $\lambda_i = 1$, that is,
    $\Sigma^{-1} S = I \iff \Sigma = S$.

    Thus the MLE is \[
        \what\mu_n = \wbar{X}_n, \quad
        \what\Sigma_n = \frac1n \sum_{i=1}^{n}
                \outerp{X_i - \wbar{X}_n}{X_i - \wbar{X}_n}. \qedhere
    \]
\end{solution}
