\chapter{Principal Component Analysis} \label{chp:pca}
\lecture{2024-04-08}{}
Let the data \[
    \mcD = \set{x^{(1)}, \dots, x^{(n)}} \subseteq \R^d
\] be drawn i.i.d. from a distribution $P$.

By Cauchy-Shwartz, \[
    \innerp{u}{v} \le \norm{u} \norm{v},
\] with equality achieved when $v = \lambda u$.

\textbf{THEREFORE,} the maximum value of $u^\top C u$ is achieved...
when $C u = \lambda u$.

This reminds me of
\begin{align*}
    E &= mc^2 \\
    E &= \frac{hc}{\lambda} \\
    \lambda &= \frac{h}{mv}
\end{align*}
