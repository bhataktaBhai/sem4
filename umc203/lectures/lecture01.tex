\chapter*{The Course} \label{chp:course}
\lecture{2024-01-09}{Classification and Bayes classifier}
\textbf{Instructor:} Prof.~Chiranjib~Bhattacharyya \\
\textbf{Office:} CSA, 254 \\
\textbf{Office hours:} TBD\\
\textbf{Lecture hours:} TuTh 10:00--11:20

\section*{References} \label{sec:references}
\begin{enumerate}
    \item \textit{Pattern Classification} by Duda, Hart, and Stork
    \item \textit{Probabilistic Theory of Pattern Recognition} by Devroye,
        Györfi, and Lugosi
    \item \textit{Pattern Recognition and Machine Learning} by Bishop
    \item \textit{Foundations of Machine Learning} by Mohri, Rostamizadeh,
        and Talwalkar
\end{enumerate}

\chapter{Bayes' Classifier} \label{chp:bayes}
Consider a machine which can measure the diameter of any fruit placed on it.
Can the machine distinguish between an apple and an orange?
Now suppose the machine also has the \emph{capacity} to measure the weight
of the fruit.
Can it distinguish between an apple and an orange now?
\[
    \text{Fruit} \mapsto (x_1, x_2) \mapsto \set{\text{Apple}, \text{Orange}}
\] where $x_1$ is the diameter and $x_2$ is the weight.
These are called \emph{features}.

% scatter plot of x2 vs x1
\begin{center}
    \begin{tikzpicture}
        \begin{axis}[
            xlabel=$x_1$,
            ylabel=$x_2$,
            xmin=-1, xmax=9,
            ymin=-1, ymax=9,
            axis lines=middle,
            % legend pos=north west,
            ]
            \addplot[only marks, mark=*, Red] table {
                1 4
                2 5
                2.5 5.5
                3 4.2
                2.6 4.8
                3.2 5.2
                1.6 3.6
            };
            \addplot[only marks, mark=*, orange] table {
                5 2.5
                5.5 3
                6 3.2
                5.3 3.4
                5.8 2.3
                6.2 3.5
            };
            \addplot[domain=-1:10, samples=2, blue] {0.7*x+1};
        \end{axis}
    \end{tikzpicture}
\end{center}

How do we measure how good a classifier is?
This example has very few data points, so error is zero.
Data is expensive, so accurate testing is expensive.

Let $h$ be a classifier.
We want to measure how good $h$ is.
We consider a random variable (of as yet unknown distribution) and compute
the probability of error.

We consider this slightly more formally.
Let the training data be \[
    \mcD = \set{(x^i, y^i)}
\] where $x \in \R^2$ and $y \in \set{1, -1}$, where $-1$ and $1$ represent
apples and oranges respectively.
Then $h$ is a function from $\R^2$ to $\set{-1, 1}$.
We wish to measure the probability $\Pr(h(X) \ne Y)$.

\section{Probability Review} \label{sec:probability_review}
Suppose a coin is given with unknown probability of heads $p$.
How do we estimate $p$?
We flip the coin $n$ times and count the number of heads $n_H$.
Then we estimate $p$ as $\hat{p} = n_H/n$.

The rationale behind this is the weak/strong law of large numbers.
\begin{fact}[Weak Law of Large Numbers] \label{thm:probability_review:wlln}
    Let $X_1, X_2, \ldots$ be i.i.d.~random variables with mean $\mu$.
    Then for any $\varepsilon > 0$, \[
        \Pr\left(\abs{\frac{1}{n} \sum_{i=1}^n X_i - \mu} > \varepsilon\right)
        \to 0
    \] as $n \to \infty$.
\end{fact}

\begin{fact}[Strong Law of Large Numbers] \label{thm:probability_review:slln}
    Let $X_1, X_2, \ldots$ be i.i.d.~random variables with mean $\mu$.
    Then \[
        \Pr\left(\lim_{n \to \infty} \frac{1}{n} \sum_{i=1}^n X_i = \mu\right)
        = 1
    \]
\end{fact}
We have made several assumptions here.
We know the structure of the problem at hand.
For instance, we know that there is exactly one coin tossed each time.
We have assumed that the coin tosses are independent.
We have assumed that the probability of heads is the same for each toss.

Suppose we know the following in our earlier experiment:
\begin{itemize}
    \item $\Pr(Y = 1)$
    \item $\Pr(Y = -1)$
    \item $\Pr(X = x \mid Y = 1)$
    \item $\Pr(X = x \mid Y = -1)$
\end{itemize}
Let $\eta(x) = \Pr(Y = 1 \mid X = x)$ given by Bayes' rule.
Our rule for classification is \[
    h(x) = \begin{cases}
        \hphantom{-}1 & \text{if } \eta(x) > \frac{1}{2} \\
        -1 & \text{otherwise}
    \end{cases}
\] This is called the \emph{Bayes classifier}.

\section{Multivariate Gaussians} \label{sec:multivariate_gaussians}
\[
    f(x) = \frac{1}{\sqrt{(2\pi)^d \det(\Sigma)}}
    \exp\left(-\frac{1}{2} (x - \mu)^\top \Sigma^{-1} (x - \mu)\right)
\] where $x$ is a $d$-dimensional column vector (so that the exponent is a
scalar), $\mu$ is the mean, and $\Sigma$ is the covariance matrix.
\begin{align*}
    E[X] &= \int_{x \in \R^d} x f(x) \dd x \\
    \Sigma_{ij} &= E[(X_i - \mu_i)(X_j - \mu_j)]
\end{align*}
\textbf{Reading assignment:} Let $A$ be a square symmetric real-valued matrix.
What is known about the eigenvalues and positive definiteness of $A$?

\begin{definition}[Definite matrix] \label{def:definite}
    A matrix $A_{n \times n}$ is said to be \emph{positive definite}
    if $u^\top A u > 0$ for all $u \in \R^n \setminus \set{0}$.

    $A$ is said to be \emph{positive semidefinite} if $u^\top A u \ge 0$
    for all $u \in \R^n$.

    $A$ is said to be \emph{negative definite} and \emph{negative
    semidefinite} if $-A$ is positive definite and positive semidefinite
    respectively.
\end{definition}

\begin{exercise}
    Compute the Bayes classifier under the assumption that $X$ under class 1
    and class 2 has multivariate Gaussian distribution with means $\mu_1$
    and $\mu_2$ with same covariance matrix $\Sigma$.
\end{exercise}
