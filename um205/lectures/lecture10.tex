\lecture{2024-01-24}{}

\begin{theorem}[Multinomial theorem] \label{thm:multinomial}
    Let $n, k \in \N$ and $x_1, \dots, x_k$ be indeterminates.
    Then \[
        \sum_{\substack{0\le a_1, \dots, a_k \le n \\ a_1 + \dots + a_k = n}}
        \binom{n}{a_1, \dots, a_k} x_1^{a_1} \dots x_k^{a_k} = (x_1 + \dots + x_k)^n
    \]
\end{theorem}
\begin{proof}
    When the RHS is expanded, we get $1$ term for each arrangement of $n$
    objects out of $x_1, \dots, x_k$, with repetition allowed.
    The coefficient of $x_1^{a_1} \dots x_k^{a_k}$ is the number of such
    arrangements with $a_1$ of $x_1$, $a_2$ of $x_2$, etc.
    This is given by $\binom{n}{a_1, \dots, a_k}$.

    For $a_1 + \dots + a_k \ne n$, no such term is obtained.
\end{proof}

\begin{exercise} \label{thm:half_choose}
    Compute $\binom{1/2}{n}$.
\end{exercise}
\begin{solution}
    \begin{align*}
        \binom{1/2}{n}
            &= \frac{(1/2)(-1/2) \dots (1/2 - n + 1)}{n} \\
            &= \frac{(-1)^{n-1} (2n - 3)!!}{2^n n!} \\
            &= \frac{(-1)^{n-1} (2n - 2)!}{2^n n! (2n-2)(2n-4) \dots 2} \\
            &= \frac{(-1)^{n-1} (2n - 2)!}{2^n n! 2^{n-1} (n-1)!} \\
            &= \frac{(-1)^{n-1}}{2^{2n-1}} \frac{(2n-2)!}{n! (n-1)!} \\
            &= \frac{(-1)^{n-1}}{n 2^{2n-1}} \binom{2n-2}{n-1}.
    \end{align*}
\end{solution}

\section{Compositions \& Paritions} \label{sec:comp/part}

\begin{definition*} \label{def:composition}
    A \emph{weak composition} of $n \in \N$ is a sequence $(a_i)_{i=1}^k$
    where $a_i \in \N$ and $a_1 + \dots + a_k = n$.
    If each $a_i > 0$, then it is called a \emph{(strict) composition}.
\end{definition*}
\begin{example}
    For $n = 3$, its strict compositions are $(1, 1, 1)$, $(1, 2)$, $(2, 1)$
    and $(3)$.
\end{example}

\begin{proposition} \label{thm:composition:count}
    The number of weak compositions of $n$ into $k$ parts is
    $\binom{n+k-1}{k-1}$.
\end{proposition}
\begin{proof}
    % TODO: bijection with $n$ balls in $k$ boxes.
\end{proof}
\begin{corollary}
    The number of compositions of $n$ into $k$ parts is $\binom{n-1}{k-1}$.
\end{corollary}
\begin{proof}
    Each box must get at least one ball, so use \cref{thm:composition:count}
    with $n \mapsto n - k$.
\end{proof}
\begin{corollary}
    The total number of compositions is $2^{n-1}$.
\end{corollary}
\begin{proof}
    $\sum_{k=1}^{n} \binom{n-1}{k-1} = \sum_{k=0}^{n-1} \binom{n-1}{k}
    = 2^{n-1}$.
\end{proof}

\begin{definition*}[Partitions] \label{def:partitions}
    An \emph{(integer) partition} of $n \in \N$ is a sequence
    $\lambda = (\lambda_1, \dots, \lambda_k)$ of weakly decreasing positive
    integers which sum to $n$.
    We write $\lambda \vdash n$.
    Each $\lambda_i$ is called a \emph{part} and the number of parts is
    called the \emph{length}, denoted $\ell(\lambda)$.
    We write $p(n)$ for the number of partitions of $n$.
\end{definition*}
\begin{example}
    The partitions of $5$ are $(5)$, $(4, 1)$, $(3, 2)$, $(3, 1, 1)$,
    $(2, 2, 1)$, $(2, 1, 1, 1)$ and $(1, 1, 1, 1, 1)$.
    Thus $p(5) = 7$.
\end{example}

\begin{proposition} \label{thm:partitions:conjugate}
    The number of partitions of $n$ into exactly (resp. at most) $k$ parts
    is the same as the number of partitions of $n$ with largest part exactly
    (resp. at most) $k$.
\end{proposition}

\begin{definition}[Young diagram] \label{def:young}
    The \emph{Young/Ferrers diagram} of a partition is a left-justified
    array of boxes with $\lambda_i$ boxes in the $i$th row.
\end{definition}
\begin{example}
    The Young diagrams of $(4, 1)$ and $(3, 2)$ are
    \begin{center}
        \young{{4,1}} \qquad \young{{3,2}}
    \end{center}
\end{example}

\begin{definition}[Conjugate] \label{def:partition:conjugate}
    The \emph{conjugate} of a partition $\lambda$, denoted $\lambda'$, is
    the partition whose Young diagram is the transpose of that of $\lambda$.
    That is, \[
        \lambda'_i = \size \set{j \in \N : \lambda_j \ge i}
    \]
\end{definition}

\begin{proof}[Proof of \cref{thm:partitions:conjugate}]
    If $\lambda$ has length $k$, then $\lambda'$ has largest part $k$.
    For example,
    \begin{center}
        $\vcenter{\hbox{\young[0.6]{{5,4,4,2}}}}$
        \quad has conjugate \quad
        $\vcenter{\hbox{\young[0.6]{{4,4,3,3,1}}}}$
    \end{center}
    \vspace{-2em}
\end{proof}

\begin{theorem*}
    The number of self-conjugate partitions of $n$ is equal to the number
    of partitions of $n$ into distinct odd parts.
\end{theorem*}
\begin{proof}
    % TODO
\end{proof}

\begin{fact*}[Euler] \label{thm:partitions:euler}
    The number of partitions of $n$ into odd parts is equal to the number
    of partitions of $n$ into distinct parts.
\end{fact*}

\begin{fact}[Hardy-Ramanujan Formula] \label{thm:partitions:ramanujan}
    \[
        p(n) \sim \frac{1}{4n\sqrt{3}} e^{\pi \sqrt{\frac{2n}{3}}}
    \]
\end{fact}

\begin{definition*} \label{def:set_partitions}
    A \emph{set partition} of $[n]$ is a collection of pairwise disjoint
    non-empty subsets/blocks whose union is $[n]$.
    The number of set partitions of $[n]$ into $k$ (non-empty) blocks is
    called the \emph{Stirling number of the second kind} and denoted
    $\stirling{n}{k}$, read ``$n$ set $k$''.
\end{definition*}
\begin{example}
    The set partitions of $[3]$ are $123$, $12|3$, $13|2$, $1|23$ and
    $1|2|3$.
\end{example}
