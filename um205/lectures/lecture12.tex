\lecture{2024-01-31}{}
\begin{notation}
    We write a cycle type $\lambda$ in \emph{frequency notation} as \[
        \lambda = \angled{1^{a_1}, 2^{a_2}, \dots, n^{a_n}}
    \] where $a_i$ is the number of times $i$ appears in $\lambda$.
\end{notation}
\begin{theorem*}
    The number of permutations in $S_n$ with cycle type
    $\lambda = \angled{1^{a_1}, 2^{a_2}, \dots, n^{a_n}}$ is given by \[
        \frac{n!}{(1^{a_1}a_1!)(2^{a_2}a_2!)\dots (n^{a_n}a_n!)}
    \]
\end{theorem*}
\begin{proof}
    For every permutation in $S_n$ in one-line notation, insert parentheses
    so that we first form $a_1$ cycles of length $1$, $a_2$ cycles of length
    $2$, and so on.
    This gives another permutation in cycle notation.

    How many times does a given permutation $\sigma$ occur?
    For a cycle of length $j$, the same cycle occurs in $j$ ways.
    Since we have $a_j$ such cycles, we get a factor of $j^{a_j}$.
    Next, among all $j$-cycles, we get $\sigma$ if we permute these cycles,
    and this happends in $j!$ ways.
    Finally, note that for different values of $j$, these rearrangements are
    independent.
\end{proof}
\begin{example}
    There are $\frac{n!}{n^1 1!} = (n-1)!$ permutations with cycle type
    $\lambda = (n) = \angled{n^1}$.
\end{example}
\begin{notation}
    We write $\cycle{n}{k}$ for the number of permutations in $S_n$ with
    exactly $k$ cycles.
\end{notation}

\begin{proposition*}
    For $1 \le k \le n$,
    $\cycle{n}{k} = \cycle{n-1}{k-1} + (n-1)\cycle{n-1}{k}$.
\end{proposition*}
\begin{proof}
    Consider a permutation $\sigma \in S_n$ with $k$ cycles.
    If $n$ is a fixed point, then $\sigma$ restricted to $[n-1]$ has $k-1$
    cycles.
    This gives the first term.

    If $n$ is not a fixed point, then removing $n$ gives a permutation in
    $S_{n-1}$ with $k$ cycles.
    There are $n-1$ ways to insert $n$ \emph{after} any element of any
    cycle.
\end{proof}
\begin{table}
    \centering
    \begin{tabular}{c|cccc}
        \diagbox[width=2.5em,height=2em]{$n$}{$k$} & 1 & 2 & 3 & 4 \\
        \hline
        1 & 1 \\
        2 & 1 & 1 \\
        3 & 2 & 3 & 1 \\
        4 & 6 & 11 & 6 & 1
    \end{tabular}
    \caption{The first few values of $\cycle{n}{k}$.}
    \label{tab:cycle}
\end{table}

\begin{proposition*} \label{thm:cycle:sum}
    Let $n \in \N$.
    Then \[
        \sum_{k=0}^{n} \cycle{n}{k} x^k = x^{\wbar{n}}
    \]
\end{proposition*}
\begin{proof}
    Let $G_n(x) = x^{\wbar{n}} = (x+n-1) G_{n-1}(x)$.
    Suppose the result holds for $n-1$.
    Then \begin{align*}
        G_n(x) &= (x + n - 1) \sum_{k=0}^{n} \cycle{n-1}{k} x^k \\
            &= \sum_{k=0}^{n} \cycle{n-1}{k} x^{k+1}
                + \sum_{k=0}^{n} (n-1) \cycle{n-1}{k} x^k \\
            &= \sum_{k=1}^{n} \cycle{n-1}{k-1} x^k
                + (n-1) \sum_{k=0}^{n} \cycle{n-1}{k} x^k \\
            &= \sum_{k=1}^{n} \left(\cycle{n-1}{k-1} + (n-1) \cycle{n-1}{k}
                \right) x^k \tag{because $\cycle{n}{0} = 0$} \\
            &= \sum_{k=0}^{n} \cycle{n}{k} x^k
    \end{align*}
    The base case is $G_1(x) = x$. % TODO: or is it $G_0(x) = 1$?
\end{proof}

\begin{definition*}
    The \emph{(signed) Stirling numbers of the first kind} are
    $(-1)^{n-k} \cycle{n}{k}$.
\end{definition*}

\textbf{Recall:} $V = \Q[x]$ is the space of polynomials with rational
coefficients.
This is a vector space over $\Q$.
The natural basis for $V$ is $\mcB_1 = \set{1, x, x^2, \dots}$.
But we also have another basis, $\mcB_2 = \set{x^{\underline{0}},
x^{\underline{1}}, x^{\underline{2}}, \dots} = \set{1, x, x(x-1), \dots}$.

Let $S$ be the $\N \times \N$ matrix whose $(n, k)^{\text{th}}$ entry is
$\stirling{n}{k}$.
Then \cref{thm:stirling:sum} shows that $S$ is the transition matrix from
$\mcB_2$ to $\mcB_1$.
Similarly, let $s$ be the $\N \times \N$ matrix whose $(n, k)^{\text{th}}$
entry is $(-1)^{n-k} \cycle{n}{k}$.
Then $s$ is the transition matrix from $\mcB_1$ to $\mcB_2$.
% TODO: why?
Thus we have shown that:
\begin{proposition}
    $Ss = sS = \id$.
\end{proposition}

\section{Inclusion-Exclusion formula} \label{sec:pie}
Given sets $A_1, \dots, A_n$ so that we know the size of the intersection of
any $k$ of them, we can compute the size of the union of all of them.
\begin{theorem*}[Principle of Inclusion-Exclusion] \label{thm:pie}
    \[
        \bigg\lvert\bigcup_{i\in[n]} A_i\bigg\rvert
            = \sum_{j=1}^{n} (-1)^{j-1} \sum_{\substack{S \subseteq [n] \\
            \abs{S} = j}} \bigg\lvert\bigcap_{i \in S} A_i\bigg\rvert
    \]
\end{theorem*}
