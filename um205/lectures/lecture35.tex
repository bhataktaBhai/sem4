\lecture{2024-04-05}{}
\begin{proposition}
    Let $N \le G$.
    Then the set of left cosets partitions $G$.
    Moreover, $uN = vN$ iff $u^{-1}v \in N$,
    and $uN = vN$ iff $u$ and $v$ represent the coset.
\end{proposition}
\begin{proof}
    Since $N \le G$, $e \in N$ and so $g = g \cdot 1 \in g N$.
    Thus $G = \bigcup_{g \in G} gN$.

    Suppose $u, v \in G$ such that $uN \cap vN \ne \O$.
    Let $x = um = vn$ be an element of the intersection.
    Then $u = vnm^{-1}$, so $uN \subseteq vN$.
    Similarly, $vN \subseteq uN$, so that $uN = vN$.

    Thus the set of left cosets $\set{gN \mid g \in G}$ is a partition of
    $G$.

    Now $uN = vN \iff um = vn \iff u^{-1}v = mn^{-1} \in N$.

    Finally, $v \in uN$ means that $v$ is a representative for $uN$.
    So $vN = uN$.
    Thus they both represent the same coset.
\end{proof}

\begin{definition*}[Normal] \label{def:group:normal}
    The element $gng^{-1}$ is called the \emph{conjugate} of $n$ by $g$,
    and $gNg^{-1} = \set{gng^{-1} \mid n \in N}$ is called the conjugate
    of $N$ by $g$.

    The element $g$ is said to \emph{normalize} $N$ if $gNg^{-1} = N$.
    A subgroup $N$ is said to be \emph{normal} if every $g \in G$
    normalizes $N$.
    We write $N \nsub G$.
\end{definition*}
\begin{proposition}
    Let $G$ be a group and $N \le G$.
    Then
    \begin{enumerate}
        \item The operation $(uN) \cdot (vN) = (uv)N$ is well-defined iff
            $N$ is normal in $G$.
        \item If so, the set of left cosets form a group with the above
            product, with identity $eN$ and inverse $(gN)^{-1} = g^{-1}N$.
    \end{enumerate}
\end{proposition}
\begin{proof} \leavevmode
    \begin{enumerate}
        \item Suppose the product is well-defined.
        That is, for any $u' \in uN$ and $v' \in vN$,
        we have $u'v' \in uvN$.

        Let $g \in G$ and $n \in N$.
        Taking the product of $eN$ with $gN$, we have
        $ng \in gN \implies g^{-1}ng \in N$.

        Now suppose $N$ is normal in $G$.
        Then $g^{-1}ng \in N$ for all $g \in G$ and $n \in N$.

        Let $u' = um$, $v' = vn$, $m, n \in N$.
        Then \[
            u'v' = umvn = uv(v^{-1}mvm) = uv(m'm) \in uvN
        \] since $v^{-1}mv = m' \in N$.
        \item Associativity follows from the associativity of $G$.
        Identity is borrowed from $G$.
        Inverse is borrowed from $G$. \qedhere
    \end{enumerate}
\end{proof}
\begin{example}
    If $G$ is abelian, then every subgroup is normal.
    So $G/N$ is the quotient group for any $N \le G$.
\end{example}

\begin{theorem}
    Let $N \le G$.
    Then the following are equivalent.
    \begin{enumerate}
        \item $N \nsub G$.
        \item $N_G(N) = G$, where $N_G(N)$ is the set of normalizers
            of $N$ in $G$.
        \item $gN = Ng$ for all $g \in G$.
        \item The multiplication of left cosets makes $G/N$ a group.
        \item $gNg^{-1} \subseteq N$ for all $g \in G$.
    \end{enumerate}
\end{theorem}

\begin{proposition}
    $N \le G$ is normal iff it is the kernel of some homomorphism.
\end{proposition}
\begin{proof}
    We have shown that left and right cosets are equal for kernels of
    homomorphisms.
    This gives that kernels are normal subgroups.

    Conversely, suppose $N \nsub G$.
    Let $H = G/N$.
    Define $\pi\colon G \to H$ by $\pi(g) = gN$.
    Then $\pi$ is a homomorphism.

    What is the kernel of $\pi$?
    \begin{align*}
        \ker \pi &= \set{g \in G \mid \pi(g) = eN} \\
               &= \set{g \in G \mid gN = N} \\
               &= \set{g \in G \mid g \in N} = N. \qedhere
    \end{align*}
\end{proof}

\begin{definition}
    Let $N \le G$.
    The homomorphism $g \mapsto gN$ is the \emph{natural projection}
    of $G$ onto $G/N$.
\end{definition}
