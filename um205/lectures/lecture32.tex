\section{Group Actions} \label{sec:group_actions}
\lecture{2024-03-27}{}
\begin{definition}[Group action] \label{def:group_action}
    A (left) \emph{group action} of a group $G$ on a set $A$ is a map
    $G \times A \to A$ written $(g, a) \mapsto g \cdot a$, satisfying
    \begin{enumerate}
        \item $g_1 \cdot (g_2 \cdot a) = (g_1 g_2) \cdot a$ for all
            $g_1, g_2 \in G$ and $a \in A$.
        \item $e \cdot a = a$ for all $a \in A$.
    \end{enumerate}
    If such a map exists, we say that $G$ \emph{acts on} $A$.
    Equivalently, $A$ has the \emph{symmetries} of $G$.
\end{definition}
\begin{examples}
    \item The regular $n$-gon is acted on by $D_{2n}$.
    \item $(g, a) \mapsto a$ is the \emph{trivial action} of $G$ on $A$.
    \item Let $V$ be a vector space over a field $F$.
    Then $(F^*, *)$ is a group, and $F^*$ acts on $V$ by scalar
    multiplication.
    \item Let $A$ be a set and $S_A$ denote the group of permutations of
    $A$ (\ie, bijection from $A$ to $A$).
    Then $S_A$ acts on $A$ by $\sigma \cdot a = \sigma(a)$.
\end{examples}

Suppose $G$ acts on $A$.
Then define for all $g \in G$, the map $\sigma_g\colon A \to A$ by \[
    \sigma_g(a) = g \cdot a.
\]
\begin{proposition}
    \begin{enumerate}
        \item For all $g \in G$, $\sigma_g$ is a permutation of $A$.
        That is, $\sigma_g \in S_A$.
        \item The map $\varphi\colon G \to S_A$ given by
        $\varphi(g) = \sigma_g$ is a group homomorphism
        (where the group operation on $S_A$ is composition).
    \end{enumerate}
\end{proposition}
\begin{proof} \leavevmode
    \begin{enumerate}
        \item We will show that $\sigma_{g^{-1}}$ is the (two-sided)
        inverse of $\sigma_g$.
        \begin{align*}
            (\sigma_{g^{-1}} \circ \sigma_g)(a)
                &= \sigma_{g^{-1}}(g \cdot a) \\
                &= g^{-1} \cdot (g \cdot a) \\
                &= (g^{-1} g) \cdot a \\
                &= e \cdot a \\
                &= a.
        \end{align*}
        Interchange the roles of $g$ and $g^{-1}$ to get \[
            (\sigma_g \circ \sigma_{g^{-1}})(a) = a.
        \]
        \item Consider \begin{align*}
            \varphi(g_1 g_2)(a)
                &= \sigma_{g_1 g_2}(a) \\
                &= (g_1 g_2) \cdot a \\
                &= g_1 \cdot (g_2 \cdot a) \\
                &= \sigma_{g_1}(\sigma_{g_2}(a)) \\
                &= \varphi(g_1)(\varphi(g_2)(a)) \\
                &= (\varphi(g_1) \circ \varphi(g_2))(a).
        \end{align*}
        Since this holds for all $a \in A$, \[
            \varphi(g_1 g_2) = \varphi(g_1) \circ \varphi(g_2).
        \]
    \end{enumerate}
\end{proof}

\begin{definition}[Permutation representation] \label{def:perm_rep}
    Let $G$ act on $A$ by $g \cdot a$.
    The map $\varphi\colon G \to S_A$ given by $\varphi(g) = \sigma_g$,
    where $\sigma_g(a) = g \cdot a$, is called the \emph{permutation
    representation} of $G$.
\end{definition}

\begin{definition}[Faith] \label{def:group_actions:faith}
    A group action is called \emph{faithful} if every $g \in G$ gives a
    different permutation of $A$, \ie, the permutation representation
    is injective.
\end{definition}

\begin{definition}[Kernel] \label{def:group_actions:kernel}
    The \emph{kernel} of a group action of $G$ on $A$ is the set \[
        \set{g \in G \mid g \cdot a = a \text{ for all } a \in A}.
    \]
\end{definition}

\begin{example}
    The trivial action is unfaithful (except for the trivial group)
    and the kernel is the group itself.
\end{example}
\begin{proposition}
    Any action is faithful if and only if its kernel is $\set{e}$.
\end{proposition}
\begin{proof}
    We know that $\varphi(e) = \sigma_e = \id_A$.
    If the action is faithful, then there is no other $g \in G$ such that
    $g \cdot a = a \;\forall a \in A$.
    Thus, the kernel is $\set{e}$.

    Conversely, suppose the kernel is $\set{e}$.
    Let $g_1, g_2$ be such that $\sigma_{g_1} = \sigma_{g_2}$.
    Then $\sigma_{g_1 g_2^{-1}} = \id_A$, so $g_1 g_2^{-1} = e$,
    and $g_1 = g_2$.
\end{proof}

\begin{example}
    $G$ acts on itself by left multiplication.
    This is faithful, because $g_1 e = g_2 e \iff g_1 = g_2$.
\end{example}

\begin{definition*}[Subgroup] \label{def:group_actions:subgroup}
    Let $G$ be a group.
    A non-empty subset $H \subseteq G$ is called a \emph{subgroup},
    denoted $H \le G$, if $H$ is closed under products and inverses.
\end{definition*}
\begin{proposition}
    Any subgroup of a group is a group.
\end{proposition}
\begin{proof}
    
\end{proof}

\begin{examples}
    \item Every group $G$ has two subgroups, $\set{e}$ and $G$.
    $\set{e}$ is called the \emph{trivial subgroup}.
    If $H \le G$ and $H \ne G$, then $H$ is called a \emph{proper
    subgroup}.
    \item $\set{1, r, \dots, r^{n-1}} < D_{2n}$.
    \item $(3\Z, +) < (\Z, +)$.
    \item $(\Q^*, \cdot) \not\le (\R, +)$.
\end{examples}

\begin{proposition}
    A non-empty subset $H \subseteq G$ is a subgroup if and only if
    for every $x, y \in H$, $xy^{-1} \in H$.
\end{proposition}
\begin{proof}
    Suppose $H$ is a subgroup.
    Then $x, y \in H \implies y^{-1} \in H$ by closure under inverses,
    and $xy^{-1} \in H$ by closure under products.

    Suppose $H$ satisfies the condition.
    Let $h \in H$.
    Then $h h^{-1} = e \in H$.
    Now taking $x = e$ and $y = h$, we have $h^{-1} = e h^{-1} \in H$.
    Thus we have closure under inverses.

    Let $h_1, h_2 \in H$.
    Then $h_2^{-1} \in H$, so $h_1 h_2 = h_1 {(h_2^{-1})}^{-1} \in H$.
\end{proof}

\begin{exercise}
    A finite non-empty subset $H$ of a group $G$ is a subgroup iff for
    all $x, y \in H$, $xy \in H$.
\end{exercise}
\begin{proof}
    If $H$ is a subgroup, this is by definition.

    Suppose $H$ satisfies the condition.
    Let $h \in H$.
    $H$ is closed under products, so $\set{h, h^2, \dots} \subseteq H$.
    By the pigeonhole principle, $e$ and $h^{-1}$ are in this set.
\end{proof}
