\lecture{2024-03-20}{}
\begin{remark}
    By \labelcref{thm:legendre:p-1}, $\lgnd 1 p = 1$.
\end{remark}

\begin{corollary} \label{thm:quad_res:half}
    The number of quadratic residues modulo $p$ is equal to
    the number of quadratic non-residues modulo $p$.
\end{corollary}
\begin{proof}
    
\end{proof}

\begin{corollary} \label{thm:quad_res:product}
    The multiplication table of residues modulo $p$ is as follows:
    \begin{center}
        \begin{tabular}{c|c|c}
            $\cdot$ & Residue & Non-residue \\
            \hline
            Residue & Residue & Non-residue \\
            Non-residue & Non-residue & Residue
        \end{tabular}
    \end{center}
\end{corollary}
\begin{proof}
    Follows from definition and \labelcref{thm:legendre:product}.
\end{proof}

\begin{corollary}
    $(-1)^{\frac{p-1}{2}} = \lgnd{-1}{p}$.
\end{corollary}
\begin{proof}
    Set $a = -1$ in \labelcref{thm:legendre:p-1}.
\end{proof}

\begin{exercise}
    Use this result to show that there are infinitely many primes
    congruent to $1 \pmod{4}$.
\end{exercise}

\begin{definition}[Least residues] \label{def:quad_res:least_res}
    Let $p$ be a prime. The set \[
        S = \set{-\frac{p-1}{2}, -\frac{p-3}{2},
            \dots, -1, 1, \dots,
            \frac{p-3}{2}, \frac{p-1}{2}}
    \] is called the set of \emph{least residues modulo $p$}.
    These together with $0$ form a complete set of residues modulo $p$.
\end{definition}

\begin{example}
    For $p = 7$, $S = \set{-3, -2, -1, 1, 2, 3}$.
\end{example}
\begin{definition} \label{def:quad_res:mu}
    Let $p$ be a prime that does not divide $a$.
    We define \begin{align*}
        \mu(p, a) = \text{number of negative least residues modulo $p$ of
        the integers }\\
        \set*[\Big]{a, 2a, \dots, \frac{p-1}{2} a}.
    \end{align*}
\end{definition}

\begin{theorem}[Gauss' lemma] \label{thm:quad_res:gauss}
    \[
        \lgnd a p = (-1)^{\mu(p, a)}
    \] for $p$ prime and $p \nmid a$.
\end{theorem}
\begin{proof}
    For each $1 \le l \le \frac{p-1}{2}$, let the least residue of $la$ be
    $\pm m_l$, $m_l > 0$.

    \textbf{Claim:} $m_l = m_k$ iff $l = k$.
    \begin{proof}
        If $l \ne k$ and $m_l = m_k$,
        then $la \equiv \pm ka \pmod p$.

        But $p \nmid a$, so $l \pm k \equiv 0 \pmod p$,
        But $\abs{l \pm k} \le p - 1$, so $l \not\equiv \pm k \pmod p$.
        Contradiction.
    \end{proof}

    Thus $\set{m_1, \dots, m_{\frac{p-1}{2}}}$ is the complete set
    $\set{1, 2, \dots, \frac{p-1}{2}}$.

    Then \begin{align*}
        \prod_{l=1}^{(p-1)/2} la
            &\equiv (-1)^{\mu(p, a)} \prod_{l=1}^{(p-1)/2} m_l  \pmod p \\
        a^{\frac{p-1}{2}} \paren{\frac{p-1}{2}}!
            &\equiv (-1)^{\mu(p, a)} \paren{\frac{p-1}{2}}! \\
        a^{\frac{p-1}{2}} &\equiv (-1)^{\mu(p, a)}.
    \end{align*}
    Using \labelcref{thm:legendre:p-1}, we get the result.
\end{proof}
\begin{remark}
    This is a powerful result, but not useful for computation.
\end{remark}

\begin{proposition}
    $2$ is a residue (resp. non-residue) for primes of the form $8k \pm 1$
    (resp. $8k \pm 3$).
    Equivalently, $\lgnd 2 p = (-1)^{\frac{p^2 - 1}{8}}$.
\end{proposition}
\begin{proof}
    Let $p$ be an odd prime and $a = 2$.
    By Gauss' lemma, look at $\set{2, 4, \dots, p-1}$.
    Thus, $\mu(p, 2)$ is equal to the number of elements which are more than
    $\frac{p-1}{2}$.

    Define $m \in \N$ as the unique natural number such that
    $2m \le \frac{p-1}{2}$ and $2(m + 1) > \frac{p-1}{2}$.
    Then $\mu = \frac{p-1}{2} - m$.
    \begin{itemize}
        \item If $p = 8k + 1$, then $\frac{p-1}{2} = 4k$ so $m = 2k$.
        This gives that $\mu = 4k - 2k = 2k$ is even.
        \item If $p = 8k + 3$, then $\frac{p-1}{2} = 4k + 1$ so $m = 2k$.
        This gives that $\mu = 4k + 1 - 2k = 2k + 1$ is odd.
        \item If $p = 8k + 5$, then $\frac{p-1}{2} = 4k+2$ so $m = 2k+1$.
        This gives $\mu = 4k+2 - 2k - 1 = 2k + 1$ is odd.
        \item If $p = 8k + 7$, then $\frac{p-1}{2} = 4k + 3$ so $m = 2k+1$.
        This gives $\mu = 4k+3 - 2k - 1 = 2k + 2$ is even.
    \end{itemize}
    Thus $\lgnd 2 p = (-1)^\mu$ is $1$ for $p = 8k \pm 1$ and $-1$ for
    $p = 8k \pm 3$.
\end{proof}

\begin{theorem}[Law of quadratic reciprocity] \label{thm:quad_res:reciprocity}
    Let $p$ and $q$ be distinct odd primes.
    Then
    \begin{enumerate}
        \item $\lgnd{-1}{p} = (-1)^{\frac{p-1}{2}}$.
        \item $\lgnd 2 p = (-1)^{\frac{p^2 - 1}{8}}$.
        \item $\lgnd{p}{q} \lgnd{q}{p}
            = (-1)^{\frac{p-1}{2} \frac{q-1}{2}}$.
    \end{enumerate}
\end{theorem}
\begin{lemma}
    Let $p$ be an odd prime and $(a, 2p) = 1$.
    Then \[
        \lgnd a p = (-1)^t, \quad \text{where } t
            = \sum_{j=1}^{\frac{p-1}{2}} \floor{\frac{ja}{p}}
    \]
\end{lemma}
\begin{proof}
    Note that $a$ is odd.
    Let $\mu$ be as in Gauss' lemma.
    Let $-r_1, \dots, -r_\mu$ be negative least residues and
    $s_1, \dots, s_k$ be positive least residues ($k = \frac{p-1}{2} - \mu$).
    Then \[
        \sum_{j=1}^{\frac{p-1}{2}} ja = \sum_{j=1}^{\frac{p-1}{2}} p \floor{\frac{ja}{p}} + \sum_{j=1}^{k} s_j + \sum_{j=1}^{\mu} (p - r_j).
    \] But $\sum_{j=1}^{(p-1)/2} j = \frac{p^2 - 1}{8}$.
    Also, from the proof of \cref{thm:quad_res:gauss}, we have
    $\sum r_j + \sum s_j = \sum_{j=1}^{\frac{p-1}{2}} j$.

    Thus we have \begin{align}
        \frac{p^2 - 1}{8} a
            &= p \sum_{j=1}^{\frac{p-1}{2}} \floor{\frac{ja}{p}}
            + \sum s_j + \mu p - \sum r_j \label{eq:quad_res:sum} \\
        \frac{p^2 - 1}{8} &= \sum s_j + \sum r_j \label{eq:quad_res:sum2}
    \end{align}
    Subtracting \cref{eq:quad_res:sum2} from \cref{eq:quad_res:sum} gives
    \begin{align*}
        (a - 1) \frac{p^2 - 1}{8}
            &= p \sum \floor{\frac{ja}{p}} + p \mu - 2 \sum r_j \\
            &\equiv \sum \floor{\frac{ja}{p}} + \mu \pmod 2.
    \end{align*}
\end{proof}
