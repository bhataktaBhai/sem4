\section{Integers} \label{sec:Z}
\lecture{2024-01-17}{}
\begin{definition*}
    An \emph{integer} is an expression of the form $a \lm b$, where
    $a, b \in \N$.
    Two integers are said to be equal, $a \lm b = c \lm d$, if
    $a + d = b + c$.
    Let \Z\ denote the set of all integers.
\end{definition*}
\begin{exercise}
    On $\N \times \N$, $(a, b) \sim (c, d)$ defined by $a + d = b + c$
    is an equivalence relation.
\end{exercise}

\begin{definition}
    The \emph{sum} and \emph{product} of two integers $a \lm b$ and
    $c \lm d$ is defined to be \begin{align*}
        (a \lm b) + (c \lm d) &\coloneq (a + c) \lm (b + d) \\
        (a \lm b) \times (c \lm d) &\coloneq (ac + bd) \lm (ad + bc)
    \end{align*}
\end{definition}

\begin{proposition}
    These operations are well-defined.
\end{proposition}
Since $n \lm 0$ behaves like $n$, we identify \N\ with the set of all such
integers.
\begin{definition}
    If $a \lm b \in \Z$, its \emph{negation}, denoted $-(a \lm b)$, is
    defined to be $b \lm a$.
    In particular, if $n \in \N$, then $-n = 0 \lm n$.
\end{definition}
\begin{definition}
    We define $(a \lm b) \le (c \lm d)$ if there exists an $n \in \N$ such
    that $(a \lm b) + n = (c \lm d)$.
\end{definition}

\begin{lemma}[Trichotomy] \label{thm:Z:trichotomy}
    
\end{lemma}

\begin{definition}[Ring] \label{def:ring}
    A \emph{ring} is a set $S$ with two binary operations $+$ and $\cdot$ such
    that for all $a, b, c \in S$,
    \begin{enumerate}[label=\small(R\arabic*)]
        \item addition is associative, \label{def:ring:asso}
        \item addition is commutative, \label{def:ring:comm}
        \item there exists an additive identity $0$, \label{def:ring:zero}
        \item there exists an additive inverse $-a$, \label{def:ring:inverse}
        \item multiplication is associative, \label{def:ring:mult_asso}
        \item there exists a multiplicative identity $1$, \label{def:ring:one}
        \item multiplication is distributive over addition (on both sides).
        \label{def:ring:dist}
    \end{enumerate}
    If the multiplicative identity does not exist, we call $S$ a
    \emph{rng}.

    For a \emph{commutative ring}, we require additionally that
    \begin{enumerate}[label=\small(CR\arabic*)]
        \item multiplication is commutative. \label{def:ring:mult_comm}
    \end{enumerate}
\end{definition}
Note that inverses are unique, since if $a + b = 0$ and $a + b' = 0$, then
$b = (b' + a) + b = b' + (a + b) = b'$.

\begin{definition}[Ordered Ring] \label{def:ordered_ring}
    An \emph{ordered ring} is a ring $S$ with a total order $\le$ such that
    for all $a, b, c \in S$,
    \begin{enumerate}[label=\small(OR\arabic*)]
        \item $a \le b$ implies $a + c \le b + c$,
            \label{def:ordered_ring:sum}
        \item $0 \le a$ and $0 \le b$ implies $0 \le ab$.
            \label{def:ordered_ring:prod}
    \end{enumerate}
\end{definition}

\begin{theorem}[Algebra of \Z]
    \Z\ is an ordered commutative ring.
\end{theorem}

\section{Rationals} \label{sec:Q}
\begin{definition*}
    A \emph{rational number} is an expression of the form $a \doubleslash b$
    where $a \in \Z$ and $b \in \Z^* = \Z \setminus \{0\}$.
    $a \doubleslash 0$ is not a rational number for any $a \in \Z$.
    Two rational numbers are said to be equal,
    $a \doubleslash b = c \doubleslash d$, if $ad = bc$.

    \Q\ denotes the set of all rational numbers.
\end{definition*}

\begin{definition}
    The \emph{sum} and \emph{product} of two rational numbers
    $a \doubleslash b$ and $c \doubleslash d$ is defined to be
    \begin{align*}
        (a \doubleslash b) + (c \doubleslash d)
            &\coloneq (ad + bc) \doubleslash bd \\
        (a \doubleslash b) * (c \doubleslash d)
            &\coloneq (ac) \doubleslash (bd)
    \end{align*}
\end{definition}

Since $z \doubleslash 1$ behaves like $z$, we identify \Z\ with the set of
all such rational numbers.

\begin{definition}
    The \emph{reciprocal} of a non-zero rational $a \doubleslash b$, denoted
    $(a \doubleslash b)^{-1}$, is defined to be $b \doubleslash a$.
\end{definition}
\begin{remark}
    The numerator and denominator of a rational number are not well-defined.
\end{remark}

\begin{definition} \label{def:field}
    A field is a set $F$ with 2 operations $+ : F \times F \to F$ and
    $\times : F \times F \to F$ such that
    \begin{enumerate}[label=\small(F\arabic*)]
        \item \label{def:field:commutativity}
            $+$ \& $\times$ are commutative on $F$.
        \item \label{def:field:associativity}
            $+$ \& $\times$ are associative on $F$.
        \item \label{def:field:distributivity}
            $+$ \& $\times$ satisfy distributivity on $F$, \textit{i.e.},
            $a \times (b + c) = a \times b + a \times c$ and
            $(b + c) \times a = a \times b + a \times c$ for all
            $a, b, c \in F$. 
        \item \label{def:field:identity}
            There exist 2 \emph{distinct} elements, called 0 and 1 such that
            for all $x \in F$,
            \begin{align*}
                x + 0 &= x \\
                x \times 1 = 1 \times x &= x
            \end{align*}
        \item \label{def:field:negative}
            For every $x \in F$, there is a $y \in F$ such that $x + y = 0$.
        \item \label{def:field:reciprocal}
            For every $x \in F^*$, there is a $z \in F$ such that
            $xz = zx = 1$.
    \end{enumerate}
    If multiplication is not commutative, we call $F$ a 
    \emph{division ring}.
\end{definition}

\begin{definition} \label{def:ordered_field}
    An \emph{ordered field} is a set that admits two operations $+$ and
    $\cdot$ and relation $<$ so that $(F, +, \cdot)$ is a field and $(F, <)$
    is an ordered set and:
    \begin{enumerate}[label=\small(OF\arabic*)]
        \item \label{def:order:sum}
            For $x, y, z \in F$, if $x < y$ then $x + z < y + z$.
        \item \label{def:order:product}
            For $x, y \in F$, if $0 < x$ and $0 < y$ then $0 < x \cdot y$.
    \end{enumerate}
\end{definition}

\begin{theorem}[Algebra of \Q]
    \Q\ is an ordered field.
\end{theorem}

\begin{definition}
    The quotient of $x \in \Q$, $y \in \Q^*$, denoted $x / y$, is defined to
    be $x y^{-1}$.
\end{definition}
This gives $a \doubleslash b = a / b$ for all $a \in \Z$ and $b \in \Z^*$.

\begin{definition}
    $x \in \Q$ is a positive (resp. negative) rational number if $x = a / b$
    for $a, b \in \N^*$ (resp. $x = -y$ for some positive $y$).
\end{definition}

\begin{proposition}
    Let $x \in \Q$.
    Then there exists a unique $n \in \Z$ such that $n \le x < n + 1$.
    In particular, there exists an $N \in \Z$ such that $x < N$.
    This unique $n$ is denoted $\floor{x}$.
\end{proposition}

\begin{proposition}
    If $x < y$ are rational numbers, then there is a rational $z$ such that
    $x < z < y$.
\end{proposition}

\begin{fact}
    There is no rational number whose square is $2$.
\end{fact}
