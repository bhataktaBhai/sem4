\lecture{2024-01-03}{}

\setcounter{section}{-1}
\section{The Course} \label{sec:course}
\textbf{MS Teams Code:} \texttt{ucwewgr}\\
\textbf{Textbook:} \textit{Differential Equations and Dynamical Systems}
by Lawrence Perko.\\
\textbf{TA:} Babhrubahan Bose (PhD student)\\
\textbf{Prerequisites:} Real analysis, specifically convergence of sequences and
series of functions.
Linear algebra.
Some topology.

\subsection{Introduction} \label{sec:introduction}
We wish to solve systems of the form \[
    \dot{x} = f(x)
\] where $x \colon (a, b) \subseteq \R \to \R^n$, $x(t) = (x_1(t), \dots, x_n(t))$,
$f \colon \R^n \to \R^n$, $f = (f_1, \dots, f_n)$ and \[
    \dot{x}(t) \coloneq (x_1'(t), \dots, x_n'(t)).
\] So the system is shorthand for \begin{align*}
    x_1'(t) &= f_1(x_1(t), \dots, x_n(t)) \\
    &\;\;\vdots \\
    x_n'(t) &= f_n(x_1(t), \dots, x_n(t)).
\end{align*}
We will later show that higher order derivatives and equations like
$\dot{x} = f(t, x)$ can be reduced to this form.

\section{Linear Systems} \label{sec:linear}
Where $f$ is linear, \ie, \[
    f(x) = Ax \text{ where } A \in M(n, \R).
\] $M(n, \R)$ is the set of all $n \times n$ matrices with real entries.

For $n = 1$ this reduces to $\dot{x} = ax$, $a \in \R$.
A (The) solution is $x(t) = ce^{at}$, $c \in \R$.

\begin{theorem}[Uniqueness] \label{thm:linear:uniqueness}
    All solutions to $\dot{x} = ax$ are of the form $x(t) = ce^{at}$ for some
    $c \in \R$.
\end{theorem}
\begin{proof}
    For any solution $x$, $(x e^{-at})' = \dot{x} e^{-at} - a x e^{-at} = 0$.
    Thus $x = c e^{at}$ for some $c \in \R$.
\end{proof}

Consider $n = 2$.
\begin{definition}[Uncoupled system] \label{def:linear:uncoupled_system}
    A system $\dot{x} = f(x)$ is \emph{uncoupled} if $f_j$ does not depend on
    $x_i$ for $i \neq j$.
\end{definition}
Consider an uncoupled linear system \[
    \dot{x} = \begin{pmatrix}
        -1 & 0 \\
        0 & 2
    \end{pmatrix} x.
\] The unique solution is $x(t) = \begin{pmatrix}
    c_1 e^{-t} \\
    c_2 e^{2t}
\end{pmatrix}$, $c_1, c_2 \in \R$.

\subsection*{Phase Portrait of an ODE System} \label{sec:phase_portrait}
We can plot the solution curves of this system in $\R^2$ for various values of
the parameters.
This plot is called the \emph{phase portrait} of the system.
The paths of the solutions are called \emph{orbits}.

We can also write the solution as \[
    x(t) = \begin{pmatrix}
        e^{-t} & 0 \\
        0 & e^{2t}
    \end{pmatrix} \begin{pmatrix}
        c_1 \\
        c_2
    \end{pmatrix}.
\] Consider the map $\phi \colon \R \times \R^2 \to \R^2$ given by \[
    \phi(t, c) = \begin{pmatrix}
        e^{-t} & 0 \\
        0 & e^{2t}
    \end{pmatrix} c.
\] $\phi$ is then called the \emph{flow} or the \emph{dynamical system}
associated with the ODE system.

An interpretation of the phase portrait requires viewing the tangent vectors at
any point $x(t)$ in a path, as $\dot{x}(t)$ or $f(x)$.
The system can be viewed through the lens of its vector field, which is given by
$f(x)$.
Any solution to the system is simply a curve which lies tangent to the vector
field at every point, which is called an \emph{integral curve} of the vector
field.
