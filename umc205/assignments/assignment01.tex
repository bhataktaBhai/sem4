\documentclass[12pt]{article}
\input{~/IISc/sem4/preamble}
\input{~/IISc/sem4/preamble/mvc}
\usepackage{pgfplots}
\pgfplotsset{compat=1.18}
\usepackage{tikz}
\usetikzlibrary{graphs, graphs.standard, quotes, positioning, arrows.meta}

\newcommand{\ind}[1]{\bm{1}_{#1}}
\newcommand{\given}{\mid}
\newcommand{\pms}{\{-1, 1\}}
\DeclareMathOperator*{\argmax}{argmax}
\DeclareMathOperator*{\argmin}{argmin}
\DeclareMathOperator*{\E}{\mathbb E}

\usepackage{amsmath}
\DeclareMathOperator\divv{div}
\DeclareMathOperator\hess{Hess}
% \DeclareMathOperator\Tr{Tr}

\usepackage{pgfplots}
\pgfplotsset{compat=1.18}
\usepackage{tikz}
\usetikzlibrary{graphs, graphs.standard, quotes, positioning, arrows.meta}

\newcommand{\ind}[1]{\bm{1}_{#1}}
\newcommand{\given}{\mid}
\newcommand{\pms}{\{-1, 1\}}
\DeclareMathOperator*{\argmax}{argmax}
\DeclareMathOperator*{\argmin}{argmin}
\DeclareMathOperator*{\E}{\mathbb E}

\usepackage{amsmath}
\DeclareMathOperator\divv{div}
\DeclareMathOperator\hess{Hess}
% \DeclareMathOperator\Tr{Tr}

\usepackage{pgfplots}
\pgfplotsset{compat=1.18}
\usepackage{tikz}
\usetikzlibrary{graphs, graphs.standard, quotes, positioning, arrows.meta}

\newcommand{\ind}[1]{\bm{1}_{#1}}
\newcommand{\given}{\mid}
\newcommand{\pms}{\{-1, 1\}}
\DeclareMathOperator*{\argmax}{argmax}
\DeclareMathOperator*{\argmin}{argmin}
\DeclareMathOperator*{\E}{\mathbb E}

\usetikzlibrary{bbox}

\title{Assignment 1} \setcounter{assignment}{1}
\author{Naman Mishra}
\date{16 January, 2024}
\newcommand\state[3]{%
    % #1#2#3
    \begin{tikzpicture}[scale=0.13]
        \ifthenelse{\equal{#1}{L}}{\draw (0, 0) -- (1, 1);}{\draw (0, 1) -- (1, 0);}
        \ifthenelse{\equal{#2}{L}}{\draw (2, 0) -- (3, 1);}{\draw (2, 1) -- (3, 0);}
        \ifthenelse{\equal{#3}{L}}{\draw (1, -2) -- (2, -1);}{\draw (1, -1) -- (2, -2);}
    \end{tikzpicture}
}

\begin{document}
\maketitle

\begin{problem}
    Give a DFA for the language of all strings over the alphabet
    $\set{0, 1}$ which contain an occurrence of $010$ or $100$ as a
    contiguous substring.
\end{problem}
\begin{solution} \leavevmode
    \begin{center}
        \begin{tikzpicture}
            \node[state, initial] (e) {};
            \node[state, above right=of e] (1) {};
            \node[state, right=of 1] (10) {};
            \node[state, accepting, below right=of 10] (F) {};
            \node[state, below right=of e] (0) {};
            \node[state, right=of 0] (01) {};
            \path[->]
                (e) edge node {$1$} (1)
                    (e) edge node[swap] {$0$} (0)
                (1) edge node {$0$} (10)
                    (1) edge[loop above] node {$1$} (1)
                (10) edge node {$0$} (F)
                    (10) edge node {$1$} (01)
                (F) edge[loop right] node {$0, 1$} (F)
                (0) edge node[swap] {$1$} (01)
                    (0) edge[loop below] node {$0$} (0)
                (01) edge node[swap] {$0$} (F)
                    (01) edge node {$1$} (1);
        \end{tikzpicture}
    \end{center}
\end{solution}

\begin{problem}
\end{problem}
\begin{solution}
    We take the following liberties while drawing the DFA.
    \begin{itemize}
        \item If two states $q$ and $q'$ are such that $\delta(q, a) =
        \delta(q', a)$ for all $a \in \set{A, B}$, where $\delta$ is the
        transition function, we join them to a ``junction'' in between, and
        draw a single path from the junction instead of one from each state.
        \item Instead of labelling each edge with the input symbol, we color
        the edges red for input $A$ and blue for input $B$.
    \end{itemize}
    \begin{center}
        \begin{tikzpicture}[every node/.style={font=\scriptsize},
        % every edge.style={every node/.style={very near start}},
        node distance=2cm and 2cm,
        bezier bounding box]
            \node[state, initial] (LLLC) {$\epsilon$};
            \node[state, accepting, right=0.5cm of LLLC] (LLLD){\state LLL};
            \coordinate (LLL) at ([yshift=-1cm]$(LLLC)!0.5!(LLLD)$);
            \draw (LLLC) -- (LLL) -- (LLLD);

            \node[state, below left=of LLL] (RLLC) {\state RLL};
            \node[state, accepting, below=0.5cm of RLLC] (RLLD) {\state RLL};
            \coordinate (RLL) at ([xshift=1cm]$(RLLC)!0.5!(RLLD)$);
            \draw (RLLC) -- (RLL) -- (RLLD);

            \node[state, below right=of LLL] (LRR) {\state LRR};

            \node[state, accepting, below right=of LRR] (LLRD){\state LLR};
            \node[state, left=0.5cm of LLRD] (LLRC) {\state LLR};
            \coordinate (LLR) at ([yshift=-1cm]$(LLRC)!0.5!(LLRD)$);
            \draw (LLRC) -- (LLR) -- (LLRD);

            \node[state, left=of LLRC] (RRR) {\state RRR};

            \node[state, accepting, below=of LLR] (LRL) {\state LRL};

            \node[state, left=1cm of LRL] (RLRC) {\state RLR};
            \node[state, accepting, below left=0.5cm and 0.5cm of RLRC] (RLRD) {\state RLR};
            \coordinate (RLR) at ([xshift=-0.2cm, yshift=0.2cm]$(RLRC)!0.5!(RLRD)$);
            \draw (RLRC) -- (RLR) -- (RLRD);


            \node[state, accepting, below=1.5cm of LRL] (RRLD) {\state RRL};
            \node[state, above right=0.1cm and 0.5cm of RRLD] (RRLC){\state RRL};
            \coordinate (RRL) at ([xshift=0.4cm, yshift=-1cm]$(RRLC)!0.5!(RRLD)$);
            \draw (RRLC) -- (RRL) -- (RRLD);

            \path[->]
                (LLL) edge[color=Red] (RLLC)
                    (LLL) edge[color=blue] (LRR)
                (RLL) edge[color=Red] (LLRC)
                    (RLL) edge[color=blue] (RRR)
                (LRR) edge[color=Red] (RRR)
                    (LRR) edge[color=blue] (LLRD)
                (LLR) edge[color=Red] (RLRC)
                    (LLR) edge[color=blue] (LRL)
                (RRR) edge[color=Red] (LRL)
                    (RRR) edge[color=blue] (RLRD)
                (LRL) edge[color=Red] (RRLC)
                    (LRL) edge[color=blue] [bend right=3cm] (LLLD)
                (RRL) edge[color=Red] [bend right=3cm] (LRR)
                    (RRL) edge[color=blue] [bend left=2cm] (RLLD)
                (RLR) edge[color=Red] [bend left=1.5cm] (LLLD)
                    (RLR) edge[color=blue] (RRLD);

            \node[circle, draw, inner sep=0.6pt, fill=black] at (LLL) {};
            \node[circle, draw, inner sep=0.6pt, fill=black] at (RLL) {};
            \node[circle, draw, inner sep=0.6pt, fill=black] at (LLR) {};
            \node[circle, draw, inner sep=0.6pt, fill=black] at (RRL) {};
            \node[circle, draw, inner sep=0.6pt, fill=black] at (RLR) {};
        \end{tikzpicture}
    \end{center}
\end{solution}

\begin{problem}
    Show that the set of strings in $\set{0, 1, 2}^*$ which are base $3$
    representations of even numbers, is regular.
\end{problem}
\begin{proof}
    Let $s \in \set{0, 1, 2}^*$ be a non-empty string.
    We zero-index the string from right to left.
    Then $s$ encodes the number $n = \sum_{i=0}^n s_i 3^i$ in base $3$.
    \begin{align*}
        n &= \sum_{i=0}^n s_i 3^i \\
        n &\equiv \sum_{i=0}^n s_i 1^i \pmod{2} \\
        n &\equiv \#_1(s) \pmod{2}
    \end{align*} where $\#_1(s)$ is the number of $1$'s in $s$.
    Thus $n$ is even iff $s$ contains an even number of $1$'s.

    We first give a DFA which accepts $Z$, the set of valid ternary
    representations of all natural numbers, \ie, the set of all non-empty
    strings over $A$ which either start with $1$ or $2$,
    or are exactly ``$0$''.
    \begin{center}
        \begin{tikzpicture}
            \node[state, initial] (e) {};
            \node[state, accepting, above right=of e] (0) {};
            \node[state, accepting, below right=of e] (1) {};
            \node[state, right=of 0] (0x) {};
            \path[->]
                (e) edge node {$0$} (0)
                    (e) edge node[swap] {$1$, $2$} (1)
                (1) edge[loop right] node {$0$, $1$, $2$} (1)
                (0) edge node {$0$, $1$, $2$} (0x)
                (0x) edge[loop right] node {$0$, $1$, $2$} (0x);
        \end{tikzpicture}
    \end{center}
    This gives us that $Z$ is regular.

    Let $E$ be the set of all strings over $\set{0, 1, 2}$ which are
    base $3$ representations of even numbers, possibly with leading zeroes
    and with the empty string counting as a representation of $0$.
    This is a regular language, as it is accepted by the following DFA.
    \begin{center}
        \begin{tikzpicture}[every node/.style={font=\scriptsize}]
            \node[state, initial, accepting] (even) {even};
            \node[state, right=of e] (odd) {odd};
            \path[->]
                (even) edge[bend right] node[swap] {$1$} (odd)
                    (even) edge[loop above] node {$0$, $2$} (even)
                (odd) edge[loop above] node {$0$, $2$} (odd)
                    (odd) edge[bend right] node[swap] {$1$} (even);
        \end{tikzpicture}
    \end{center}
    Since regular languages are closed under intersection, $E$ intersected
    with $Z$ is also regular.
    However, this is precisely the set of all valid ternary representations
    of even numbers as required in the problem.
\end{proof}

\begin{problem}
    Let $\mcA = (Q, s, \delta, F)$ be a DFA over alphabet $A$.
    Give a formal proof that for any strings $x, y \in A^*$: \[
        \widehat{\delta}(s, xy)
            = \widehat{\delta}(\widehat{\delta}(s, x), y)
    \]
\end{problem}
\begin{proof}
    We will prove this by induction on the length of $y$.
    For $\size y = 0 \iff y = \epsilon$, we have \begin{align*}
        \widehat{\delta}(s, x\epsilon)
            &= \widehat{\delta}(s, x) \\
            &= \widehat{\delta}(\widehat{\delta}(s, x), \epsilon)
    \end{align*} by definition of $\widehat\delta$.

    Suppose the statement holds for all $y$ of length $\size y = n \geq 0$.
    Let $y$ be a string of length $\size y = n + 1$.
    Then $y = y' a$ for some $y'$ of length $n$ and $a \in A$.
    \begin{align*}
        \widehat{\delta}(s, xy)
            &= \widehat{\delta}(s, xy'a) \\
            &= \delta(\widehat{\delta}(s, xy'), a) \tag{definition} \\
            &= \delta(\widehat{\delta}(\widehat{\delta}(s, x), y'), a)
                \tag{induction hypothesis} \\
            &= \widehat{\delta}(\widehat{\delta}(s, x), y'a) \tag{definition} \\
            &= \widehat{\delta}(\widehat{\delta}(s, x), y).
    \end{align*}
    Thus by induction, the statement holds for all $y \in A^*$.
    Since $x$ was arbitrary, the statement holds for all $x, y \in A^*$.
\end{proof}

\begin{problem}
    Consider the language of nested C-style comments.
    The aphabet comprises characters ``\verb|/|'', ``\verb|*|'', and
    ``\verb|c|'' (the latter symbol representing any ASCII character apart
    from ``\verb|/|'' and ``\verb|*|'').
    The language allows all well-nested and complete comments.
    Thus strings like ``\verb|cc/*ccc*/c|'' and
    ``\verb|cc/*cc/*ccc*/c*/ccc|'' are in the language, but not 
    ``\verb|cc/*c/*cc*/cc|''.
    Is this language regular? Justify your answer. 
\end{problem}
\begin{proof}
    Suppose this language $L$ is regular.
    Let $k$ be a pumping length for $L$.
    That is, let $k \in \N$ be such that for any word $t \in L$ of the form
    $xyz$ with $\abs{y} \ge k$, there exist strings $u$, $v$ and $w$ with
    $y = uvw$, $v \ne \epsilon$, and $xuv^iwz \in L$ for each $i \ge 0$.

    Let $t = xyz$ where $x = (\verb|/*|)^k$, $y = (\verb|*/|)^k$, and
    $z = \epsilon$.
    Then $t \in L$ since all comments are well-nested and closed.
    Let $u$, $v$ and $w$ be strings following the above statement for this
    $y$.

    Since the first $2k$ characters of $xuwz$ constitute $k$ comment opening
    delimiters, but the rest of the string has less than $2k$ characters,
    it cannot contain $k$ comment closing delimiters.
    Thus $xuwz \notin L$, a contradiction.

    Therefore $L$ can not be regular.
\end{proof}

\begin{problem}
    For a set of natural numbers $X$, define $binary(X)$ to be the set of
    binary representations of numbers in $X$.
    Similarly define $unary(X)$ to be the set of ``unary'' representations
    of numbers in $X$: $unary(X) = \set{1^n \mid n \in X}$.
    Thus for $X = \set{2, 3, 6}$, $binary(X) = \set{10, 11, 110}$ and
    $unary(X) = \set{11, 111, 111111}$.
    Consider the two propositions below:
    \begin{enumerate}[label=(\alph*)]
        \item For all $X$, if $binary(X)$ is regular then so is $unary(X)$.
        \item For all $X$, if $unary(X)$ is regular then so is $binary(X)$.
    \end{enumerate}
    One of the statements above is true and the other is false.
    Which is which?
    Justify your answer.
\end{problem}
The second statement is true and the first is false.
We first prove the falsity of the first statement.
One crucial observation for both proofs is that \[
    lengths(unary(X)) = \set{\abs{1^n} : n \in X} = X.
\]
\begin{proof}
    Let $X = \set{2^n \mid n \in \N}$.
    Then $binary(X) = \set{10^n \mid n \in \N}$ is regular.
    A DFA for this language is
    \begin{center}
        \begin{tikzpicture}
            \node[state, initial] (e) {};
            \node[state, accepting, right=of e] (1) {};
            \node[state, right=of 1] (11) {};
            \path[->]
                (e) edge node {$1$} (1)
                    (e) edge[loop above] node {$0$} (e)
                (1) edge node {$1$} (11)
                    (1) edge[loop above] node {$0$} (1)
                (11) edge[loop above] node {$0$, $1$} (11);
        \end{tikzpicture}
    \end{center}
    However, $unary(X) = \set{1^{2^n} \mid n \in \N}$ is not regular as its
    lengths, that is $X$, is not ultimately periodic.
    This is because for any alleged period $p \in \N^+$ and any $N \in \N$,
    we have $2^{N+p} \in X$ but $p + 2^{N+p} \notin X$ since
    $2^{N+p} < p + 2^{N+p} < 2^{N+p+1}$ (note also that $2^{N+p} > N$).
\end{proof}
We now prove the truth of the second statement.
\begin{proof}
    Let $X$ be such that $unary(X)$ is regular.
    Then $X$ is ultimately periodic.
    Let $N, p \in \N^+$ be such that for all $n \ge N$, $n \in X \iff n + p
    \in X$.

    Then $X = E \cup \set{n \in \N \mid n \ge N, n \bmod p \in K}$ for
    some finite subset $E \subseteq \set{0, 1, \ldots, N-1}$ and some
    $K \subseteq \set{0, 1, \dots, p-1}$.
    We write $X$ as $E \cup (M \cap B)$, where \begin{align*}
        M &= \set{n \in \N \mid n \bmod p \in K} \\
        B &= \set{n \in \N \mid n \ge N}
    \end{align*}
    Since there exists a unique binary representation of each natural number
    (without leading zeroes), $binary(X) = binary(E) \cup (binary(M) \cap
    binary(B))$.
    We show that each of these three sets is regular by constructing DFAs
    over the alphabet $A = \set{0, 1}$ that accept them.

    We first give a DFA which accepts $Z$, the set of valid binary
    representations of all natural numbers, \ie, the set of all non-empty
    strings over $A$ which either start with $1$ or are exactly ``$0$''.
    \begin{center}
        \begin{tikzpicture}
            \node[state, initial] (e) {};
            \node[state, accepting, above right=of e] (0) {};
            \node[state, accepting, below right=of e] (1) {};
            \node[state, right=of 0] (0x) {};
            \path[->]
                (e) edge node {$0$} (0)
                    (e) edge node[swap] {$1$} (1)
                (1) edge[loop right] node {$0$, $1$} (1)
                (0) edge node {$0$, $1$} (0x)
                (0x) edge[loop right] node {$0$, $1$} (0x);
        \end{tikzpicture}
    \end{center}
    This gives us that $Z$ is regular.

    We define $binary^\infty(S)$ to be the set of all strings which are
    any binary representation of some number in $S$, possibly with leading
    zeroes and with the empty string counting as a representation of $0$.
    Note that though one natural number may have multiple binary
    representations, each binary representation is a representation of a
    unique natural number.

    Then $binary(S) = binary^\infty(S) \cap Z$ for all $S \subseteq \N$.
    Thus it suffices to show that $binary^\infty(E)$, $binary^\infty(M)$ and
    $binary^\infty(B)$ are regular, since regular languages are closed under
    intersection.

    Let
    \begin{align*}
        Q &= \set{0, 1, \ldots, N} \\
        \delta(q, a) &= \begin{cases}
            \min\set{2q, N} & \text{if } a = 0, \\
            \min\set{2q + 1, N} & \text{if } a = 1.
        \end{cases} \\
        \mcE &= (Q, 0, \delta, E) \\
        \mcB &= (Q, 0, \delta, \set{N})
    \end{align*}
    Here, each state $q \in \set{0, 1, \dots, N-1}$ corresponds to the
    property that the string read so far represents $q$, whereas state $N$
    corresponds to the property that it represents some number greater than
    or equal to $N$.
    $\delta$ preserves each property and thus $\mcE$ and $\mcB$ accept
    $binary^\infty(E)$ and $binary^\infty(B)$ respectively.
    This gives regularity of $binary(E)$ and $binary(B)$.

    Finally, $binary^\infty(M)$ is regular since it is accepted by \[
        \mcM = (\set{0, \dots, p-1}, 0, \delta', K),
    \] where
    \begin{align*}
        \delta'(q, 0) &= 2q \bmod p \\
        \delta'(q, 1) &= (2q + 1) \bmod p
    \end{align*}
    This gives regularity of $binary(M)$.

    Since regular languages are closed under union and intersection,
    $binary(X)$ is also regular.
\end{proof}

\end{document}
