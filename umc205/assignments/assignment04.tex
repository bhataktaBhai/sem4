\documentclass[12pt]{article}
\input{~/IISc/sem4/preamble}
\input{~/IISc/sem4/preamble/mvc}
\usepackage{pgfplots}
\pgfplotsset{compat=1.18}
\usepackage{tikz}
\usetikzlibrary{graphs, graphs.standard, quotes, positioning, arrows.meta}

\newcommand{\ind}[1]{\bm{1}_{#1}}
\newcommand{\given}{\mid}
\newcommand{\pms}{\{-1, 1\}}
\DeclareMathOperator*{\argmax}{argmax}
\DeclareMathOperator*{\argmin}{argmin}
\DeclareMathOperator*{\E}{\mathbb E}

\usepackage{amsmath}
\DeclareMathOperator\divv{div}
\DeclareMathOperator\hess{Hess}
% \DeclareMathOperator\Tr{Tr}

\usepackage{pgfplots}
\pgfplotsset{compat=1.18}
\usepackage{tikz}
\usetikzlibrary{graphs, graphs.standard, quotes, positioning, arrows.meta}

\newcommand{\ind}[1]{\bm{1}_{#1}}
\newcommand{\given}{\mid}
\newcommand{\pms}{\{-1, 1\}}
\DeclareMathOperator*{\argmax}{argmax}
\DeclareMathOperator*{\argmin}{argmin}
\DeclareMathOperator*{\E}{\mathbb E}

\usepackage{amsmath}
\DeclareMathOperator\divv{div}
\DeclareMathOperator\hess{Hess}
% \DeclareMathOperator\Tr{Tr}

\usepackage{pgfplots}
\pgfplotsset{compat=1.18}
\usepackage{tikz}
\usetikzlibrary{graphs, graphs.standard, quotes, positioning, arrows.meta}

\newcommand{\ind}[1]{\bm{1}_{#1}}
\newcommand{\given}{\mid}
\newcommand{\pms}{\{-1, 1\}}
\DeclareMathOperator*{\argmax}{argmax}
\DeclareMathOperator*{\argmin}{argmin}
\DeclareMathOperator*{\E}{\mathbb E}

\usepackage{geometry}
\DeclareMathOperator{\pref}{pref}
% \geometry{a4paper, margin=1in}

\title      {Assignment 4}
\setcounter{assignment}{4}
\setcounter   {section}{4}
\author{Naman Mishra}
\date{14 March, 2024}

\begin{document}
\maketitle
% Problem 1
\begin{problem*}
    Construct a context-free grammar for the language \[
        \set{xy \in \set{0, 1}^* \mid \abs{x} = \abs{y} \text{ and } y
        \text{ contains a } 1}.
    \]
\end{problem*}
\begin{solution}
    \begin{align*}
        S &\to 0 S 0 \mid 1 S 0 \\
        S &\to 0 R 1 \mid 1 R 1 \\
        R &\to 0 R 0 \mid 0 R 1 \mid 1 R 0 \mid 1 R 1 \mid \epsilon \qedhere
    \end{align*}
\end{solution}

% Problem 2
\begin{problem*}
    Convert the following context-free grammar to Chomsky Normal Form (CNF):
    \begin{align*}
        S &\to AAA \mid B \\
        A &\to aA \mid B \\
        B &\to \epsilon
    \end{align*}
\end{problem*}
\begin{solution}
    We construct the new set of productions using these.

    We add the transition $A \to \epsilon$, since $A \to B$ and
    $B \to \epsilon$.
    This allows us to add the transition $S \to AA$ and $A \to a$.

    But $S \to AA$ and $A \to \epsilon$ means we can add $S \to A$.
    Now that $S \to A$ is a production, we can add $S \to a$.

    We then drop unit- and $\epsilon$-productions to get the rules
    \begin{align*}
        S &\to a \mid AA \mid AAA \\
        A &\to a \mid aA
    \end{align*}
    The problematic productions are $S \to AAA$ and $A \to aA$.
    We can fix these by introducing non-terminals $X$ and $Y$ with rules
    \begin{align*}
        X &\to AA \\
        Y &\to a
    \end{align*}
    This gives the CNF
    \begin{align*}
        S &\to a \mid AA \mid XA \\
        X &\to AA \\
        A &\to a \mid YA \\
        Y &\to a \qedhere
    \end{align*}
\end{solution}

% Problem 3
\begin{problem*}
    Consider the language $L = \set*{a^nb^{n^2} \mid n \ge 0}$.
    Use the Pumping Lemma for CFLs to show that $L$ is not a CFL. 
\end{problem*}
\begin{solution}
    Suppose $L$ is a CFL.
    Let $k$ be the pumping constant.
    Consider the string $z = a^kb^{k^2}$.
    Then $z = uvwxy$ such that
    \begin{itemize}
        \item $\abs{vwx} \le k$,
        \item $vx \ne \epsilon$, and
        \item $uv^iwx^iy \in L$ for all $i \ge 0$.
    \end{itemize}
    Suppose $v = a^\alpha b^\beta$ with $\alpha \beta \ne 0$.
    Then $uv^2wx^2y = ua^{\alpha} b^{\beta} a^{\alpha} b^{\beta} wx^2y$
    contains at least two runs of $a$'s and hence cannot be in $L$.
    Thus $v$ is not of this form.
    Similarly, neither is $x$.

    Thus $v$ and $x$ can only contain either all $a$'s or all $b$'s.
    If they were both all $a$'s, then $uv^2wx^2y$ would contain more than
    $k$ $a$'s but only $k^2$ $b$'s.
    Similarly, if they were both all $b$'s, then $uv^2wx^2y$ would contain
    more than $k^2$ $b$'s but only $k$ $a$'s.

    Thus $v = a^\alpha$ and $x = b^\beta$.
    Then $uv^2wx^2y = a^{k + \alpha} b^{k^2 + \beta} \in L$.
    This gives $\beta = 2k\alpha + \alpha^2$.
    Also, $uwy = a^{k - \alpha} b^{k^2 - \beta} \in L$.
    This gives $\beta = 2k\alpha - \alpha^2$.
    Thus $\alpha^2 = 0 \implies \alpha = 0$ and $\beta = 0$.
    But then $vx = \epsilon$, a contradiction.

    Thus $L$ is not a CFL.
\end{solution}

% Problem 4
\begin{problem*}
    Consider the CFG $G$ below:
    \begin{align*}
        S &\to XC \mid AY \\
        X &\to aXb \mid ab \\
        Y &\to bYc \mid bc \\
        A &\to aA \mid a \\
        C &\to cC \mid c
    \end{align*}
    \begin{enumerate}[label=(\alph*)]
        \item Describe the language accepted by $G$.
        \item Use the construction in Parikh’s theorem to construct a
        semi-linear expression for $\psi(L(G))$.
        That is \begin{enumerate}[label=\roman*.]
            \item Identify the basic pumps for $G$.
            \item Identify the $\le$-minimal trees.
            \item Use those to obtain an expression for $\psi(L(G))$.
        \end{enumerate}
        \item Use the semi-linear expression obtained above to give a
        regular expression that is letter-equivalent to $L(G)$.
    \end{enumerate}
\end{problem*}
\begin{solution} \leavevmode
    \begin{enumerate}[label=(\alph*)]
        \item \[
            L(G) = \set{a^l b^m c^n \mid l, m, n \ge 0 \text{ and }
                m = l \text{ or } m = n}.
        \]
        \item \begin{enumerate}[label=\roman*.]
            \item The basic pumps are $A \to aA$, $B \to bB$, $C \to cC$,
            $X \to aXb$, and $Y \to bYc$.
            \begin{center}
                \begin{forest}
                    [$A$
                        [$a$]
                        [$A$]
                    ]
                \end{forest}
                \qquad \qquad
                \begin{forest}
                    [$B$
                        [$b$]
                        [$B$]
                    ]
                \end{forest}
                \qquad \qquad
                \begin{forest}
                    [$C$
                        [$c$]
                        [$C$]
                    ]
                \end{forest}
                \\
                \begin{forest}
                    [$X$
                        [$a$]
                        [$X$]
                        [$b$]
                    ]
                \end{forest}
                \qquad \qquad
                \begin{forest}
                    [$Y$
                        [$b$]
                        [$Y$]
                        [$c$]
                    ]
                \end{forest}
            \end{center}
            \item The $\le$-minimal trees are
            $S \to XC \Rightarrow abc$ and $S \to AY \Rightarrow abc$.
            \begin{center}
                \begin{forest}
                    [$S$
                        [$X$
                            [$a$]
                            [$b$]
                        ]
                        [$C$
                            [$c$]
                        ]
                    ]
                \end{forest}
                \qquad \qquad
                \begin{forest}
                    [$S$
                        [$A$
                            [$a$]
                        ]
                        [$Y$
                            [$b$]
                            [$c$]
                        ]
                    ]
                \end{forest}
            \end{center}
            \item The basic pumps, in the order they are listed, are
            of lengths $1$, $1$, $1$, $2$ and $2$.

            For the first tree, we can apply the pumps $X \to aXb$,
            $B \to bB$ or $C \to cC$
            Thus that ``bucket'' gives Parikh map $\set{3 + 2m + n}$.

            For the second tree, we can apply the pumps $A \to aA$,
            $B \to bB$ or $Y \to bYc$.
            Thus that bucket gives the same Parikh map.

            Thus the Parikh map of the language is \[
                \psi(L(G)) = 3 + \Angled{2, 1} = 3 + \Angled{1}
            \] or simply $\set{3, 4, \dots}$. \qedhere
        \end{enumerate}
        \item $aaaa^*$ also has Parikh map $\set{3, 4, \dots}$.
    \end{enumerate}
\end{solution}

% Problem 5
\begin{problem*}
    Give the transition diagram of a PDA for the language
    $L = \set{a^l b^m c^n \mid l = m + n}$.
\end{problem*}
\begin{solution} \leavevmode
    \begin{center}
        \begin{tikzpicture}
            \node[state, initial] (a) {$a$};
            \node[state, right=2cm of a] (b) {$b$};
            \node[state, right=2cm of b] (c) {$c$};
            \path[->]
            (a) edge[loop above] node[above] {\begin{tabular}{c}
                $a, \bot/A$ \\
                $a, A/AA$
            \end{tabular}} (a)
            (a) edge[loop below] node[below] {$\epsilon, \bot/\epsilon$} (a)
            (a) edge node[above]
                {$b, A/\epsilon$} (b)
            (a) edge[bend right] node[below]
                {$c, A/\epsilon$} (c)
            (b) edge[loop above] node[above] {$b, A/\epsilon$} (b)
            (b) edge node[above]
                {$c, A/\epsilon$} (c)
            (c) edge[loop above] node[above] {$c, A/\epsilon$} (c);
        \end{tikzpicture}
    \end{center}
    accepts $L$ via empty stack.
\end{solution}

% Problem 6
\begin{problem*}
    Is the class of context-free languages closed under the
    prefix operation?
    Justify your answer.
\end{problem*}
\begin{solution}
    Yes.

    Let $\pref(L)$ denote the set of prefixes of strings in $L$. \[
        \pref(L) = \set{u \mid \exists v \in A^*, uv \in L}.
    \]
    We first do away with the case of the empty language.
    $\pref(\O) = \O$ is a CFL iff $\O$ is a CFL.
    Whether $\O$ is a CFL or not (it is) is not even relevant.

    Next, notice that $\pref(L)$ contains the empty string for any non-empty
    $L$.
    Thus $\pref(L) = \pref(L \setminus \set{\epsilon})$.
    So we can assume that $L$ does not contain the empty string.

    Now let $G = (N, A, S, P)$ be a CFG for $L$ in CNF.
    Assume that there are no non-terminals from which \emph{no}
    terminal string can be derived.%
    \footnote{If there are, we can remove them, and all productions
    containing them, from $G$ without affecting the language generated.
    We handled $L = \O$ separately because the starting symbol cannot
    be removed.}

    For each non-terminal $X \in N$, introduce a new non-terminal
    $X_\bot \notin N$.
    Denote the set of these new non-terminals
    $N_\bot = \set{X_\bot \mid X \in N}$.
    We will call these ``vanishing'' non-terminals, or simply
    ``vanisher''s.

    For each $X_\bot$, we add the following productions to $P$:
    \begin{itemize}
        \item $X_\bot \to \epsilon$.
        \item $X_\bot \to a$, for each production $X \to a$.
        \item $X_\bot \to Y Z_\bot$ and $X_\bot \to Y_\bot$,
        for each production $X \to Y Z$.
    \end{itemize}
    Call this new set of productions $P'$

    Intuitively, vanishing non-terminals are allowed to go to the empty
    string directly.
    They represent truncation, and as such can only be allowed to appear
    at the end of any sentential form.

    We claim that $G' = (N \cup N_\bot, A, S_\bot, P')$
    is a CFG for the prefix of $L$,
    \ie, \[
        L(G') = \pref(L).
    \]

    We will use ``non-terminal sentential form'' to mean a sentential form
    which contains at least one non-terminal.
    We use $\Rightarrow$ to denote derivation steps in both $G$ and $G'$.
    The context will make it clear which grammar we are referring to
    (if any vanisher is present, we are referring to $G'$).
    % \begin{lemma}
    %     If a sentential form of $G'$ contains a vanishing non-terminal,
    %     it occurs as the last symbol.
    % \end{lemma}
    % \begin{proof}
    %     We induct on the length $n$ of the derivation.
    %     The base case $n = 0$ is direct.
    %
    %     Suppose the claim holds for all sentential forms derived in $n$
    %     steps.
    %     Let $S \xRightarrow{n} \nu \xRightarrow{1} \omega$.
    %
    %     Suppose the last step is $X \to \gamma$.
    %     This is an original production, from $P'$, so $\gamma$ does not
    %     contain any vanishing non-terminals.
    %
    %     Then $\nu = \alpha X \beta \xRightarrow{1} \alpha \gamma \beta
    %     = \omega$.
    %     By the induction hypothesis, only the last symbol of $\beta$ can be
    %     a vanishing non-terminal.
    %     Hence $w$ can only contain a vanishing non-terminal at the end.
    %
    %     If the last step is $X_\bot \to \gamma$, then $\nu = \alpha X_\bot$
    %     by the induction hypothesis, so $\omega = \alpha \gamma$,
    %     where $\alpha$ is free of vanishing non-terminals.
    %     But looking at the productions, we see that $\gamma$ either contains
    %     no vanishing non-terminals, or contain them as the last symbol.
    %     Thus the claim holds for $n + 1$.
    %
    %     By induction, the claim holds for all $n$.
    % \end{proof}

    \begin{lemma} \label{thm:trail}
        Let $S_\bot \xRightarrow{n} \omega \xRightarrow{1} w$ be a
        leftmost derivation of $w \in L(G')$.
        Then $\omega = z X_\bot$ for some $z \in A^*$ and $X \in N$.

        Moreover, each intermediate sentential form is of the form
        $\gamma X_\bot$ for some $\gamma \in (A \cup N)^*$ and $X \in N$.
    \end{lemma}
    \begin{proof}
        The last symbol in $S_\bot$ is vanishing.
        From the definition of $P'$ we see that for any production
        $X_\bot \to \gamma$, $\gamma$ is either a terminal string or
        ends with a vanisher.

        If a production from the rightmost non-terminal contains no
        non-terminals, then it must be the last step in the derivation
        (since it is leftmost).

        Combining these two, we can induct on the length $n$ of the
        derivation.

        For $n = 0$, the only sentential form is $S_\bot$,
        and the claim holds.
        In each step before the last, it is not possible for the last
        non-terminal to go to a terminal string.
        But then each step must maintain that the last symbol is a
        vanisher.

        By induction, each intermediate sentential form (including $\omega$)
        is of the form $\gamma X_\bot$ for some $\gamma \in (A \cup N)^*$
        and $X \in N$.

        But $\omega$ contains exactly one non-terminal, since
        it derives a terminal string in a single step.
        Thus $\omega = z X_\bot$ for some $z \in A^*$ and $X \in N$.
    \end{proof}

    \begin{lemma} \label{thm:reverse}
        Suppose $z X_\bot$ is a sentential form in $G'$.
        Then there is a sentential form $zX\gamma$ in $G$.
    \end{lemma}
    \begin{proof}
        Consider a leftmost derivation $S_\bot \xRightarrow{*} zX_\bot$.
        By the previous lemma, each intermediate sentential form is of the
        form $\gamma Y_\bot$.
        For any production $Z \to \zeta$, keep it as is.
        For any production $Z_\bot \to \zeta$, replace it with the
        corresponding production in $P$.
        That is,
        \begin{center}
            \begin{tabular}{lcl}
                $Z_\bot \to a$ & $\implies$ & $Z \to a$ \\
                $Z_\bot \to Z_1 (Z_2)_\bot$ & $\implies$ & $Z \to Z_1 Z_2$
                    \\
                $Z_\bot \to (Z_1)_\bot$ & $\implies$ & $Z \to Z_1 Z_2$
            \end{tabular}
        \end{center}
        where in the last case,
        $Z \to Z_1 Z_2$ must exist in $P$ for some $Z_2$,
        in order for $Z_\bot \to (Z_1)_\bot$ to be in $P'$.

        Then for any $\gamma Y_\bot \xRightarrow{1} \gamma'$, this process
        preserves the first $\abs{\gamma'}$ symbols, but with each vanishing
        non-terminal replaced by its non-vanishing counterpart.

        Thus, this yields a derivation of some sentential form $\gamma''$,
        whose $\abs{zX_\bot}$-length prefix is $zX$.
    \end{proof}

    \begin{proposition}
        $L(G') \subseteq \pref(L(G))$.
    \end{proposition}
    \begin{proof}
        Let $S_\bot \xRightarrow{n} \omega \xRightarrow{1} w$ be a leftmost
        derivation of $w \in L(G')$.

        By \cref{thm:trail}, $\omega = z X_\bot$ for some $z \in A^*$ and
        $X \in N$.
        By \cref{thm:reverse}, there is a sentential form
        $z X \gamma$ of $G$.

        We assumed that each non-terminal derives some terminal string.
        By extension, each sentential form of $G$ derives a terminal string.
        Let $\gamma$ derive a terminal string $y$,
        and $X$ derive a terminal string $x$.

        The last step in the derivation of $w$ must involve $X_\bot$.
        If it is $X_\bot \to a$, then $X \to a$ is a production in $P$
        and so $zay \in L(G)$.
        Then $w = za \in \pref(L(G))$.

        If it is $X_\bot \to \epsilon$, then $w = x$.
        But $z x y \in L(G)$, so $w \in \pref(L(G))$.

        Thus $L(G') \subseteq \pref(L(G))$.
    \end{proof}

    \begin{proposition}
        $\pref(L(G)) \subseteq L(G')$.
    \end{proposition}
    \begin{proof}
        Let $w \in \pref(L(G))$.

        Look at the parse tree of a derivation of $w$ in $G$.
        Since $w \in \pref(L(G))$, the leftmost $\abs{w}$ leaves of
        the parse tree give $w$.

        We will give a scheme to derive $w$ in $G'$.

        If $w = \epsilon$, then apply the production $S_\bot \to \epsilon$.
        Otherwise, take the subtree rooted at the root which contains
        only the leaves for $w$.
        Look at the first level.

        If the root production is $S \to a$, then $w = \epsilon$ or
        $w = a$, which can both be derived from $S_\bot$ in a single step.

        Otherwise, the root production is $S \to XY$, and the $w$-subtree
        must contain at least one child of the root.
        \begin{itemize}
            \item If the subtree for $w$ only contains left child,
            then apply the production $S_\bot \to X_\bot$.
            This reduces the problem to deriving $w$ from $X_\bot$,
            which can be done by repeating this process on the subtree
            rooted at $X$.
            \item If it contains both, then $w = xy$, where $x$ is derived
            from $X$, and $y$ is the prefix of a terminal string derived
            from $Y$.
            Keep the productions from $X$ as is.
            Apply this same process to the subtree rooted at
            $Y$ to derive $y$.
        \end{itemize}
        For example, consider the grammar $G$ in CNF with productions
        \begin{align*}
            S &\to XY \\
            X &\to AB \\
            Y &\to CD
        \end{align*} \[
            A \to a \qquad
            B \to b \qquad
            C \to c \qquad
            D \to d
        \] The parse tree for $abcd$ is
        \begin{center}
            \begin{forest}
                [$S$
                    [$X$
                        [$A$
                            [$a$]
                        ]
                        [$B$
                            [$b$]
                        ]
                    ]
                    [$Y$
                        [$C$
                            [$c$]
                        ]
                        [$D$
                            [$d$]
                        ]
                    ]
                ]
            \end{forest}
        \end{center}
        For the prefix $abc$, the subtree is
        \begin{center}
            \begin{forest}
                [$S$
                    [$X$
                        [$A$
                            [$a$]
                        ]
                        [$B$
                            [$b$]
                        ]
                    ]
                    [$Y$
                        [$C$
                            [$c$]
                        ]
                    ]
                ]
            \end{forest}
        \end{center}
        Then the above algorithm gives the first step as
        $S_\bot \to X Y_\bot$ and asks to keep the tree rooted at $X$
        fixed.
        \begin{center}
            \begin{forest}
                [$S_\bot$
                    [$X$
                        [$A$
                            [$a$]
                        ]
                        [$B$
                            [$b$]
                        ]
                    ]
                    [$Y_\bot$]
                ]
            \end{forest}
        \end{center}
        Then applying this process to derive $c$ from $Y_\bot$ gives
        \begin{center}
            $\vcenter{\hbox{\begin{forest}
                [$S_\bot$
                    [$X$
                        [$A$
                            [$a$]
                        ]
                        [$B$
                            [$b$]
                        ]
                    ]
                    [$Y_\bot$
                        [$C_\bot$]
                    ]
                ]
            \end{forest}}}$
            \quad and then \quad
            $\vcenter{\hbox{\begin{forest}
                [$S_\bot$
                    [$X$
                        [$A$
                            [$a$]
                        ]
                        [$B$
                            [$b$]
                        ]
                    ]
                    [$Y_\bot$
                        [$C_\bot$
                            [$c$]
                        ]
                    ]
                ]
            \end{forest}}}$
        \end{center}
        which is a derivation of $abc$ in $G'$.
    \end{proof}

    Combining the two propositions, we have $L(G') = \pref(L(G))$.
\end{solution}

\end{document}
