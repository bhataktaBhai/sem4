\documentclass[12pt]{article}
\input{~/IISc/sem4/preamble}
\input{~/IISc/sem4/preamble/mvc}
\usepackage{pgfplots}
\pgfplotsset{compat=1.18}
\usepackage{tikz}
\usetikzlibrary{graphs, graphs.standard, quotes, positioning, arrows.meta}

\newcommand{\ind}[1]{\bm{1}_{#1}}
\newcommand{\given}{\mid}
\newcommand{\pms}{\{-1, 1\}}
\DeclareMathOperator*{\argmax}{argmax}
\DeclareMathOperator*{\argmin}{argmin}
\DeclareMathOperator*{\E}{\mathbb E}

\usepackage{amsmath}
\DeclareMathOperator\divv{div}
\DeclareMathOperator\hess{Hess}
% \DeclareMathOperator\Tr{Tr}

\usepackage{pgfplots}
\pgfplotsset{compat=1.18}
\usepackage{tikz}
\usetikzlibrary{graphs, graphs.standard, quotes, positioning, arrows.meta}

\newcommand{\ind}[1]{\bm{1}_{#1}}
\newcommand{\given}{\mid}
\newcommand{\pms}{\{-1, 1\}}
\DeclareMathOperator*{\argmax}{argmax}
\DeclareMathOperator*{\argmin}{argmin}
\DeclareMathOperator*{\E}{\mathbb E}

\usepackage{amsmath}
\DeclareMathOperator\divv{div}
\DeclareMathOperator\hess{Hess}
% \DeclareMathOperator\Tr{Tr}

\usepackage{pgfplots}
\pgfplotsset{compat=1.18}
\usepackage{tikz}
\usetikzlibrary{graphs, graphs.standard, quotes, positioning, arrows.meta}

\newcommand{\ind}[1]{\bm{1}_{#1}}
\newcommand{\given}{\mid}
\newcommand{\pms}{\{-1, 1\}}
\DeclareMathOperator*{\argmax}{argmax}
\DeclareMathOperator*{\argmin}{argmin}
\DeclareMathOperator*{\E}{\mathbb E}

\usepackage{geometry}
\DeclareMathOperator{\pref}{pref}
\tikzset{font=\scriptsize}
\newcommand{\TOTAL}{\textsc{Total}}
% \geometry{a4paper, margin=1in}
\DeclarePairedDelimiterX{\oldmatch}[2]{\langle}{\rangle}%
{#1\;\delimsize\vert\;\mathopen{}#2}
\def\match{\oldmatch*}
\newcommand{\triple}[6]{%
    \begin{matrix}
        #1 & #2 & #3 \\
        #4 & #5 & #6 \\
    \end{matrix}%
}

\title      {Assignment 5}
\setcounter{assignment}{5}
\setcounter   {section}{5}
\author{Naman Mishra\\
(22223)}
\date{13 April, 2024}

\begin{document}
\maketitle
% Problem 1
\begin{problem*}
    Show that the function $square\colon \N \to \N$, given by
    $square(n) = n^2$ is computable by a Turing machine in the sense
    discussed in class.
    Give a complete description of the moves of the TM in a modular way.
\end{problem*}
\begin{solution}
    A Turing machine for this computation is shown in \cref{fig:square}.
    That is, a machine over the input alphabet $\set{0}$ that inputs
    $0^n$ and outputs $0^{n^2}$.

    Transitions that are not shown are assumed to go to the reject state,
    as discussed in class.
    However, these never occur.
    We have ommitted the reject state as it is never reached.

    $-$ in the transitions represents any character.
    The transitions $\neg x/{-}, L$ or $\neg x/{-}, R$ denote that upon
    reading any character $y \ne x$, replace it with $y$ itself and move
    left or right, respectively (as in the TM given in problem 5).

    The machine works as follows:
    \begin{itemize}
        \item Mark the first $0$ as `final' by changing it to $1$.
        \item For each remaining $0$, append two $1$s to the right.
        When this is done, change another one of the $0$s to $1$.
        \item When no more $0$s are left, change all $1$s to $0$s
        and accept.
    \end{itemize}
    This uses the fact that \[
        n^2 = n + \sum_{i=1}^{n-1} 2i. \qedhere
    \]
    \begin{figure}
        \centering
        \begin{tikzpicture}[node distance=2.6cm]
            \node[state, initial] (s) {$s$};
            \node[state, right=of s] (mark) {mark};
            \node[state, right=of mark] (append) {1};
            \node[state, below right=2.6cm and 1.3 cm of append] (append2) {11};
            \node[state, right=of append] (unmark) {unmark};
            \node[state, below=of mark] (rw) {rewind};
            \node[state, right=of rw] (MARK) {MARK};
            \node[state, below=of MARK] (forward) {forward};
            \node[state, left=of forward] (fixup) {fixup};
            \node[state, accepting, left=of fixup] (accept) {$t$};

            \path[->]
            (s) edge[loop above] node {$\leftend / \leftend, R$} (s)
                edge node {$0 / 1, R$} (mark)
                edge node[swap] {$\blank / \blank, L$} (accept)
            (mark) edge node {$0 / \hat{0}, R$} (append)
                   edge node[swap] {$\neg 0 / {-}, L$} (rw)
            (append) edge[loop right] node {$\neg\blank / {-}, R$} (append)
                     edge node[swap] {$\blank / 1, R$} (append2)
            (append2) edge node[swap] {$\blank / 1, L$} (unmark)
            (unmark) edge[loop above] node {$\neg \hat0 / {-}, L$} (unmark)
                     edge[bend right] node[swap] {$\hat{0} / 0, R$} (mark)
            (rw) edge[loop left] node {$\neg 1 / {-}, L$} (rw)
                 edge node[swap] {$1 / 1, R$} (MARK)
            (MARK) edge node[swap] {$0 / 1, R$} (mark)
                   edge node {$\neg 0 / {-}, L$} (forward)
            (forward) edge[loop right] node {$\neg\blank / {-}, R$} (forward)
                      edge node {$\blank / \blank, L$} (fixup)
            (fixup) edge[loop above] node {$\neg\leftend / 0, L$} (fixup)
                    edge node {$\leftend / \leftend, R$} (accept);
        \end{tikzpicture}
        \caption{A Turing machine that computes the square function.}
        \label{fig:square}
    \end{figure}
\end{solution}

\begin{problem*}
    Is the following question decidable:
    Given a Turing machine $M$ and a state $q$ of $M$,
    does $M$ ever enter state $q$ on some input? Justify your answer.
\end{problem*}
\begin{solution}
    No.
    The halting problem can be reduced to this.

    Given any Turing machine $M$ and input $x$, we can construct a new
    Turing machine $P_{M, x}$ that erases its input and runs $M$ on $x$.
    Thus $P_{M, x}$ halts on all inputs iff $M$ halts on $x$.
    Let $t$ be the accept state of $P_{M, x}$.

    This map $M\#x \mapsto P_{M, x}\#t$ is computable, which gives a reduction
    from $\HP$ to $\set{M\#q \mid M \text{ enters state } q}$.

    Thus, if it were decidable whether $P_{M, x}$ enters state $t$ on some
    input, we could decide the halting problem.
\end{solution}

\begin{problem*}
    Let $L, K \subseteq \Sigma^*$.
    Define \[
        L / K = \set{x \mid \exists y \in K, xy \in L}.
    \]
    \begin{enumerate}[label=(\alph*)]
        \item Show that if $L$ is regular and $K$ is \emph{any} language,
            then $L / K$ is regular.
        \item Show that even if we are given a DFA for $L$ and a Turing
            machine for $K$, we cannot always construct an automaton for
            $L / K$.
    \end{enumerate}
\end{problem*}
\begin{solution}
    Let $L$ be regular.
    Its canonical relation $\equiv_L$ has finite index.

    Consider the relation $\equiv_{L / K}$ defined by \[
        x \equiv_{L/K} y \iff (\forall z \in \Sigma^*, xz \in L/K \iff yz \in L/K.)
    \]

    We will show that \[
        x \equiv_L y \implies x \equiv_{L/K} y.
    \] If this is true, then $\equiv_L$ is a refinement of $\equiv_{L/K}$,
    so $\equiv_{L/K}$ has finite index.

    Let $x \equiv_L y$.
    Then for all $z \in \Sigma^*$, $xz \in L \iff yz \in L$.
    Fix a $z \in \Sigma^*$.
    Suppose $xz \in L/K$.
    Then there exists $w \in K$ such that $xzw \in L$.
    But then $yzw \in L$ (by the canonical relation), so $yz \in L/K$.
    Symmetrically, if $yz \in L/K$, then $xz \in L/K$.
    Thus for all $z \in \Sigma^*$, $xz \in L/K \iff yz \in L/K$,
    so that $x \equiv_{L/K} y$.
    This proves the claim.
\end{solution}

\begin{problem*}
    Show that neither the language \[
        \TOTAL = \set{M \mid M \text{ halts on all inputs}}
    \] nor its complement is recursively enumerable.
\end{problem*}
\begin{solution}
    Both proofs are by reducing $\nHP$ to the given language.
    The reduction for $\neg \TOTAL$ is easier,
    so we present that proof first.
    \begin{proof}[$\nHP \le \neg \TOTAL$]
        Given a Turing machine $M$ and input $x$, we can construct a new
        Turing machine $P_{M, x}$ that erases its input and runs $M$ on $x$.
        Thus if $M$ halts on $x$, then $P_{M, x}$ halts on all inputs.
        If M does not halt on $x$, then $P_{M, x}$ does not halt on any input.

        Let $\varphi$ be the map that sends $M\#x$ to $P_{M, x}$.
        Then $\varphi$ is a computable function, and \[
            M\#x \in \nHP \iff \varphi(M\#x) \in \neg \TOTAL.
        \] This proves the claim.
    \end{proof}

    The reduction for $\TOTAL$ is more involved.
    \begin{proof}[$\TOTAL$]
        Given a Turing machine $M$ and input $x$, we can construct a new
        Turing machine $Q_{M, x}$ that does the following:
        \begin{enumerate}
            \item For input $w$, simulate $M$ on $x$ for $|w|$ steps.
            \item If $M$ halts on $x$ within $|w|$ steps, then enter a
                looping state.
            \item If $M$ does not halt on $x$ within $|w|$ steps, then halt.
        \end{enumerate}
        This is easy to achieve by storing the input $w$ on one tape, and
        simulating $M$ on $x$ on another tape, blanking one letter from
        $w$ after each step of the simulation.
        This is a computable function.

        If $M$ halts on $x$ in $n$ steps, then $Q_{M, x}$ enters an
        infinite loop for inputs longer than $n$.
        But if $M$ does not halt on $x$, then $Q_{M, x}$ halts on every input.
        Thus \[
            M\#x \in \nHP \iff Q_{M, x} \in \TOTAL.
        \] This proves the claim.
    \end{proof}
    Since $\nHP$ is not recursively enumerable, neither $\TOTAL$ nor its
    complement is recursively enumerable.
\end{solution}

\begin{problem*}
    Consider the TM $M$ given in the assignment, with input alphabet
    $\set{a, b}$.
    \begin{enumerate}[label=(\alph*)]
        \item Give any string in $Valcomp_{M, baabb}$.
        \item Recall the notion of matching triples of symbols used in class.
        Give the entire set of matching triples for $M$.
        \item Justify the claim that for two valid configurations $c_1$ and
        $c_2$ of $M$, which are of the same length,
        we have $c_1 \xRightarrow{1} c_2$ iff for each position in $c_1$,
        the triple of symbols in $c_1$ and the corresponding triple in $c_2$
        match.
        \item Define an analagous notion of matching pairs of symbols.
        What if we weaken the criterion above to say that at each position
        the pairs of symbols in $c_1$ and $c_2$ ``match''?
    \end{enumerate}
\end{problem*}
\begin{solution} \leavevmode
    \begin{enumerate}[label=(\alph*)]
        \item \small\begin{gather*}
            \begin{matrix}
                \leftend & b & a & a & b & b & \blank \\
                       s & - & - & - & - & - & - \\
            \end{matrix} \;\,\#\;\,
            \begin{matrix}
                \leftend & b & a & a & b & b & \blank \\
                       - & s & - & - & - & - & - \\
            \end{matrix} \;\,\#\;\,
            \begin{matrix}
                \leftend & b & a & a & b & b & \blank \\
                       - & - & p & - & - & - & - \\
            \end{matrix} \;\,\#\\
            \begin{matrix}
                \leftend & b & a & a & b & b & \blank \\
                       - & - & - & p & - & - & - \\
            \end{matrix} \;\,\#\;\,
            \begin{matrix}
                \leftend & b & a & a & b & b & \blank \\
                       - & - & - & - & p & - & - \\
            \end{matrix} \;\,\#\;\,
            \begin{matrix}
                \leftend & b & a & a & b & b & \blank \\
                       - & - & - & - & - & p & - \\
            \end{matrix} \;\,\#\\
            \begin{matrix}
                \leftend & b & a & a & b & b & \blank \\
                       - & - & - & - & - & - & p \\
            \end{matrix} \;\,\#\;\,
            \begin{matrix}
                \leftend & b & a & a & b & b & c \\
                       - & - & - & - & - & q & - \\
            \end{matrix} \;\,\#\;\,
            \begin{matrix}
                \leftend & b & a & a & b & c & c \\
                       - & - & - & - & t & - & - \\
            \end{matrix} \;\,\#
        \end{gather*} \normalsize
        \item We assume the unspecified transitions to be ${-} / {-}, R$ into
        the reject state $r$ (as discussed on MS Teams).
        We split the matching triples by state.
        We let $x$, $y$, $z$ denote any tape symbol, and $\beta$ denote any
        non-blank symbol.
        Also let $B$ denote any symbol that is not $b$.
        We denote a matching triple as
        $\match{\text{triple 1}}{\text{triple 2}}$ for clarity.

        The matching triples for $s$ are \small \begin{align*}
            \match{\triple \leftend x y s--}{\triple \leftend x y -s-}
            && \match{\triple x \leftend y -s-}{\triple x \leftend y --s}
            && \match{\triple x y \leftend --s}{\triple x y \leftend ---} \\
            \match{\triple a x y s--}{\triple a x y ---}
            && \match{\triple x a y -s-}{\triple x a y r--}
            && \match{\triple x y a --s}{\triple x y a -r-} \\
            \match{\triple b x y s--}{\triple b x y -p-}
            && \match{\triple x b y -s-}{\triple x b y --p}
            && \match{\triple x y b --s}{\triple x y b ---} \\
            \match{\triple c x y s--}{\triple c x y -r-}
            && \match{\triple x c y -s-}{\triple x c y --r}
            && \match{\triple x y c --s}{\triple x y c ---} \\
            \match{\triple \blank x y s--}{\triple \blank x y -r-}
            && \match{\triple x \blank y -s-}{\triple x \blank y --r}
            && \match{\triple x y \blank --s}{\triple x y \blank ---}
        \end{align*} \normalsize
        For $p$, the matching triples are \small \begin{align*}
            \match{\triple \blank x y p--}{\triple c x y ---}
            && \match{\triple x \blank y -p-}{\triple x c y q--}
            && \match{\triple x y \blank --p}{\triple x y c -q-} \\
            \match{\triple \beta x y p--}{\triple \beta x y -p-}
            && \match{\triple x \beta y -p-}{\triple x \beta y --p}
            && \match{\triple x y \beta --p}{\triple x y \beta ---}
        \end{align*} \normalsize
        For $q$, they are \small \begin{align*}
            \match{\triple b x y q--}{\triple c x y ---}
            && \match{\triple x b y -q-}{\triple x c y t--}
            && \match{\triple x y b --q}{\triple x y c -t-} \\
            \match{\triple B x y q--}{\triple B x y -r-}
            && \match{\triple x B y -q-}{\triple x B y --r}
            && \match{\triple x y B --q}{\triple x y B ---}
        \end{align*} \normalsize
        For $r$, they are \small \begin{align*}
            \match{\triple x y z r--}{\triple x y z -r-}
            && \match{\triple x y z -r-}{\triple x y z --r}
            && \match{\triple x y z --r}{\triple x y z ---}
        \end{align*} \normalsize
        The same triples apply for the accept state $t$, with $r$ replaced by $t$.
        \small \begin{align*}
            \match{\triple x y z t--}{\triple x y z -t-}
            && \match{\triple x y z -t-}{\triple x y z --t}
            && \match{\triple x y z --t}{\triple x y z ---}
        \end{align*}
        In addition to these, we have the matching triples \small \begin{align*}
            \match{\triple x y z ---}{\triple x y z \theta--}
            && \match{\triple x y z ---}{\triple x y z --\theta}
        \end{align*} \normalsize for every state $\theta$, since the read
        head could be position right outside the triple and move in.
        Lastly, we have the matching triples \small \begin{align*}
            \match{\triple x y z ---}{\triple x y z ---}
        \end{align*} \normalsize where the read head is not in the triple
        either before or after the transition.
        \item The construction of the triples makes it clear that if
        $c_1 \xRightarrow{1} c_2$, then each triple in $c_1$ matches the
        corresponding triple in $c_2$.

        The converse is also true.
        Since $c_1$ and $c_2$ are valid configurations, they have exactly
        one state symbol in the bottom row.
        The triples matching ensures that the state symbol moves at most
        one position, either left or right.
        Thus there are at most $6$ triples which contain any state symbol
        (before or after the transition).
        The matching triples ensure that these have arisen from valid
        transitions.

        For the other triples, the matching ensures that no tape symbol
        has changed, since the only matching triple with no state symbols
        is \[
            \match{\triple x y z ---}{\triple x y z ---}.
        \]

        Notice that having some length greater that $1$ is essential.
        Consider the transition \[
            \begin{pmatrix}
                u & v & w & x & y & z \\
                - & - & q & - & - & -
            \end{pmatrix} \xRightarrow{1} \begin{pmatrix}
                u & v & w' & x & y & z \\
                - & - & -  & q & - & -
            \end{pmatrix}.
        \] The match \[
            \match{\triple u v w --q}{\triple u v w' ---}
        \] ensures that the letter change and move are consistent.
        But what forces the matching triple \[
            \match{\triple x y z ---}{\triple x y z q--}?
        \] The state could `vanish' in the false transition \[
            \begin{pmatrix}
                u & v & w & x & y & z \\
                - & - & q & - & - & -
            \end{pmatrix} \xRightarrow{1} \begin{pmatrix}
                u & v & w' & x & y & z \\
                - & - & -  & - & - & -
            \end{pmatrix}
        \] and both the initial and final triples would match.
        But the triples in between reject this possibility, since
        \[
            \match{\triple v w x -q-}{\triple v w' x ---}
        \] is never a matching triple.
        The length provides immediate context to each position.
        This is more clear from the next part.
        \item The desirable properties still hold.
        If $c_1$ and $c_2$ are valid configurations of the same length,
        then $c_1 \xRightarrow{1} c_2$ iff for each position in $c_1$,
        the pair of symbols in $c_1$ and the corresponding pair in $c_2$
        match.

        This is because for any transition which changes a portion of the
        tape as \[
            \begin{pmatrix}
                u & v & w & x & y \\
                - & - & q & - & -
            \end{pmatrix} \xRightarrow{1} \begin{pmatrix}
                u & v & w' & x & y \\
                - & - & -  & q & -
            \end{pmatrix},
        \] the pair at $w$ ensures that the head has moved right by
        following the correct transition.
        The pair at $v$ ensures that the head has \emph{not} moved left.
        \qedhere
    \end{enumerate}
\end{solution}

\begin{problem*}
    Show that it is undecidable whether the intersection of two given CFLs
    is a CFL.
\end{problem*}
\begin{solution}
    We will reduce the halting problem to this problem.

    Let $M$ be a Turing machine and $x$ be an input.
    Define two PDAs $M_1$ and $M_2$ as in class:
    Given a string $c_0 \# c_1 \# \dots \# c_n \#$ of (reversed)
    configurations,
    \begin{itemize}
        \item $M_1$ checks that the even-numbered configurations are correctly
        followed by their successors.
        \item $M_2$ checks that the odd-numbered configurations are correctly
        followed by their successors.
    \end{itemize}
    Then $L(M_1) \cap L(M_2)$ is the set of valid computations of $M$ on $x$.
    This is precisely $Valcomp_{M, x}$.
    If $x$ does not halt on $M$, this is empty, and hence a CFL.
    If $x$ halts on $M$, this is not a CFL.

    Thus the map $M\#x \mapsto (M_1, M_2)$ is a reduction from the halting
    problem to the problem of determining whether the intersection of two
    given CFLs is a CFL.
\end{solution}

\end{document}
