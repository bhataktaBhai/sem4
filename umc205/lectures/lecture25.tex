\chapter{G\"odel's Incompleteness Theorem} \label{chp:godel}
\lecture{2024-04-09}{}

\begin{theorem*}[Godel (1931)] \label{thm:godel1}
    There is no sound and complete proof system for arithmetc, \ie,
    first-order logic of natural numbers with addition and multiplication
    $(\N, +, \cdot)$.
\end{theorem*}

\begin{definition}[Validity] \label{def:validity}
    A formula $\varphi$ is \emph{valid} if it is true in every model.
\end{definition}
\begin{example}
    $\forall x (x = x)$ is valid.
    $(p \rightarrow q) \rightarrow (\neg q \rightarrow \neg p)$ is valid.
\end{example}

\section{Arithmetic} \label{sec:arithmetic}
The first order logic of $(\N, +, \cdot)$ is defined by
\begin{itemize}
    \item Domain $\N = \set{0, 1, 2, \dots}$
    \item Terms $0$, $1$, $0 + 1$, $1 \cdot x$, $x + y$, $x \cdot y$, etc
    \item Atomic formulas $t_1 = t_2$
    \item Formulas derived as
    \begin{itemize}
        \item Atomic formulas,
        \item Quantification $\forall x \varphi$, $\exists x \varphi$,
        where $\varphi$ is a formula,
        \item Connectives $\neg$, $\land$, $\lor$, etc.
    \end{itemize}
\end{itemize}

\noindent We can build up more complex formulas using these building blocks.
\begin{itemize}
    \item ``$x$ gives a quotient $q$ and remainder $r$ when divided by $y$''
    can be expressed as \[
        \mathit{intdiv}(x, y, q, r)
            \coloneq x = (q \cdot y) + r \land r < y.
    \]
    \item ``$y$ divides $x$'' can be expressed as \[
        \mathit{div}(x, y)
            \coloneq \exists q\; \mathit{intdiv}(x, y, q, 0)
    \] or more directly as \[
        \mathit{div}(x, y)
            \coloneq \exists q (x = q \cdot y).
    \]
    \item $x$ is prime. \[
        \mathit{prime}(x)
            \coloneq \neg (x = 1) \land
            \forall y (\mathit{div}(y, x) \rightarrow y = 1 \lor y = x).
    \]
    \item $x$ is a power of $2$. \[
        \mathit{power_2}(x)
            \coloneq \forall y (\mathit{prime}(y) \land \mathit{div}(y, x) \rightarrow y = 2).
    \]
    \item Every number has a successor. \[
        \forall x \exists y (y = x + 1).
    \]
    \item Every number has a predecessor. \[
        \forall x \exists y (y + 1 = x).
    \]
    \item There are only finitely many primes. \[
        \exists x \forall y (\mathit{prime}(y) \rightarrow y \leq x).
    \]
    \item There are infinitely many primes. \[
        \forall x \exists y (y > x \land \mathit{prime}(y)).
    \]
\end{itemize}

\noindent Let $\Th(\N, +, \cdot)$ be the set of sentences of
$\FO(\N, +, \cdot)$ that are true.
\section{Peano's Proof System for Arithmetic} \label{sec:peano}
Peano introduced the axioms
\begin{itemize}
    \item $\forall x \neg (x + 1 = 0)$
    \item $\forall x \forall y (x + 1 = y + 1 \rightarrow x = y)$
    \item $\forall x (x + 0 = x)$
    \item $\forall x \forall y \forall z (x + (y + 1) = (x + y) + 1)$
    \item $\forall x (x \cdot 0 = 0)$
    \item $\forall x \forall y \forall z (x \cdot (y + 1) = (x \cdot y) + x)$.
    \item Axiom schema of induction:
    \[
        \varphi(0) \land
        \forall x \big(\varphi(x) \rightarrow \varphi(x + 1)\big)
            \rightarrow \forall x \varphi(x).
    \]
\end{itemize}
Along with the inference rules from first-order logic, such as modus ponens,
universal instantiation, etc.,
this gives a proof system for arithmetic.

\begin{definition}[Proofs] \label{def:proofs}
    A \emph{proof} of a sentence $\varphi$ is a finite sequence of
    sentences $\varphi_1, \varphi_2, \dots, \varphi_n$ such that
    each $\varphi_i$ is either an axiom or follows from previous
    sentences by an inference rule, and $\varphi_n = \varphi$.
\end{definition}

\begin{definition*}[Soundness and Completeness] \label{def:peano:sound}
    A proof system is \emph{sound} if every provable sentence is true,
    and \emph{complete} if every true sentence is provable.
\end{definition*}

\begin{lemma}
    If $\Th(\N, +, \cdot)$ is not recursively enumerable,
    then there is no sound and complete proof system for arithmetic.
\end{lemma}
\begin{proof}
    Suppose there is a sound and complete proof system for arithmetic.
    Then we can decide whether a sentence is true by enumerating all
    proofs.
\end{proof}

Thus we need to show that $\Th(\N, +, \cdot)$ is not recursively enumerable,
which of course we will do by reduction to the negative halting problem.

Let $\Delta$ be the alphabet defined for valcomp.
Let $p > \abs{\Delta}$ be a prime number.
We can view each string $w \in \Delta^*$ as a number in base $p$.
We will not assign the digit $0$ to any letter.

Define valid computations as \begin{align*}
    \mathit{valcomp}_{M, x}(v)
    &\coloneq \exists c \exists d \big(
        \mathit{power}_p(c) \land \mathit{power}_p(d) \\
        &\qquad\qquad\land \mathit{length}(v, d) \\
        &\qquad\qquad\land \mathit{start}(v, c) \\
        &\qquad\qquad\land \mathit{move}(v, c, d) \land \mathit{halt}(v, d)
    \big).
\end{align*} Define $\varphi_{M, x}$ as \[
    \varphi_{M, x} \coloneq \exists v\; \mathit{valcomp}_{M, x}(v).
\]
