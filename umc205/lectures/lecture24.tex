\lecture{2024-03-28}{}
\begin{theorem*}[Rice 1956] \label{thm:rice_1956}
    Any \emph{non-monotone} property of recuresive enumerable languages
    is not even recursively enumerable.
    That is, if $P$ is a non-monotone propertty of r.e. languages,
    then $L_P$ is not even r.e.
\end{theorem*}
\begin{remark}
    Languages that are not recursively enumerable are sometimes called
    \emph{highly undecidable}.
\end{remark}
\begin{proof}
    We wish to reduce $\neg\HP$ to $L_P$.
    That is, $\neg\HP \le L_P$.

    That is, we want a computable function $M\#x \xmapsto{\sigma} Q_{M, x}$
    such that $Q_{M, x}$ is in $L_P$ iff $M$ does not halt on $x$.

    We will define $Q_{M, x}$ as follows:
    \begin{itemize}
        \item Since $P$ is non-monotone, there exist r.e. languages
        $A \subseteq B$ such that $A \in P$ but $B \notin P$.
        \item Take three tapes.
        Write the input on the first two tapes.
        Write $x$ on the third tape.
        \item Run a Turing machine for $A$ on the first tape,
        a TM for $B$ on the second tape,
        and $M$ on the third tape,
        all in parallel.
        \item Accept if the first tape accepts, \ie, if $x \in A$.
        \item Also accept if $M$ halts on $x$, and the second tape accepts.
        That is, if $M\#x \notin \neg\HP$ and $x \in B$.
    \end{itemize}
    Then the language accepted by $Q_{M, x}$ is \[
        L(Q_{M, x}) = \begin{cases}
            A & \text{if $M$ does not halt on $x$,} \\
            B & \text{if $M$ halts on $x$.}
        \end{cases}
    \]
\end{proof}

\section{Decidability Problems Regarding CFLs} \label{sec:cfl:decidability}
\begin{exercise}
    Is it decidable whether a given CFG accepts a non-empty language?
\end{exercise}
\begin{solution}
    Yes.
    One way is to use the bound for the height of minimal parse trees
    (\cref{TODO}).
    Check all trees up to that height.

    Another way is to check whether there is a way to reach a terminal
    string from each non-terminal.
    First, mark all terminals.
    Then, iteratively mark each non-terminal $X$ such that $X \to \gamma$
    is a rule with each symbol in $\gamma$ marked.
    If the start symbol is marked at the end,
    the language is non-empty.
\end{solution}

\begin{exercise}
    Is it decidable whether a given CFG accepts a finite language?
\end{exercise}
\begin{solution}
    Yes.
    Check all trees up to height $n + 2n$. % TODO
\end{solution}

\begin{exercise}
    Is it decidable whether a given CFG $G$ is universal?
    That is, is $L(G) = A^*$?
\end{exercise}
\begin{solution}
    No!

    We will reduce $\neg\HP$ to this problem.
    Highly undecidable!

    Let $M = (Q, A, \Gamma, s, \delta, \leftend, \blank, t, r)$.
    Let $x = a_1 a_2 \dots a_n$ be an input to $M$.

    We can represent a configuration of $M$ as follows:
    \begin{center}
        \begin{tabular}{cccccc}
            $\leftend$ & $b_1$ & $b_2$ & $b_3$ & $\dots$ & $b_n$ \\
            $-$ & $-$ & $-$ & $q$ & \dots & $-$
        \end{tabular}
    \end{center}
    Thus we define the double-decker alphabet \[
        \Delta = \Gamma \times \set{Q \cup \set{-}}
                \cup \set{\#}
    \]

    Define a \emph{valid computation} of $M$ on $x$ as a string \[
        c_0 \# c_1 \# \dots \# c_N \#
    \] where each $c_i$ is a valid configuration,
    $c_0$ is the initial configuration,
    $c_N$ is a \emph{halting} configuration,
    and $c_i \xRightarrow{1} c_{i+1}$ for all $i$.
    Further, all $c_i$ are of the same length, and minimal.
    That is, the length is either the length of the input,
    or the last cell that is ever read
    (basically every cell that matters).
\end{solution}
