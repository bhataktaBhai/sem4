\chapter{Turing Machines} \label{chp:tm}
\lecture{2024-03-07}{}

\section{A Brief History of Logic \& Computability} \label{sec:history}

In 1928, David Hilbert posed the Entscheidungsproblem -- deciding validity
of first-order logic formulas.

Kurt G\"odel showed
\begin{itemize}
    \item in 1929, the completeness of first-order logic;
    \item in 1931, the incompleteness of first-order arithmetic;
    \item in 1931, primitive recursive functions. % TODO: what?
\end{itemize}

In 1936, Alan Turing proved the unsolvability of the Entscheidungsproblem
using Turing machines.
In the same year, Alonzo Church proved the same result using $\lambda$-calculus.

\section{Introduction} \label{sec:tm:intro}
A Turing machine reads a tape that is infinite to the right.
The tape is divided into cells,
each of which contains a symbol from a finite alphabet $A$.

In each step:
\begin{itemize}
    \item In current state $p$, read current symbol under the tape head,
    say $a$.
    \item Change state to $q$.
    \item Replace $a$ with $b$.
    \item Move the tape head left or right. \[
        \delta(p, a) = (q, b, L/R)
    \]
\end{itemize}

The machine has special designated \emph{accept} state $t$
and \emph{reject} state $r$.
These states are assumed to be ``sink'' states.
The machine accepts if it enters the accept state,
and rejects if it enters the reject state.
The machine never ``falls off'' the left end of the tape.
That is, on reading `$\leftend$', it always moves right.

\section{Turing Machines Formally} \label{sec:tm:formal}
\begin{definition*}[Turing machine] \label{def:tm}
    A \emph{Turing machine} is a tuple \[
        \mcM = (Q, A, \Gamma, s, \delta, \leftend, \blank, t, r)
    \] where
    \begin{itemize}
        \item $Q$ is a finite set of states;
        \item $A$ is the finite input alphabet;
        \item $\Gamma \supseteq A$ is the finite tape alphabet;
        \item $s \in Q$ is the start state;
        \item $\delta: Q \times \Gamma \to Q \times \Gamma \times \set{L, R}$
        is the (deterministic) transition function;
        \item $\leftend \in \Gamma \setminus A$ is the left end marker;
        \item $\blank \in \Gamma \setminus A$ is the blank symbol;
        \item $t \in Q$ is the accept state; and
        \item $r \in Q \setminus \set{t}$ is the reject state.
    \end{itemize}
    $\delta$ must follow the following constraints:
    \begin{itemize}
        \item For any state $p$,
        $\delta(p, \leftend) = (q, \leftend, R)$ for some $q \in Q$;
        \item For any $a \in \Gamma$, $\delta(t, a) = (t, b, \sigma)$ for 
        some $b$ and $\sigma$;
        \item For any $a \in \Gamma$, $\delta(r, a) = (r, b, \sigma)$ for
        some $b$ and $\sigma$.
    \end{itemize}
\end{definition*}

\begin{definition}[Configuration] \label{def:tm:configuration}
    A \emph{configuration} of a Turing machine \mcM\ as above is a triple \[
        (p, y\blank^\omega, n) \in Q \times \Gamma^\omega \times \N
    \] which specifies that the machine is in state $p$,
    with ``non-blank'' tape contents $y$ (which cannot end with $\blank$),
    and the read head positioned at the $n$th cell.
\end{definition}
\begin{remark}
    The initial configuration of \mcM\ on input $w \in A^*$ is
    $(s, \leftend w \blank^\omega, 0)$.
\end{remark}

\begin{definition}[Acceptance] \label{def:tm:acceptance}
    We define the $1$-step transition of \mcM\ as follows:
    If $\delta(p, a) = (q, b, \sigma)$, and $z_n = a$, then \[
        (p, z, n) \xRightarrow{1} (q, s_b^n(z), n+\sigma).
    \] where we identify $L$ with $-1$ and $R$ with $+1$.

    We define $\xRightarrow{n}$ and $\xRightarrow{*}$ recursively as
    before.

    We say that \mcM\ \emph{accepts} $w \in A^*$ if \[
        (s, \leftend w \blank^\omega, 0) \xRightarrow{*} (t, z, i)
    \] for some $z \in \Gamma^*$, $i \in \N$.

    We say that \mcM\ \emph{rejects} $w \in A^*$ if \[
        (s, \leftend w \blank^\omega, 0) \xRightarrow{*} (r, z, i)
    \] for some $z$ and $i$.
\end{definition}

\begin{definition}[Halting] \label{def:tm:halting}
    A Turing machine \mcM\ is said to \emph{halt} on an input $w$ if
    \mcM\ either accepts or rejects $w$.
    Otherwise, we say that \mcM\ \emph{loops} on $w$.

    The \emph{language accepted by \mcM} is \[
        L(\mcM) = \set{w \in A^* \mid \mcM \text{ accepts } w}.
    \]

    A language $L \subseteq A^*$ is said to be \emph{recursively enumerable}
    if it is the language accepted by some Turing machine.

    It is said to be \emph{recursive} if it is the language accepted by some
    Turing machine that halts on all inputs.
\end{definition}
Why do we use the terms ``recursively enumerable'' and ``recursive''?
\begin{proposition}[Recursively enumerable]
\label{thm:tm:recursively_enumerable}
    Let \mcM\ be a Turing machine.
    Then there exists a procedure to list all strings in $L(\mcM)$
    (possibly with repetitions).
\end{proposition}
\begin{proof}
    
\end{proof}
