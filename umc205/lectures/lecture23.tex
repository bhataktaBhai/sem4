\section{Reductions} \label{sec:reductions}
\lecture{2024-03-26}{Reductions and Rice's Theorems}
\begin{definition}[Computable map] \label{def:reductions:computable_map}
    A map $\sigma\colon A^* \to B^*$ is \emph{computable} if there exists a
    Turing machine \mcM{} that, for every input $w \in A^*$,
    halts with output $\sigma(w)$.
\end{definition}
\begin{definition*}[Reduction] \label{def:reduction}
    Let $L \subseteq A^*$ and $M \subseteq B^*$ be languages.
    We say that $L$ \emph{reduces} to $M$, denoted $L \le M$, if there
    exists a computable map $\sigma\colon A^* \to B^*$ such that \[
        w \in L \iff \sigma(w) \in M.
    \]
\end{definition*}
\begin{examples}
    \item Let $L = 2\N$ and $L' = \set{x\#y\#m \mid x \equiv y \pmod m}$.
    Then $L \le L'$ via the computable map $n \mapsto n\#2\#0$.

    Is $L' \le L$? Yes!
    $L'$ is computable, so we can simply take the computable map \[
        w \mapsto \begin{cases}
            0 & \text{if } w \in L', \\
            1 & \text{otherwise}.
        \end{cases}
    \]
    \item Let $L = \set{M \mid M \text{ accepts } \epsilon}$.
    Then $\HP \le L$ by \cref{thm:decidability:epsilon}, where the map is
    $\enc(M)\#x \mapsto \enc(P_{M,x})$. \\
    $L \le \HP$ by the map $\enc(M) \mapsto \enc(M)\#$.
\end{examples}

\begin{theorem*} \label{thm:reductions:recursion}
    Let $L \le M$.
    \begin{enumerate}
        \item If $M$ is recursively enumerable, then so is $L$.
        \item If $M$ is recursive, then so is $L$.
    \end{enumerate}
\end{theorem*}
\begin{proof}
    For each case, let $\mcM$ be a Turing machine that accepts/deicdes $M$.

    Let $\sigma\colon A^* \to B^*$ be the computable map such that $w \in L
    \iff \sigma(w) \in M$, computed by a Turing machine $\Sigma$.

    Define the Turing machine $\mcL$ that first simulates $\Sigma$
    on input $w$, and then simulates $\mcM$ on $\sigma(w)$.
\end{proof}

\begin{exercise}
    Show that the language \[
        L = \set{M \mid M \text{ accepts a regular language}}
    \] is not recursively enumerable.
\end{exercise}
\begin{solution}
    We will show that $\neg\HP \le L$.
    Use the computable map $M\#x \mapsto Q_{M,x}$, where $Q_{M,x}$ does the
    following:
    \begin{itemize}
        \item First simulate $M$ on $x$.
        \item If $M$ halts, check if the input is of the form $a^n b^n$.
    \end{itemize}
    Then \[
        L(Q_{M,x}) = \begin{cases}
            \O & \text{if } M \text{ does not halt on } x, \\
            \set{a^n b^n} & \text{if } M \text{ halts on } x.
        \end{cases}
    \]
    Then $M\#x \in \neg\HP \iff Q_{M,x} \in L$ so $\neg\HP \le L$.
    Since $\neg\HP$ is not recursively enumerable, neither is $L$.

    This is heavily inspired by \cref{thm:decidability:regular}.
\end{solution}

\begin{definition}[Properties] \label{def:reductions:prop}
    A \emph{property} $P$ of languages over an alphabet $A$ is a subset
    of languages over $A$.

    A property $P$ is \emph{trivial} if $P = \O$ or $P = \set{A^*}$.
\end{definition}
\begin{examples}
    \item ``is empty'' is non-trivial.
    \item ``contains $\epsilon$'' is non-trivial.
    \item ``is accepted by a TM'' is non-trivial.
\end{examples}

A non-trivial property of r.e. languages is any property $P$ such that
$P$ contains at least one r.e. language and at least one non-r.e. language.

From now on, ``trivial'' will mean ``trivial with respect to recursively
enumerable languages''.
\begin{examples}
    \item ``is empty'' is non-trivial.
    \item ``contains $\epsilon$'' is non-trivial.
    \item ``is accepted by a TM'' is trivial, since each recursively
    enumerable language is by definition accepted by a TM.
    So $P = \RE$.
\end{examples}

\begin{definition}[Monotonicity] \label{def:monotonicity}
    A property $P$ is \emph{monotone} (with respect to $\RE$) if for all
    languages $L, L'$, \[
        L \subseteq L' \implies L \in P \implies L' \in P.
    \]
\end{definition}
\begin{examples}
    \item ``is infinite'' is monotone.
    \item ``is finite'' is not monotone.
\end{examples}

\begin{definition}
    For a property $P$, we define \[
        L_P = \set{\enc(M) \mid L(M) \in P}.
    \]
\end{definition}
\begin{theorem*}[Rice 1953] \label{thm:rice}
    Any non-trivial property of recursively enumerable languages is
    undecidable.
    That is, if $P$ is a non-trivial property of r.e. languages, then
    $L_P$ is undecidable.
\end{theorem*}
