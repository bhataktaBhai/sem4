\lecture{04}{Thu 11 Jan '24}{}

\subsection{Two Necessary Conditions for Regular Languages} \label{sec:pumping_periodic}
In a given DFA $\mcA$ with $n$ states,
any path of length greater than $n-1$ must contain a cycle.
Let $u$ be the string of symbols on the path from the start state
to the beginning of the cycle,
let $v$ be the (non-empty) string of symbols on the cycle,
and let $w$ be the string of symbols on the path from the end of the cycle
to the final state.

Then if $uvw$ is accepted by $\mcA$, then so is $uv^kw$ for any
$k \geq 0$.

\begin{theorem*}[Pumping Lemma]
    For any regular language $L$, there exists a constant $k$, such that
    for any word $t \in L$ of the form $xyz$ with $\abs{y} \ge k$, there
    exist strings $u$, $v$ and $w$ such that
    \begin{enumerate}
        \item $y = uvw$, $v \ne \epsilon$, and
        \item $xu v^i wz \in L$ for each $i \ge 0$.
    \end{enumerate}
\end{theorem*}
\begin{proof}
    Let $L$ be accepted by a DFA $\mcA = (Q, s, \delta, F)$ with $n$ states.
    Let $t$ be a word in $L$ of the form $xyz$ with $\abs{y} \ge n$.
    Let $y = a_1 a_2 \dots a_m$.
    Let $q_0 = \what{\delta}(s, x)$ and $q_i = \what{\delta}(q_{i-1}, a_n)$
    for $1 \le i \le m$.
    By the pigeonhole principle, there exist $i < j$ such that $q_i = q_j$.
    Then letting $u = a_1 \dots a_i$, $v = a_{i+1} \dots a_j$, and
    $w = a_{j+1} \dots a_m$, we have that $y = uvw$, $v \ne \epsilon$,
    $\what{\delta}(s, xu) = \what{\delta}(s, xuv) = q_i$.
    Since \[
        \what{\delta}(q, xy) = \what{\delta}(\what{\delta}(q, x), y),
    \] (which we will prove in assignment 1) we have that \begin{align*}
        \what{\delta}(q_i, v) &= q_i
        \intertext{and so by induction}
        \what{\delta}(q_i, v^k) &= q_i
        \intertext{which gives}
        \what{\delta}(s, xuv^kwz)
            &= \what{\delta}\bigg(\what{\delta}\Big(\what{\delta}(s,xu), v^k\Big), wz\bigg) \\
            &= \what{\delta}\bigg(\what{\delta}(q_i, v^k), wz\bigg) \\
            &= \what{\delta}(q_i, wz) \\
            &= \what{\delta}\Big(\what{\delta}(s, xuv), wz\Big) \\
            &= \what{\delta}(s, t) \in F. \qedhere
    \end{align*}
\end{proof}
\begin{proposition}
    The language $\set{a^n b^n \mid n \ge 0}$ is not regular.
\end{proposition}
\begin{proof}
    Let $k \in \N$.
    Choose $t = a^k b^k = xyz$ where $x = \epsilon$, $y = a^k$, and $z = b^k$.
    Let $y = uvw$ for some non-empty $v$.
    Then $v = a^j$ for some $j \ge 1$.
    Then $x u v^2 w z = a^{k + j} b^k$, which is not in the language.
    Therefore, the language is not regular.
\end{proof}

\begin{exercise}
    Show that $\set{a^{2^n} \mid n \ge 0}$ is not a regular language.
\end{exercise}
\begin{solution}
    Let $k \in \N$.
    Choose $t = a^{2^k} = xyz$ where $x = \epsilon$, $y = a^{2^k} - 1$, and
    $z = a$.
    Let $y = uvw$ for some non-empty $v$.
    Then $v = a^j$ for some $1 \le j < 2^k$.
    Then $x u v^2 w z = a^{2^k + j}$, which is not in the language
    since $2^k < 2^k + j < 2^{k + 1}$.
\end{solution}

\begin{exercise}
    Is the language $\set{w \cdot w \mid w \in \set{0, 1}^*}$ regular?
\end{exercise}
\begin{solution}
    Let $k \in \N$.
    Choose $t = 0^k 1^k 0^k 1^k = xyz$ where $x = 0^k$, $y = 1^k$, and
    $z = 0^k 1^k$.
    Let $y = uvw$ for some non-empty $v$.
    Then $v = 1^j$ for some $1 \le j \le k$.
    If $j$ is odd, we are done.
    Otherwise, $x u v^2 w z = 0^k 1^{k + m} 1^m 0^k 1^k$, where $j = 2m$.
    This is not in the language since the second half starts with a $1$.
\end{solution}
