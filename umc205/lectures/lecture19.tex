\section{Computability} \label{sec:computability}
\lecture{19}{Tue 12 Mar '24}{}
\begin{definition*}[Computability] \label{def:computability}
    A function $f\colon \N \to \N$ is said to be \emph{computable} if there
    exists a Turing machine $M$ such that for all $n \in \N$,
    $M$ given input $\leftend 0^n$ halts with output $\leftend 0^{f(n)}$.
\end{definition*}

We can view a function on the naturals as a language in the following
manner.
\begin{definition}[Language of a function] \label{def:language}
    Let $f\colon \N \to \N$ be a function. The \emph{language of $f$} is
    \[
        L_f = \set{0^n \# 0^{f(n)} \mid n \in \N}.
    \]
\end{definition}

\begin{theorem*} \label{thm:computable}
    A function $f\colon \N \to \N$ is computable if and only if its language
    $L_f$ is recursive.
\end{theorem*}
\begin{proof}
    Suppose $f$ is computable.
    Let $M_f$ be the Turing machine that computes $f$.

    We define an $M$ that decides $L_f$ as follows.
    First, reject all strings not of the form $0^i \# 0^j$.
    Then, simulate $M_f$ on input $0^i$ (push the hash and $0^j$ forward
    as needed).
    Finally, check that the tape is in the form $0^j \# 0^j$ and accept if
    so.

    For the converse, suppose $L_f$ is recursive, with Turing machine $M$.
    We define a Turing machine $M_f$ that computes $f$ as follows.
    We append a \# to the input, then simulate $M$ on the input.
    If $M$ rejects, we append a $0$ and repeat.
    When $M$ finally accepts (which it must, since $M$ halts on all inputs),
    we output the part of the tape after the \#.
\end{proof}

\section{Equivalent Models} \label{sec:tm:equiv}
\subsection{Multiple Tapes} \label{sec:tm:equiv:multiple_tapes}
Consider a Turing machine with $k$ tapes.
That is, the transition function is of the form
\[
    \delta\colon Q \times \Gamma^k \to Q \times \Gamma^k \times \set{L, R}^k.
\]
\begin{theorem*} \label{thm:tm:k_tape}
    A $k$-tape Turing machine is equivalent to a single-tape Turing machine.
\end{theorem*}
\begin{proof}
    We give a recipe to simulate a $k$-tape Turing machine with a
    single-tape Turing machine.
    Let the $k$-tape machine be \[
        M = (Q, A, \Gamma, s, \delta, \vdash, \blank, t, r).
    \]

    Define a new alphabet $\Gamma' = (\Gamma)^k \set{\vDash}$.
    Define new states 
\end{proof}

Consider a Turing machine with a two-way infinite tape.
That is, the tape extends infinitely to the left and right.
There is no left-end marker, and the input is given as
$\blank^\omega x \blank^\omega$, with the read head initially on the
first letter of $x$.
\begin{theorem*} \label{thm:tm:two_way}
    A two-way infinite tape Turing machine is equivalent to a one-way
    infinite tape Turing machine.
\end{theorem*}
\begin{proof}
    We can simulate a two-way tape using two one-way tapes.
    In each configuration we also store whether the read head is on the
    upper or lower tape.
\end{proof}

Consider a Turing machine with a two-dimensional tape.
That is, the tape is a grid of cells, and the read head can move in
four directions.

\begin{theorem*} \label{thm:tm:2d}
    A two-dimensional tape Turing machine is equivalent to a one-dimensional
    tape Turing machine.
\end{theorem*}
\begin{proof}
    We can simulate a two-dimensional tape using two one-dimensional tapes.
    The first tape stores the cells in a diagonally ennumerated pattern,
    and the second stores the index of the current cell.
    The mapping from the current index to any of its neighbours is
    computable, so we can gather the number of cells to move the read head
    using another Turing machine.
\end{proof}

\section{Non-deterministic Turing Machine} \label{sec:tm:ntm}
A non-deterministic Turing machine is a Turing machine with several
possible transitions from each state.
A word is accepted if there exists a sequence of transitions that leads
to an accepting state.
\begin{theorem*} \label{thm:tm:ntm}
    A non-deterministic Turing machine is equivalent to a deterministic
    Turing machine.
\end{theorem*}
