\lecture{14}{Thu 15 Feb '24}{}
\section{Pumping Lemma} \label{sec:cfl:pumping}
\begin{theorem}[Pumping lemma for CFLs] \label{thm:cfl:pumping}
    For every CFL $L$ there is a constant $k \ge 0$ such that such that for
    any word $z$ in $L$ of length at least $k$,
    there are strings $u$, $v$, $w$, $x$, $y$ such that
    \begin{itemize}
        \item $z = uvwxy$,
        \item $vx \ne \epsilon$,
        \item $|vwx| \le k$, and
        \item for each $i \ge 0$, the string $uv^iwx^iy$ belongs to $L$.
    \end{itemize}
\end{theorem}
Consider a parse tree for any string.
Note that subtrees hanging at the same non-terminal can be replaced by
each other.
For example, if
\begin{center}
    \begin{forest}
        [S
            [a]
            [S
                [a]
                [S
                    [a]
                    [S
                        [$\epsilon$]
                    ]
                    [b]
                ]
                [b]
            ]
            [b]
        ]
    \end{forest}
\end{center} is a derivation, then so are
\begin{center}
    \begin{forest}
        [S
            [a]
            [S
                [$\epsilon$]
            ]
            [b]
        ]
    \end{forest}
\end{center}
and
\begin{center}
    \begin{forest}
        [S
            [a]
            [S
                [a]
                [S
                    [a]
                    [S
                        [a]
                        [S
                            [a]
                            [S
                                [$\epsilon$]
                            ]
                            [b]
                        ]
                        [b]
                    ]
                    [b]
                ]
                [b]
            ]
            [b]
        ]
    \end{forest}
\end{center}
\begin{proof}
    Let $G = (N, A, S, P)$ be a CNF grammar for $L$.
    A full binary tree of height $h$ has $2^h$ leaves.
    A parse tree in $G$ with height $h$ has a terminal string of length at
    most $2^{h-1}$, since a terminal node has no sibling.
    Thus a string of length $2^n$ or more, must have a parse tree of at
    least height $n+1$.
    Let $k = 2^{\abs{N}}$.
    Consider a parse tree in $G$ of a string $z$ of length at least $k$.
    The longest path from the root to a leaf has length at least
    $\abs{N}+1$, and thus has at least $\abs{N} + 2$ nodes, or $\abs{N} + 1$
    non-terminal nodes.
    By the pigeonhole principle, two of these nodes must have the same
    label.
    Let $X$ be the lowest repeated non-terminal, and let $X_\bot$ and
    $X^\top$ be the two lowest occurrences of $X$, with $X_\bot$ closer to
    the leaves.

    Let $w$ be the string of terminals derived from $X_\bot$, and let
    $vwx$ be the string of terminals derived from $X^\top$.
    Let $z = uvwxy$.
    Since the longest path from $X^\top$ down to a leaf has length at most
    $\abs{N} + 1$, we have $\abs{vwx} \le 2^{\abs{N}} = k$.

    One of the strings $v$ and $x$ must be non-empty, since $G$ is CNF
    ($X^\bot$ must have a sibling, which must lead to a terminal).
    We can then replace the subtree at $X^\top$ by the subtree at $X_\bot$,
    and obtain a parse tree for $uwy$.
    We can also replace the subtree at $X^\bot$ by the subtree at $X_\top$,
    and obtain a parse tree for $uv^2wx^2y$.

    Continuing in this way, we can obtain a parse tree for $uv^iwx^iy$ for
    any $i \ge 0$.
\end{proof}

\begin{exercise}
    Show that the following languages are not context-free.
    \begin{itemize}
        \item $\set{a^n b^n c^n \mid n \ge 0}$.
        \item $\set{ww \mid w \in \set{a, b}^*}$.
    \end{itemize}
\end{exercise}
\begin{solution} \leavevmode
    \begin{itemize}
        \item Let $k$ be the pumping lemma constant.
        Then $a^k b^k c^k$ is in the language, and so there are strings
        $u$, $v$, $w$, $x$, $y$ such that $uvwxy = a^k b^k c^k$, $vx \ne
        \epsilon$, $\abs{vwx} \le k$, and $uv^2wx^2y$ is in the language.
        Since $\abs{vwx} \le k$, it cannot contain all three letters.
        Since $vx \ne \epsilon$, $uwy$ will have more of the letters that
        $vwx$ does not contain, and less of the letters it does contain.
        \item Suppose it is.
        Let $k$ be the pumping lemma constant.
        Then \[
            a^{k+1} b^{k+1} a^{k+1} b^{k+1}
        \] is in the language,
        and so there are strings $u$, $v$, $w$, $x$, $y$ as above.
        Since $\abs{vwx} \le k$, it cannot overlap with $3$ or $4$ of the
        homogenous blocks.

        Thus $uwx$ will still be of the form $a^p b^q a^r b^s$, with
        none of $p$, $q$, $r$, or $s$ being zero.
        Since $uwy$ is in the language, $(p, q) = (r, s)$.
        But removing $v$ and $x$ will change the number of $a$s and $b$s
        in only one of the blocks, so $uwy \ne a^p b^q a^p b^q$.
    \end{itemize}
\end{solution}

\section{Closure properties} \label{sec:cfl:closure}
\begin{table}[h]
    \centering
    \begin{tabular}{cc}
        \toprule
        & Closed? \\
        \midrule
        Union & \cmark \\
        Intersection & \xmark \\
        Complement & \xmark \\
        Concatenation & \cmark \\
        \bottomrule
    \end{tabular}
    \caption{Closure properties of CFLs.}
    \label{tab:cfl:closure}
\end{table}
We have the following closure properties for CFLs.
\begin{theorem}
    The class of context-free languages is closed under union and
    concatenation.
\end{theorem}
\begin{proof}
    Let $G_1 = (N_1, A, S_1, P_1)$ and $G_2 = (N_2, A, S_2, P_2)$ be CFGs
    for $L_1$ and $L_2$.
    Let $N = N_1 \cup N_2 \cup \set{S}$, where $S$ is a new start symbol.
    Let $G = (N, A, S, P)$, where \[
        P = P_1 \cup P_2 \cup \set{S \to S_1 \mid S_2}.
    \] Then $G$ is a CFG for $L_1 \cup L_2$.
\end{proof}
