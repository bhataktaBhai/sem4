\lecture{2024-02-29}{}
\begin{exercise}
    Consider the CFG $G_1$ again.
    \begin{align*}
        S &\to XC \mid AY \\
        X &\to aXb \mid ab \\
        Y &\to bYc \mid bc \\
        A &\to aA \mid a \\
        C &\to cC \mid c
    \end{align*}
    \begin{enumerate}
        \item What is the language of $G_1$?
        \item What is the Parikh image of this language?
        \item Write it as a semi-linear set.
    \end{enumerate}
\end{exercise}

\begin{definition}[Pump] \label{def:cfg:pump}
    Let $G = (N, A, S, P)$ be a context free grammar in CNF.
    A \emph{pump} of $G$ is a derivation tree $s$ with at least $2$ nodes,
    such that $\yield(s) = u \cdot \groot(s) \cdot v$ for some
    $u, v \in A^*$.
\end{definition}
\begin{example}
    For the grammar $S \to \epsilon \mid SS \mid aSb$, example pumps are
    \vspace{3em}
    \begin{center}
        \begin{tikzpicture}
            \begin{forest}
                [$S$ [$a$] [$S$] [$b$]]
            \end{forest}
        \end{tikzpicture}
    \end{center}
\end{example}

\begin{definition}[Basic pumps] \label{def:cfg:pump:order}
    Let $s$ and $t$ be pumps of a context free grammar $G$.
    Then we say $s \pumple t$ iff $t$ can be grown from $s$ by replacing
    a non-terminal $X$ in $s$ by a pump of $X$.

    A pump which is $\pumple$-minimal is called \emph{basic}.
\end{definition}
\begin{lemma}
    The height of a basic pump is at most $2 \abs{N}$.
\end{lemma}
\begin{proof}
    Let $s$ be a basic pump.
    Let $L$ be the longest path in $s$.
    If $L$ has more than $2 \abs{N} + 1$ nodes, then $L$ has more than
    $2 \abs{N}$ nodes excluding the root.
    So by the pigeonhole principle, there are two nodes $x$ and $y$ in $L$
    with the same label.
\end{proof}

\begin{definition}[Order on parse trees] \label{def:cfg:tree:order}
    Let $s$ and $t$ be derivation trees of terminal strings starting from
    the start symbol $S$.
    Then we say $s \le t$ iff $t$ can be grown from $s$ by basic pumps
    whose non-terminals are comtained in those of $s$.
\end{definition}

\section{Pushdown Automata} \label{sec:pda}
CFG's were introduced by Noam Chomsky in 1956.
Oettinger introduced pushdown automata in 1961.
Chomsky, Schützenberger, and Evey showed equivalence of CFG's and PDA's in 1962.

A pushdown automaton reads a string from left to right and uses a stack to
store information about the string it has read so far.
Each step of the PDA looks like:
\begin{itemize}
    \item read and pop the top-of-stack symbol;
    \item read current symbol and advance the read head;
    \item push in a string of symbols onto the stack;
    \item change state.
\end{itemize}
Thus each transition of a PDA is of the form \[
    (p, a, X) \to (q, Y_1 Y_2 \dots Y_k)
\]

There are two commonly-used mechanisms of acceptance for PDA's:
\begin{itemize}
    \item \emph{Empty stack acceptance}: accept if the input is exhausted
        and the stack is empty.
    \item \emph{Final state acceptance}: accept if the input is exhausted
        and the PDA is in a final state.
\end{itemize}

\begin{example}
    An acceptance-by-empty-stack PDA for the language
    $\set{a^n b^n \mid n \ge 0}$ is \begin{align*}
        (s, \epsilon, \bot) &\to (s, \epsilon) \\
        (s, a, \bot) &\to (p, A) \\
        (p, a, A) &\to (p, AA) \\
        (p, b, A) &\to (q, \epsilon) \\
        (q, b, A) &\to (q, \epsilon)
    \end{align*}
    This is represented by the following diagram:
    \begin{center}
        \begin{tikzpicture}[node distance=2cm]
            \node[state, initial] (s) {$s$};
            \node[state, right=of s] (p) {$p$};
            \node[state, right=of p] (q) {$q$};
            \path[->]
            (s) edge[loop above] node{$\epsilon, \bot/\epsilon$} (s)
            (s) edge node{$a, \bot/A$} (p)
            (p) edge[loop above] node{$a, A/AA$} (p)
            (p) edge node{$b, A/\epsilon$} (q)
            (q) edge[loop above] node{$b, A/\epsilon$} (q);
        \end{tikzpicture}
    \end{center}
    The automaton is in state $p$ while reading the run of $a$'s,
    and moves to state $q$ when it starts reading the run of $b$'s.
\end{example}

\begin{definition*}[Pushdown automaton] \label{def:pda}
    A \emph{pushdown automaton} is a tuple
    $\mcM = (Q, A, \Gamma, s, \delta, \bot, F)$ where
    \begin{itemize}
        \item $Q$ is a finite set of states,
        \item $A$ is the input alphabet,
        \item $\Gamma$ is the stack alphabet,
        \item $s \in Q$ is the start state,
        \item $\delta \subseteq Q
            \times (A \cup \set{\epsilon})
            \times \Gamma
            \times Q
            \times \Gamma^*$ is a finite set of (non-deterministic)
            transitions,
        \item $\bot \in \Gamma$ is the bottom-of-stack symbol,
        \item $F \subseteq Q$ is the set of final states.
    \end{itemize}
    A \emph{configuration} of \mcM\ is of the form
    $(q, u, \gamma) \in Q \times A^* \times \Gamma^*$,
    which encodes the current state,
                  the input read so far,
              and the stack contents.
\end{definition*}
The initial configuration of \mcM\ is $(s, w, \bot)$ for input $w \in A^*$.
We can define transition relations on configurations:
\begin{definition*}
    For a $1$-step transition of \mcM, we define \[
        (p, au, X\beta) \xRightarrow{1} (q, u, \alpha\beta)
    \] iff $(p, a, X) \to (q, \alpha)$ is a transition in $\delta$.

    We recursively define $\xRightarrow{*}$ as for NFAs.
    Then \mcM is said to accept a word $w$
    \begin{itemize}
        \item by empty stack iff
        $(s, w, \bot) \xRightarrow{*} (q, \epsilon, \epsilon)$
        for some $q \in F$, and
        \item by final state iff
        $(s, w, \bot) \xRightarrow{*} (f, \epsilon, \gamma)$
        for some $f \in F$ and $\gamma \in \Gamma^*$.
    \end{itemize}
\end{definition*}

\begin{exercise}
    Design PDA's for the following languages:
    \begin{itemize}
        \item Balanced parentheses.
        \item $\set{a, b}^* \setminus \set{ww \mid w \in \set{a, b}^*}$.
    \end{itemize}
\end{exercise}
\begin{solution}
    For balanced parentheses, we can use the following PDA:
    \begin{align*}
        (s, \epsilon, \bot) &\to (s, \epsilon) \\
        (s, \texttt{(}, c) &\to (s, c A) \\
        (s, \texttt{)}, \texttt{(}) &\to (s, \epsilon)
    \end{align*}
    or diagrammatically:
    \begin{center}
        \begin{tikzpicture}
            \node[state, initial] (s) {$s$};
            \path[->]
            (s) edge[loop above] node{$\epsilon, \bot/\epsilon$} (s)
            (s) edge[loop right] node{$\texttt{(}, c/cA$} (s)
            (s) edge[loop below] node{$\texttt{)},\texttt{(}/\epsilon$} (s);
        \end{tikzpicture}
    \end{center}
\end{solution}
