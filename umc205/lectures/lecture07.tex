\lecture{2024-01-23}{}

\begin{theorem*}[Glushkov construction] \label{thm:glushkov}
    For every regular language $L$, there exists an NFA with exactly one
    initial and final state that accepts $L \setminus \set{\epsilon}$.
\end{theorem*}
\begin{proof}
    Let $\mcA = (Q, S, \Delta, F)$ be an NFA accepting $L$.
    Let $s$ and $f$ be new states.
    We construct an NFA \[
        \mcA' = (Q \cup \set{s, f}, \set{s}, \Delta', \set{f})
    \] where \begin{align*}
        \Delta'(s, a) &= \bigcup_{q \in S} \Delta(q, a), \\
        \Delta'(q, a) &= \begin{cases}
            \Delta(q, a) & \text{if } \Delta(q, a) \cap F = \O \\
            \Delta(q, a) \cup \set{f} & \text{otherwise}.
        \end{cases}
    \end{align*}
    Let $w \in L \setminus \set{\epsilon}$.
    Then there exists a path from some initial state $s_i$ to some final
    state $f_j$ in $\mcA$.
    This path is preserved in $\mcA'$, only with $s$ in place of $s_i$ and
    $f$ in place of $f_j$.

    Conversely, every path from $s$ to $f$ in $\mcA'$ corresponds to a
    path from some initial state to some final state in $\mcA$.
\end{proof}

\begin{theorem*}[Closure under Kleene star] \label{thm:closure:kleene}
    The class of regular languages is closed under Kleene star.
\end{theorem*}
\begin{proof}
    Let $L$ be a regular language with NFA $\mcA = (Q, S, \Delta, F)$.
    Let $\mcA'$ be the NFA constructed using the Glushkov construction.
    Now fuse the initial and final states of $\mcA'$ to obtain an NFA
    accepting $L^*$.
\end{proof}

\section{Regular Expressions} \label{sec:regex}
Consider the alphabet $\set{a, b}$.
A regular expression over this language is built from $a$, $b$, $\epsilon$
and $\O$ using operators $+$, $\cdot$, $*$, and parentheses.
\begin{examples}
    \item $(a^*)\cdot b$ is ``any number of $a$'s followed by one $b$''.
    \item $(a + b)^*abb(a + b)^*$ is ``contains $abb$ as a subword''.
    \item $(a+b)^*b(a+b)(a+b)$ is ``third last letter is $b$''.
    \item $(b^*ab^*a)b^*$ is ``has even number of $a$s''.
    \item (Exercise) Give a regular expression for ``Every $4$ bit block of
    the the form $4i$, $4i+1$, $4i+2$, $4i+3$ has an even number of $1$s''.
    \\
    (Answer) $(0000 + 0011 + \dots + 1111)^*\cdot(\epsilon + 0 + 1 + 00 + \dots + 111)$.
\end{examples}

\begin{definition*} \label{def:regex}
    The syntax of regular expressions over an alphabet $A$ is defined by \[
        r ::= \O \mid a \mid r + r \mid r \cdot r \mid r^*
    \] with $a \in A$.

    The semantics of regular expressions over an alphabet $A$ is defined by
    \begin{enumerate}
        \item $L(\O) = \O$.
        \item $L(a) = \set{a}$.
        \item $L(r_1 + r_2) = L(r_1) \cup L(r_2)$.
        \item $L(r_1 \cdot r_2) = L(r_1) \cdot L(r_2)$.
        \item $L(r^*) = L(r)^* = \bigcup_{i = 0}^\infty L(r)^i$.
    \end{enumerate}
    We give precedence to $^*$, $\cdot$ and $+$ in that order.
\end{definition*}
Parentheses can be used to override precedence.
Note that we do not give $\epsilon$ as an atom, but it can be represented
using $\O^*$.

\begin{theorem*}[Kleene's theorem] \label{thm:regex:kleene}
    The class of languages defined by regular expressions is exactly the
    class of regular languages.
\end{theorem*}
We prove the two directions separately.
\begin{proof}[RE to NFA]
    Let $L$ be a language corresponding to a regular expression $r$.
    We prove by induction on the structure of $r$ that $L$ is regular.
    For the base cases $r = \O$ and $r = a$, we have $L = \O$ and
    $L = \set{a}$, both of which are regular.
    The inductive cases follow from the closure properties of regular
    languages.
\end{proof}
