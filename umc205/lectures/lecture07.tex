\lecture{07}{Tue 23 Jan '24}{}

\begin{theorem}[Closure under Kleene star] \label{thm:closure:kleene}
    The class of regular languages is closed under Kleene star.
\end{theorem}
\begin{proof}
    Let $L$ be a regular language accepted by a DFA
    $\mcA = (Q, s, \delta, F)$.
    We construct an NFA $\mcA' = (Q', S, \Delta, F')$ for $L^*$ by
    \begin{itemize}
        \item $Q' = Q \cup \set{Q}$;
        \item $S = \set{Q}$;
        \item $F' = \set{Q}$;
        % \item $\delta'(s', a) = \begin{cases}
        %     \set{\delta(s, a)} & \text{if } \delta(s, a) \notin F, \\
        %     \set{s'} & \text{otherwise}.
        % \end{cases}$
        \item For $q \in Q$, $\Delta(q, a) = \begin{cases}
            \set{\delta(q, a)} & \text{if } \delta(q, a) \notin F, \\
            \set{\delta(q, a), Q} & \text{otherwise}.
        \end{cases}$
        \item $\Delta(Q, a) = \Delta(s, a)$.
    \end{itemize}
\end{proof}

\section{Regular Expressions} \label{sec:regex}
Consider the alphabet $\set{a, b}$.
A regular expression over this language is built from $a$, $b$, $\epsilon$,
using operators $+$, $\cdot$, $*$, and parentheses.
\begin{examples}
    \item $(a^*)\cdot b$ is ``any number of $a$'s followed by one $b$''.
    \item $(a + b)^*abb(a + b)^*$ is ``contains $abb$ as a subword''.
    \item $(a+b)^*b(a+b)(a+b)$ is ``third last letter is $b$''.
    \item $(b^*ab^*a)b^*$ is ``has even number of $a$s''.
    \item (Exercise) Give a regular expression for ``Every $4$ bit block of
    the the form $4i$, $4i+1$, $4i+2$, $4i+3$ has an even number of $1$s''.
    \\
    (Answer) $(0000 + 0011 + \dots + 1111)^*\cdot(\epsilon + 0 + 1 + 00 + \dots + 111)$.
\end{examples}

\begin{definition} \label{def:regex}
    The syntax of regular expressions over an alphabet $A$ is defined by \[
        r ::= \O \mid a \mid r + r \mid r \cdot r \mid r^*
    \] with $a \in A$.

    The semantics of regular expressions over an alphabet $A$ is defined by
    \begin{enumerate}
        \item $L(\O) = \O$.
        \item $L(a) = \set{a}$.
        \item $L(r_1 + r_2) = L(r_1) \cup L(r_2)$.
        \item $L(r_1 \cdot r_2) = L(r_1) \cdot L(r_2)$.
        \item $L(r^*) = L(r)^* = \bigcup_{i = 0}^\infty L(r)^i$.
    \end{enumerate}
    We give precedence to $*$, $\cdot$, $+$, in that order.
\end{definition}

\begin{theorem}[Kleene's theorem] \label{thm:regex:kleene}
    The class of languages defined by regular expressions is exactly the
    class of regular languages.
\end{theorem}
We prove the two directions separately.
\begin{proof}[RE to NFA]
    Let $L$ be a language corresponding to a regular expression $r$.
    We prove by induction on the structure of $r$ that $L$ is regular.
    For the base cases $r = \O$ and $r = a$, we have $L = \O$ and
    $L = \set{a}$, both of which are regular.
    The inductive cases follow from the closure properties of regular
    languages.
\end{proof}
