\lecture{09}{Tue 30 Jan '24}{}
\section{Myhill-Nerode Theorem} \label{sec:mn}
We will see several %TODO: how to finish this sentence?
\begin{itemize}
    \item A language $L$ is regular iff a certain equivalence relation
    induced by $L$ (called $\equiv_L$) has a finite number of equivalence
    classes.
    \item Every language $L$ has a ``canonical'' deterministic automaton
    accepting it.
    Every other DA for $L$ is a ``refinement'' of this canonical DA.
    For regular languages, there is a unique DA for $L$ with the minimal
    number of states.
    \begin{remark}
        \emph{Every} language $L$ has a DA accepting it, the ``free'' DA for
        $L$, which has one state for each string over the alphabet.
    \end{remark}
\end{itemize}

\begin{definition}[Refinement] \label{def:mn:refinement}
    Let $\mcA = (Q, s, \delta, F)$ be a deterministic automaton over an
    alphabet $A$.
    We say that \mcA' is a \emph{refinement} of \mcA\ if there exists a
    $Q$-indexed partition $\set{Q'_p}_{p \in Q}$ of $Q'$ such that for all
    $p, q \in Q$ and $a \in A$ such that $\delta(p, a) = q$, there exists a
    $p' \in Q'_p$ and a $q' \in Q'_q$ such that $\delta'(p', a) = q'$.
\end{definition}

\begin{definition}
    For any language $L \subseteq A^*$, we define the canonical equivalence
    relation $\equiv_L$ on $A^*$ as \[
        x \equiv_L y \iff \forall z \in A^*(xz \in L \iff yz \in L).
    \]
\end{definition}

\begin{exercise}
    Describe the equivalence classes for $L = $ ``odd number of $a$'s''.
\end{exercise}
\begin{solution}
    $L$ and $L^c$.
\end{solution}

\begin{exercise}
    Describe precisely the equivalence classes of $\equiv_L$ for the
    language $L \subseteq \set{a, b}^*$ comprising strings in which the 2nd
    last letter is a $b$.
\end{exercise}
\begin{solution}
    $\epsilon + a + (.^*)aa$, $b + (.^*)ab$, $(.^*)ba$,
    $(.^*)bb$.
\end{solution}

\begin{exercise}
    Descrive the equivalence classes of $\equiv_L$ for the language
    $L = \set{a^nb^n \mid n \ge 0}$.
\end{exercise}
\begin{solution}
    $L$ is the disjoint union of $\bigsqcup_{0 \le m \le n} \set{a^nb^m}$
    and its complement.
\end{solution}
% \pagebreak
% \begin{definition}[Limits] \label{def:mn:limits}
%     Let $f\colon \R \to \R$ be a function and let $p \in \R$.
%     We say that $\lim\limits_{x \to p} f(x) = L$ if for all
%     $\varepsilon > 0$, there exists a $\delta_\varepsilon > 0$ such that
%     whenever $0 < \abs{x - p} < \delta_\varepsilon$, we have
%     $\abs{f(x) - L} < \varepsilon$.
% \end{definition}
% We are writing $\delta_\varepsilon$ to emphasize that $\delta$ depends on
% $\varepsilon$.
% One can choose a different $\delta$ for each $\varepsilon$.
% \begin{exercise}
%     Complete the following proof of $\lim_{x \to 0} 1 = 1$.
% \end{exercise}
% \begin{proof}
%     The function at hand is $f(x) = 1$ for all $x \in \R$.
%     To show that $\lim_{x \to 0} f(x) = 1$, we will assume that
%     $\varepsilon > 0$ is given and produce a $\delta_\varepsilon > 0$ that
%     satisfies the definition.
%
%     Let $\varepsilon > 0$.
%     Choose $\delta_\varepsilon = $ \underline{\hspace{1cm}}.
%     Suppose $x \in \R$ such that $0 < \abs{x - 0} < \delta_\varepsilon$.
%     Then \underline{\hspace{5cm}}.
%
%     Since $\varepsilon > 0$ was \emph{arbitrary} (we allowed it to be any
%     positive real number), and we constructed a $\delta_\varepsilon$ that
%     works, we have shown that
%     \begin{center}
%         For any $\varepsilon > 0$, there exists a $\delta_\varepsilon > 0$
%         such that whenever $0 < \abs{x - 0} < \delta_\varepsilon$, we have
%         $\abs{f(x) - 1} < \varepsilon$.
%     \end{center}
%     This is precisely the definition of $\lim_{x \to 0} f(x) = 1$.
% \end{proof}
