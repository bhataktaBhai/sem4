\lecture{2024-01-16}{}

\begin{definition}
    Let $L \subseteq A^*$ be a language.
    The \emph{Kleene closure} of $L$, denoted $L^*$, is defined as \[
        L^* = \bigcup_{n \in \N} L^n
    \] where $L^0 = \set{\epsilon}$ and $L^{n+1} = L^n \cdot L$.
\end{definition}
In other words, \[
    L^* = \set{s \in A^* \mid \exists w \in L^\N \text{ and }
        n \in \N \text{ such that } s = w_0 \cdots w_n}
\]
\begin{exercise}
    If $L \subseteq \set{a}^*$, show that $L^*$ is regular.
\end{exercise}
\begin{exercise}
    Show that there exists a language $L \subseteq A^*$ such that neither
    $L$ nor its complement $A^* \setminus L$ contains an infinite regular
    subset.
\end{exercise}
\begin{solution}
    Assignment 2.
\end{solution}

\begin{definition}[Ultimate periodicity] \label{def:ultimate_periodicity}
    A subset $X$ of \N\ is said to be \emph{ultimately periodic} if there
    exist $n_0 \in \N$, $p \in \N^*$ such that for all $m \ge n_0$,
    $m \in X$ iff $m + p \in X$.
\end{definition}
\begin{proposition}
    A subset $X$ being ultimately periodic is equivalent to either
    \begin{itemize}
        \item there exist $n_0 \in \N$, $p \in \N^*$ such that for all
            $m \ge n_0$, $m \in X \implies m + p \in X$, or
        \item 
    \end{itemize}
\end{proposition}

\begin{definition}
    For a language $L \subseteq A^*$, define $\lengths(L)$ to be
    $\set{\size w \mid w \in L}$.
\end{definition}
\begin{theorem*}
    If $L$ is a regular language, then $\lengths(L)$ is ultimately periodic.
\end{theorem*}
