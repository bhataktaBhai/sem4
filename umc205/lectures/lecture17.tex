\lecture{2024-03-05}{}
\subsection{Equivalence of acceptance criteria}
\label{sec:criteria_equivalence}
\begin{theorem*}
    Any language is accepted by a PDA via empty stack if and only if
    it is accepted by another PDA via final state.
\end{theorem*}
\begin{proof}
    Suppose $L$ is accepted by a PDA \mcM\ via final state.
    Then we can add a new state \textsc{Hole} and a new
    $\epsilon$-transition from every final state to \textsc{Hole}.
    For \textsc{Hole}, we add an $\epsilon$-transition loop to itself
    that pops the stack and goes to the final state.

    But what if the stack is emptied without reaching a final state?
    We introduce a new bottom-of-stack symbol $\bbot$.
    Let $\mcM = (Q, A, \Gamma, s, \delta, \bot, F)$.
    Define \[
        \mcM' = (Q \cup \set{s', \textsc{Hole}}, A, \Gamma \cup \set{\bbot}, s', \delta', \bbot, \set{\textsc{Hole}})
    \] where $\delta'$ has transitions from $\delta$ and the new transitions
    \begin{align*}
        (s', \epsilon, \bbot) &\to (s, \bot\bbot) \\
        (f, \epsilon, X) &\to (\textsc{Hole}, \epsilon) \quad \text{for all } f \in F, X \in \Gamma \cup \set{\bbot} \\
        (\textsc{Hole}, \epsilon, X) &\to (\textsc{Hole}, \epsilon) \quad \text{for all } X \in \Gamma \cup \set{\bbot}
    \end{align*}

    Suppose a word $w$ is accepted by \mcM\ via final state.
    Then once \mcM'\ reads $w$, it will land on a final state in \mcM\ and
    will have a non-empty stack (since $\mcM$ transitions cannot pop the
    new $\bbot$ symbol).
    Then it will transition to \textsc{Hole} and empty the stack.

    Suppose a word $w$ is accepted by \mcM'\ via empty stack.
    The stack can only be emptied by the transitions from \textsc{Hole}
    at the end of the input tape.
    But to get to \textsc{Hole}, \mcM\ must have been in a final state
    after reading the entire input.
\end{proof}

\section{CFGs and PDAs} \label{sec:cfg:pda_equivalence}
\begin{theorem}[Chomsky] \label{thm:cfg:chomsky}
    The class of languages accepted by a PDA is precisely the class of
    context-free languages.
\end{theorem}
\begin{proof}[CFG to PDA]
    Let $G = (N, A, S, P)$ be a CFG.
    Assume WLOG that all rules of $G$ are of the form \[
        X \to c B_1 B_2 \dots B_k
    \] where $c \in A \cup \set{\epsilon}$ and $B_i \in N$.
    This can always be done, since for any $a \in A$ we can introduce a new
    non-terminal $A'$ and a rule $A' \to a$.

    We construct a PDA \mcM\ that simulates a leftmost derivation of $G$ on
    the given input.
    Define $\mcM = (\set{s}, A, N, s, \delta, S)$ where $\delta$ is given
    by \[
        (s, c, X) \to (s, B_1 B_2 \dots B_k)
    \] for any rule $X \to c B_1 B_2 \dots B_k$ in $P$.
    Then \mcM\ accepts $L(G)$ by empty stack.
\end{proof}

\begin{exercise}
    Construct a PDA for the CFG \begin{align*}
        S &\to \texttt{(}SR \mid SS \mid \epsilon \\
        R &\to \texttt{)}
    \end{align*}
\end{exercise}
\begin{proof}
    We have the following transitions in $\delta$:
    \begin{align*}
        (s, \texttt{(}, S) &\to (s, SR) \\
        (s, \epsilon, S) &\to (s, SS) \\
        (s, \epsilon, S) &\to (s, \epsilon) \\
        (s, \texttt{)}, R) &\to (s, \epsilon) \qedhere
    \end{align*}
    % Diagrammatically, the PDA looks like this:
    % \begin{center}
    %     \begin{tikzpicture}
    %         \node[state, initial] (s) {$s$};
    %         \path[->] (s) edge[loop above] node {\texttt{(}, 
    %     \end{tikzpicture}
    % \end{center}
\end{proof}

\begin{lemma}
    For any PDA \mcM\ that accepts via empty stack, there is a PDA \mcM'
    with a single state that accepts the same language via empty stack.
\end{lemma}
\begin{proof}
    Let $\mcM = (Q, A, \Gamma, s, \delta, \bot, F)$.
    Define \[
        \mcM' = (\set{Q}, A, Q \times \Gamma \times Q, Q, \delta', (s, \bot), \O)
    \] where we will define $\delta'$ as follows.
    For every original transition \[
        (p, a, X) \to (q, B_1 B_2 \dots, B_k),
    \] we add the transitions \[
        (Q, a, \angled{p, X, r}) \to (Q, \angled{q B_1 q_1}\angled{q_1 B_2 q_2}\dots\angled{q_{k-1} B_k r})
    \] for every $q_1, q_2, \dots, q_{k-1}, r \in Q$.
    In particular, for every original transition \[
        (p, \epsilon, X) \to (q, \epsilon),
    \] we add the transition \[
        (Q, \epsilon, (p, X, r)) \to (Q, \epsilon)
    \] for every $r \in Q$.
\end{proof}
