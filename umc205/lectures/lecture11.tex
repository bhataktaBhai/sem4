\lecture{2024-02-06}{}
\begin{definition*}[Stable partitioning] \label{def:stable_partitioning}
    Let $\mcA = (Q, q_0, \delta, F)$ be a DFA.
    An equivalence relation $\sim$ on $Q$ is said to be a
    \emph{stable partitioning} of \mcA\ if for all $p, q \in Q$,
    if $p \sim q$, then
    \begin{itemize}
        \item $p \in F \iff q \in F$, and
        \item $\delta(p, a) \sim \delta(q, a)$ for all $a \in A$.
    \end{itemize}
\end{definition*}
\begin{definition*}[DFA refinement] \label{def:dfa_refinement}
    We say that a DFA $\mcB$ refines a DFA $\mcA$ iff there exists a stable
    partitioning $\sim$ of $\mcB$ such that $\mcB/{\sim}$ is isomorphic to
    $\mcA$.
\end{definition*}
Here, $\mcB/{\sim}$ is the DFA obtained by merging all states in the same
equivalence class of $\sim$.
How is this merging done?
Let $p \sim q$.
Then for all $a \in A$, we have that $[\delta(p, a)] = [\delta(q, a)]$.
Thus we define $\delta'([p], a) = [\delta(p, a)]$.
\begin{definition*}
    Let $\mcA = (Q, s, \delta, F)$ be a DFA and let $\sim$ be a stable
    partitioning of \mcA.
    We define $\mcA/{\sim}$ as follows. \[
        \mcA/{\sim} = \big(
            Q/{\sim},
            [s],
            [p] \xrightarrow{a} [\delta(p, a)],
            \set{[p] \mid p \in F}
        \big).
    \]
\end{definition*}

\begin{proposition}
    Let $\sim$ be a stable partitioning of a DFA \mcA.
    Then $\mcA/{\sim}$ accepts the same language as \mcA.
\end{proposition}
\begin{proof}
    Let $\mcA = (Q, q_0, \delta, F)$ and $\mcA' = \mcA/{\sim}$.
    Then we claim that $\what{\delta'}([q_0], w) = [p]$.
    This is immediate from the definition of $\delta'$.
    Thus the two automata accept the same language.
\end{proof}

\begin{definition}
    Let $\mcA = (Q, s, \delta, F)$ be a DA for $L$ with no unreachable
    states.
    We define $\approx_L$ as follows. \[
        p \approx_L q \iff \exists x \equiv_L y \text{ such that }
            \what{\delta}(s, x) = p \text{ and } \what{\delta}(s, y) = q.
    \]
\end{definition}
\begin{proposition}
    $\approx_L$ is a stable partitioning of \mcA.
\end{proposition}
\begin{proof}
    Suppose $p \approx_L q$.
    Then $p \in F$ iff $x \in L$ iff $y \in L$ iff $q \in F$.
    Also, since $xa \equiv_L ya$ for all $a \in A$, we have that
    $\what{\delta}(s, xa) \approx_L \what{\delta}(s, ya)$ and so
    $\delta(p, a) \approx_L \delta(q, a)$ for all $a \in A$.
\end{proof}

\begin{theorem}
    Let $\mcA = (Q, s, \delta, F)$ be a DFA for $L$ with no unreachable
    states.
    Define $\approx$ as follows. \[
        p \approx q \iff \forall z \in A^* \left(
            \what{\delta}(p, z) \in F \iff \what{\delta}(q, z) \in F
        \right).
    \] Then ${\approx} = {\approx_L}$.
\end{theorem}
\begin{proof}
    Let $p \approx q$.
    Then for all $z \in A^*$, $\what{\delta}(p, z) \in F$ iff
    $\what{\delta}(q, z) \in F$.
    Since no state in \mcA\ is unreachable, let $p = \what{\delta}(s, x)$
    and $q = \what{\delta}(s, y)$.
    Then $\what{\delta}(s, xz) \in F \iff \what{\delta}(s, yz) \in F$ so
    that $x \equiv_L y$.
    Thus $p \approx_L q$.

    Conversely, let $p \approx_L q$.
    Let $x \equiv_L y$ such that $\what{\delta}(s, x) = p$ and
    $\what{\delta}(s, y) = q$.
    Then since $xz \in L$ iff $yz \in L$, we have that
    $\what{\delta}(p, z) \in F$ iff $\what{\delta}(q, z) \in F$.
\end{proof}

\begin{exercise}
    Run algorithm to compute $\approx$ for the DFA below
\end{exercise}
\begin{solution} \leavevmode
    \begin{table}[h]
        \centering
        \begin{tabular}{c|cccc}
              & $s$ & $p$ & $q$ & $r$ \\
            \hline
            $s$ &   &   &   &   \\
            $p$ & \checkmark & & & \\
            $q$ & & \checkmark & & \\
            $r$ & \checkmark & & \checkmark &
        \end{tabular}
    \end{table}
\end{solution}

\subsection{Analysis of the minimization algorithm}
\label{sec:min_alg:analysis}
We claim that the algorithm always terminates.
Let $n = \abs{Q}$.
Since the algorithm terminates when the table is unchanged, it must
terminate after at most $\binom{n}{2}$ iterations.

In fact, the algorithm terminates after at most $n$ iterations.
In each iteration, we mark a pair if they lead to some states upon reading
the same input, which were marked in the previous iteration.

We argue that at the end of the $i$\textsuperscript{th} iteration,
the algorithm has marked all pairs of states that are distinguishable
by a string of at most $i$ symbols.

$i = 0$ is trivial.
Suppose this holds for some $i \ge 0$.
Let $p, q \in Q$ be such that $p$ and $q$ are distinguishable by a string
$w$ of at most $i + 1$ symbols.
If $\abs{w} \le i$, then $p$ and $q$ are marked at the end of the
$i$\textsuperscript{th} iteration.
Otherwise, $w = av$ for some $a \in A$ and $v \in A^i$.
Then since $\what{\delta}(p, w) = \what{\delta}(\delta(p, a), v)$ and
$\what{\delta}(q, w) = \what{\delta}(\delta(q, a), v)$, we have that
$\delta(p, a)$ and $\delta(q, a)$ are distinguishable by a string of $i$
symbols, and so they will be marked at the end of the
$i$\textsuperscript{th} iteration.
This means that $p$ and $q$ will be marked at the end of the
$(i + 1)$\textsuperscript{th} iteration.
