\lecture{08}{Thu 25 Jan '24}{}
\begin{proof}
    Let $L$ be a regular language over an alphabet $A$,
    accepted by a DFA $\mcA = (Q, s, \delta, F)$.
    If the set of final states is empty, then $L = \O$.
    Otherwise, we induct on the number of states of $\mcA$.
    For the base case, there is a DFA with one state which accepts
    all strings in $L$.

    This is the regular expression $r = (a_1 + \dots + a_k)^*$, where
    $A = \set{a_1, \dots, a_k}$.
    Define $L_{pq}$ to be
    $\set{w \in A^* \mid \what{\delta}(p, w) = q}$.
    Then $L(\mcA) = \bigcup_{f \in F} L_{sf}$.
    For $X \subseteq Q$, define \begin{multline*}
        L^X_{pq} \coloneq \{w \in A^* \mid \what{\delta}(p, w) = q
        \text{ and } \what{\delta}(p, x) \in X \\
        \text{ for all non-empty proper prefixes } x \text{ of } w\}.
    \end{multline*}
    \textbf{Claim:}
        $L^{X+r}_{pq} = L^X_{pq} \cup L^X_{pr} (L^X_{rr})^* L^X_{rq}$.
    This is easily converted to regular expressions by substituting $+$
    for $\cup$.

    For the base case of $X = \O$, observe that \[
        L^\O_{pq} = \set*{a \in A + \epsilon \mid
            \what{\delta}(p, a) = q}
    \] is finite, and hence can be converted to a regular expression.

    By induction on $\abs{X}$ with base case $\O$, we go upto $X = Q$
        to prove that $L = \bigcup_{f \in F} L^Q_{sf}$ can be converted to
        a regular expression. \qedhere
\end{proof}

\begin{exercise}
    Convert the following automata to regular expressions.
    \begin{center}
        \begin{tikzpicture}
            \node[state, initial] (s) {$s$};
            \node[state, accepting] (f) [right=of s] {$f$};
            \path[->]
                (s) edge [loop above] node {$a$, $b$} (s)
                    (s) edge node {$b$} (f);

            \node[state, initial, right=2cm of f] (even) {0};
            \node[state, accepting] (odd) [right=of even] {1};
            \path[->]
                (even) edge [loop above] node {$b$} (even)
                    (even) edge [bend left] node {$a$} (odd)
                (odd) edge [loop above] node {$b$} (odd)
                    (odd) edge [bend left] node {$a$} (even);
        \end{tikzpicture}
    \end{center}
\end{exercise}
\begin{solution}
    For the first one, the regex of $L^\O_{sf}$ is $b$.
    The regex of $L^{\set{s}}_{sf}$ is $b + (a+b+\epsilon)(a+b+\epsilon)^*
    b$.
    This can be simplified as \begin{align*}
        b + (a + b)(a + b)^* b + (a + b)^* b
            &= b + (a + b)^+ b + (a + b)^* b \\
            &= (a + b)^* b + (a + b)^* b \\
            &= (a + b)^* b
    \end{align*}
    The regex of $L^{\set{s,f}}_{sf}$ is also $(a + b)^* b$ since
    $L^{\set{s}}_{ff} = \O$.

    For the second one, $L^\O_{01}$ has regex $a$.
    Then $L^{\set{0}}_{01}$ has regex $a + b(b + \epsilon)^* a = b^* a$.

    We also need $L^{\set{0}}_{11}$.
    So we compute $L^\O_{11} = b + \epsilon$ and
    \[
        L^{\set{0}}_{11} = b + \epsilon + a (b + \epsilon)^* a
            = \epsilon + b + a b^* a.
    \]
    So \begin{align*}
        L^{\set{0, 1}}_{01} &= b^* a + (b^* a) (b + \epsilon)^* 
    \end{align*}
\end{solution}

We can also view this as a system of equations.
For example, for the second automaton of the exercise above, we set up
equations to capture $L_q = \set{w \in A^* \mid \what{\delta}(q, w) \in F}$
as \begin{align*}
    x_0 &= b \cdot x_0 + a \cdot x_1 \\
    x_1 &= \epsilon + a \cdot x_0 + b \cdot x_1
\end{align*}

In general, such equations can have many solutions.
But for equations arising from DFAs, we can show that there is a unique
solution.
We can write the equations above as \[
    \begin{pmatrix}
        x_0 \\
        x_1
    \end{pmatrix} = \begin{pmatrix}
        b & a \\
        a & b
    \end{pmatrix} \begin{pmatrix}
        x_0 \\
        x_1
    \end{pmatrix} + \begin{pmatrix}
        \O \\
        \epsilon
    \end{pmatrix}
\]
For any such system $X = AX + B$, $A^*B$ is the \emph{least} solution.
When $A^*$ is a $2 \times 2$ matrix, \[
    \begin{pmatrix}
        a & b \\
        c & d
    \end{pmatrix}^* = \begin{pmatrix}
        (a + b d^* c)^* & (a + b d^* c)^* b d^* \\
        (d + c a^* b)^* c a^* & (d + c a^* b)^*
    \end{pmatrix}
\]
\textbf{Reading:} Kozen, Supplementary Lecture A.
