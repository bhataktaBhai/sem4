\chapter{Languages} \label{chp:languages}
\lecture{02}{Thu 04 Jan '24}{Deterministic Finite-State Automata}

\begin{definition*}
    An \emph{alphabet} is a non-empty finite set of symbols or ``letters''.

    A \emph{string} or \emph{word} over an alphabet $A$ is a finite sequence
    of letters from $A$.
    Equivalently, a string is a map from a prefix (possibly empty) of $\N$
    to $A$.
    The length of a string $s$, denoted $\abs{s}$,
    is the cardinality of its domain.
    The empty string is denoted $\epsilon$.

    The set of all strings over $A$ is denoted $A^*$.
\end{definition*}
\begin{example}
    $A = \set{a, b, c}$ and $\Sigma = \set{0, 1}$ are both alphabets.
    $aaba$ is a string over $A = \set{a, b, c}$.
\end{example}

\begin{proposition}
    Let $A$ be an alphabet.
    Then $A^*$ is countably infinite.
\end{proposition}
\begin{proof}
    Let $n = \size A$.
    Let $f \colon A \to \set{1, \dots, n}$ be a numbering of $A$.
    Replacing each letter in a string with its number gives a representation
    of the string as a natural number in base $n + 1$.
    This gives an injection $A^* \to \N$ and so $A^*$ is countable.
    Infiniteness is obvious.

    Alternatively, consider the strings in their
    \nameref{def:languages:lexicographic}.
\end{proof}

\begin{definition*}[Language] \label{def:languages:language}
    A \emph{language} over an alphabet $A$ is a subset of $A^*$.
\end{definition*}
\begin{example}
    Let $A = \set{a, b, c}$.
    Then $\set{abc, aaba}$,
    $\set{\epsilon, b, aa, bb, aab, aba, bbb, \dots}$, $\set{\epsilon}$,
    $\set{}$ are all languages over $A$.
\end{example}

\begin{definition*}[Concatenation] \label{def:languages:concatenation}
    Let $u$, $v$ be strings over an alphabet $A$.
    Then $u \cdot v$ or simply $uv$ is the string obtained by appending $v$
    to the end of $u$.

    For two languages $L_1$, $L_2$ over $A$, define their concatenation \[
        L_1 \cdot L_2 \coloneq \set{uv \mid u \in L_1, v \in L_2}.
    \]
\end{definition*}
We will also write $ua$ where $u$ is a string and $a$ is a letter to mean
$u$ concatenated with the string of length 1 consisting of the letter $a$.

\begin{definition*}[Lexicographic order] \label{def:languages:lexicographic}
    Let $(A, <)$ be a totally ordered alphabet.
    We say $u < v$ for $u, v \in A^*$ if either $\size u < \size v$ or
    $\size u = \size v$ and $u = pxu'$, $v = pyv'$ for some $p, u', v' \in A^*$
    and $x, y \in A$ with $x < y$.

    This is called the \emph{lexicographic order} on $A^*$.
\end{definition*}

\begin{proposition}
    Let $A$ be an alphabet.
    Then the set of all languages over $A$ is uncountable.
\end{proposition}
\begin{proof}
    Diagonilization.
    Let $\Phi\colon A^* \to 2^{A^*}$ be a map.
    Then define $L_\Phi = \set{s \in A^* \mid s \notin \Phi(s)}$.
    Then $L_\Phi$ is a language over $A$ that is not in the image of $\Phi$.
\end{proof}

\chapter{Finite-State Automata} \label{chp:fa}
\begin{center}
    \begin{tikzpicture}
        \node[state, initial] (q0) {$e$};
        \node[state, accepting, right=of q0] (q1) {$o$};
        \path[->]
            (q0) edge [bend left] node {$a$} (q1)
            (q0) edge [loop above] node {$b$} (q0)
            (q1) edge [loop above] node {$b$} (q1)
            (q1) edge [bend left] node {$a$} (q0);
    \end{tikzpicture}
\end{center}
This is a deterministic finite state automaton.
The machine starts at state $e$,
and as it reads a (finite) string consiting of $a$'s and $b$'s,
it moves to the state specified by the arrows from the current state.
The state $o$ is an ``accepting state''.
Each string whose evaluation ends at state $o$ is said to be accepted by the
automaton.

Each state represents a property of the input string read so far.
In this case, State $e$ corresponds to an even number of $a$'s read,
and State $o$ corresponds to an odd number of $a$'s read.
This can be proven by induction to conclude that the automaton accepts the
language $\set{w \in \set{a, b}^* \mid \#_a(w)}$.

\begin{example}
    Let $A = \set{a, b}$.
    Consider the DFA
    \begin{center}
        \begin{tikzpicture}
            \node[state, initial] (e) {};
            \node[state, right=of e] (a) {};
            \node[state, right=of a] (ab) {};
            \node[state, accepting, right=of ab] (abb) {};
            \path[->]
                (e) edge node {$a$} (a)
                (e) edge [loop above] node {$b$} (e)
                (a) edge node {$b$} (ab)
                (a) edge [loop above] node {$a$} (a)
                (ab) edge node {$b$} (abb)
                (ab) edge [bend left] node {$a$} (a)
                (abb) edge [loop above] node {$a, b$} (abb);
        \end{tikzpicture}
    \end{center}
    We label the nodes as $\epsilon$, $a$, $ab$ and $abb$
    \begin{center}
        \begin{tikzpicture}
            \node[state, initial] (e) {$\epsilon$};
            \node[state, right=of e] (a) {$a$};
            \node[state, right=of a] (ab) {$ab$};
            \node[state, accepting, right=of ab] (abb) {$abb$};
            \path[->]
                (e) edge node {$a$} (a)
                (e) edge [loop above] node {$b$} (e)
                (a) edge node {$b$} (ab)
                (a) edge [loop above] node {$a$} (a)
                (ab) edge node {$b$} (abb)
                (ab) edge [bend left] node {$a$} (a)
                (abb) edge [loop above] node {$a, b$} (abb);
        \end{tikzpicture}
    \end{center}
    and consider the property corresponding to each state.
    \begin{itemize}
        \item State $abb$: the string seen so far contains $abb$.
    \end{itemize}
    Every other state will have the property that the string seen so far
    does not contain $abb$, in addition to some other property.
    \begin{itemize}
        \item State $a$: the string seen so far ends in $a$.
        \item State $ab$: the string seen so far ends in $ab$.
        \item State $\epsilon$: every other string seen so far.
    \end{itemize}
\end{example}

Here is another example of a DFA, which accepts strings over $\set{0, 1}$
that have even parity of $1$s in each completed length 4 block.
\begin{center}
    \begin{tikzpicture}[scale=0.9, transform shape]
        \node[state, initial, accepting] (0e) {$0, e$};
        \node[state, accepting, above right=of 0e] (1e) {$1, e$};
        \node[state, accepting, below right=of 0e] (1o) {$1, o$};
        \node[state, accepting, right=of 1e] (2e) {$2, e$};
        \node[state, accepting, right=of 1o] (2o) {$2, o$};
        \node[state, accepting, right=of 2e] (3e) {$3, e$};
        \node[state, accepting, right=of 2o] (3o) {$3, o$};
        \node[state, below right=of 3e] (0o) {$0, o$};

        % apply near start to all nodes
        \begin{scope}[every node/.style={very near start}]
            \path[->] (0e) edge node {$0$} (1e)
            (0e) edge node {$1$} (1o)
            (1e) edge node {$0$} (2e)
            (1e) edge node {$1$} (2o)
            (1o) edge node {$0$} (2o)
            (1o) edge node {$1$} (2e)
            (2e) edge node {$0$} (3e)
            (2e) edge node {$1$} (3o)
            (2o) edge node {$0$} (3o)
            (2o) edge node {$1$} (3e)
            (3e) edge node {$1$} (0o)
            (3o) edge node {$0$} (0o);
        \end{scope}
        \path[->]
            (0o) edge [loop right] node {$0, 1$} (0o)
            (3e) edge [bend right=2.5cm] node[above, very near start] {$0$} (0e)
            (3o) edge [bend left=2.5cm] node[very near start] {$1$} (0e);
    \end{tikzpicture}
\end{center}

\begin{exercise}
    Give a DFA that accepts strings over the alphabet $\set{a, b}$ containing an
    even number of $a$'s and an odd number of $b$'s.
\end{exercise}
\begin{solution} \leavevmode
    \begin{center}
        \begin{tikzpicture}
            \node[state, initial] (00) {$e, e$};
            \node[state, above right=of 00] (10) {$o, e$};
            \node[state, accepting, below right=of 00] (01) {$e, o$};
            \node[state, below right=of 10] (11) {$o, o$};

            \path
                (00) edge node[below] {$a$} (10)
                (10) edge node[below] {$b$} (11)
                (11) edge node[above] {$a$} (01)
                (01) edge node[above] {$b$} (00)
                (00) edge [bend right] node[below] {$b$} (01)
                (01) edge [bend right] node[below] {$a$} (11)
                (11) edge [bend right] node[above] {$b$} (10)
                (10) edge [bend right] node[above] {$a$} (00);
        \end{tikzpicture}
    \end{center}
\end{solution}

\begin{exercise}
    Give a DFA that accepts strings over $\set{a, b, /, *}$ which don't end
    inside a C-style comment, \ie, comments of the form \texttt{/* ... */}.
\end{exercise}
\begin{solution} \leavevmode
    \begin{center}
        \begin{tikzpicture}
            \node[state, initial, accepting] (out) {};
            \node[state, accepting, right=of out] (slash) {};
            \node[state, right=of slash] (in) {};
            \node[state, right=of in] (star) {};

            \path[->]
                (out) edge [loop below] node {$a, b, *$} (out)
                (out) edge node {$/$} (slash)
                (slash) edge node {$*$} (in)
                (slash) edge [loop below] node {$/$} (slash)
                (slash) edge [bend left] node {$a, b$} (out)
                (in) edge [loop below] node {$a, b, /$} (in)
                (in) edge node {$*$} (star)
                (star) edge [loop right] node {$*$} (star)
                (star) edge [bend right] node[above] {$/$} (out)
                (star) edge [bend left] node {$a, b$} (in);
        \end{tikzpicture}
    \end{center}
\end{solution}
