\section{Chomsky Normal Form} \label{sec:cnf}
\lecture{13}{Tue 13 Feb '24}{}
\begin{definition}[Chomsky normal form] \label{def:cnf}
    A context-free grammar $G$ is said to be in \emph{Chomsky normal form}
    if all of its production rules are of the form \begin{align*}
        X &\to YZ \\
        X &\to a
    \end{align*} where $Y$ and $Z$ are non-terminals, and $a$ is a
    terminal.
\end{definition}
\begin{example}
    Consider $G_4$ in the examples above, which generates balanced
    parentheses. \begin{align*}
        S &\to \texttt{(}S\texttt{)} \mid SS \mid \epsilon
    \intertext{This can be converted into a CNF as follows}
        L &\to \texttt{(} \\
        R &\to {\texttt{)}} \\
        S &\to LR \mid SS \mid LX \\
        X &\to SR
    \end{align*}
\end{example}
Why do we care about Chomsky normal form?
Suppose we have a context-free grammar $G$ and a string $w$.
We want to know if $w \in L(G)$.
This is hard to do in general, but it is trivial if $G$ is in CNF.
Any production rule applied to an intermediate string $w$ cannot decrease
the length of $w$, so we will know to terminate in finitely many steps.

\begin{theorem}
    Every context-free grammar $G$ can be converted into a Chomsky normal
    form grammar $G'$ such that $L(G') = L(G) \setminus \set{\epsilon}$.
\end{theorem}
Choose any problematic production rule in $G$.
If the RHS has more than two (say $n$) non-terminals, we can introduce a new
non-terminal in place of $n-1$ of them, from which we generate those $n-1$
non-terminals in sequence. \\
If the RHS has more than one terminal, we can introduce a new non-terminal
for each of those, and we have just shown how to deal with that case. \\
In fact, if the RHS is of length at least 2,
we can replace its terminals with non-terminals. \\
Thus the only problematic case is when the RHS is either a single terminal,
a single non-terminal, or the empty string.

\begin{theorem}
    Let $G$ be a context-free grammar.
    Then there is a context-free grammar $G'$ such that $L(G') = L(G)$ and
    $G'$ has no unit- or $\epsilon$-productions.
\end{theorem}
We give an algorithm to achieve this, and will prove its correctness later.

Let $G = (N, A, S, P)$ be a context-free grammar.
Create a new set of productions $P'$ as follows:
First add all the productions from $P$ to $P'$.
Then,
\begin{itemize}
    \item if $P$ has productions $X \to \alpha Y \beta$ and
    $Y \to \epsilon$, add the rule $X \to \alpha \beta$ to $P'$.
    \item if $X \to Y$ and $Y \to \gamma$, add $X \to \gamma$ to $P'$.
\end{itemize}
This gives us a new grammar $G' = (N, A, S, P')$.
Finally, drop all unit- and $\epsilon$-productions from $P'$ to get
$P''$.
Then $G'' = (N, A, S, P'')$ is an ``equivalent'' grammar without
unit- or $\epsilon$-productions.
Equivalence is in the sense that
$L(G') = L(G) \setminus \set{\epsilon}$.

\begin{example}
    We apply this to $G_4$ from above.
    The inital grammar is \begin{align*}
        S &\to \texttt{(}S\texttt{)} \mid SS \mid \epsilon.
        \intertext{We can apply the first rule to add the production}
        S &\to \texttt{()}.
    \end{align*}
    There are no more productions to add, so we remove the
    $\epsilon$-production to get the grammar \[
        S \to \texttt{()} \mid \texttt{(}S\texttt{)} \mid SS.
    \]
\end{example}

\begin{exercise}
    Put the following grammar into CNF.
    \begin{align*}
        S &\to aSbb \mid T \\
        T &\to bTaa \mid S \mid \epsilon
    \end{align*}
\end{exercise}
\begin{solution}
    By rule 2, we can add all productions of $S$ to $T$, and vice versa.
    This gives the grammar \begin{align*}
        S, T &\to aSbb \mid T \mid bTaa \mid S \mid \epsilon
    \end{align*}
    By rule 1, we can add the productions \begin{align*}
        S &\to abb \quad
        \text{since } S \to aSbb \text{ and } S \to \epsilon \\
        S &\to baa \quad
        \text{since } S \to bTaa \text{ and } T \to \epsilon
    \end{align*}
    And add both of these to $T$ as well.
    This gives the grammar \begin{align*}
        S, T &\to aSbb \mid T \mid bTaa \mid S \mid \epsilon \mid
        abb \mid baa.
    \end{align*}
    We cannot add any more productions, so we have our grammar $G'$.
    Dropping unit- and $\epsilon$-productions gives us $G''$ as
    \begin{align*}
        S &\to aSbb \mid bTaa \mid abb \mid baa \\
        T &\to aSbb \mid bTaa \mid abb \mid baa
    \end{align*}
    We can obviously omit $T$ and replace it with $S$ to get \begin{align*}
        S &\to aSbb \mid bSaa \mid abb \mid baa
    \end{align*}
\end{solution}

We now prove that the algorithm terminates with a correct grammar.
\begin{proof}[Termination]
    The algorithm terminates because any new production added has an RHS
    that is a subsequence of the RHS of an original production.
    Only finitely many such subsequences exist.
\end{proof}
\begin{proof}[Correctness]
    We first show that $L(G') = L(G)$.
    Let $G_i'$ be the grammar after the $i$th iteration of the algorithm.
    Clearly $L(G_0') = L(G)$.
    It is easy to see that $L(G_{i+1}') = L(G_i')$.
    Thus $L(G') = L(G)$.

    We need to show that $L(G'') = L(G') \setminus \set{\epsilon}$.
    We do this by first showing that for any
    $w \in L(G') \setminus \set{\epsilon}$,
    any minimal length derivation of $w \in G'$ does not use unit- or
    $\epsilon$-productions.

    % Suppose the derivation uses an $\epsilon$-production $Y \to \epsilon$.
    % This cannot be the first step, since otherwise $w = \epsilon$.
    % Let $n$ be the first step at which $Y \to \epsilon$ is used.
    % That is, the $n$th step is
    % $\alpha Y \beta \xRightarrow{1} \alpha \beta$.
    % Now if the $(n-1)$th step is
    % $\alpha' X \beta' \xRigharrow{1} \alpha' \gamma Y \delta \beta'$
    % where $\alpha' \gamma = \alpha$ and $\delta \beta' = \beta$,
    % then we can replace the $(n-1)$th and $n$th steps with
    % $\alpha' X \beta' \to \alpha' \gamma \delta \beta'$.

    % Suppose the previous step was not of this form.
    % Then it was either 
    Suppose the derivation uses an $\epsilon$-production $Y \to \epsilon$.
    Since $w \ne \epsilon$, this cannot be the first step.
    So the derivation looks like \[
        S \xRightarrow{l} \alpha X \beta \xRightarrow{1}
        \alpha \alpha' Y \beta' \beta \xRightarrow{m}
        \gamma Y \delta \xRightarrow{1} \gamma \delta \xRightarrow{n} w
    \] where $\alpha \alpha' \leadsto \gamma$ and
    $\beta' \beta \leadsto \delta$.
    But then we can give a shorter derivation \[
        S \xRightarrow{l} \alpha X \beta \xRightarrow{1} \alpha \alpha'
        \beta' \beta \xRightarrow{m} \gamma \delta \xRightarrow{n} w.
    \] Similarly, a derivation which contains a unit-production \[
        S \xRightarrow{l} \alpha X \beta \xRightarrow{1} \alpha Y \beta
        \xRightarrow{m} \gamma Y \delta \xRightarrow{1} \gamma \phi \delta
        \xRightarrow{n} w
    \] can be shortened to \[
        S \xRightarrow{l} \alpha X \beta \xRightarrow{1} \alpha \phi \beta
        \xRightarrow{m} \gamma \phi \delta \xRightarrow{n} w.
    \]
    This proves that for any $w \in L(G') \setminus \set{\epsilon}$,
    $w \in L(G'')$.
    Thus $L(G') \setminus \epsilon \subseteq L(G'')$ and since
    $P'' \subseteq P'$, $L(G'') \subseteq L(G)$.
    All that remains to be shown is that $L(G'')$ does not contain
    $\epsilon$.

    Since each production in $P''$ (weakly) increases the length of any
    intermediate string, no derivation in $G''$ can produce $\epsilon$.
\end{proof}
