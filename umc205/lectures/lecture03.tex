\lecture{03}{Tue 09 Jan '24}{}
\section{A good way to construct DFAs} \label{sec:dfa:ideas}
Suppose we have to construct a DFA for a language $L$ over an alphabet $A$.
\begin{itemize}
    \item Think of a finite number of properties of strings that you might
    want to keep track of.
    For example, ``number of $a$'s seen so far is even''.
    \item Identify an initial property that is true of the empty string,
    say $p_0$.
    \item Make sure there is a rule to update the properties which are being
    tracked for a string $wa$, based purely on the properties for $w$ and
    the last input $a$.
    \item The properties should imply membership in $L$ or non-membership
    in $L$.
\end{itemize}

\section{DFAs Formally} \label{sec:dfa:formal}
\begin{definition*}[DFA] \label{def:dfa}
    A \emph{deterministic finite-state automaton} $\mathcal{A}$ over an
    alphabet $A$ is a tuple $(Q, s, \delta, F)$ where
    \begin{itemize}
        \item $Q$ is a finite set of \emph{states},
        \item $s \in Q$ is the \emph{start state},
        \item $\delta: Q \times A \to Q$ is the \emph{transition function},
        \item $F \subseteq Q$ is the set of \emph{final states}.
    \end{itemize}
\end{definition*}
For example, the first example in this \namecref{chp:fa} can be written as
\begin{align*}
    A &= \set{a, b} \\
    Q &= \set{e, o} \\
    s &= e \\
    F &= \set{o}
\end{align*} and \begin{align*}
    \delta &= \begin{aligned}
        (e, a) &\mapsto o, & (o, a) &\mapsto e, \\
        (e, b) &\mapsto e, & (o, b) &\mapsto o.
    \end{aligned}
\end{align*}
We further define $\what{\delta} : Q \times A^* \to Q$ as the extension of
$\delta$ to strings.
\begin{align*}
    \what{\delta}(q, \epsilon) &= q \\
    \what{\delta}(q, wa) &= \delta(\what{\delta}(q, w), a)
\end{align*}
\begin{definition*}[Language of a DFA] \label{def:dfa:lang}
    The \emph{language of a DFA} $\mathcal{A}$ is
    \begin{equation*}
        L(\mathcal{A}) = \set{w \in A^* \mid \what{\delta}(s, w) \in F}
    \end{equation*}
\end{definition*}

\section{Regular Languages} \label{sec:dfa:regular}
\vspace{1em}
\begin{definition*}[Regular Language] \label{def:regular}
    A language $L$ is \emph{regular} if there exists a DFA $\mathcal{A}$
    over $A$ such that $L(\mathcal{A}) = L$.
\end{definition*}
For example, the exercises we have done so far.
Another example is \emph{any} finite language.

\begin{theorem*}
    The class of regular languages over an alphabet is countable.
\end{theorem*}
\begin{proof}
    We partition the set of all DFAs over $A$ by their number of states.
    For each $n \in \mathbb{N}$, there are finitely many DFAs with $n$
    states.
    A countable union of finite sets is countable.
    Thus the set of all DFAs over $A$ is countable.
    Since each regular language corresponds to at least one DFA, the set
    of all regular languages over $A$ is countable.
\end{proof}
However, we have seen that there are uncountably many languages over any
alphabet.
This immediately yields the following.
\begin{corollary}
    There are uncountably many languages that are not regular.
\end{corollary}

\begin{theorem*}[Closure under set operations] \label{thm:dfa:regular:closure1}
    The class of regular languages is closed under union, intersection and
    complementation.
\end{theorem*}
\begin{proof}
    For complementation, simply invert the set of final states.
    That is, given $\mathcal{A} = (Q, s, \delta, F)$, let
    $\mathcal{A}' = (Q, s, \delta, Q \setminus F)$.
    Then $L(\mathcal{A}') = A^* \setminus L(\mathcal{A})$, since
    \begin{align*}
        w \in L(\mathcal{A}') &\iff \what{\delta}(s, w) \in Q \setminus F \\
        &\iff \what{\delta}(s, w) \notin F \\
        &\iff w \notin L(\mathcal{A}) \\
        &\iff w \in A^* \setminus L(\mathcal{A})
    \end{align*}

    For intersection and union, define the \emph{product} of two DFAs.
    \begin{definition}[Product] \label{def:dfa:product}
        Given two DFAs $\mathcal{A} = (Q, s, \delta, F)$ and
        $\mathcal{B} = (Q', s', \delta', F')$ over the same alphabet $A$, the
        \emph{product} of $\mathcal{A}$ and $\mathcal{B}$ is \[
            \mathcal{A} \times \mathcal{B} = (Q \times Q', (s, s'), \Delta,
            F \times F')
        \] where $\Delta((q, q'), a) = (\delta(q, a), \delta'(q', a))$.
    \end{definition}
    Note that in the above definition, the extension of $\Delta$ to strings
    $\what{\Delta}$ is given by \[
        \what{\Delta}((q, q'), w) = (\what{\delta}(q, w), \what{\delta'}(q', w))
    \]
    This is easily proved by induction on the length of $w$ (or structural
    induction on $w$).
    \begin{align*}
        \what{\Delta}((q, q'), \epsilon) &= (q, q') \\
        &= (\what{\delta}(q, \epsilon), \what{\delta'}(q', \epsilon)) \\
        \shortintertext{and if}
        \what{\Delta}((q, q'), w) &= (\what{\delta}(q, w), \what{\delta'}(q', w)) \\
        \shortintertext{then}
        \what{\Delta}((q, q'), wa) &= \Delta(\what{\Delta}((q, q'), w), a) \\
        &= \Delta((\what{\delta}(q, w), \what{\delta'}(q', w)), a) \\
        &= (\delta(\what{\delta}(q, w), a), \delta'(\what{\delta'}(q', w), a)) \\
        &= (\what{\delta}(q, wa), \what{\delta'}(q', wa)).
    \end{align*}

    Now let $\mathcal{A}$, $\mathcal{B}$ be DFAs over $A$.
    Then $L(\mathcal{A} \times \mathcal{B})
    = L(\mathcal{A}) \cap L(\mathcal{B})$, since
    \begin{align*}
        w \in L(\mathcal{A} \times \mathcal{B})
            &\iff \what{\Delta}((s, s'), w) \in F \times F' \\
            &\iff (\what{\delta}(s, w), \what{\delta'}(s', w)) \in F \times F' \\
            &\iff \what{\delta}(s, w) \in F \land \what{\delta'}(s', w) \in F' \\
            &\iff w \in L(\mathcal{A}) \land w \in L(\mathcal{B})
    \end{align*}
    Since $X \cup Y = \comp{\comp{X} \cap \comp{Y}}$, closure under union
    follows from closure under complementation and intersection.

    More directly, the DFA
    $(Q \times Q', (s, s'), \Delta, F \times Q' \cup F' \times Q)$
    accepts the language $L(\mathcal{A}) \cup L(\mathcal{B})$.
\end{proof}
