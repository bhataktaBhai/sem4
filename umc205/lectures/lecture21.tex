\lecture{21}{Tue 19 Mar '24}{}
\section{More Decidability Problems} \label{sec:prb:decidability}
\begin{exercise}
    Is it decidable whether a given Turing machine has at least $481$
    states?
    Assume that the TM is given using our standard encoding.
\end{exercise}
\begin{solution}
    Yes.
    Simply check the appropriate section of zeros in the encoding.
\end{solution}

\begin{exercise}
    Is it decidable whether a given Turing machine takes more than $481$
    steps on input $\epsilon$ without halting?
\end{exercise}
\begin{solution}
    Yes.
    Simply simulate the TM for $481$ steps using the universal TM,
    while storing the number of steps on a separate tape.
\end{solution}

\begin{exercise}
    Is it decidable whether a given Turing machine takes more than $481$
    steps on \emph{some} input without halting?
\end{exercise}
\begin{solution}
    Yes.
    If the TM halts on some input $w$ of length more than $481$ in at most
    $481$ steps,
    then it must halt on $w_1w_2\dots w_{481}$ in at most $481$ steps,
    since the read head cannot proceed to any letter beyond this.

    Thus, we can simulate the TM on all inputs of length at most $481$ for
    $481$ steps,
    and reject if it halts on all of them, accepting otherwise.
\end{solution}

\begin{exercise}
    Is it decidable whether a given Turing machine takes more than $481$
    steps on \emph{all} inputs without halting?
\end{exercise}
\begin{solution}
    Yes.
    Again check words of length up to $481$.
    Reject if it halts on any of them, accept otherwise.
\end{solution}

\begin{exercise}
    Is it decidable whether a given Turing machine moves its head more than
    $481$ cells away from the left-end marker, on input $\epsilon$?
\end{exercise}
\begin{solution}
    Yes.
    The number of possible configurations without ever reading the $482$nd
    cell is finite (precisely $n \cdot m^{481} \cdot 482$, where $n$ and $m$
    are the number of states and tape symbols, respectively).
    Simulate the TM for one more step than this number of configurations.

    If it ever reads the $482$nd cell, accept.
    If not, then by the pigeonhole principle it must have entered a loop,
    and we reject.
\end{solution}

\begin{exercise} \label{thm:decidability:epsilon}
    Is it decidable whether a given Turing machine accepts the null-string
    $\epsilon$?
\end{exercise}
\begin{solution}
    No.
    This is equivalent to the halting problem.

    Suppose $N$ is a TM that decides whether a given TM $M$ accepts
    $\epsilon$.
    We can construct a new TM $\mcH$ as follows:
    \begin{enumerate}
        \item On input $\enc(M)\#x$, construct a new TM $P_{M, x}$ that
            erases the input, writes $x$ on the tape,
            and then simulates $M$ on the input $x$.
            Accept if $M$ ever halts on $x$.
        \item Run $N$ on $P_{M, x}$.
        \item If $N$ accepts, then $M$ halts on $x$. \\
        If $N$ rejects, then $M$ does not halt on $x$.
    \end{enumerate}
    Then this \mcH{} is the fabled decider for the halting problem.
\end{solution}

Note that the language of $P_{M, x}$ is \[
    L(P_{M, x}) = \begin{cases}
        A^* & \text{if } M \text{ halts on } x, \\
        \O & \text{otherwise}.
    \end{cases}
\] This will be useful for the coming exercises.

\begin{exercise} \label{thm:decidability:any}
    Is it decidable whether a given Turing machine accepts any string
    at all?
    That is, is $L(M) \ne \O$?

    Similarly, is it decidable whether a given Turing machine accepts
    all strings?
    That is, is $L(M) = A^*$?
\end{exercise}
\begin{solution}
    Neither.
    If either of these were decidable,
    we could decide the halting problem by deciding whether
    $L(P_{M, x}) \ne \O$ or equivalently $L(P_{M, x}) = A^*$.
\end{solution}

\begin{exercise} \label{thm:decidability:finite}
    Is it decidable whether a given Turing machine accepts a finite set?
\end{exercise}
\begin{solution}
    No.
    Again, decide whether $L(P_{M, x})$ is finite.
\end{solution}

\begin{exercise} \label{thm:decidability:regular}
    Is it decidable whether a given Turing machine accepts a regular set?
\end{exercise}
\begin{solution}
    No.
    Construct the Turing machine $Q_{M, x}$ which does the following:
    \begin{itemize}
        \item First, check if $x = a^nb^nc^n$ for some $n$.
            If not, reject.
        \item Then, simulate $M$ on $x$.
            If it halts, accept.
    \end{itemize}
    Then \[
        L(Q_{M, x}) = \begin{cases}
            \set{a^n b^n c^n \mid n \in \N}
                & \text{if } M \text{ halts on } x, \\
            \O
                & \text{otherwise}.
        \end{cases}
    \] Then deciding whether $L(Q_{M, x})$ is regular ammounts to deciding
    whether $M$ halts on $x$.
\end{solution}

\begin{exercise} \label{thm:decidability:context_free}
    Is it decidable whether a given Turing machine accepts a CFL?
\end{exercise}
\begin{solution}
    No.
    Same construction as before.
\end{solution}

\begin{exercise}
    Is it decidable whether a given Turing machine accepts a recursive set?
\end{exercise}
