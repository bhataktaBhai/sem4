\section{Universal Turing Machine} \label{sec:utm}
\lecture{2024-03-14}{}
We can construct a Turing Machine $U$ that takes the encoding of a turing
machine $M$, its input $x$, and simulates $M$ on $x$.
$U$ accepts, rejects, or loops as $M$ does on $x$.

\subsection{Encoding} \label{sec:tm:encoding}
Represent a Turing Machine with
\begin{itemize}
    \item states $\set{1, 2, \dots, n}$,
    \item tape alphabet $\set{1, 2, \dots, m}$,
    \item input alphabet $\set{1, 2, \dots, k}$ (where $k \geq m$),
    \item start, accept and reject states $s, t, r \in [n]$ (distinct),
    \item left end marker $u$ and blank symbol $v \in \set{k + 1, \dots, m}$,
    \item transitions $(i_z, a_z) \to (j_z, b_z, s_z), z \in [l]$
    $L$ and $R$ are taken to be $0$ and $1$ respectively
\end{itemize} as \[
0^n 1 0^m 1 0^k 1 0^s 1 0^t 1 0^r 1 0^u 1 0^v 1 0^{i_1} 1 0^{a_1} 1 0^{j_1} 1 0^{b_1} 1 0^{s_1} \dots 0^{i_l} 1 0^{a_l} 1 0^{j_l} 1 0^{b_l} 1 0^{s_l}.
\]
\begin{exercise}
    Give the Turing machine encoded by \begin{multline*}
        00010000100101001000100010000\ 1\\
        01000101000100\ 1\ 0100100100100\ 1\ 010101010.
    \end{multline*}
\end{exercise}
\begin{solution} \leavevmode
    % The encoding is \begin{multline*}
    %     0^3 1 0^4 1 0^2 1 0^1 1 0^2 1 0^3 1 0^3 1 0^4 \ 1\\
    %     0^1 1 0^3 1 0^1 1 0^3 1 0^2 \ 1\ 
    % \end{multline*}
    \begin{center}
        \begin{tikzpicture}
            \node[state, initial] (1) {1};
            \node[state, accepting, right=of 1] (2) {$t$};
            \node[state, right=of 2] (3) {$r$};
            \path[->]
            (1) edge[loop above] node[above] {$\leftend/\leftend, R$} (1)
            (1) edge node {$b/b, R$} (2)
            (1) edge[loop below] node[below] {$a/a, L$} (1);
        \end{tikzpicture}
    \end{center}
    \vspace{-2em}
\end{solution}

\subsection{How does the Universal Turing Machine work?}
\label{sec:utm:working}
Use $3$ tapes: for the input $M\#x$, for current tape contents of $M$,
and for current state and position of head.

We assume that the input $x$ is written on the tape in the form
$0^{x_1} 1 0^{x_2} 1 \dots 0^{x_k} 1$.
The third tape is in the form $0^q 1 0^n 1$, where $q$ is the current state
and $n$ is the position of the head.

To simulate one step of $M$ on $x$:
\begin{enumerate}
    \item Read the current position $n$ from the third tape.
    Move to that position on the second tape (skip over $n$ many $1$s).
    \item Move to the beginning of the third tape.
    \item Find the transition which matches the current state and the symbol
    under the head.
    \item Write the new state and the new symbol under the head
    on the third and second tapes.
    \item Increment or decrement the position on the third tape.
    \item Compare $q$ against $t$ and $r$, and change state accordingly.
\end{enumerate}

\section{The Halting Problem} \label{sec:hp}
Fix an encoding $\enc$ of Turing Machines.
Define the language \[
    \HP = \set{\enc(M)\#\enc(x) \mid M \text{ halts on } x}.
\] As an example, the string $\enc(M)\#010010 \notin \HP$, where $M$ is the
Turing machine in the previous exercise.

What can we say about $\HP$?
\begin{theorem}
    $\HP$ is recursively enumerable.
\end{theorem}
\begin{proof}
    We have already constructed a universal Turing machine $U$.
    Modify it so that it accepts the input whether it is accepted or
    rejected by $M$.
    Then $U$ accepts $\enc(M)\#\enc(x)$ if and only if $M$ halts on $x$.
\end{proof}

\begin{theorem*}
    $\HP$ is undecidable.
\end{theorem*}
\begin{proof}
    Suppose there is a Turing machine $\mcH$ that decides $\HP$.
    For each binary string $\gamma$, define the Turing machine $M_\gamma$ as
    \begin{itemize}
        \item $M$, if $\gamma = \enc(M)$, and
        \item $M_{loop}$, if $\gamma$ is not a valid encoding,
        where $M_{loop}$ is a Turing machine that loops on every input.
    \end{itemize}

    For all binary strings $\gamma$ and $x$, we can then compute \[
        H(\gamma, x) = [M_\gamma \text{ halts on } x]
    \] by running $\mcH$ on $\enc(M_\gamma)\#x$.
    What if $x$ is not a valid encoding? We deem $H(\gamma, x) = 0$.

    We can then define a new Turing machine $N$ as follows:
    \begin{enumerate}
        \item For a binary input $x$, transform it to $\enc(M_x)\#x$.
        \item Run $\mcH$ on this transformed input.
        \item If $\mcH$ accepts, then reject $w$.
            If $\mcH$ rejects, then accept $w$.
    \end{enumerate}

    Then $N$ halts on $x$ if and only if $M_x$ does not halt on $x$.
    Thus for all $x$, $N \ne M_x$.
    But $N$ must occur in the list of all Turing machines, so we have a
    contradiction.

    In particular, $N$ halts on $\enc(N)$ iff $M_{\enc(N)}$
    does not halt on $\enc(N)$.
    This is a contradiction.
\end{proof}

\begin{corollary*}
    The language $\neg\HP$ is not even recursively enumerable.
\end{corollary*}
We first prove the following lemma.
\begin{lemma}
    If $A$ and $\wbar{A}$ are both recursively enumerable, then $A$ is
    recursive.
\end{lemma}
\begin{proof}
    Let $M$ and $\wbar{M}$ be Turing machines that enumerate $A$ and
    $\wbar{A}$ respectively.
    Construct the Turing machine $N$ with two tapes, that simulates $M$ on
    one and $\wbar{M}$ on the other, alternating between the two.
    Eventually, one of these must halt, and $N$ will make the appropriate
    decision.
\end{proof}

\begin{proof}[Proof of Corollary]
    Suppose $\neg\HP$ is recursively enumerable.
    Then $\HP$ and $\neg\HP$ are both recursively enumerable.
    By the lemma, $\HP$ is recursive, which is a contradiction.
\end{proof}
