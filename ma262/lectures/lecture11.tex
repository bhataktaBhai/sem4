\chapter{Branching Processes} \label{chp:branching}
\lecture{2024-02-08}{Branching Processes}
\begin{definition*}[Branching process] \label{def:branching_process}
    Let $\ubar{p} = (p_i)_{i \in \N}$ be a probability distribution on $\N$ and let
    $X_{n, i} \sim \ubar{p}$ be iid random variables for each $n, i \in \N$.
    We define the \emph{branching process} $(Z_n)_{n \in \N}$ by
    \[
        Z_0 = 1, \quad Z_{n+1} = \sum_{i=1}^{Z_n} X_{n, i}.
    \]
    $\ubar{p}$ is called the \emph{offspring/progeny distribution}.
    The associated random tree (the $i$th node on the $n$th level, if it
    exists, having $X_{n, i}$ children) is called the
    \emph{Galton-Watson tree} or the \emph{Bienaym\'e tree}.
\end{definition*}

Clearly $Z_{n+1}$ depends only on $Z_n$, so the process is a Markov chain.
\begin{definition}[Extinction] \label{def:extinction}
    The \emph{extinction probability} of a branching process
    $(Z_n)_{n \in \N}$ is \[
        \eta = \Pr(Z_n = 0 \text{ for some } n \in \N).
    \]
\end{definition}

\begin{proposition*}[Expectation] \label{thm:expectation}
    Let $\mu = \E_{X \sim \ubar{p}}[X] = \E[Z_1]$.
    Then $\E[Z_n] = \mu^n$.
\end{proposition*}
\begin{proof}
    By induction, $\E[Z_0] = 1 = \mu^0$.
    Then \[
        \E[Z_{n+1} \given Z_n]
            = \E\left[\sum_{i=1}^{Z_n} X_{n, i} \given Z_n\right]
            = Z_n \E[X_{n, 1}]
            = Z_n \mu.
    \] So \[
        \E[Z_{n+1}]
            = \E[\E[Z_{n+1} \given Z_n]]
            = \E[Z_n \mu]
            = \mu \E[Z_n]
    \] and by induction follows the result.
\end{proof}

\begin{proposition}
    If $\E[Z_1] < 1$, then the process becomes extinct with probability 1.
\end{proposition}
\begin{proof}
    Markov's inequality gives \[
        \Pr(Z_n \geq 1) \leq \E[Z_n] = \mu^n.
    \] So \[
        \lim_{n \to \infty} \Pr(Z_n \ne 0) = 0. \qedhere
    \]
\end{proof}

\begin{theorem*}
    Consider a branching process with $p_1 < 1$.
    Let $G$ be the pgf of $Z_1$.
    Then the extinction probability $\eta$ is the smallest solution to
    $G(s) = s$ in $[0, 1]$.

    Further, \begin{align*}
        \eta &= 1 \text{ if } \E[Z_1] \leq 1, \\
        \eta &< 1 \text{ if } \E[Z_1] > 1.
    \end{align*}
\end{theorem*}
\begin{proof}
    Let $A$ be the event that the branching process goes extinct.
    Then \[
        \eta = \Pr(A) = \E[\Pr(A \given Z_1)] = \E[\eta^{Z_1}] = G(\eta).
    \] Thus the extinction probability is a fixed point of $G$.
    The second last equality is because \[
        \Pr(A \given Z_1 = k) =
            \Pr\set{\text{each of the $k$ children goes extinct}},
    \] which are independent events.

    Let $\eta_n \coloneq \Pr(Z_n = 0)$.
    Since $\set{Z_n = 0} \subseteq \set{Z_{n+1} = 0} \subseteq A$,
    $\eta_n \le \eta_{n+1} \le \eta$
    and $\eta_n \uparrow \eta$.
\end{proof}

\begin{definition}[Criticality] \label{def:criticality}
    An offspring distribution is called \emph{critical} if $\E[Z_1] = 1$.
    It is called \emph{subcritical} (resp. \emph{supercritical}) if
    $\E[Z_1] < 1$ (resp. $\E[Z_1] > 1$).
\end{definition}
