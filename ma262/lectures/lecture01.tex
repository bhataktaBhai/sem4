\lecture{01}{Thu 04 Jan '24}{}

\setcounter{section}{-1}
\section{The Course} \label{sec:course}
\textbf{Texts:} \begin{itemize}
    \item \textit{Markov Chains}, J. R. Norris
    \item \textit{Introduction to Stochastic Processes}, Hoel, Port, Stone
    \item Karlin and Taylor
\end{itemize}

\textbf{Grading:} \begin{itemize}
    \item[(20\%)] 2 quizzes
    \item[(30\%)] 1 midterm
    \item[(50\%)] Final
\end{itemize}

\section{Discrete time Markov Chains} \label{sec:dtmc}
\begin{definition}
    Let $S$ be a state set (at most countable).
    A matrix $P = (p_{xy}; x, y \in S)$ is called a \emph{stochastic matrix}
    if $p_{xy} \ge 0$ for all $x, y \in S$ and $\sum_{y \in S} p_{xy} = 1$
    for all $x \in S$.
\end{definition}

\begin{definition}
    Let $S$ be a state set, $P = (p_{xy})$ a stochastic matrix, and $\mu_0$
    a probability distribution on $S$, \ie, $\mu_0(x) \ge 0$ for all
    $x \in S$ and $\sum_{x \in S} \mu_0(x) = 1$.

    Suppose $X_0, X_1, \dots$ are random variables defined on the same
    probability space taking values in $S$.
    Then $(X_n; n \ge 0)$ is called a Markov chain with initial distribution
    $\mu_0$ and transition matrix $P$, denoted $MC(\mu_0, P)$, if $X_0$ has
    distribution $\mu_0$ and for all $n \ge 0$, \[
        P(X_{n+1} = x_{n+1} \mid X_n = x_n, \dots, X_0 = x_0)
            = p_{x_n x_{n+1}}
    \] whenever $P(X_n = x_n, \dots, X_0 = x_0) > 0$.
\end{definition}
\begin{notation}
    Whenever writing $P(X_n \in A \mid (X_0, \dots, X_{n-1}) \in B)$, it is
    understood that only $P((X_0, \dots, X_{n-1}) \in B) > 0$ is considered.
\end{notation}

\begin{theorem}
    $(X_n; 0 \le n \le N)$ is $MC(\mu_0, P)$ iff \[
        P(X_0 = x_0, \dots, X_N = x_N) = \mu_0(x_0) p_{x_0 x_1} \dots p_{x_{N-1} x_N}
    \] for all $x_0, \dots, x_N \in S$.
\end{theorem}
\begin{proof}
    Both directions are proven by induction.

    Suppose $(X_n; 0 \le n \le N)$ is $MC(\mu_0, P)$.
    $P(X_0 = x_0) = \mu_0(x_0)$.
    If $P(X_0 = x_0) > 0$, then
    $P(X_0 = x_0, X_1 = x_1) = \mu_0(x_0) p_{x_0 x_1}$.
    If $P(X_0 = x_0) = 0$, then $P(X_0 = x_0, X_1 = x_1) \le P(X_0 = x_0) = 0$,
    and so $P(X_0 = x_0, X_1 = x_1) = 0 = \mu_0(x_0) p_{x_0 x_1}$.

    \textbf{Induction:} Suppose \[
        P_j \coloneq P(X_0 = x_0, \dots, X_j = x_j)
        = \mu_0(x_0) p_{x_0 x_1} \dots p_{x_{j-1} x_j}.
    \] If this is zero, so is $P_{j+1}$, and so it is equal to
    $\mu_0(x_0) p_{x_0 x_1} \dots p_{x_{j-1} x_j} p_{x_j x_{j+1}}$. \\
    If not, then \begin{align*}
        P_{j+1}
            &= P_j P(X_{j+1} = x_{j+1} \mid X_0 = x_0, \dots, X_j = x_j) \\
            &= P_j p_{x_j x_{j+1}} \\
            &= \mu_0(x_0) p_{x_0 x_1} \dots p_{x_{j-1} x_j} p_{x_j x_{j+1}},
    \end{align*} closing the induction.
    In particular, \[
        P(X_0 = x_0, \dots, X_N = x_N) = \mu_0(x_0) p_{x_0 x_1} \dots
        p_{x_{N-1} x_N}.
    \]

    Now suppose \[
        P(X_0 = x_0, \dots, X_N = x_N)
        = \mu_0(x_0) p_{x_0 x_1} \dots p_{x_{N-1} x_N}
    \]
    for all $x_0, \dots, x_N \in S$.
    Then for any $x_0, \dots, x_{N-1} \in S$,
    \begin{align*}
        P(X_0 = x_0, \dots, X_{N-1} = x_{N-1})
            &= \sum_{x_N \in S} P(X_0 = x_0, \dots, X_N = x_N) \\
            &= \sum_{x_N \in S} \mu_0(x_0) p_{x_0 x_1} \dots 
                p_{x_{N-2} x_{N-1}} p_{x_{N-1} x_N} \\
            &= \mu_0(x_0) p_{x_0 x_1} \dots p_{x_{N-2} x_{N-1}}.
    \end{align*}
    We have by backwards induction that for all $1 \le i \le N$, \[
        P(X_0 = x_0, \dots, X_i = x_i)
        = \mu_0(x_0) p_{x_0 x_1} \dots p_{x_{i-1} x_i}
    \] and $P(X_0 = x_0) = \mu_0(x_0)$.
    This allows us to deduce that \[
        P(X_{i+1} = x_{i+1} \mid X_0 = x_0, \dots, X_i = x_i)
        = p_{x_i x_{i+1}}
    \] by definition of conditional probability.
\end{proof}

\begin{theorem}[Strong Law of Large Numbers] \label{thm:dtmc:slln}
    Suppose $Z_1, Z_2, \dots$ are iid \R-valued random variables and $E[Z_1]$
    exists.
    Then \[
        \frac{Z_1 + \dots + Z_n}{n} \to E[Z_1]
    \] as $n \to \infty$, that is, \[
        P\left\{\omega \in \Omega : \lim_{n \to \infty}
        \frac{Z_1(\omega) + \dots + Z_n(\omega)}{n} = E[Z_1]\right\} = 1.
    \]
\end{theorem}
\begin{theorem}[Weak Law of Large Numbers] \label{thm:dtmc:wlln}
    
\end{theorem}
\begin{theorem}[Central Limit Theorem] \label{thm:dtmc:clt}
    Suppose $Z_1, Z_2, \dots$ are iid \R-valued random variables and $E[Z_1^2]$
    exists.
    Then \[
        \frac{\sqrt n}{\sqrt{V(Z_1)}} \left(\frac{Z_1 + \dots + Z_n}{n}
        - E[Z_1]\right) \stackrel{d}{\to} N(0, 1).
    \]
\end{theorem}
\begin{examples}
    \item A two-state Markov chain.
    \begin{center}
        \begin{tikzpicture}
            \node[state] (1) {$1$};
            \node[state, right=of 1] (2) {$2$};
            \path
                (1) edge[loop left] node {$p$} (1)
                (2) edge[loop right] node {$q$} (2)
                (1) edge[bend left, above] node {$1-p$} (2)
                (2) edge[bend left, below] node {$1-q$} (1);
        \end{tikzpicture}
    \end{center}
    This corresponds to the matrix \[
        P = \begin{pmatrix}
            p & 1-p \\
            1-q & q
        \end{pmatrix}.
    \]
    \item A three-state Markov chain.
    \begin{center}
        \begin{tikzpicture}
            \node[state] (1) {$1$};
            \node[state, above right=of 1] (2) {$2$};
            \node[state, below right=of 2] (3) {$3$};
            \path
                (1) edge node[below] {$\frac12$} (2)
                (2) edge node[below] {$\frac14$} (3)
                (3) edge node[above] {$\frac14$} (1)
                (1) edge[bend right] node[below] {$\frac12$} (3)
                (3) edge[bend right] node[above] {$\frac34$} (2)
                (2) edge[bend right] node[above] {$\frac12$} (1);
            \path
                (2) edge[loop above] node {$\frac14$} (2);
        \end{tikzpicture}
    \end{center}
    This has transition matrix \[
        P = \begin{pmatrix}
            1/2 & 1/2 & 0 \\
            1/2 & 1/4 & 1/4 \\
            1/4 & 3/4 & 0
        \end{pmatrix}.
    \]
    \item Simple random walk on $\Z$.
    Staring from $0$, at each step, move right with probability $p$ and left
    with probability $q = 1-p$.
    $P(X_{n+1} = x + 1 \mid X_n = x) = p$ and
    $P(X_{n+1} = x - 1 \mid X_n = x) = q$.
    All other probabilities are $0$.

    Such a simple random walk is called symmetric if $p = q = \frac12$.
    A special case is where $\mu_0 = \delta_x$ for some $x \in \Z$, where
    $\delta_x$ is the Kr\"onecker delta.

    Aside: Suppose $Z_1, \dots, Z_k$ are random variables taking values in a
    state set $S$ defined on a probability space $(\Omega, \mathcal F, P)$.
    and $\tilde{Z}_1, \dots, \tilde{Z}_k$ are rvs taking values in a
    state set $S$ defined on a probability space
    $(\tilde{\Omega}, \tilde{\mathcal F}, \tilde{P})$.
    Then $(Z_1, \dots, Z_k)$ and $(\tilde{Z}_1, \dots, \tilde{Z}_k)$ are said to
    be identically distributed if \[
        P(Z_1 = x_1, \dots, Z_k = x_k)
        = P(\tilde{Z}_1 = x_1, \dots, \tilde{Z}_k = x_k).
    \] This is notated as \[
        (Z_1, \dots, Z_k) \stackrel{d}{=} (\tilde{Z}_1, \dots, \tilde{Z}_k).
    \]

    Suppose that $Y_1, Y_2, \dots$ are iid with distribution
    $\begin{pmatrix}
        1 & -1 \\
        p & 1-p
    \end{pmatrix}$.
    We have that $(X_n; n \ge 0) \stackrel{d}{=} (\sum_{j=1}^{n} Y_j; n \ge 0)$.
    Then from the weak law of large numbers, \[
        \frac{X_n}{n} \to E[Y_1] = 2p - 1.
    \] From the central limit theorem, \[
        \frac{X_n - n(p - q)}{\sqrt{n} \sqrt{1 - (p-q)^2}} \stackrel{d}{\to}
        N(0, 1).
    \]

    On a graph, a simple symmetric random walk is a random walk on a graph
    where each \[
        p_{xy} = \begin{cases}
            \frac1{\mathrm{deg}(x)} & \text{if } y \sim x, \\
            0 & \text{otherwise}.
        \end{cases}
    \]

    On $\Z^2$, a simple random walk is given by $p_N$, $p_E$, $p_S$, $p_W$,
    where $p_N + p_E + p_S + p_W = 1$.
    At each step, move up with probability $p_N$, right with probability $p_E$,
    down with probability $p_S$, and left with probability $p_W$.

    \item Consider a shooting game with 4 modes: $N$ (normal), $D$ (distance),
    $W$ (windy) and $DW$ (distance and windy).
    The game changes mode randomly to a mode different from the current mode
    with direcred graph $K_4$ with some edge weights.
\end{examples}

\begin{theorem}
    If $(X_n; n \ge 0)$ is a DTMC with transition matrix $P$, then \[
        P_{\mu_0}(X_n = y) = (\mu_0 P^n)_y.
    \] In particular, $P_x(X_n = y) = (P^n)_{x,y} = p_{xy}^{(n)}$.
\end{theorem}
Here, $\mu_0$ is viewed as a row vector, and $P_{\mu_0}$ is the distribution
under the assumption that $X_0 \sim \mu_0$.
Also, $P_x$ is under the assumption that $\mu_0 = \delta_x$.
\begin{proof}
    \begin{align*}
        P_{\mu_0}(X_n = y)
            &= \sum_{\substack{x_j \in S \\ 0 \le j < n}}
                P(X_0 = x_0, \dots, X_{n-1} = x_{n-1}, X_n = y) \\
            &= \sum_{\substack{x_j \in S \\ 0 \le j < n}}
                (\mu_0)_{x_0} p_{x_0 x_1} \dots p_{x_{n-1} y} \\
            &= (\mu_0 P^n)_y \qedhere
    \end{align*}
\end{proof}

\begin{theorem}
    Let $(X_n; n \ge 0)$ be $MC(\mu_0, P)$.
    Then for any $n \ge 0$, $l \ge 1$, $x_n, \dots, x_{n+l} \in S$ and
    $A \subseteq S^n$, \begin{multline*}
        P_{\mu_0}(X_i = x_i, n < i \le n + l \mid X_n = x_n, (X_0, \dots, X_{n-1}) \in A) \\
        = P_{x_n}(X_1 = x_{n+1}, \dots, X_l = x_{n+l})
    \end{multline*}
\end{theorem}
In other words, conditioning on $X_n = x_n$ and $(X_0, \dots, X_{n-1}) \in A$,
the process $(X_n, X_{n+1}, \dots)$ is $MC(\delta_{x_n}, P)$.
\begin{proof}
    \begin{align*}
        P(X_{n+l} = x_{n+l}, &\dots, X_n = x_n, (X_0, \dots, X_{n-1}) \in A) \\
            &= p_{x_{n} x_{n+1}} \dots p_{x_{n+l-1} x_{n+l}}
            \sum_{(x_0, \dots, x_{n-1}) \in A} \mu_0(x_0) p_{x_0 x_1} \dots p_{x_{n-1} x_n} \\
            &= P_{\mu_0}(X_n = x_n, (X_0, \dots, X_{n-1}) \in A) p_{x_n x_{n+1}} \dots p_{x_{n+l-1} x_{n+l}} \\
    \end{align*}
\end{proof}
