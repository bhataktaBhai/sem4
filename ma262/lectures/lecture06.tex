\section{Stationary distributions} \label{sec:inv}
Let $\mcS$ be a countable state space and $P$ a transition matrix on $\mcS$.
Then any function $\lambda\colon \mcS \to [0, \infty]$ corresponds to a
measure on $\mcS$, by setting \[
    \lambda(A) = \sum_{x \in A} \lambda(x).
\] Conversely, any measure on $\mcS$ is of this form.
(\textcolor{Red}{Why?}) %TODO

A measure $\lambda$ can be thought of as a row vector
$(\lambda(x))_{x \in \mcS}$.
Then \[
    (\lambda P)_y = \sum_{x \in \mcS} \lambda(x) P(x, y)
\] also gives a measure on $\mcS$.

A \emph{probability distribution} $\pi$ on $\mcS$ is a measure such that
$\pi(\mcS) = 1$.

\begin{definition*} \label{def:inv:measure}
    A measure $\lambda$ on $\mcS$ is \emph{invariant} for a DTMC with
    transition matrix $P$ is $\lambda P = \lambda$.
    That is, \[
        \sum_{x \in \mcS} \lambda(x) P(x, y) = \lambda(y)
            \quad \forall y \in \mcS.
    \]
    If $\lambda$ is a probability distribution, it is called a
    \emph{stationary distribution}.
\end{definition*}

\begin{proposition}
    If $(X_n)_{n \in \N}$ is $MC(\pi, P)$ where $\pi$ is a stationary
    distribution, then \[
        X_0 \eqd X_1 \eqd \dots \sim \pi.
    \]
\end{proposition}
