\lecture{2024-02-13}{Structure of GWB Trees}
\section{The Structure of GWB Trees}
\begin{definition}[Conjugate distribution] \label{def:conjugate}
    Let $\ubar{p} = (p_i)_{i \in \N}$ be a pmf on $\N$ with $p_0 > 0$
    (hence $\eta > 0$).
    Then $\ubar{\tilde{p}} = (\eta^{i-1} p_i)_{i \in \N}$ is called the
    \emph{conjugate distribution} of $\ubar{p}$.
\end{definition}

\begin{exercise}
    Show that $\ubar{\tilde{p}}$ is a pmf.
\end{exercise}
\begin{solution}
    \begin{align*}
        \sum_{i=1}^\infty \eta^{i-1} p_i
            &= \frac1{\eta} \sum_{i=1}^\infty \eta^i p_i \\
            &= \frac1{\eta} \E_{X \sim \ubar{p}}[\eta^X] \\
            &= \frac1{\eta} G(\eta) \\
        \intertext{but $G(\eta) = \eta$}
            &= 1.
    \end{align*}
    For a more intuitive proof, recall that $G(\eta) = \eta$ \emph{because}
    $\E[\eta^Z_1] = \eta$.
\end{solution}

\begin{exercise}
    Show that $\ubar{\tilde{p}}$ is a critical or subcritical offspring
    distribution.
\end{exercise}
\begin{proof}
    If $\eta = 1$, then $\ubar{\tilde{p}} = \ubar{p}$.
    But $\eta = 1 \iff \iff \E[Z_1] \le 1$.
    Thus $\ubar{\tilde{p}}$ is critical.

    Suppose $\eta < 1$, so that $p_0 + p_1 < 1$
    (otherwise the process would die off, since each node produces either
    one child, or, with positive probability, no children.
    Alternatively, $\E[Z_1] < 1$).
    Then \begin{align*}
        \E[Z_1] &= \sum k \eta^{k-1} p_k \\
                &= \dv{}{\eta} \E[\eta^X] \\
                &= G'(\eta)
    \end{align*}
    If $G'(\eta) > 1$, then $G'(s) > 1 \;\forall s \in (\eta, 1)$.
    Why? Because $p_0 + p_1 < 1$, so $G$ is strictly convex.
    But $G(\eta) = \eta$, so there exists a $\xi \in (\eta, 1)$ such that
    $G'(\xi) = 1$, a contradiction.
    Thus $G'(\eta) \le 1$ so $\E[Z_1] \le 1$.
\end{proof}

\begin{theorem*}
    Let $\ubar{\tilde{p}}$ be the conjugate distribution of $\ubar{p}$.
    Let $\mcT_{\ubar p}$ and $\mcT_{\ubar{\tilde p}}$
    be the GWB trees with offspring distributions
    $\ubar{p}$ and $\ubar{\tilde{p}}$ respectively.

    Then \[
        (\mcT_{\ubar{p}} \given \text{it is finite})
            \eqd \mcT_{\ubar{\tilde{p}}}.
    \]
\end{theorem*}

\begin{exercise}
    Find the conjugate distribution of $Bin(r, p)$ where
    $p \in (\frac1r, 1)$.
\end{exercise}

\begin{definition*}[Breadth-first walk] \label{def:bfs}
    Let $\ubar t$ be a plane (rooted) tree.
    Label its vertices $1, 2, \dots, n$ in breadth-first order.
    Let $C_j(\ubar t)$ be the number of children of vertex $j$ in $\ubar t$.
    Then the \emph{breadth-first walk} of $\ubar t$ is the sequence
    \begin{align*}
        S_j(\ubar t) &= \begin{cases}
            1 & \text{if } j = 0, \\
            S_{j-1} + C_j(\ubar t) - 1 & \text{otherwise}.
        \end{cases} \\
        &= 1 + \sum_{i=1}^j (C_i(\ubar t) - 1).
    \end{align*}
\end{definition*}
It is obvious that $S_n(\ubar t) = 0$.

\begin{theorem*}
    There exists a bijection between the set of finite plane trees and the
    set $\mcS$ of sequences $(s_n)_{n \in \N}$ such that
    \begin{itemize}
        \item $s_0 = 1$,
        \item $s_{n+1} \ge s_n - 1$,
        \item there is an $n_0$ such that $s_n = 0 \iff n \ge n_0$.
    \end{itemize}
    This bijection is such that each tree corresponds to its breadth-first
    walk.
\end{theorem*}
