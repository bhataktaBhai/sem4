\lecture{2024-03-13}{}
\begin{theorem} \label{thm:cont:compact_uniform}
    A continuous function on a compact set is uniformly continuous.
\end{theorem}
\begin{proof}
    Let $f\colon X \to Y$ be continuous where $X$ is compact.

    Let $\varepsilon > 0$ be arbitrary.
    For each $x \in X$, choose $\delta_x > 0$ such that
    $f(B_X(x; \delta_x)) \subseteq B_Y(f(x); \varepsilon)$.
    The collection $\set{B_X(x; \frac12 \delta_x)}_{x \in X}$ is an open
    cover of $X$.
    Since $X$ is compact, only finitely many of these are needed to cover
    $X$.
    Label the centers $x_1, \dots, x_n$ and the corresponding radii
    $\frac{\delta_1}{2}, \dots, \frac{\delta_n}{2}$.
    Let $\delta > 0$ be the smallest of these radii.

    Let $p, q \in X$ be such that $d_X(p, q) < \delta$.
    Then there exists an $i$ such that $d_X(p, x_i) < \frac{\delta_i}{2}$.
    But $d_X(p, q) < \delta < \frac{\delta_i}{2}$,
    so $d_X(q, x_i) < \delta_i$ by the triangle inequality.

    Thus $f(p)$ and $f(q)$ are both at most $\varepsilon$ away from
    $f(x_i)$, so they are at most $2\varepsilon$ away from each other.

    Thus for any $2\varepsilon > 0$, we have produced a $\delta > 0$ such
    that any points within $\delta$ of each other are mapped to points
    within $2\varepsilon$ of each other.
\end{proof}

\begin{theorem} \label{thm:cont:connected}
    Continuous functions map connected sets to connected sets.
\end{theorem}
\begin{proof}
    Let $X$ be connected and $f\colon X \to Y$ continuous.
    Suppose $f(X)$ is not connected.
    That is, there exist nonempty sets $A, B \subseteq f(X)$ such that
    $A \cup B = f(X)$ and $\wbar{A} \cap B = A \cap \wbar{B} = \O$.

    Let $A_* = f^{-1}(A)$ and $B_* = f^{-1}(B)$.
    They are nonempty since $A$ and $B$ are nonempty and in the range.
    Also,
    $X = f^{-1}(f(X)) = f^{-1}(A \cup B) = f^{-1}(A) \cup f^{-1}(B)$.
    So $A_*$ and $B_*$ cover $X$.

    Recall that $f$ is continuous implies that the preimage of a closed
    set is closed.
    $A_* = f^{-1}(A) \subseteq f^{-1}(\wbar{A})$, so
    $\wbar{A_*} \subseteq f^{-1}(\wbar{A})$.

    Then \begin{align*}
        \wbar{A_*} \cap B_* &\subseteq f^{-1}(\wbar{A}) \cap f^{-1}(B) \\
        &\subseteq f^{-1}(\wbar{A} \cap B) \\
        &= f^{-1}(\O) = \O.
    \end{align*}
    Similarly, $A_* \cap \wbar{B_*} = \O$.

    Thus $A_*$ and $B_*$ are a separation of $X$, contradicting the
    connectedness of $X$.
\end{proof}

\section{Discontinuities of functions on \R} \label{sec:cont:dis}
\begin{examples}
    \item The \emph{Heaviside function} defined by \[
        H(x) = \begin{cases}
            1 & \text{if } x \ge 0, \\
            0 & \text{if } x < 0.
        \end{cases}
    \]
    \item The \emph{sign function} defined by \[
        \sgn(x) = \begin{cases}
            \hphantom{-}1  & \text{if } x > 0, \\
            \hphantom{-}0  & \text{if } x = 0, \\
            -1 & \text{if } x < 0.
        \end{cases}
    \]
    \item \[
        x \mapsto \begin{cases}
            \frac{3x - x^2}{x(x^2 + 2)} & \text{if } x \ne 0, \\
            0 & \text{if } x = 0.
        \end{cases}
    \]
    \item \[
        x \mapsto \begin{cases}
            \frac1{x} & \text{if } x \ne 0, \\
            0 & \text{if } x = 0.
        \end{cases}
    \]
    \item \[
        x \mapsto \begin{cases}
            \sin\frac1x & \text{if } x \ne 0, \\
            0 & \text{if } x = 0.
        \end{cases}
    \]
    \item The \emph{Dirichlet function} \[
        \bm{1}_{\Q}(x) = \begin{cases}
            1 & \text{if } x \in \Q, \\
            0 & \text{otherwise}.
        \end{cases}
    \] This is discontinuous \emph{everywhere}.
\end{examples}

\begin{definition*}
    Given $f\colon (a, b) \to \R$ and $c \in (a, b)$, we say that
    \begin{enumerate}
        \item $f$ has a \emph{simple discontinuity} or a \emph{discontinuity
            of the first kind} at $c$ if the left-hand and right-hand limits
            $\lim_{x \to c^-} f(x)$ and $\lim_{x \to c^+} f(x)$ exist but
            are either unequal, or unequal to $f(c)$.
        \item $f$ has a \emph{discontinuity of the second kind} at $c$ if
            either the left-hand or right-hand limit does not exist.
    \end{enumerate}
\end{definition*}

The first three examples above have simple discontinuities at $0$.
The last three have discontinuities of the second kind.
The third example has a \emph{removable} discontinuity at $0$, since
both one-sided limits exist and are equal.

\begin{theorem} \label{thm:discont:monotone}
    Monotone functions do not have discontinuities of the second kind.
\end{theorem}
