\begin{definition}[Ternary expansion] \label{def:ternary_expansion}
\lecture{2024-01-31}{The Cantor set}
    Let $x \in [0, 1]$.
    A \emph{ternary expansion} of $x$ is a sequence $(d_1, d_2, \dots)
    \subseteq \set{0, 1, 2}$ such that \[
        x = \sup \set{D_k = \sum_{j=1}^{k-1} \frac{d_j}{3^j} : k \ge 1}
    \] which is equivalent to \[
        \sum_{j=1}^{\infty} \frac{d_j}{3^j} = x
    \] We write $x = 0.d_1d_2d_3\dots$ to denote this.
\end{definition}
\begin{example}
    For $x = \frac13$, we have both $x = 0.1000\dots$ and
    $x = 0.0222\dots$, so ternary expansions are not unique.
\end{example}

Let $I_0 = [0, \frac13]$, $I_1 = [\frac13, \frac23]$ and
$I_2 = [\frac23, 1]$.
Let $x \in [0, 1]$.
Choose $d_1 = j$ such that $x \in I_j$ (in ambiguous cases,
pick any one).
Then \begin{align*}
    x \in \left[\frac{d_1}{3}, \frac{d_1 + 1}{3}\right] \\
    \implies 0 \le x - \frac{d_1}{3} \le \frac13 \\
    \implies D_1 \le x \le D_1 + \frac13
\end{align*}
Let $I_{j 0}, I_{j 1}, I_{j 2}$ be the subdivisions of $I_j$.
Choose $d_2 = l$, where $x \in I_{jl}$ iff \begin{align*}
    x \in \left[\frac{d_1}{3} + \frac{d_2}{9},
        \frac{d_1}{3} + \frac{d_2 + 1}{9}\right] \\
    \implies D_2 \le x \le D_2 + \frac19
\end{align*}
How do we break ties?
\begin{description}
    \item[Scheme A] If at the $k^{\text{th}}$ state, $x \in [0, 1)$ is
    an endpoint of 2 intervals, pick the right interval.
    This gives a unique expansion.
    That is, pick $d_k$ such that $D_k \le x < D_k + \frac13$.
    \item[Scheme B] For $x \in (0, 1]$, always pick the left interval.
    That is, pick $d_k$ such that $D_k < x \le D_k + \frac13$.
\end{description}
\begin{table}
    \centering
    \begin{tabular}{c|c|c|c|c|c|c}
        \toprule
        & $0$ & $1$ & $\frac13$ & $\frac23$ & $\frac49$ & $\frac12$ \\
        \midrule
        A & $0.000\dots$ & & $0.100\dots$ & $0.200\dots$ &
            $0.1100\dots$ & $0.111\dots$ \\
        B & & $0.222\dots$ & $0.022\dots$ & $0.122\dots$ &
            $0.1022\dots$ & $0.111\dots$ \\
        \bottomrule
    \end{tabular}
    \caption{Scheme A vs Scheme B}
    \label{tab:ternary_expansion}
\end{table}
We make the following observations:
\begin{itemize}
    \item Ambiguity only occurs at endpoints of ``middle thirds''.
    \item Say $x$ is an endpoint of a middle third.
    Let $k$ be the first stage where ambiguity occurs.
    Then if $x$ is the left endpoint, scheme A gives
    $x = 0.d_1d_2\dots d_{k-1}1000\dots$ and scheme B gives
    $x = 0.d_1d_2\dots d_{k-1}0222\dots$.
    If $x$ is the right endpoint, scheme A gives
    $x = 0.d_1d_2\dots d_{k-1}2000\dots$ and scheme B gives
    $x = 0.d_1d_2\dots d_{k-1}1222\dots$.

    Note that this ambiguity can be resolved by a scheme C, which picks
    the expansion which has no $1$ starting from the point of ambiguity.
\end{itemize}

\begin{theorem*} \label{thm:cantor_set}
    There exists a non-empty $E \subseteq [0, 1]$ such that
    \begin{enumerate}
        \item $E$ is compact.
        \item $E = \set{\text{limit points of } E}$.
        \item $E^\circ = \wbar{E}^\circ = \O$.
        \item $E$ is uncountable.
    \end{enumerate}
\end{theorem*}
\begin{proof}
    \begin{align*}
        E = \{x \in [0, 1] : x \text{ admits at least one ternary} \\
        \text{expansion with only $0$'s and $2$'s}\}
    \end{align*}
    We can construct this set by removing the middle thirds.
    \begin{align*}
        E_0 &= [0, 1] \\
        E_1 &= E_0 \setminus \left(\frac13, \frac23\right) \\
           &= \left[0, \frac13\right] \cup \left[\frac23, 1\right] \\
        E_2 &= E_1 \setminus \left[\left(\frac19, \frac29\right) \cup
            \left(\frac49, \frac59\right) \cup \left(\frac79, \frac89\right)\right] \\
        E_m &= E_{m-1} \setminus \bigcup_{k=0}^{3^{m-1} - 1}
            \left(\frac{3k + 1}{3^m}, \frac{3k + 2}{3^m}\right)
    \end{align*}
    We claim that $E = \bigcap_{m=1}^\infty E_m$ satisfies the conditions of
    the theorem.
    First note that $E_1 \subseteq E_2 \subseteq \dots$,
    so for any $m_1 < m_2 < \dots < m_k$,
    $\bigcap_{i = 1}^k E_{m_i} = E_{m_k}$ is non-empty.
    By \cref{thm:compact:intersection}, $E$ is non-empty

    Since $E$ is the intersection of closed sets, $E$ is closed.
    Since $E$ is bounded, $E$ is compact.

    We have that $E^\circ = \O$ since $E$ does not contain any open
    intervals.
    \emph{Formally}, we will show that for any interval $(a, b)$, there
    exist $k$ and $m$ such that $\left(\frac{3k+1}{3^m}, \frac{3k+2}{3^m}
    \right)$ is contained in $(a, b)$. % TODO

    \emph{Heuristically}, we see that the length of the removed intervals
    is $\frac13 + \frac19 + \dots = 1$, so that the remaining set cannot
    contain any interval of positive length.

    Uncountability is by a diagonal argument.
\end{proof}
