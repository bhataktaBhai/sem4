\section{Metric Topology} \label{sec:topology}
\lecture{08}{Thu 18 Jan '24}{}
\begin{definition*}
    Let $(X, d)$ be a metric space.
    \begin{enumerate}
        \item The \emph{open ball} centered at $p$ or radius
        $\varepsilon > 0$ is the set \[
            B_d(p; \varepsilon) \coloneq \set{x \in X : d(p, x)
                < \varepsilon}
        \] This set is also called the $\varepsilon$-neighborhood of $p$.
        Similarly, the \emph{closed ball} centered at $p$ or radius
        $\varepsilon > 0$ is the set \[
            \set{x \in X : d(p, x) \le \varepsilon}
        \]
        \item Given a set $E \subseteq X$ and $p \in X$, $p$ is an
        \emph{interior point} of $E$ if there exists some $\varepsilon > 0$
        such that the $\varepsilon$-neighborhood $B(p; e)$ is contained in
        $E$.
        The collection of all interior points of $E$, denoted $E^\circ$, is
        called the \emph{interior} of $E$.
        \item A set $E \subseteq X$ is said to be \emph{open} if it is
        equal to its interior.
        \item The collection of all open sets of $(X, d)$ is called the
        $d$-topology on $X$.
    \end{enumerate}
\end{definition*}

\begin{remark}
    The empty set is always open.
\end{remark}

\begin{examples}
    \item The open ball on \R\ is an interval
    $(p - \varepsilon, p + \varepsilon)$.
    \item
    % \begin{center}
    %     % plot the unit circle
    %     \begin{tikzpicture}
    %         \begin{axis}[
    %             axis lines = center,
    %             xlabel = $x$,
    %             ylabel = $y$,
    %             xmin = -1.5,
    %             xmax = 1.5,
    %             ymin = -1.5,
    %             ymax = 1.5,
    %             xtick = {-1, 0, 1},
    %             ytick = {-1, 0, 1},
    %             xticklabels = {$-1$, $0$, $1$},
    %             yticklabels = {$-1$, $0$, $1$},
    %         ]
    %             \addplot[
    %                 domain = 0:360,
    %                 samples = 100,
    %                 color = blue,
    %             ] ({cos(x)}, {sin(x)});
    %             % \addplot[
    %             %     domain = -1:1,
    %             %     samples = 100,
    %             %     color = red,
    %             % ] {1};
    %         \end{axis}
    %     \end{tikzpicture}
    % \end{center}
    \item For the discrete metric, \[
        B_d(p; \varepsilon) = \begin{cases}
            \set{p} & \varepsilon < 1 \\
            X & \varepsilon \ge 1
        \end{cases}
    \] Every set is open, by taking any $\varepsilon = 1$.
\end{examples}

\begin{proposition}
    Every open ball is an open set.
\end{proposition}
\begin{proof}
    Let $(X, d)$ be the metric.
    Let $p \in X$, $\varepsilon > 0$, and $q \in B(p; \varepsilon)$.
    Choose $\delta = \varepsilon - d(p, q) > 0$ works.
    We show that $B(q; \delta) \subseteq B(p; \varepsilon)$.
    Let $r \in B(q; \delta)$.
    Then from the triangle inequality, \begin{align*}
        d(p, r) &\le d(p, q) + d(q, r) \\
        &< d(p, q) + \delta \\
        &= \varepsilon \qedhere
    \end{align*}
\end{proof}

\begin{proposition} \label{thm:open_union}
    The union of any collection of open sets is open, and the intersection
    of any finite collection of open sets is open.
\end{proposition}
\begin{proof}
    Let $\mathscr{U}$ be a collection of open sets.
    Let $E = \bigcup_{U \in \mathscr{U}} U$.
    For any $p \in E$, $p$ is contained in some $U \in \mathscr{U}$.
    Then there exists some $\varepsilon > 0$ such that $B(p; \varepsilon)
    \subseteq U \subseteq E$.

    Let $U_1, \dots, U_n$ be open sets and let $E = \bigcap_{i=1}^n U_i$.
    For any $p \in E$, $p \in U_i$ for all $i$.
    Then there exist $\varepsilon_1, \dots, \varepsilon_n > 0$ such that
    $B(p; \varepsilon_i) \subseteq U_i$ for all $i$.
    Letting $\varepsilon$ be the minimum of the $\varepsilon_i$'s, we have
    $B(p; \varepsilon) \subseteq U_i$ for all $i$.
    So $B(p; \varepsilon) \subseteq E$.
\end{proof}

\begin{definition*}
    Let $(X, d)$ be a metric space and $E \subseteq X$.
    \begin{enumerate}
        \item Given $p \in X$, we say that $p$ is an \emph{accumulation
        point} of $E$ if for every $\varepsilon > 0$, $B(p; \varepsilon)$
        contains a point $q \in E$ such that $q \ne p$.
        \item A point $p \in E$ is said to be \emph{isolated} in $E$ if it
        is not an accumulation point of $E$.
        \item $E$ is said to be \emph{closed} if it contains all its
        accumulation points.
        \item The \emph{closure} of $E$, denoted $\wbar{E}$, is the union
        of $E$ with all its accumulation points.
        \item The boundary of $E$ is the set
        $\partial E = \wbar{E} \setminus E^\circ$.
    \end{enumerate}
\end{definition*}

\begin{examples}
    \item In the discrete metric, every point is isolated in every subset.
    \item Finite subsets have no accumulation points.
\end{examples}

\begin{remarks}
    \item $p$ need not lie in $E$ to be an accumulation point.
    \item If $p$ is an accumulation point of $E$, then every neighborhood of
    $p$ contains infinitely many points of $E$.
\end{remarks}
