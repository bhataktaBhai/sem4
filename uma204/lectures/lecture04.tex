\lecture{04}{Wed 10 Jan '24}{}
\begin{definition}[Archimedean property] \label{def:archimedean}
    An ordered field $F$ is said to have the \emph{Archimedean property} if
    for every $x, y > 0$, there exists an $n \in \N \subseteq F$ such that
    $nx > y$.
\end{definition}

\begin{theorem} \label{thm:Q:archimedean}
    $\Q$ has the Archimedean property.
\end{theorem}
\begin{proof}
    Let $x, y > 0$ be rationals.
    If $x > y$, $n = 1$ works.
    Suppose $x \le y$.
    It suffices to show that $\exists n \in \N (nr > 1)$, where $r = x/y$.
    Since $r$ is positive, we have $p, q \in \N^*$ such that $r = p/q$.
    Let $n = 2q$.
    This gives $nr > 1$.
\end{proof}
\begin{remark}
    Not all ordered fields have the Archimedean property.
    % TODO: Construct such a field.
\end{remark}

\begin{theorem}
    Let $F$ be an ordered field with the LUB property.
    Then $F$ has the Archimedean property.
\end{theorem}
\begin{proof}
    Let $x, y > 0$.
    Suppose $\forall n \in \N (nx \le y)$.
    Let $A = \set{nx \mid n \in \N}$.
    Clearly $A$ is non-empty and bounded above.
    Then $\sup A$ exists and so there exists an $m \in \N$ such that
    $\sup A - x < m x$.
    Thus $\sup A < (m + 1)x \in A$, a contradiction.
\end{proof}

\begin{theorem}
    Let $F$ be an ordered field with the LUB property.
    Then $\Q$ is dense in $F$, \ie, given $x < y \in F$, there exists a
    rational $r \in \Q$ such that $x < r < y$.
\end{theorem}
\begin{proof}
    Follows from \cref{thm:Q:archimedean} and
    \refifdef{prb:archimedean=dense}{\cref}{problem 4 on assignment 1}.
\end{proof}

\subsection{The Reals} \label{sec:R}
\vspace{0.5em}
\begin{theorem}[Dedekind/Cauchy] \label{thm:R:unique}
    There exists a unique (up to isomorphism) ordered field with the LUB
    property.
\end{theorem}
\begin{proof}[Proof of uniqueness.]
    Let $F$ and $G$ be OFWLUB.
    Let $h$ be identity on $\Q \subseteq F, G$.
    Let $z \in F$ and \[
        % A_z = \set{w \in \Q \mid w \le_F z}.
        A_z = \set{w \in \Q \mid w <_F z}.
    \]
    \textbf{Claim:} $A_z$ is non-empty and bounded above when viewed as a
    subset of $G$, and therefore has a supremum in $G$. \\
    First, $A_z$ is non-empty by density applied to $(z-1_F, z)$ o
    Archimedean applied to $-z$.
    Secondly, by Archimedean (or density) there exists a \emph{rational}
    upper bound $q$ of $A_z$ in $F$.
    This $q$ is also an upper bound of $A_z$ in $G$. \\
    By LUB, $A_z$ has a supremum in $G$.

    We define $h(z) \coloneq \sup_G A_z$.
    For this we need to show that $h(r) = r$ for all $r \in \Q$, so that the
    definitions coincide.
    Let $r \in \Q$ so that $A_r = \set{w \in \Q \mid w <_F r}$.
    Clearly $r$ is an upper bound of $A_r$ in $G$.
    For any $g \in G$, there is some $q \in \Q$ such that $g <_G q <_G r$
    (by density of $\Q$ in $G$).
    Thus $g$ cannot be an upper bound of $A_r \subseteq G$.
    Thus $r = \sup_G A_r = h(r)$.

    \textbf{Claim:} $h$ preserves order. \\
    Let $z < w \in F$.
    By density of $\Q$ in $F$, there exist rationals $r$, $s$, $t$ such that
    $z < r < s < t < w$.
    Then $A_z \subsetneq A_w$ as subsets of $F$ and hence of $G$.
    Thus \[
        h(z) = \sup_G A_z \le_G r < s < t \le_G \sup_G A_w = h(w).
    \]
    \textbf{Claim:} $h$ preserves addition. \\
    % Let $B_z$ be the set $\set{w \in \Q \mid w <_F z}$.
    % Since $h$ preserves order, $\sup_G B_z = \sup_G \set{h(w) \mid w
    % \in B_z} = h(\sup_F B_z) = h(z) = \sup_G A_z$.
    % This is by the following subclaim:
    %
    % \textbf{Subclaim:} If $h$ is an order-preserving bijection, then for any
    % non-empty bounded subset $S$ of $F$, $\sup h(S) = h(\sup S)$. \\
    % Let $s_F$ be the supremum of $S$.
    % Then $h(s_F)$ is an upper bound of $h(S)$.
    % Let $g \in G$ be less than $h(s_F)$.
    % Then $h^{-1}(g) < s_F$ and so there is some $s \in S$ such that
    % $h^{-1}(g) < s < s_F$, but then $g < h(s)$ so that $g$ is not an upper
    % bound of $h(S)$.
    It is sufficient to show that $A_{x+y} = A_x + A_y$, where set addition
    is defined pairwise.
    If a rational $q \in A_x + A_y$, then clearly $q <_F x + y$ and so
    $q \in A_{x+y}$.
    Let $q \in A_{x+y} \iff q <_F x + y$.
    Then $q - x \in A_y$.
    Since $A_y$ has no largest element (by density), there exists an
    $r \in A_y$ with $q - x < r < y$.
    Then $q - r < x$ and so $q - r \in A_x$.
    Thus $q = (q - r) + r \in A_x + A_y$ which gives equality of the sets.

    Since $\sup A_x + \sup A_y = \sup (A_x + A_y) = \sup A_{x+y}$, $h$
    preserves addition.

    \textbf{Claim:} $h$ preserves multiplication. \\

\end{proof}

\subsubsection{Dedekind's Construction} \label{sec:R:dedekind}
\begin{definition}[Dedekind cut] \label{def:R:dedekind:cut}
    A \emph{Dedekind cut} is a non-empty proper subset $A \subsetneq \Q$
    such that
    \begin{enumerate}
        \item if $a \in A$, then $b \in A$ for all $b \in \Q$ with $b < a$.
        \item if $a \in A$, then there exists a $c \in A$ such that $a < c$.
    \end{enumerate}
\end{definition}

\begin{definition}[\R] \label{def:R:dedekind}
    We define \[
        \R \coloneq \set{A \in 2^\Q \mid A \text{ is a Dedekind cut}}.
    \] Further,
    \begin{enumerate}
        \item $A \le B \iff A \subseteq B$;
        \item $A + B = \set{a + b \mid a \in A, b \in B}$.
        The additive identity $0 = \set{x \in \Q \mid x < 0}$;
        \item for $A, B > 0$, \[
            A \cdot B = \set{q \in \Q \mid q \le rs \text{ for some }
                r \in A, s \in B}.
        \] If $A < 0$ but $B > 0$, then $A \cdot B = -((-A) \cdot B)$.
        If $B < 0$ but $A > 0$, then $A \cdot B = -(A \cdot (-B))$.
        If $A < 0$ and $B < 0$, then $A \cdot B = (-A) \cdot (-B)$.
    \end{enumerate}
\end{definition}
