\lecture{04}{Wed 10 Jan '24}{}
\begin{definition*}[Archimedean property] \label{def:archimedean}
    An ordered field $F$ is said to have the \emph{Archimedean property} if
    for every $x, y > 0$, there exists an $n \in \N \subseteq F$ such that
    $nx > y$.
\end{definition*}

\begin{theorem*} \label{thm:Q:archimedean}
    $\Q$ has the Archimedean property.
\end{theorem*}
\begin{proof}
    Let $x, y > 0$ be rationals.
    If $x > y$, $n = 1$ works.
    Suppose $x \le y$.
    It suffices to show that $\exists n \in \N (nr > 1)$, where $r = x/y$.
    Since $r$ is positive, we have $p, q \in \N^*$ such that $r = p/q$.
    Let $n = 2q$.
    This gives $nr > 1$.
\end{proof}
\begin{remark}
    Not all ordered fields have the Archimedean property.
    % TODO: Construct such a field.
\end{remark}
% \begin{proposition}
%     There exists an ordered field without the Archimedean property.
% \end{proposition}
% \begin{proof}
%     Define addition, multiplication and order on $\Q^2$ as \begin{align*}
%         (a, b) + (c, d) &= (a + c, b + d) \\
%         (0, 1) \cdot (0, 1) &= (1, 0) \\
%         (1, 0) &< (0, 1)
%         \intertext{so that}
%         (a, b) \cdot (c, d) &= (ac + bd, ad + bc) \\
%         (a, b) < (c, d) &\mathrel{\text{iff}}
%             a < c \text{ or } a = c \land b < d
%     \end{align*}
%     We check that this is an ordered field.
%     \begin{enumerate}[label=\small(F\arabic*)]
%         \item $+$ and $\cdot$ are commutative:
%         This is immediate.
%         \item $+$ and $\cdot$ are associative:
%         Addition is immediate.
%         For multiplication, \begin{align*}
%             (a, b) \cdot ((c, d) \cdot (e, f))
%             &= (a, b) \cdot (ce + df, cf + de) \\
%             &= (ace + adf + bcf + bde, acf + ade + bce + bdf) \\
%             &= ((ac + bd)e + (ad + bc)f, \\
%             &\qquad\qquad(ac + bd)f + (ad + bc)e) \\
%             &= ((a, b) \cdot (c, d)) \cdot (e, f).
%         \end{align*}
%         \item Distributivity:
%         \begin{align*}
%             (a, b) \cdot ((c, d) + (e, f))
%             &= (a, b) \cdot (c + e, d + f) \\
%             &= (a(c + e) + b(d + f), a(d + f) + b(c + e)) \\
%             &= (ac + bd, ad + bc) + (ae + bf, af + be) \\
%             &= (a, b) \cdot (c, d) + (a, b) \cdot (e, f).
%         \end{align*}
%         \item Identities:
%         $(0, 0) \ne (1, 0)$ are the additive and multiplicative identities
%         respectively.
%         \item Additive inverse:
%         $(a, b) + (-a, -b) = (0, 0)$.
%         \item Multiplicative inverse:
%         Let $(a, b) \ne 0$.
%         If $a \ne 0$, choose $c = \frac1a$ and $d = $
%     \end{enumerate}
% \end{proof}

\begin{theorem*}
    Let $F$ be an ordered field with the LUB property.
    Then $F$ has the Archimedean property.
\end{theorem*}
\begin{proof}
    Let $x, y > 0$.
    Suppose $\forall n \in \N (nx \le y)$.
    Let $A = \set{nx \mid n \in \N}$.
    Clearly $A$ is non-empty and bounded above.
    Then $\sup A$ exists and so there exists an $m \in \N$ such that
    $\sup A - x < m x$.
    Thus $\sup A < (m + 1)x \in A$, a contradiction.
\end{proof}

\begin{theorem*}
    Let $F$ be an ordered field with the LUB property.
    Then $\Q$ is dense in $F$, \ie, given $x < y \in F$, there exists a
    rational $r \in \Q$ such that $x < r < y$.
\end{theorem*}
\begin{proof}
    Follows from \cref{thm:Q:archimedean} and
    \refifdef{prb:archimedean=dense}{\cref}{problem 4 on assignment 1}.
\end{proof}

\section{The Reals} \label{sec:R}
\vspace{0.5em}
\begin{theorem*}[Dedekind/Cauchy] \label{thm:R:unique}
    There exists a unique (up to isomorphism) ordered field with the LUB
    property.
\end{theorem*}
We first recover some properties of supremums.
\begin{lemma} \label{thm:R:sup}
    Let $F$ be an ordered field with the LUB property.
    Let $A$ and $B$ be non-empty bounded above subsets of $F$.
    Then $\sup A + \sup B = \sup (A + B)$.
    Further, if all elements of $A$ and $B$ are non-negative, then
    $\sup A \sup B = \sup (A B)$.

    Here $A + B \coloneq \set{a + b \mid a \in A, b \in B}$ and
    $A B \coloneq \set{ab \mid a \in A, b \in B}$.
\end{lemma}
\begin{proof}
    Let $\alpha = \sup A$ and $\beta = \sup B$.
    For all $a \in A$ and $b \in B$, $a + b \le \alpha + \beta$.
    Thus $\alpha + \beta$ is an upper bound of $A + B$.

    Let $c < \alpha + \beta$.
    Since $c - \beta < \alpha$, there exists an $a \in A$ larger than
    $c - \beta$.
    Then $c - a < \beta$ and so there exists a $b \in B$ larger than
    $c - a$.
    Thus $c < a + b \in A + B$ and so $\alpha + \beta = \sup (A + B)$.

    Now suppose all elements of $A$ and $B$ are non-negative.
    If $\alpha = 0$ or $\beta = 0$, then $\alpha \beta = 0$ and so
    $\alpha \beta = \sup (A B)$.

    For all $a \in A$ and $b \in B$, $ab \le \alpha \beta$.
    Let $c < \alpha \beta$.
    Since $c/\beta < \alpha$, there exists an $a \in A$ larger than
    $c/\beta$.
    Then $c/a < \beta$ and so there exists a $b \in B$ larger than
    $c/a$.
    Thus $c < ab \in A B$ and so $\alpha \beta = \sup (A B)$.
\end{proof}

\begin{proof}[Proof of uniqueness.]
    Let $F$ and $G$ be OFWLUB.
    Let $h$ be identity on $\Q \subseteq F, G$.
    For $z \in F$ let \[
        % A_z = \set{w \in \Q \mid w \le_F z}.
        A_z = \set{w \in \Q \mid w <_F z}.
    \]

    \textbf{Claim:} $A_z$ is non-empty and bounded above when viewed as a
    subset of $G$, and therefore has a supremum in $G$.

    \hfill\begin{minipage}{0.9\textwidth}
        First, $A_z$ is non-empty by density applied to $(z-1_F, z)$ or
        Archimedean applied to $-z$.
        Secondly, by Archimedean (or density) there exists a \emph{rational}
        upper bound $q$ of $A_z$ in $F$.
        This $q$ is also an upper bound of $A_z$ in $G$. \\
        By LUB, $A_z$ has a supremum in $G$.
    \end{minipage}

    We define $h(z) \coloneq \sup_G A_z$.
    For this we need to show that $h(r) = r$ for all $r \in \Q$, so that the
    definitions coincide.
    Let $r \in \Q$ so that $A_r = \set{w \in \Q \mid w <_F r}$.
    Clearly $r$ is an upper bound of $A_r$ in $G$.
    For any $g \in G$ less than $r$, there is some $q \in \Q$ such that
    $g <_G q <_G r$ (by density of $\Q$ in $G$).
    Thus $g$ cannot be an upper bound of $A_r \subseteq G$.
    Thus $r = \sup_G A_r = h(r)$.

    \textbf{Claim:} $h$ preserves order.

    \hfill\begin{minipage}{0.9\textwidth}
        Let $z < w \in F$.
        By density of $\Q$ in $F$, there exist rationals $r$, $s$, $t$ such that
        $z < r < s < w$.
        Then $A_z \subsetneq A_w$ as subsets of $F$ and hence of $G$.
        Thus \[
            h(z) = \sup_G A_z \le_G r < s \le_G \sup_G A_w = h(w).
        \]
    \end{minipage}

    \textbf{Claim:} $h$ preserves addition.

    % Let $B_z$ be the set $\set{w \in \Q \mid w <_F z}$.
    % Since $h$ preserves order, $\sup_G B_z = \sup_G \set{h(w) \mid w
    % \in B_z} = h(\sup_F B_z) = h(z) = \sup_G A_z$.
    % This is by the following subclaim:
    %
    % \textbf{Subclaim:} If $h$ is an order-preserving bijection, then for any
    % non-empty bounded subset $S$ of $F$, $\sup h(S) = h(\sup S)$. \\
    % Let $s_F$ be the supremum of $S$.
    % Then $h(s_F)$ is an upper bound of $h(S)$.
    % Let $g \in G$ be less than $h(s_F)$.
    % Then $h^{-1}(g) < s_F$ and so there is some $s \in S$ such that
    % $h^{-1}(g) < s < s_F$, but then $g < h(s)$ so that $g$ is not an upper
    % bound of $h(S)$.
    \hfill\begin{minipage}{0.9\textwidth}
        It is sufficient to show that $A_{x+y} = A_x + A_y$, where set
        addition is defined pairwise.
        If a rational $q \in A_x + A_y$, then clearly $q <_F x + y$ and
        so $q \in A_{x+y}$.
        Let $q \in A_{x+y} \iff q <_F x + y$.
        Then $q - x \in A_y$.
        Since $A_y$ has no largest element (by density), there exists an
        $r \in A_y$ with $q - x < r < y$.
        Then $q - r < x$ and so $q - r \in A_x$.
        Thus $q = (q - r) + r \in A_x + A_y$ which gives equality of
        the sets.

        From the previous lemma, $\sup A_x + \sup A_y =
        \sup (A_x + A_y) = \sup A_{x+y}$ and so $h$ preserves addition.
    \end{minipage}

    \textbf{Claim:} $h$ preserves multiplication.

    \hfill\begin{minipage}{0.9\textwidth}
        Let $0 < x, y \in F$.
        Let $A^+_z = \set{w \in \Q \mid 0 < w <_F z}$.
        We will show that $A^+_{xy} = A^+_x A^+_y$, where set product is
        defined pairwise.
        % \begin{description}
        %     \item[$\bm{A^+_x A^+_y \subseteq A^+_{xy}}$:]
        %     Suppose $w \in A^+_x A^+_y$.
        %     Then $w = qr$ for some $0 < q <_F x$ and $0 < r <_F y$.
        %     This immediately gives $0 < w <_F xy$, so that $w \in A^+_{xy}$.
        %     \item[$\bm{A^+_{xy} \subseteq A^+_x \cdot A^+_y}$:]
        %     Suppose $w \in A^+_{xy}$.
        %     Then $0 < w/x <_F y$.
        %     Thus $w/x \in A^+_y$.
        %     Since $A^+_y$ has no largest element, there exists an $r \in A^+_y$
        %     such that $w/x <_F r <_F y$.
        %     Then $w/r <_F x$ and so $w/r \in A^+_x$.
        %     Thus $w = (w/r) \cdot r \in A^+_x \cdot A^+_y$.
        % \end{description}
        If a rational $q \in A^+_x A^+_y$, then clearly $0 < q <_F xy$
        and so $q \in A^+_{xy}$.
        Let $q \in A^+_{xy} \iff 0 < q <_F xy$.
        Then $q/x \in A^+_y$.
        Since $A^+_y$ has no largest element, there exists an
        $r \in A^+_y$ with $q/x < r < y$.
        Then $q/r < x$ and so $q/r \in A^+_x$.
        Thus $q = (q/r) \cdot r \in A^+_x A^+_y$ which gives equality of
        the sets.

        From the previous lemma, $\sup A^+_x \sup A^+_y =
        \sup (A^+_x A^+_y) = \sup A^+_{xy}$ and so $h$ preserves
        multiplication of positive elements.

        Since $h$ preserves addition, $h$ preserves additive inverses.
        So $h$ preserves multiplication of all elements.
    \end{minipage}

    Thus $h$ is an isomorphism between $F$ and $G$.
\end{proof}

\subsection{Dedekind's Construction} \label{sec:R:dedekind}
\begin{definition*}[Dedekind cut] \label{def:R:dedekind:cut}
    A \emph{Dedekind cut} is a non-empty proper subset $A \subsetneq \Q$
    such that
    \begin{enumerate}
        \item if $a \in A$, then $b \in A$ for all $b \in \Q$ with $b < a$.
        \item if $a \in A$, then there exists a $c \in A$ such that $a < c$.
    \end{enumerate}
\end{definition*}

\begin{definition*}[\R] \label{def:R:dedekind}
    We define \[
        \R \coloneq \set{A \in 2^\Q \mid A \text{ is a Dedekind cut}}.
    \] Further,
    \begin{enumerate}
        \item $A \le B \iff A \subseteq B$;
        \item $A + B = \set{a + b \mid a \in A, b \in B}$.
        \item for $A, B > 0$, \[
            A \cdot B = \set{q \in \Q \mid q \le rs \text{ for some }
                r \in A, s \in B}.
        \] If $A < 0$ but $B > 0$, then $A \cdot B = -((-A) \cdot B)$.
        If $B < 0$ but $A > 0$, then $A \cdot B = -(A \cdot (-B))$.
        If $A < 0$ and $B < 0$, then $A \cdot B = (-A) \cdot (-B)$.
    \end{enumerate}
\end{definition*}

\begin{proposition*} \label{thm:R:dedekind:negative}
    $O = \set{z \in \Q \mid z < 0}$ is the additive identity of \R.
    For any $A \in \R$, \[
        B = \set{x \in \Q \mid \exists r \in O (r - x \notin A)}
    \] is an additive inverse of $A$.
\end{proposition*}
\begin{proof}
    Let $A \in \R$.
    For all $a \in A$, there exists $a' \in A$ larger than $a$.
    So $a - a' \in O$ and thus $a' + (a - a') = a \in A + O$.

    For all $a \in A + O$, there exists $a' \in A$ and $o \in O$ such that
    $a = a' + o$.
    But then $a' > a$, so $a \in A$.
    Thus $A + O = A$.

    Let $B$ be as defined.
    Let $a + b \in A + B$ where $a \in A$ and $b \in B$.
    Then there exists $r \in O$ such that $r - b \notin A$.
    So $r - b > a$ and thus $a + b < r < 0$.

    Now let $o \in O$.
    Since $O$ has no largest element, there exists an $o' \in O$ such that
    $o' > o$.
    Let $a \in A$.
    Consider the set $\alpha = \set{n \in \Z \mid a + n(o' - o) \in A}$.
    By archimedean property of $\Q$, $\alpha$ is bounded.
    It is obviously non-empty fucker.
    Thus it has a supremum $n$.
    Let $a' = a + n(o' - o)$.
    $a' + (o' - o) = o' - (o - a') \notin A$ because $n$ was supremum.
    This gives $o - a' \in B$.
    Thus $o \in A + B$.
\end{proof}
