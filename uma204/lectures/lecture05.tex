\lecture{2024-01-11}{$\R$ via Cauchy sequences}
\begin{theorem*}
    $\R$ has the least upper bound property.
\end{theorem*}
\begin{proof}
    Let $\alpha$ be a non-empty subset of $\R$ that is bounded above.
    We claim that $S = \bigcup_{A \in \alpha} A$ is the supremum of $\alpha$.
    \begin{description}
        \item[$s$ is a cut:] Since $S$ is a union of a non-empty set of
        non-empty sets, it is non-empty.
        Since $S$ is bounded above, say by some cut $C$, we have
        $S \subseteq C \subsetneq \Q$ and so $S \ne \Q$.
        If $a \in S$, then $a \in A$ for some
        $A \in \alpha$. Since $A$ is a cut, every rational smaller than $a$
        is contained in $A$ and thereby in $S$.
        Moreover, there exists an $a' \in A$ which is larger than $a$.
        Thus $a' \in S$ is larger than $a$.
        \item[upper bound:] $A \subseteq S$ for all $A \in \alpha$.
        \item[least upper bound:] For any $D \subsetneq S$,
        let $b \in S \setminus D$.
        But since $b \in A$ for some $A \in \alpha$, $D$ is not an upper
        bound of $\alpha$. \qedhere
    \end{description}
\end{proof}
Dedekind's construction is an ``order completion''.
Thus all the order properties (LUB, density) are nice, but arithmetic is
ugly.

\subsection{Cauchy's Construction} \label{sec:R:cauchy}
There seem to be sequences in $\Q$ that ``should'' have a limit (\eg, a
monotone and bounded sequence) but do not (within $\Q$).
We construct equivalence classes of sequences which ``converge'' to the same
number, and define reals by those classes.
\begin{definition}[Sequence]
    A sequence of rational numbers is a $f \colon \N \to \Q$.
    We usually denote $f(k)$ by $a_k$ and call it the $k$-th term of the
    sequence.
    The function $f$ is usually written as $(a_k)_{k \in \N}$.
\end{definition}

\begin{definition*}
    A sequence $(a_k)_{k \in \N} \subseteq \Q$ is said to be
    \begin{enumerate}
        \item \Q-bounded if there exists an $M \in \Q$ such that
        $|a_k| \le M$ for all $k \in \N$.
        \item \Q-Cauchy if for every rational $\epsilon > 0$, there exists
        an $N \in \N$ such that $|a_m - a_n| < \epsilon$ for all
        $m, n \ge N$.
        \item convergent in \Q\ if there exists an $L \in \Q$ such that
        for all (rational) $\varepsilon > 0$, there exists an $N \in \N$
        such that $|a_n - L| < \varepsilon$ for all $n \ge N$.
    \end{enumerate}
\end{definition*}

\begin{exercise}
    Show that if a sequence is convergent in \Q, then it is \Q-Cauchy,
    and if it is \Q-Cauchy, then it is \Q-bounded.
\end{exercise}
\begin{remark}
    From UMA101, we know that if a sequence is convergent in \Q, the limit
    is unique.
    We also know arithmetic laws of limits (which we proved over \R, but
    they hold over \Q as well).
\end{remark}

\begin{definition}
    Two sequences $a = (a_n)_{n \in \N}$ and $b = (b_n)_{n \in \N}$ are said
    to be \emph{equivalent} if their difference converges to $0$.
\end{definition}

\begin{proposition}
    Let $\mathcal{C}$ denote the space of \Q-cauchy sequences.
    Then $\sim$ given by $a \sim b$ if $a$ and $b$ are equivalent (as per
    the previous definition) is an equivalence relation.
\end{proposition}
\begin{proof}
    Reflixivity and symmetry are immediate.
    Transitivity follows from the triangle inequality.
\end{proof}

\begin{definition*}[\R] \label{def:R:cauchy}
    We define \[
        \R \coloneq \mathcal{C} /{\sim}.
    \] Further,
    \begin{enumerate}
        \item $[(a_n)_n] +_\R [(b_n)_n] \coloneq [(a_n + b_n)_n]$.
        \item The additive identity $0 = [(0)_{n \in \N}]$.
        \item $[(a_n)_n] \cdot_\R [(a_n)_n] \coloneq [(a_n \cdot b_n)_n]$.
        \item $[a]$ is positive iff there exists a rational $c > 0$ and an
        $N \in \N$ such that $a_n > c$ for all $n \ge N$.
    \end{enumerate}
\end{definition*}
Recall how we define order from a set of positive elements.
Assuming the existence of additive inverses, we define $a <_R b$ iff
$b - a$ is positive.
The set of positive elements $\R^+$ must satisty (Apostol order axioms):
\begin{itemize}
    \item $0 \notin \R^+$.
    \item For $x \ne 0$, exactly one of $x \in \R^+$ or $-x \in \R^+$ holds.
    \item For $x, y \in \R^+$, $x + y \in \R^+$ and $x \cdot y \in \R^+$.
\end{itemize}
The first two give trichotomy, and the third is the field order properties.
\begin{proposition*}
    The operations $+\R$ and $\cdot_\R$ and the relation $>_\R$ are
    well-defined.
\end{proposition*}
\begin{proof}
    Let $a \sim a'$ and $b \sim b'$ be \Q-Cauchy sequences.
    Then $(a + b) - (a' + b') = (a - a') + (b - b') \to 0$.
    So $a + b \sim a' + b'$.

    For multiplication, \begin{align*}
        a b - a' b' &= a b - a b' + a b' - a' b' \\
        &= a(b - b') + b' (a - a') \\
        &\to 0.
    \end{align*}
    So $a b \sim a' b'$.

    Finally, for any positive $c$, $a_n - a'_n$ is eventually smaller than
    $c/2$, so $a_n > c$ implies $a'_n > c/2$ for sufficiently large $n$.
\end{proof}

\begin{proposition*}
    The relation $<_\R$ makes $\R$ an ordered field.
\end{proposition*}
\begin{proof}
    First note that each $[x] \in \R$ has additive inverse $[-x]$ where
    each term is the additive inverse of the corresponding term in $x$.

    $[0] \notin \R^+$ is obvious.

    Let $[x] \ne 0$.
    Suppose $[x] \in \R^+$ and $-[x] \in \R^+$.
    Then there exists $c, d > 0$ and $N \in \N$ such that
    $x_n > c$ and $-x_n > d$ for all $n \ge N$.
    This is a contradiction.

    Now suppose neither is positive.
    Then for any $\varepsilon > 0$, there exists an $N \in \N$ such that
    $x_n < \varepsilon$ and $-x_n < \varepsilon$ for all $n \ge N$.
    But this means that $x_n \to 0$ so $[x] = 0$, a contradiction.

    Finally, let $[x], [y] \in \R^+$.
    Then for $c, d$ bounding $x$ and $y$, we have $x_n + y_n > c + d$ for
    sufficiently large $n$.
    Similarly, $x_n y_n > c d$ for the same $n$.
\end{proof}
