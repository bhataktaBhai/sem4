\chapter{Sequences and Series of Functions} \label{chp:ssfn}
\lecture{2024-03-28}{Sequences and series of functions}
In this chapter, all functions, unless stated otherwise, are assumed
to be complex-valued.
\begin{definition}[Pointwise convergence] \label{def:ssfn:pointwise}
    A sequence of functions $(f_n)_{n \in \N}$ between metric spaces
    $X$ and $Y$ is said to \emph{converge pointwise} to a function $f$
    if for all $x \in X$, the sequence $(f_n(x))_{n \in \N}$ converges to
    $f(x)$.
\end{definition}
\begin{definition}[Pointwise sum] \label{def:ssfn:pointwise_sum}
    Let $(f_n)_{n \in \N}$ be a sequence of functions from some metric
    space $X$ to a normed vector space $Y$.
    Then the sum $\sum_{n=1}^{\infty} f_n$ converges pointwise to a function
    $s$ if \[
        s(x) = \sum_{n \in \N} f_n(x)
    \] for all $x \in X$.
\end{definition}

\begin{definition*}[Uniform convergence] \label{def:ssfn:unif}
    A sequence of functions $(f_n)_{n \in \N}$ between metric spaces
    $X$ and $Y$ is said to \emph{converge uniformly} to a function $f$
    if for every $\varepsilon > 0$, there exists an $N \in \N$ such that
    for all $x \in X$ and $n \ge N$, $d_Y(f_n(x), f(x)) < \varepsilon$.

    We denote this by $f_n \unifto f$.

    If $Y$ is a normed space, we say that the sum $\sum_{n=1}^{\infty} f_n$
    converges uniformly to a function $s$ if its sequence of partial sums
    converges uniformly to $s$.
\end{definition*}

\begin{exercise}[Cauchy criterion] \label{thm:ssfn:cauchy}
    State and prove the Cauchy criterion for uniform convergence
    for functions onto a complete metric space.
\end{exercise}
\begin{solution}
    $(f_n)_{n \in \N}$ converges uniformly iff for every $\varepsilon > 0$
    there exists an $N \in \N$ such that $\forall x \forall m \forall n$
    with $m, n \ge N$, \[
        d(f_m(x), f_n(x)) < \varepsilon.
    \]
    $\sum_{n \in \N} f_n$ converges uniformly iff for every
    $\varepsilon > 0$ there exists an $N \in \N$ such that
    $\forall x \forall m \forall n$ with $m, n \ge N$, \[
        \abs{\sum_{i = m+1}^n f_i(x)} < \varepsilon.
    \]
    \begin{proof}
        The statement for the series follows from that for the sequence.
        Let $Y$ be complete.

        Let $(f_n\colon X \to Y)_{n \in \N} \unifto f$.
        Let $\varepsilon > 0$.
        Choose $N$ such that $\forall x \forall n \ge N$, \[
            d(f_n(x), f(x)) < \varepsilon/2.
        \] Then by the triangle inequality, we have that
        $\forall x \forall m, n \ge N$, \[
            d(f_m(x), f_n(x)) < \varepsilon/2.
        \]

        Now suppose that $(f_n)_n$ satisfies the Cauchy criterion.
        For every $\varepsilon$ there exists an $N$ such that
        $\forall x \forall m, n \ge N$, \[
            d(f_m(x), f_n(x)) < \varepsilon.
        \] Thus for any $x$, the sequence $f_n(x)$ is Cauchy and hence
        convergent.
        Let $f$ be the pointwise limit of $(f_n)_n$.
        We claim that $f_n \unifto f$.

        Let $\varepsilon > 0$.
        Choose $N$ such that $\forall x \forall m, n \ge N$, \[
            d(f_m(x), f_n(x)) < \varepsilon/2.
        \] Fix an $x \in X$.
        Let $N_x$ be such that $\forall n \ge N_x$, \[
            d(f_n(x), f(x)) < \varepsilon/2.
        \] Let $N' = \max(N, N_x)$.
        Then for all $n \ge N$, \[
            d(f_n(x), f(x)) \le d(f_n(x), f_{N'}(x)) + d(f_{N'}(x), f(x))
            < \varepsilon.
        \] Since $x$ was arbitrary, we have that $f_n \unifto f$.
        \renewcommand{\qedsymbol}{\ensuremath{\square\blacksquare}}
    \end{proof}%
    \renewcommand{\qedsymbol}{}%
\end{solution}

\begin{exercise*}[Weierstrass M-test] \label{thm:ssfn:mtest}
    Let $(f_n)_{n \in \N}$ be a sequence of functions from a metric space
    $X$ to the complex numbers.
    Suppose there is a sequence $(M_n)_{n \in \N} \subseteq \R$
    such that $\forall x \forall n$, \[
        \abs{f_n(x)} \le M_n.
    \] If $\sum_{n \in \N} M_n$ converges, then $\sum_{n \in \N} f_n$
    converges absolutely and uniformly on $X$.
\end{exercise*}
\begin{proof}
    Let $\varepsilon > 0$.
    Choose $N$ such that $\sum_{n = N}^{\infty} M_n < \varepsilon$.
    Then for all $x \in X$ and $m, n \ge N$, \[
        \abs{\sum_{i = m+1}^n f_i(x)}
            \le \sum_{i = m+1}^n \abs{f_i(x)}
            \le \sum_{i = m+1}^n M_i
            < \varepsilon.
    \] Thus by \cref{thm:ssfn:cauchy}, $\sum_{n \in \N} f_n$ converges
    absolutely and uniformly.
\end{proof}
