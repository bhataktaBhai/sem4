\section{Compactness} \label{sec:compactness}
\lecture{10}{Wed 24 Jan '24}{}

\begin{definition*} \label{def:bounded}
    A subset $E \subseteq (X, d)$ is said to be \emph{bounded} if there
    exists a $p \in X$ and $M > 0$ such that $E \subseteq B(p; M)$.
\end{definition*}
Equivalently, $E$ is bounded if for all $x \in X$, there exists an $M > 0$
such that $E \subseteq B(x; M)$.
Or, $E$ is bounded if there exists an $M > 0$ such that
$d(x, y) < M$ for all $x, y \in E$.

Consider $E = \set{p \in \Q : 2 < p^2 < 3}$.
Then $E$ is both closed and bounded in $(\Q, \abs{\cdot})$.
However, continuous functions on $E$ are neither uniformly continuous nor
bounded.

\begin{definition}[Open cover] \label{def:open_cover}
    Let $E \subseteq (X, d)$.
    An open cover $\set{\mcU_\alpha}_{\alpha \in \Lambda}$ of $E$ in $X$ is
    a collection of open sets $\mcU_\alpha$ such that
    $E \subseteq \bigcup_{\alpha \in \Lambda} \mcU_\alpha$.
\end{definition}

\begin{definition*}[Compact set] \label{def:compact}
    A subset $E \subseteq (X, d)$ is said to be compact if any open cover
    $\mcU = \set{\mcU_\alpha}_{\alpha \in \Lambda}$ of $E$ in $X$ admits a
    finite subcover of $E$, \ie, there exist
    $\alpha_1, \dots, \alpha_k \in \Lambda$ such that $E \subseteq
    \bigcup_{i = 1}^k \mcU_{\alpha_i}$.
\end{definition*}

\begin{examples}
    \item $E \subseteq (X, d)$ is finite.
    Let $\mcU$ be an open cover of $E = \set{p_1, \dots, p_n}$.
    Then for each $p_j \in E$, there exists $\alpha_j \in \Lambda$ such that
    $p_j \in \mcU_{\alpha_j}$.
    Then $E \subseteq \bigcup_{j = 1}^n \mcU_{\alpha_j}$.
    \item $E = (0, 1)$ is not compact in $(\R, \abs{\cdot})$.
    \begin{proof}
        Let $\mcU_n = (\frac1{n+2}, \frac1n)$ for $n \in \N^*$.
        Then $\mcU = \set{\mcU_n}_{n \in \N^*}$ is an open cover of $E$.
        However, $\mcU$ does not admit a finite subcover of $E$.
        For any finite $\set{\mcU_{n_1}, \dots, \mcU_{n_k}}$, let
        $n_0 = \max\set{n_j : 1 \le j \le k}$.
        Then $\bigcup \mcU_{n_j} \subseteq (\frac1{n_0 + 2}, 1)$ and thus is
        not a cover of $E$.
    \end{proof}
    \item $E = [0, 1]$ is compact in $(\R, \abs{\cdot})$.
    In fact, all rectangles (sets of the form
    $[a_1, b_1] \times \dots \times [a_n, b_n]$) are compact in
    $(\R^n, \norm{\cdot})$.
\end{examples}

\begin{theorem*} \label{thm:compactness}
    Let $E \subseteq (\R^n, \norm{\cdot})$.
    Then the following are equivalent:
    \begin{enumerate}[label=(\arabic*)]
        \item $E$ is compact.
        \item $E$ is closed and bounded.
        \item Every infinite subset of $E$ admits a limit point in $E$.
    \end{enumerate}
\end{theorem*}
\begin{remark}
    The equivalence between (1) and (2) is known as the Heine-Borel theorem.
\end{remark}
We first show that (1) $\implies$ (2) in a general metric space.
This is not necessary for the theorem, since we will later show
$(1) \implies (3) \implies (2) \implies (1)$ anyway.
But $(2) \implies (1)$ is only valid in $\R^n$, so that won't yield
$(1) \implies (2)$ in a general space.
\begin{proof}[Proof that (1) $\implies$ (2) in general]
    Let $E \subseteq (X, d)$ be compact.
    Let $z \in E^c$.
    For any $y \in E$, let $\delta_y = d(y, z) / 2$.
    Note that $B(z, \delta_y) \cap B(y, \delta_y) = \O$.

    Then $\mcU = \set{B(y; \delta_y) : y \in E}$ is an open cover of $E$.
    Since $E$ is compact, $\mcU$ admits a finite subcover of $E$.
    That is, there exist $y_1, \dots, y_k \in E$ such that
    $E \subseteq \bigcup_{i = 1}^k B(y_i; \delta_{y_i})$.
    Let $\delta = \min\set{\delta_{y_i}}$.
    Then $B(z; \delta) \cap E = \O$, so $B(z; \delta) \subseteq E^c$.
    This shows that $E^c$ is open, so $E$ is closed.

    For boundedness, take the largest ball from finite subcover of
    $E$ in $\bigcup_{R > 0} B(p; R)$ for some $p \in E$.
\end{proof}
To show that $(2) \implies (1)$ in $(\R^n, \norm{\cdot})$,
we first show that for any $R \in \R$, the set $[-R, R]^n$ is compact.
\begin{theorem}
    All rectangles of the form $[-R, R]^n$ are compact in
    $(\R^n, \norm{\cdot})$.
\end{theorem}
\begin{proof}
    Fix an $R > 0$.
    Let $I_0 = [-R, R]^n$.
    Note that \[
        \operatorname{diam}(I_0)
            = \max\set{\norm{x - y} \mid x, y \in I_0} = 2R \sqrt n.
    \] Let $\mcU$ be an open cover of $I_0$.
    Suppose this has no finite subcover of $I_0$.
    We partition $I_0$ into $2^n$ equal rectangles of the form
    $J_1 \times J_2 \times \dots J_n$, where each $J_i$ is either $[-R, 0]$
    or $[0, R]$.
    The diameter of each of these rectangles is $R \sqrt n$.
    By the pigeonhole principle, there exists $I_1$ among these rectangles
    that does not admit a finite subcover in $\mcU$.

    We continue this process to obtain a sequence of rectangles
    $I_0 \supseteq I_1 \supseteq I_2 \supseteq \dots$, none of which admit a
    finite subcover in $\mcU$, with
    $\operatorname{diam}(I_k) = \frac{2R}{2^k} \sqrt n$.

    We use the nested interval property (\cref{thm:nested_intervals}) to
    obtain a point $x \in \bigcap_{k = 1}^\infty I_k$.
    Since $\mcU$ covers $I_0$, there exists $U \in \mcU$ that contains $x$.
    Since $U$ is open, it contains some $\varepsilon$-ball around $x$.
    But the diameters of the $I_k$s decrease to $0$, so $I_k \subseteq U$ for
    some $k$.
    This contradicts that $I_k$ does not admit a finite subcover in $\mcU$.

    Thus the original rectangle $I_0$ admits a finite subcover in $\mcU$.
\end{proof}

\begin{lemma}[Nested interval theorem] \label{thm:nested_intervals}
    Let $I_0 \supseteq I_1 \supseteq \dots$ be a sequence of closed
    rectangles in $\R^n$.
    Then $\bigcap_{k = 1}^\infty I_k \ne \O$.
\end{lemma}
\begin{proof}
    Let each $I_k = [a_{1k}, b_{1k}] \times \dots \times [a_{nk}, b_{nk}]$.
    Then $(a_{ik})_{k \in \N}$ and $(b_{ik})_{k in \N}$ are
    bounded sequences in $\R$.
    Let $A_i = \sup\set{a_{ik} \mid k \in \N}$.
    Note that each $b_{ik}$ is an upper bound for the set
    $\set{a_{ik} \mid k \in \N}$, so $A_i \le b_{ik}$ for all $k$.

    Let $A = (A_1, \dots, A_n)$.
    For each $k$, $A \in I_k$ since $A_i \in [a_{ik}, b_{ik}]$ for all $i$.
    Thus $\bigcap_{k = 1}^\infty I_k \ni A$.
\end{proof}
