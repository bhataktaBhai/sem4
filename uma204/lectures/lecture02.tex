\lecture{02}{Tue 02 Jan '24}{}
\begin{proposition}
    The operations $+_\Z$, $\cdot_\Z$ and the order $\leq_\Z$ are well-defined.
\end{proposition}
\begin{proof}
    Suppose $[(a, b)] = [(a', b')]$ and $[(c, d)] = [(c', d')]$.
    Then \begin{align*}
        a + b' &= a' + b \\
        c + d' &= c' + d \\
        (a + c) + (b' + d') &= (a' + c') + (b + d) \\
        [(a + c, b + d)] &= [(a' + c', b' + d')]
    \end{align*}
    Since $\le_\Z$ is defined in terms of $+_\Z$, it is also well-defined.
    % TODO: multiplication
    % Similarly \begin{align*}
    %     (a + b')(c + d') &= (a' + b)(c' + d) \\
    %     ac + ad' + b'c + b'd' &= a'c' + a'd + bc' + bd \\
    %     (ac + bd) + (a'd' + b'c') &= (a'c' + b'd') + (a'd + b'c) \\
    %     [(ac + bd, ad + bc)] &= [(a'c' + b'd', a'd' + b'c')]
    % \end{align*}
\end{proof}

\begin{definition}[Ring] \label{def:ring}
    A \emph{ring} is a set $S$ with two binary operations $+$ and $\cdot$ such
    that for all $a, b, c \in S$,
    \begin{enumerate}
        \item addition is associative,
        \item addition is commutative,
        \item there exists an additive identity $0$,
        \item there exists an additive inverse $-a$,
        \item multiplication is associative,
        \item there exists a multiplicative identity $1$,
        \item multiplication is distributive over addition (on both sides).
    \end{enumerate}
    A ring in which multiplication is commutative is called a
    \emph{commutative ring}.
\end{definition}
Note that inverses are unique, since if $a + b = 0$ and $a + b' = 0$, then
$b = (b' + a) + b = b' + (a + b) = b'$.

\begin{definition}[Ordered Ring] \label{def:ordered_ring}
    An \emph{ordered ring} is a ring $S$ with a total order $\leq$ such that
    for all $a, b, c \in S$,
    \begin{enumerate}
        \item $a \leq b$ implies $a + c \leq b + c$,
        \item $0 \leq a$ and $0 \leq b$ implies $0 \leq ab$.
    \end{enumerate}
\end{definition}

\begin{theorem} \leavevmode
    \begin{itemize}
        \item $(\Z, +_\Z, \cdot_\Z, \leq_\Z)$ is an ordered (commutative) ring.
        \item The map $f = n \mapsto [(n, 0)]$ from $\N$ to $\Z$ is an
        injective map that respects $+$, $\cdot$ and $\leq$.
        That is, for all $n, m \in \N$,
        \begin{enumerate}
            \item $f(n + m) = f(n) +_\Z f(m)$,
            \item $f(nm) = f(n) \cdot_\Z f(m)$,
            \item $n \leq m \iff f(n) \leq_\Z f(m)$.
        \end{enumerate}
        In other words, $f$ is an isomorphism onto a subset of $\Z$.
    \end{itemize}
\end{theorem}
Thus, we may view $(\N, +, \cdot, \le)$ as a subset of $(\Z, +_\Z, \cdot_\Z,
\leq_\Z)$, denote $[(n, 0)]$ as $n$ and drop $\Z$ in the subscript.
We further define $-[(a, b)] \coloneq [(b, a)]$ and $z_1 - z_2 \coloneq z_1 +
(-z_2)$.

Moreover, we have the following properties.
\begin{proposition} \leavevmode
    \begin{itemize}
        \item There are no zero divisors in $\Z$.
        That is, for all $a, b \in \Z$, $ab = 0$ implies $a = 0$ or $b = 0$.
        \item The cancellation laws hold:
        for all $a, b, c \in \Z$, $a + b = a + c$ implies $b = c$,
        and $ab = ac$ implies $a = 0$ or $b = c$.
        \item (trichotomy) For all $z \in \Z$, $z = n$ or $z = -n$ for some
        $n \in \N$.
    \end{itemize}
\end{proposition}

\section{Rationals} \label{sec:rationals}
We cannot solve $3x = 2$ in $\Z$.
\begin{proof}
    Suppose $3x = 2$ for some $x = [(a, b)] \in \Z$.
    Then \begin{align*}
        3x &= 2 \\
        [(3, 0)] \cdot [(a, b)] &= [(2, 0)] \\
        [(3a, 3b)] &= [(2, 0)] \\
        3a &= 3b + 2
    \end{align*}
    % TODO: what now?
    What now?
\end{proof}
We define $\Z^*$ to be $\Z \setminus \set{0}$ and define the relation $R$ on
$\Z \times \Z^*$ by $(a, b) R (c, d)$ if $ad = bc$.
Then $R$ is an equivalence relation on $\Z \times \Z^*$.

\begin{definition}
    We define $\Q$ to be the set of equivalence classes of $R$, notated
    $\Z \times \Z^* / R$.
\end{definition}
We define operations $+_\Q$ and $\cdot_\Q$ on $\Q$ by
\begin{align*}
    [(a, b)] +_\Q [(c, d)] &\coloneq [(ad + bc, bd)] \\
    [(a, b)] \cdot_\Q [(c, d)] &\coloneq [(ac, bd)]
\end{align*}
Since there are no zero divisors in $\Z$, $bd \neq 0$.

We define an order $\leq_\Q$ on $\Q$ by \[
    [(a, b)] \leq_\Q [(c, d)] \iff (ad - bc) bd \leq 0.
\]
