\section{Mean Value Theorems \& Applications} \label{sec:mvt}
\lecture{28}{Mon 18 Mar '24}{}
\begin{example}
    Let \[
        f(x) = \begin{cases}
            x^2 \sin \frac1x & \text{if } x \ne 0, \\
            0 & \text{if } x = 0.
        \end{cases}
    \]
    We show that $f$ is differentiable on \R, but $f'$ is discontinuous
    at $0$.
    \begin{proof}
        When $x \ne 0$, we use the fact that polynomials and trignometric
        functions are differentiable, so that \[
            f'(x) = 2x \sin \frac1x - \cos \frac1x.
        \] At $x = 0$, we have \[
            \lim_{h \to 0} \frac{f(h) - f(0)}{h}
            = \lim_{h \to 0} h \sin \frac1h = 0.
        \] Thus the derivative is well-defined with \[
            f'(x) = \begin{cases}
                2x \sin \frac1x - \cos \frac1x & \text{if } x \ne 0, \\
                0 & \text{if } x = 0.
            \end{cases}
        \]
    \end{proof}
\end{example}

\begin{theorem} \label{thm:stationary}
    Let $f\colon [a, b] \to \R$ be a function.
    Suppose $f$ attains a local extremum at $c \in (a, b)$ and $f$ is
    differentiable at $c$.
    Then $f'(c) = 0$.
\end{theorem}
\begin{proof}
    Suppose WLOG that $f$ attains a local maximum at $c$.
    Then we have $f(c) \ge f(x)$ for all $x$ close to $c$.
    But then for $h > 0$, \[
        \frac{f(c + h) - f(c)}{h} \le 0
    \] and \[
        \frac{f(c - h) - f(c)}{-h} \ge 0.
    \] If the left hand and right hand limits of the difference quotient
    exist and are equal, then they must both be zero.
\end{proof}

\begin{exercise}
    Let $f\colon (a, b) \to \R$ be a function.
    Suppose $f$ is differentiable at $c \in (a, b)$ and $f'(c) > 0$.
    Then does there exist an interval around $c$ such that $f$ is
    increasing on that interval?
\end{exercise}

\begin{exercise}
    Let $f\colon (a, b) \to \R$ be differentiable.
    Suppose $f'(x) \ge 0$ for all $x \in (a, b)$, and furthermore that
    \begin{itemize}
        \item $f'$ is not identically zero on any interval.
        Is $f$ strictly increasing on $(a, b)$?
        \item $f'$ is zero on a discrete set of points.
        Is $f$ strictly increasing on $(a, b)$?
    \end{itemize}
\end{exercise}

\begin{theorem}[Intermediate value property] \label{thm:diff:ivt}
    Suppose $f\colon (p, q) \to \R$ is differentiable and
    $[a, b] \subseteq (p, q)$.
    Suppose $f'(a) < \lambda < f'(b)$.
    Then there exists a $c \in (a, b)$ such that $f'(c) = \lambda$.
\end{theorem}

\begin{exercise}
    If $g\colon (a, b) \to \R$ has a simple discontinuity at $c \in (a, b)$,
    then $g$ does not satisfy the intermediate value property on some
    neighbourhood of $c$.
\end{exercise}
\begin{proof}
    We have two cases.
    \begin{description}
        \item[$\bm{f(c^-) = f(c^+) = L \ne f(c)}$.]
        Choose $\lambda = \frac12 (L + f(c))$.
        In some $\delta$ neighbourhood of $c$ (excluding $c$ itself),
        we have $f(x) - L < \frac12 (f(c) - L)$ so that
        $f(x) < \lambda$.
        At $c$ itself, we have $f(c) > \lambda$.
        Thus $\lambda$ is never attained.
        \item[$\bm{f(c^-) \ne f(c^+)}$.]
        Choose $\lambda$ between the two limits, but unequal to $f(c)$.
        \qedhere
    \end{description}
\end{proof}

\begin{corollary}
    Let $f$ and $[a, b]$ be as in \cref{thm:diff:ivt}.
    Then $f'$ only has discontinuities of the second kind.
\end{corollary}

\begin{proof}[Proof of \cref{thm:diff:ivt}]
    Let $g(x) = f(x) - \lambda x$ for $x \in (p, q) \supseteq [a, b]$.
    Then $g'(x) = f'(x) - \lambda$ so that $g'(a) < 0 < g'(b)$.
    This means that $g$ is strictly decreasing from $a$
    and strictly increasing to $b$.

    Thus $g$ attains a minimum at some $c \in (a, b)$.
    Then $g'(c) = 0$ so that $f'(c) = \lambda$.
\end{proof}

\begin{theorem*}[Generalised mean value theorem] \label{thm:mvt:general}
    Let $f, g\colon [a, b] \to \R$ be continuous functions that are
    differentiable on $(a, b)$.
    Then there exists a $c \in (a, b)$ such that \[
        (f(b) - f(a)) g'(c) = (g(b) - g(a)) f'(c).
    \]
\end{theorem*}
\begin{remark}
    If $g(x) = x$, we recover the mean value theorem.
    If furthermore $f(a) = f(b)$, we recover Rolle's theorem.
\end{remark}
