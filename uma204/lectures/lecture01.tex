\lecture{01}{Mon 01 Jan '24}{}

We assume the following.
\begin{itemize}
    \item Basics of set theory
    \item Existence of $\N = \set{0, 1, 2, \dots}$ with the usual operations
    $+$ and $\cdot$
\end{itemize}
For a recap, refer lectures 1 to 3 of UMA101.

\section{Number Systems} \label{sec:number_systems}
\[
    \N \subseteq \Z \subseteq \Q \subseteq \R \subseteq \C
\]
\subsection{Recap of the Naturals} \label{sec:naturals}
$\N$ is the unique minimal inductive set granted by the ZFC axioms.
Addition and multiplication are defined by the recursion principle and showed
that they
\begin{itemize}
    \item are associative and commutative,
    \item admit identity elements $0$ and $1$ respectively,
    \item satisfy the distributive law,
    \item satisfy cancellation laws,
    \item \textcolor{Red}{but} do not admit inverses.
\end{itemize}

\subsection{Relations} \label{sec:relations}
(Recall) A relation on a set $A$ is a subset $R \subseteq A \times A$.
We write $a \mathrel{R} b$ to denote $(a, b) \in R$.

\begin{definition}[Partial order] \label{def:relations:partial_order}
    A relation $R$ on $A$ is called a \emph{partial order} if it is
    \begin{itemize}
        \item reflexive, \ie\ $a \mathrel{R} a$ for all $a \in A$;
        \item antisymmetric, \ie\ if $a \mathrel{R} b$ and $b \mathrel{R} a$
        then $a = b$ for all $a, b \in A$;
        \item transitive, \ie\ if $a \mathrel{R} b$ and $b \mathrel{R} c$ then
        $a \mathrel{R} c$ for all $a, b, c \in A$.
    \end{itemize}
    Additionally, if for all $x, y \in A$, $x \mathrel{R} y$ or $y \mathrel{R}$,
    then $R$ is called a \emph{total order}.

    A set $A$ equipped with a partial order $\le$ is called a \emph{partially
    ordered set} (or \emph{poset}). \\
    A set $A$ equipped with a total order $\le$ is called a \emph{totally
    ordered set} or simply an \emph{ordered set}.
\end{definition}
\begin{examples}
    \item $(\N, \le)$ where we say that $a \le b$ if $\exists c \in \N$ such
    that $a + c = b$.
    \item $(\N, \mid)$ where we say that $a \mid b$ if $\exists c \in \N$ such
    that $a \cdot c = b$.
\end{examples}

In UMA101, we defined order slightly differently, where we said that either
$a \le b$ or $b \le a$ but never both.
This is a ``strict order''.
We will denote a weak partial order by $\le$ and a strict partial order by $<$.
(the notation is suggestive of how to every order there is a corresponding
strict order and vice versa).

\begin{definition}[Equivalence] \label{def:relations:equivalence}
    An \emph{equivalence relation} on a set $A$ is a relation $R$ satisfying
    \begin{itemize}
        \item reflexivity;
        \item symmetry, \ie\ if $a \mathrel{R} b$ then $b \mathrel{R} a$ for all
        $a, b \in A$;
        \item transitivity.
    \end{itemize}
\end{definition}
\begin{notation}
    We write $[x]_R$ to denote the set $\set{y \in A \mid x \mathrel{R} y}$.
\end{notation}
\begin{proposition}
    The collection $\mathscr{A} = \set{[x]_R \mid x \in A}$ partitions $A$.
\end{proposition}
\begin{proof}
    For every $x \in A$, $x \in [x]_R$ and so $\bigcup \mathscr{A} = A$.

    Let $[x]_R \cap [y]_R \ne \O$, where $x, y \in A$.
    Then there exists $z \in A$ such that $x \mathrel{R} z$ and
    $y \mathrel{R} z$, from which it follows that $x \mathrel{R} y$ and
    $[x]_R = [y]_R$.
\end{proof}

\subsection{Integers} \label{sec:integers}
We cannot solve $3 + x = 2$ in $\N$.
We introduce $\Z$ to solve this problem.

Consider the relation $R$ on $\N \times \N$ given by \[
    (a, b) \mathrel{R} (c, d) \iff a + d = b + c.
\] (check that this is an equivalence relation \textcolor{solved}{trivial}).
\begin{definition} \label{def:integers}
    We define $\Z$ to be the set of equivalence classes of $R$, notated
    $\N \times \N / R$.
\end{definition}
Further, define
\begin{itemize}
    \item $[(a, b)] +_\Z [(c, d)] \coloneq [(a + c, b + d)]$;
    \item $[(a, b)] \cdot_\Z [(c, d)] \coloneq [(ac + bd, ad + bc)]$.
    \item $z_1 \le_\Z z_2$ iff there exists $n \in \N$ such that
    $z_1 +_\Z [(n, 0)] = z_2$ \\
    (alternatively, $[(a, b)] \le_\Z [(c, d)]$ iff $a + d \le b + c$).
\end{itemize}
We need to check that these are well-defined.
What does this mean?
Consider \begin{align*}
    [(1, 2)] +_\Z [(3, 4)] &= [(4, 6)] \\
    [(3, 4)] +_\Z [(3, 4)] &= [(6, 8)]
\end{align*}
Our definition must ensure that $[(4, 6)] = [(6, 8)]$. \\
In general, the definitions are well-defined if they are independent of the
choice of representatives.
