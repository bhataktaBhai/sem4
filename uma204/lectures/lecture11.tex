\lecture{11}{Thu 25 Jan '24}{}

\begin{theorem}
    Let $\set{K_\alpha}_{\alpha \in \Lambda}$ be a collection of compact
    sets in $(X, d)$ such that any non-empty finite subcollection has
    non-empty intersection.
    Then $\bigcap_{\alpha \in \Lambda} K_\alpha \neq \O$.
\end{theorem}
\begin{proof}
    Suppose $\bigcap_{\alpha \in \Lambda} K_\alpha = \O$.
    No element in $K_1$ is in every other $K_\alpha$.
    Let $\mcU_\alpha = K_\alpha^c$ for each $\alpha$.
    Any point in $K_1$ is in at least one $\mcU_\alpha$.
    Then $\mcU_\alpha$ is an open cover of $K_1$.
    But since $K_1$ is compact, there is a finite subcover
    $\mcU_{\alpha_1}, \dots, \mcU_{\alpha_n}$.
    But then $K_1 \subseteq (K_{\alpha_1} \cap \dots \cap K_{\alpha_n})^c$,
    so $K_{\alpha_1} \cap \dots \cap K_{\alpha_n} = \O$.
    Contradiction.
\end{proof}

\begin{theorem}
    Every closed subset of a compact set is compact.
\end{theorem}
\begin{proof}
    Let $E \subseteq Y \subseteq (X, d)$ where $Y$ is compact and $E$ is
    closed.
    Let $\mcU$ be an open cover of $E$ in $X$.
    Then $\mcU + E^c$ is an open cover of $Y$.
    Let $\mcV$ be a finite subcover of $\mcU + E^c$.
    Then $\mcV - E^c$ is a finite subcover of $\mcU$.
    This is because for any $x \in E$, $x \in \mcV$ (because $x \in Y$) but
    $x \notin E^c$, so $x \in \mcV - E^c$.
\end{proof}

\begin{theorem}
    Every infinite subset of a compact set has a limit point in the compact
    set.
\end{theorem}
\begin{proof}
    Suppose $E \subseteq (X, d)$ is compact and $F \subseteq E$ is infinite.
    Suppose $F$ has no limit point in $E$.
    Then for every $z \in E$, let $B(z, \varepsilon_z)$ be a neighbourhood
    of $z$ that contains no point of $F$ (except possibly $z$).
    Then $\set{B(z, \varepsilon_z)}_{z \in E}$ is an open cover of $E$.
    However, since $E$ is compact, there is a finite subcover.
    Since each $B(z, \varepsilon_z)$ contains at most one point of $F$,
    there are only finitely many points of $F$.
    Contradiction.
\end{proof}

\begin{proof}[Proof that (3) $\implies$ (2)]
    Suppose (3) holds on some $E \subseteq (\R^n, \norm{\cdot})$ but
    $E$ is not bounded.
    Let $x_0 \in E$.
    We can produce a sequence $(x_n)_{n \in \N} \subseteq E$ such that \[
        \norm{x_{n+1}} > \norm{x_n} + 1 \text{ for all } n \in \N.
    \] % TODO: no limit point

    Now suppose (3) holds on $E$ but $E$ is not closed.
    Then there exists a $z \in E^c$ such that $z$ is a limit point of $E$.
    Then there exists a sequence $(x_n)_{n \in \N} \subseteq E$ such that
    $\norm{x_j - z} < \frac1j$ for all $j \in \N$.
    The set $F = \set{x_n}_{n \in \N}$ is infinite (otherwise, the minimum
    distance is the infimum, which is zero, but $z \notin E$).
    Then $F$ must have a limit point in $E$.

    For any $y \in \R^n$, \begin{align*}
        \norm{x_j - y} &\ge \norm{z - y} - \norm{x_j - z} \\
                 &\ge \norm{z - y} - \frac1j.
    \end{align*} If $\norm{z - y}$ is positive, then there are only finitely
    many $x_j$ within a distance $\norm{z - y}$ of $y$.
    Hence $y$ can be a limit point of $F$ only if $y = z$.
\end{proof}

\begin{theorem}
    Let $E \subseteq Y \subseteq (X, d)$ where $Y$ is compact in $X$.
    Then $E$ is compact in $Y$ if and only if it is compact in $X$.
\end{theorem}
