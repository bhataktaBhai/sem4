\lecture{11}{Thu 25 Jan '24}{}

How do we go from the compactness of rectangles to the compactness of
arbitrary closed and bounded sets in $\R^n$?
We need the following theorem.
\begin{theorem} \label{thm:compactness:closed_subset}
    A closed subset of a compact set is compact.
\end{theorem}
\begin{proof}
    Let $E \subseteq X$ be compact and $F \subseteq E$ be closed.
    Let $\mcU$ be an open cover of $F$.
    Then $\mcU \cup \set{F^c}$ is an open cover of $E$.
    This contains a finite subcover $\mcV$.
    Then $\mcV \setminus \set{F^c} \subseteq \mcU$ is a finite subcover of $F$.
\end{proof}

We are now ready to show that closed and bounded sets in $\R^n$ are compact.
\begin{proof}[Proof that (2) $\implies$ (1) in $\R^n$]
    Let $E \subseteq \R^n$ be closed and bounded.
    There is an $R > 0$ such that $E \subseteq B(0; R)$, but
    $B(0; R) \subseteq [-R, R]^n$.
    So $E$ is a closed subset of the compact set $[-R, R]^n$.
    By \cref{thm:compactness:closed_subset}, $E$ is compact.
\end{proof}

\begin{theorem} \label{thm:compact:intersection}
    Let $\set{K_\alpha}_{\alpha \in \Lambda}$ be a collection of compact
    sets in $(X, d)$ such that any non-empty finite subcollection has
    non-empty intersection.
    Then $\bigcap_{\alpha \in \Lambda} K_\alpha \neq \O$.
\end{theorem}
\begin{proof}
    Suppose $\bigcap_{\alpha \in \Lambda} K_\alpha = \O$.
    Choose a $K \in \set{K_\alpha}_{\alpha \in \Lambda}$.
    No element in $K$ is in every other $K_\alpha$.
    Let $U_\alpha = K_\alpha^c$ for each $\alpha$.
    Any point in $K$ is in at least one $U_\alpha$.
    Then $\set{U_\alpha}_{\alpha \in \Lambda}$ is an open cover of $K$.
    Since $K$ is compact, there is a finite subcover
    $U_{\alpha_1}, \dots, U_{\alpha_n}$.
    But then \begin{align*}
        K &\subseteq U_{\alpha_1} \cup \dots \cup U_{\alpha_n} \\
          &= K_{\alpha_1}^c \cup \dots \cup K_{\alpha_n}^c \\
          &= (K_{\alpha_1} \cap \dots \cap K_{\alpha_n})^c
    \end{align*}
    so $K \cap K_{\alpha_1} \cap \dots \cap K_{\alpha_n} = \O$, a
    contradiction.
\end{proof}
\begin{corollary}
    Let $K_0 \supseteq K_1 \supseteq K_2 \supseteq \dots$ be non-empty
    compact sets in $(X, d)$.
    Then $\bigcap_{n \in \N} K_n \neq \O$.
\end{corollary}
\begin{remark}
    This cannot be used to prove the nested interval property, since
    the proof of compactness of rectangles in $\R^n$ itself relies on
    the nested interval property.
\end{remark}

\begin{theorem*}
    Every infinite subset of a compact set has a limit point in the set.
\end{theorem*}
\begin{proof}
    Let $E \subseteq (X, d)$ be compact and $F \subseteq E$ be infinite.
    Suppose $F$ has no limit point in $E$.
    Then for every $z \in E$, there exists an $\varepsilon_z > 0$ such that
    $B(z; \varepsilon_z)$ contains no point of $F$, except possibly $z$.
    Then $\set{B(z; \varepsilon_z)}_{z \in E}$ is an open cover of $E$.

    Since $E$ is compact, this contains a finite subcover.
    But each $B(z; \varepsilon_z)$ contains at most one point of $F$,
    so only finitely many points of $F$ are covered.
    Contradiction.
\end{proof}

\begin{proof}[Proof that (3) $\implies$ (2)]
    Suppose (3) holds on some $E \subseteq (\R^n, \norm{\cdot})$ but
    $E$ is not bounded.
    Let $x_0 \in E$.
    We can produce a sequence $(x_n)_{n \in \N} \subseteq E$ such that \[
        \norm{x_{n+1}} > \norm{x_n} + 1 \text{ for all } n \in \N.
    \] This cannot have a limit point in $E$ (or anywhere) since for any
    $x \in E$, $B(x, 1)$ contains at most one point of the sequence.

    Now suppose (3) holds on $E$ but $E$ is not closed.
    Then there exists a $z \in E^c$ such that $z$ is a limit point of $E$.
    Then there exists a sequence $(x_n)_{n \in \N} \subseteq E$ such that
    $\norm{x_j - z} < \frac1j$ for all $j \in \N$.
    The set $F = \set{x_n}_{n \in \N}$ is infinite (otherwise, the minimum
    distance is the infimum, which is zero, but $z \notin E$).
    Then $F$ must have a limit point in $E$.

    For any $y \in \R^n$, \begin{align*}
        \norm{x_j - y} &\ge \norm{z - y} - \norm{x_j - z} \\
                 &\ge \norm{z - y} - \frac1j.
    \end{align*} If $\norm{z - y}$ is positive, then there are only finitely
    many $x_j$ within a distance $\norm{z - y}$ of $y$.
    Hence $y$ can be a limit point of $F$ only if $y = z$.
\end{proof}

\begin{theorem} \label{thm:relative:compact}
    Let $E \subseteq Y \subseteq (X, d)$ where $Y$ is compact in $X$.
    Then $E$ is compact relative to $Y$ if and only if it is compact in $X$.
\end{theorem}
\begin{proof}
    We use \cref{thm:relative:open}.

    Suppose $E$ is compact in $Y$.
    Let $\mcU$ be an open cover of $E$ in $X$.
    Then $\mcV = \set{U \cap Y}_{U \in \mcU}$ is an open cover of $E$
    in $Y$.
    This has a finite subcover $\set{U_1 \cap Y, \dots, U_n \cap Y}$.
    Then $\set{U_1, \dots, U_n}$ is a finite subcover of $E$ in $\mcU$.

    Suppose $E$ is compact in $X$.
    Let $\mcV$ be an open cover of $E$ in $Y$.
    Then for each $V \in \mcV$,
    there exists a set $U_V$ open in $X$ such that $V = U_V \cap Y$,
    so that $\mcU = \set{U_V}_{V \in \mcV}$ is an open cover of $E$ in $X$.
    This has a finite subcover $\set{U_{V_1}, \dots, U_{V_n}}$.
    Then $\set{V_1, \dots, V_n}$ is a finite subcover of $E$ in $\mcV$.
\end{proof}
