\lecture{2024-03-04}{Mertens' theorem, rearrangements}

If $\sum_{n \in \N} a_n$ and $\sum_{n \in \N} b_n$ converge, does
$\sum_{n \in \N} c_n$ have to converge?

No.
Take the conditionally convergent series $a_n = b_n = \frac{(-1)^n}{\sqrt n}$.
Then \begin{align*}
    c_n   &= \sum_{k=1}^n a_k b_{n+1-k} \\
         &= \sum_{k=1}^n \frac{(-1)^{n+1}}{\sqrt k \sqrt{n+1-k}} \\
    \abs{c_n} &= \sum_{k=1}^n \frac{1}{\sqrt k \sqrt{n+1-k}} \\
         &\ge \sum_{k=1}^n \frac{1}{\sqrt n \sqrt n} \\
         &= \frac{n}{n} = 1.
\end{align*}
Thus $\sum_{n \in \N} c_n$ cannot converge.

\begin{theorem}[Mertens] \label{thm:mertens}
    Suppose that
    \begin{enumerate}
        \item $\sum_{n \in \N} a_n$ converges absolutely to $A$, and
        \item $\sum_{n \in \N} b_n$ converges to $B$.
    \end{enumerate}
    Then their Cauchy product $\sum_{n \in \N} c_n$ converges to $AB$.
\end{theorem}
\begin{proof}
    Let $(A_n)_{n \in \N}$, $(B_n)_{n \in \N}$ and $(C_n)_{n \in \N}$ be the
    sops of $\sum a_n$, $\sum b_n$ and $\sum c_n$ respectively.
    \begin{align*}
        C_n &= c_0 + c_1 + \dots + c_n \\
           &= a_0 b_0 + (a_0 b_1 + a_1 b_0)
                + \dots + (a_0 b_n + \dots + a_n b_0) \\
           &= a_0 B_n + a_1 B_{n-1} + \dots + a_n B_0 \\
           &= A_n B + (a_0 (B_n - B) + \dots + a_n (B_0 - B)).
        \intertext{Introduce the notation $\beta_k$ for $B_k - B$.}
           &= A_n B + (a_0 \beta_n + \dots + a_n \beta_0).
    \end{align*}
    Define $\delta_n = a_0 \beta_n + \dots + a_n \beta_0$.
    It suffices to show that \[
        \lim_{n \to \infty} \delta_n = 0.
    \] Let $\alpha = \sum_{n \in \N} \abs{a_n}$.
    Since $\lim_{n \to \infty} \beta_n = 0$,
    we have that for every $\epsilon > 0$,
    there exists $m \in \N$ such that $\abs{\beta_n} < \varepsilon$
    for all $n \ge m$.

    Fix an $\varepsilon > 0$ and choose an appropriate $m$.
    Then for $n \ge m$, \begin{align*}
        \abs{\delta_n}
        &= \abs{a_0 \beta_n + \dots + a_n \beta_0} \\
        &= \abs{a_0 \beta_n + \dots + a_{n-(m-1)} \beta_{m-1}}
            + \varepsilon \sum_{j=m}^{n} \abs{a_j}
    \end{align*}
    Taking limit as $n \to \infty$, \[
        \limsup_{n \to \infty} \abs{\delta_n} \le \varepsilon \alpha.
    \] Since $\varepsilon$ was arbitrary, we have that
    $\limsup_{n \to \infty} \abs{\delta_n} = 0$ and so
    $\abs{\delta_n} \to 0$.
    This gives $\delta_n \to 0$ so that $C_n = A_n B + \delta_n \to AB$.
\end{proof}

\section{Rearrangements} \label{sec:rearrangements}
Conditionally convergent real series have the remarkable feature that their
terms can be rearranged to converge to any real number.
\begin{example}
    Consider the series $\sum \frac{(-1)^n}{n}$.
    Let $(S_n)_{n \in \N^*}$ be its partial sums.
    Note that $S_3 > S_4, S_5, S_6, \dots$.
    So the limit is less that $1 - \frac12 + \frac13 = \frac{5}{6}$.

    Now consider the rearranged series \[
        1 + \frac13 - \frac12
        + \frac15 + \frac17 - \frac14
        + \frac19 + \frac1{11} - \frac16
        + \dots.
    \] We can show that this converges, by rewriting it as \[
        1 + \frac13 - \frac{2}{4}
        + \frac15 + \frac17 - \frac{2}{8}
        + \frac19 + \frac1{11} - \frac{2}{12}
        + \dots
    \] and applying \cref{thm:series:termwise_product} to \begin{align*}
        (a_n)_{n \in \N} &= (1, 1, -2, 1, 1, -2, 1, 1, -2, \dots) \\
        (b_n)_{n \in \N} &=
            \left(1, \frac13, \frac14,
            \frac15, \frac17, \frac18,
            \frac19, \frac1{11}, \frac1{12},
            \dots\right)
    \end{align*}
    Grouping terms in groups of three, we notice that each group is
    positive.
    So the rearranged series converges to a number greater than
    $1 + \frac13 - \frac12 = \frac{5}{6}$.
    Thus rearranging the terms has created a series that has a different
    sum.
\end{example}
