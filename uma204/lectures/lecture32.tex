\section{Vector-valued Functions} \label{sec:riemann:vector}
\lecture{2024-03-27}{Vector integration}

\begin{definition}
    Let $f\colon [a, b] \to \R^n$, $f = (f_1, \dots, f_n)$.
    Then $f$ is said to be \emph{Riemann integrable} if each $f_i$
    is Riemann integrable, and in this case we define \[
        \int_a^b f(x) \dd x
            = \left( \int_a^b f_1(x) \dd x, \dots, \int_a^b f_n(x) \dd x \right).
    \]
\end{definition}

\begin{theorem} \label{thm:riemann:triangle}
    Suppose $\vec{f} \in \mcR[a, b]$.
    Then so is $\lVert{\vec{f}}\;\rVert\colon [a, b] \to [0, \infty)$,
    and \[
        \norm{\int_a^b \vec{f}(x) \dd x}
            \le \int_a^b \norm{\vec{f}(x)} \dd x.
    \]
\end{theorem}
\begin{proof}
    \[
        \norm*[\big]{\vec f(x)} = \sqrt{f_1^2(x) + \dots + f_n^2(x)}
    \] is Riemann integrable since each $f_i$ is, and the square and
    square root functions are continuous.

    Cauchy-Schwarz.
    For any $f, g \colon [a, b] \to \R^n$, \begin{align*}
        \innerp{f(x)}{g(x)}
            &\le \norm{f(x)} \, \norm{g(x)} \\
        \implies \int_a^b \innerp{f(x)}{g(x)} \dd x
            &\le \int_a^b \norm{f(x)} \, \norm{g(x)} \dd x.
    \end{align*}
    The LHS is \[
        \sum_i \int_a^b f_i(x) g_i(x) \dd x.
    \] Take $g_j(x) = \int_{a}^{b} f_j(t) \dd t$.
    Then this becomes \[
        \sum_i \left(\int_{a}^{b} f_i(x) \dd x\right)^2
            = \norm{\int_{a}^{b} f(x) \dd x}^2\!.
    \]
    For the RHS, note that $g$ is constant, and \[
        \norm{g(x)} = \norm{\int_a^b f(t) \dd t}.
    \] So the RHS becomes \[
        \norm{\int_a^b f(x) \dd x} \int_a^b \norm{f(x)} \dd x.
    \]
    Thus \[
        \norm{\int_a^b f(x) \dd x}^2
            \le \norm{\int_a^b f(x) \dd x} \int_a^b \norm{f(x)} \dd x.
    \] If $\norm{\int_a^b f(x) \dd x} \ne 0$, cancel it out.
    Otherwise, the inequality holds anyway.
\end{proof}

Wow, that came out of nowhere!
Here is an attempt at making it less biblical.
\begin{proof}
    Let $K = \int_a^b f(x) \dd x$.
    Then \begin{align*}
        \int_a^b \innerp{f(x)}{K} \dd x
            &\le \int_a^b \norm{f(x)} \, \norm{K} \dd x \\
            &= \norm{K} \int_a^b \norm{f(x)} \dd x.
    \end{align*}
    But by linearity, \[
        \int_a^b \innerp{f(x)}{K} \dd x
            = \int_a^b \sum_i f_i(x) K_i \dd x
            = \sum_i K_i \int_a^b f_i(x) \dd x
            = \sum_i K_i^2
            = \norm{K}^2.
    \] Thus \[
        \norm{K}^2 \le \norm{K} \int_a^b \norm{f(x)} \dd x
    \] so that \[
        \norm{K} \le \int_a^b \norm{f(x)} \dd x. \qedhere
    \]
\end{proof}

\section{Rectifiable Paths} \label{sec:rectifiable_paths}
\begin{definition}[Path] \label{def:path}
    A \emph{path} or \emph{curve} in $\R^n$ is a continuous function
    $\gamma\colon [a, b] \to \R^n$.

    $\gamma$ is an \emph{arc} if it is injective.

    $\gamma$ is \emph{closed} if $\gamma(a) = \gamma(b)$.

    $\gamma$ is a simple closed curve if $\gamma(a) = \gamma(b)$ and the
    image of $\gamma$ does not self-intersect except at the endpoints.
\end{definition}

\begin{definition}[Length and rectifiable paths] \label{def:path:length}
    Let $\gamma\colon [a, b] \to \R^n$ be a path.
    Let $P = \set{a = x_0 \le x_1 \le \dots \le x_n = b}$ be a partition
    of $[a, b]$.
    Define \[
        \Lambda(P, \gamma)
            = \sum_{i=1}^n \norm{\gamma(x_i) - \gamma(x_{i-1})}.
    \] Further define \[
        \Lambda(\gamma)
        = \sup\set{\Lambda(P, \gamma) \mid P \text{ partitions } [a, b]}.
    \] If $\Lambda(\gamma) < \infty$, then $\gamma$ is said to be
    \emph{rectifiable}, and its \emph{length} is defined to be
    $\Lambda(\gamma)$.
\end{definition}

\begin{theorem} \label{thm:riemann:speed}
    Let $\gamma\colon [a, b] \to \R^n$ be continuously differentiable,
    with the derivative continuously extending to $[a, b]$.
    Then $\gamma$ is rectifiable, and \[
        \Lambda(\gamma) = \int_a^b \norm{\gamma'(t)} \dd t.
    \]
\end{theorem}
\begin{proof}
    First note that since $\norm{\gamma'}$ is continuous on $[a, b]$,
    and thus Riemann integrable (composition of continuous functions).

    Let $P = \set{x_0 \le \dots \le x_n}$ partition $[a, b]$.
    By \texttt{II}FToC, \begin{align*}
        \gamma(x_j) - \gamma(x_{j-1})
            &= \int_{x_{j-1}}^{x_j} \gamma'(t) \dd t \\
        \implies \norm{\gamma(x_j) - \gamma(x_{j-1})}
            &\le \int_{x_{j-1}}^{x_j} \norm{\gamma'(t)} \dd t.
        \intertext{by \cref{thm:riemann:triangle}.
        Summing over all $j$,}
        \Lambda(P, \gamma)
            &\le \int_a^b \norm{\gamma'(t)} \dd t.
            \yesnumber \label{eq:riemann:speed:upper}
    \end{align*}
    Thus $\Lambda(\gamma) \le \int_a^b \norm{\gamma'(t)} \dd t$.

    It remains to show that
    $\Lambda(\gamma) \ge \int_a^b \norm{\gamma'(t)} \dd t$.

    $\gamma'$ is uniformly continuous.
    Let $\varepsilon > 0$ and choose the corresponding $\delta$.

    Let $P_\varepsilon = \set{x_0 \le \dots \le x_n}$ be a partition
    with each interval of length less than $\delta$.
    Then for any $t \in [x_{j-1}, x_j]$, \begin{align*}
        \norm{\gamma'(t)}
            &\le \norm{\gamma'(x_{j-1})} + \varepsilon \\
        \implies \int_{x_{j-1}}^{x_j} \norm{\gamma'(t)} \dd t
            &\le \int_{x_{j-1}}^{x_j} \norm{\gamma'(x_{j-1})} \dd t
                + \varepsilon(x_j - x_{j-1}).
    \end{align*}
    We focus on the first term.
    \begin{align*}
        \int_{x_{j-1}}^{x_j} \norm{\gamma'(x_{j-1})} \dd t
        &= \norm{\int_{x_{j-1}}^{x_j} \gamma'(x_{j-1}) \dd t} \\
        &= \norm{\int_{x_{j-1}}^{x_j} \gamma'(x_{j-1}) - \gamma'(t)
                + \gamma'(t) \dd t} \\
        &\le \norm{\int_{x_{j-1}}^{x_j} \gamma'(x_{j-1}) - \gamma'(t) \dd t}
            + \norm{\int_{x_{j-1}}^{x_j} \gamma'(t) \dd t} \\
        &< \varepsilon(x_j - x_{j-1})
            + \norm{\gamma(x_j) - \gamma(x_{j-1})}.
        \intertext{So}
        \int_{x_{j-1}}^{x_j} \norm{\gamma'(t)} \dd t
            &< \norm{\gamma(x_j) - \gamma(x_{j-1})}
                + 2\varepsilon(x_j - x_{j-1}).
    \end{align*}
    Summing over all $j$,% in \cref{eq:riemann:velocity},}
    \begin{align*}
        \int_a^b \norm{\gamma'(t)} \dd t
            &\le \Lambda(P_\varepsilon, \gamma) + 2\varepsilon(b - a).
            \yesnumber \label{eq:riemann:speed:lower}
        \intertext{Since $\varepsilon$ was arbitrary,}
        \int_a^b \norm{\gamma'(t)} \dd t
            &\le \Lambda(\gamma).
    \end{align*}
    Or is it ``taking limits as $\varepsilon \to 0$''?

    This should be obvious to me by now, but it's not.
    So lemme do some $\varepsilon$-stuff.

    $\Lambda(\gamma)$ is the supremum of $\Lambda(P, \gamma)$ over all
    partitions $P$.
    So for any $\varepsilon > 0$, there exists a partition $P$
    such that \begin{align}
        \Lambda(P, \gamma) > \Lambda(\gamma) - \varepsilon.
        \label{eq:riemann:speed:lub}
    \end{align} In fact, it is the unique upper bound that satisfies this.

    We have already shown that the speed integral is an upper bound.
    From \cref{eq:riemann:speed:lower}, we have shown the
    property \labelcref{eq:riemann:speed:lub}.
    Thus $\int_a^b \norm{\gamma'(t)} \dd t$ is the desired supremum.

    More directly, there are partitions that make $\Lambda(\cdot, \gamma)$
    arbitrarily close to the speed integral.
    So nothing less than the speed integral can be an upper bound.
    Thus $\Lambda(\gamma) \le \int_a^b \norm{\gamma'(t)} \dd t$.

    Combining this with \labelcref{eq:riemann:speed:upper}, we have \[
        \Lambda(\gamma) = \int_a^b \norm{\gamma'(t)} \dd t. \qedhere
    \]
\end{proof}
