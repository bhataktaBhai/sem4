\begin{proof}
\lecture{2024-02-28}{Cauchy condensation test, $e$ and power series}
    Let the SOPS of the two series be $(S_n)_{n \ge 1}$ and
    $(T_n)_{n \ge 1}$ respectively.
    Note that we only need to show that $(S_n)_n$ is bounded above iff
    $(T_n)_n$ is.

    Let $k$ and $n$ be such that $n \le 2^k$.
    Then \begin{align*}
        S_n &= a_1 + a_2 + a_3 + \dots + a_n \\
        &\le a_1 + (a_2 + a_3) + \dots + (a_{2^k} + \dots + a_{2^{k+1}-1}) \\
        &\le a_1 + 2a_2 + \dots + 2^k a_{2^k} \\
        &= T_{k}
    \end{align*}
    Thus if $(T_n)_n$ is bounded then so is $(S_n)_n$.

    Now let $2^k < n$.
    Then \begin{align*}
        S_n &= a_1 + a_2 + a_3 + \dots + a_n \\
           &\ge a_1 + a_2 + a_3 + \dots a_{2^k} \\
           &= a_1 + a_2 + (a_3 + a_4) + \dots + (a_{2^{k-1}+1} + a_{2^k}) \\
           &\ge a_1/2 + a_2 + 2 a_4 + \dots + 2^{k-1} a_{2^k} \\
           &= \frac12 (a_1 + 2 a_2 + 4 a_4 + \dots 2^k a_{2^k}) \\
           &= T_k
    \end{align*}
    Thus if $(S_n)_n$ is bounded then so is $(T_n)_n$.
\end{proof}
\begin{corollary}
    $\sum_{n=1}^{\infty} \frac1{n^p}$ is convergent iff $p > 1$.
\end{corollary}
\begin{proof}
    If $p \le 0$, $\frac1{n^p} \not\to 0$, so the series cannot converge.
    If $p > 0$, then $\frac1{n^p}$ is decreasing and non-negative, so
    by the Cauchy condensation test, this converges iff
    $\sum_{n=1}^{\infty} 2^n \frac1{(2^n)^p}$ converges.
    But this is a geometric series with ratio $2^{1 - p}$.
    This converges iff $2^{1-p} < 1 \iff p > 1$.
\end{proof}

\begin{corollary}
    $\sum_{n=2}^{\infty} \frac1{n(\log n)^p}$ converges iff $p > 1$.
\end{corollary}
\begin{proof}
    If $p \le 0$, then the series is bounded below by $\frac1n$ for
    $n \ge 3$.
    So by the comparison test, the series diverges.

    For $p > 0$, the terms are decreasing and non-negative.
    $\sum_{n=1}^{\infty} \frac{2^n}{2^n n^p (\log 2)^p}$ converges iff
    $p > 1$ by the previous corollary.
\end{proof}

Recall that in UMA101 we defined $e$ as $\sum_{n=0}^{\infty} \frac1{n!}$.
\begin{theorem}
    $e = \lim_{n \to \infty} \left(1 + \frac1n\right)^n$.
\end{theorem}
\begin{proof}
    Let $s_n = \sum_{j=0}^{n} \frac1{j!}$ and
    $t_n = \left(1 + \frac1n\right)^n$.
    \begin{align*}
        t_n  &= \sum_{j=0}^{n} \binom{n}{j} \frac1{n^j} \\
            &= \sum_{j=0}^{n} \frac1{j!} \frac{n(n-1)\dots(n-j+1)}{n^j} \\
            &= 1 + 1 + \frac1{2!} \left(1 - \frac1n\right)
                + \frac1{3!} \left(1 - \frac1n\right)\left(1 - \frac{2}{n}\right) + \dots + \\
            &\qquad \frac1{n!} \left(1 - \frac1n\right) \ldots \left(1 - \frac{n-1}{n}\right)
    \end{align*}
    which clearly shows that $t_n \le s_n$ so
    $\limsup_{n \to \infty} t_n \le e$.

    Fix $m \in \N$.
    For all $n \ge m \ge 3$, \begin{align*}
        \liminf_{n \to \infty}
            &\ge 1 + 1 + \frac1{2!} \left(1 - \frac1n\right) + \dots + \\
            &\qquad\frac1{m!} \left(1 - \frac1n\right)\left(1 - \frac{2}{n}\right)\dots \left(1-\frac{m-1}{n}\right)\\
            &\hspace{-2em}\xrightarrow[m \text{ fixed}]{n \to \infty} 1 + 1 + \frac1{2!} + \dots + \frac1{m!} \\
            &= S_m
    \end{align*}
    As $m \to \infty$, \[
        \liminf_{n \to \infty} t_n \ge \lim_{m \to \infty} S_m = e
    \]
\end{proof}

\begin{definition*}[Formal power series] \label{def:fps}
    Let $(c_n)_{n \in \N} \subseteq \C$.
    Then the formal sum \[
        \sum_{n=0}^{\infty} c_n z^n
    \] is called a \emph{power series}.
    Here, $z$ is an arbitrary complex number, and the convergence of this
    series depends on $z$.
\end{definition*}
