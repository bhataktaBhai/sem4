\lecture{2024-04-03}{}
\begin{theorem}
    Let $f_n \in \mcD((a, b))$ and $f_n' \unifto g$.
    Suppose $f_n(x_0) \unifto f(x_0)$ for some $x_0 \in (a, b)$.
\end{theorem}
\begin{proof}
    WLOG assume that $x > x_0$.
    Apply MVT to $f_n - f_m$ on $[x_0, x]$.

    There exists a $\theta_{m n} \in (x_0, x)$ such that \begin{equation}
        (f_n - f_m)(x) - (f_n - f_m)(x_0)
            = (f_n' - f_m')(\theta_{m n}) (x - x_0)
            \label{eq:unif:mvt}
    \end{equation} Let $\varepsilon > 0$.
    Then for $m, n$ beyond some $N$, \begin{align*}
        \abs{f_n(x_0) - f_m(x_0)} &< \frac{\varepsilon}{2} \\
        \abs{f_n'(t) - f_m'(t)} &< \frac{\varepsilon}{2(b-a)}
                                \quad \forall t \in [x_0, x]
    \end{align*}
    Applying this to \cref{eq:unif:mvt}, \begin{align*}
        \abs{f_n(x) - f_m(x)}
        &\le \abs{f_n(x_0) - f_m(x_0)}
            + \abs{f_n'(\theta_{m n} - f_m'(\theta_{m n}} \abs{x - x_0} \\
        &< \frac{\varepsilon}{2} + \frac{\varepsilon(x-x_0)}{2(b-a)} \\
        &< \varepsilon.
    \end{align*}
    Thus by the Cauchy criterior, $f_n \unifto f$ on $(a, b)$.

    Fix a $p \in (a, b)$.
    Let $\phi_n\colon (a, b) \setminus \set{p} \to \R$ be the difference
    quotient \[
        \phi_n(x) = \lim_{x \to p} \frac{f_n(x) - f_n(p)}{x - p}
    \] and similarly $\phi(x) = \lim \frac{f(x) - f(p)}{x - p}$.

    We show that $\phi_n$ is uniformly Cauchy and $\phi_n \unifto \phi$
    on $(a, b) \setminus \set{p}$.
    \begin{align*}
        \phi_n(x) - \phi_m(x)
        &= \frac{(f_n - f_m)(x) - (f_n - f_m)(p)}{x - p} \\
        &= (f_n' - f_m')(\theta)
    \end{align*} for some $\theta$ between $x$ and $p$ (by MVT).
    Thus $\phi_n \unifto h$ for some $h$.

    But now fix an $x$. \[
        \lim_{n \to \infty} \frac{f_n(x) - f_n(p)}{x - p}
            = \frac{f(x) - f(p)}{x - p} = \phi(x),
    \] so $\phi_n \pwto \phi$.

    But uniform convergence implies pointwise convergence, and so
    $\phi_n \pwto h$ which means that $h = \phi$.

    By \cref{TODO}, \begin{align*}
        \lim_{n \to \infty} \lim_{x \to p} \phi_n(x)
        &= \lim_{x \to p} \phi(x) \\
        \implies \lim_{n \to \infty} f_n'(p) &= f'(p)
    \end{align*}
    Thus $f_n'(p) \to g(p)$.
    By IIFTOC, %TODO
\end{proof}

\subsection{The  Function} \label{sec:the_______function}

Start with $\phi(x) = \abs{x}$ on $[-1, 1]$, and extend it by periodicity
to the entire number line.

Define \[
    f_k \coloneq \left(\frac{3}{4}\right)^k \phi(4^k \,\cdot\,)
\] and \[
    f \coloneq \sum_{k=0}^{\infty} f_k
\]
By the Weierstrass $M$-test, $f$ is continuous, since \[
    \abs{f_k} \le \left(\frac{3}{4}\right)^k
        \text{ and }
    \sum \left(\frac34\right)^k < \infty.
\]
Fix an $x \in \R$.
Also fix an $m \in \N$.
There is at most one integer in the $\frac12$-ball around any point, so WLOG
assume that $(4^m x, 4^m x + \frac12)$ contains no integer.

Let $\delta_m = 4^{-m} \frac12$.

\[
    \frac{f(x + \delta_m) - f(x)}{\delta_m}
    = \sum_{k=0}^\infty \left(\frac{3}{4}\right)^k
        \frac{\phi(4^k (x + \delta_m)) - \phi(4^k x)}{\delta_m}
\]
But for $k > m$, $4^k \delta_m = 4^{k-m} \frac{1}{2}$ is an even integer,
so by the periodicity of $\phi$, the summand vanishes.

For $k = m$, the summand becomes \begin{align*}
    \left(\frac{3}{4}\right)^m \frac{\phi(4^m x + \frac12) - \phi(4^m x)}{\delta_m}
\end{align*}
whose absolute value is exactly $4^m$.
