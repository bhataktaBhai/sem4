\section{Connected Sets} \label{sec:connected}
\lecture{2024-01-29}{Connectedness}

\begin{definition*} \label{def:connected} \leavevmode
    \begin{enumerate}[label=(\alph*)]
        \item Let $(X, d)$ be a metric space.
        A pair of sets $A, B \subseteq X$ are said to be \emph{separated} in
        $X$ if $\wbar{A} \cap B = A \cap \wbar{B} = \O.$
        \item A set $E \subseteq X$ is said to be \emph{disconnected} if it
        is the union of two separated sets in $X$.
        \item $E$ is \emph{connected} if it is not disconnected.
    \end{enumerate}
\end{definition*}
\begin{examples}
    \item Sets $A = (-1, 0)$ and $B = (0, 1)$ are separated in $\R$.
    Note that $\sgn$ is continuous on $A \cup B$ but does not satisfy the
    intermediate value property.

    However, if $A = (-1, 0]$ instead, then all continuous functions on
    $A \cup B$ satisfy the intermediate value property.
    \item The empty set is connected.
    \item \Q\ is disconnected in \R.
    The partition
    $\{\Q \cap (-\infty, \sqrt 2), \Q \cap (\sqrt 2, \infty)\}$
    separates \Q.
    \item \Q\ is disconnected even in \Q.
\end{examples}
\begin{exercise}
    Let $E \subseteq Y \subseteq (X, d)$.
    Then $E$ is connected relative to $Y$ iff $E$ is connected in $X$.
\end{exercise}

\begin{theorem*}
    Let $E \subseteq \R$.
    Then $E$ is connected iff $E$ is convex, \ie, for all $x < y \in E$,
    $[x, y] \subseteq E$.
\end{theorem*}
\begin{proof}
    Suppose $E$ is connected, but not convex, \ie, there exist $x < y \in E$
    and some $r \in (x, y)$ that is not in $E$.
    Then $A = (-\infty, r] \cap E$ and $B = [r, \infty) \cap E$ separate
    $E$.

    Conversely, suppose $E$ is convex but not connected.
    Then there exist $A, B \subseteq E$ that separate $E$.
    Let $x \in A$ and $y \in B$ and suppose WLOG that $x < y$.
    Note that $A \cap [x, y]$ is non-empty and bounded.
    Let $r = \sup(A \cap [x, y])$.

    By the lemma below, $r \in \wbar{A \cap [x, y]} \subseteq \wbar{A}
    \cap [x, y]$ so $r \in \wbar{A}$.
    Disconnectedness forces that $r \notin B \iff r \in A$ so $x \le r < y$.

    But since $r$ is the supremum of $A \cap [x, y]$, $(r, y) \subseteq B$.
    This gives $r \in \wbar{B}$, violating the separation of $A$ and $B$.
\end{proof}

\section{The Cantor Set} \label{sec:cantor_set}
\begin{definition*}[Perfect set] \label{def:perfect_set}
    A set $E \subseteq (X, d)$ is said to be \emph{perfect} if every point
    of $E$ is a limit point of $E$.
\end{definition*}
Note that $E = [0, 1]$ is perfect in $\R$.
Can we produce a ``sparse'' perfect set?
Throwing away isolated points makes the set open.
Throwing away a finite number of open sets preserves perfectness, but there
are still \emph{intervals of positive length}.

\textbf{Can we produce a perfect set such that}
\begin{enumerate}
    \item it contains no intervals of positive length?
    \item $E$ is \emph{nowhere dense}, \ie, $(\wbar{E})^\circ = \O$?
\end{enumerate}
Note that the second condition implies the first.
