\begin{examples}
\lecture{09}{Mon 21 Jan '24}{}
    \item Closed intervals are closed sets.
    \item $E = [0, 1)$ is neither open nor closed.
    $\wbar{E} = [0, 1]$, $E^\circ = (0, 1)$ and $\partial E = \set{0, 1}$.
    \item Finite sets are always closed.
    \item In the discrete metric, every set is both open and closed.
\end{examples}
\begin{proposition} \label{thm:closed:open}
    A set is closed iff its complement is open.
\end{proposition}
\begin{proof}
    Let $E \subseteq X$ be closed.
    Let $x \in E^c$.
    Since $x$ is not a limit point of $E$, there exists an open ball
    around it that contains no points of $E$.
    Thus $x$ is an interior point of $E^c$.
    This gives that $E^c$ is open.

    Now let $E \subseteq X$ be open.
    Let $x$ be a limit point of $E^c$.
    Then there is no open ball around $x$ that lies entirely in $E$.
    Thus $x$ cannot lie in $E$.
    This gives that $E^c$ is closed.
\end{proof}
\begin{corollary} \label{thm:closed_intersection}
    The intersection of any collection of closed sets is closed,
    and the union of any finite collection of closed sets is closed.
\end{corollary}
\begin{proof}
    \Cref{thm:open_union,thm:closed:open} and De Morgan's laws.
\end{proof}

\begin{theorem*} \label{thm:closed:prop} \leavevmode
    \begin{enumerate}
        \item The closure of a set is closed.
        \item If $A \subseteq B$, then $\wbar{A} \subseteq \wbar{B}$.
        \item $\wbar{E}$ is the smallest closed set containing $E$.
        That is, \[
            \wbar{E} = \bigcap_{\substack{F \supseteq E \\ F \text{ closed}}} F.
        \]
    \end{enumerate}
\end{theorem*}
\begin{proof} \leavevmode
    \begin{enumerate}
        \item Let $x$ be a limit point of $\wbar{E}$.
        We will show that $x$ is a limit point of $E$.
        Let $\varepsilon > 0$.
        If $B(x; \varepsilon)$ contains no point from $E$,
        it must contain a limit point of $E$.
        But then there are points of $E$ arbitrarily close to it,
        within the ball.
        Thus $B(x; \varepsilon)$ contains points of $E$ for any
        $\varepsilon > 0$ and so $x$ is a limit point of $E$.
        \item Every point in $A$ is a point in $B$.
        Every limit point of $A$ is a limit point of $B$.
        \item Every closed set $F$ containing $E$ must contain
        \begin{itemize}
            \item every point of $E$, and
            \item every limit point of $E$.
        \end{itemize}
        Thus $\wbar{E} \subseteq F$ for every such $F$.
        so $\wbar{E}$ is the smallest closed set containing $E$. \qedhere
    \end{enumerate}
\end{proof}

\begin{definition*}[Relative topology] \label{def:relative_topology}
    Given a metric space $(X, d)$ and subsets $E \subseteq Y$,
    we say that $E$ is open (resp. closed) \emph{relative to} $Y$
    if $E$ is an open (resp. closed) set in the metric space $(Y, d\vert_Y)$.
\end{definition*}
\begin{proposition} \label{thm:relative:open}
    Let $(X, d)$ be metric space and $E \subseteq Y \subseteq X$.
    Then $E$ is open relative to $Y$ iff there exists an open set
    $F \subseteq X$ such that $E = F \cap Y$.
\end{proposition}
\begin{proof}
    Let $E$ be open relative to $Y$.
    For each $x \in E$, there exists an $\varepsilon_x > 0$ such that
    $B_Y(x, \varepsilon_x) \subseteq E$.
    Let $F = \bigcup_{x \in E} B_X(x, \varepsilon_x)$.
    Then $F$ is open relative to $X$ and $F \cap Y = E$.

    Let $E = F \cap Y$ where $F$ is open in $X$.
    For every $x \in F$ there exists an $\varepsilon > 0$ such that
    $B_X(x, \varepsilon) \subseteq F$.
    Then for any $y \in E$,
    $B_Y(y, \varepsilon) = B_X(y, \varepsilon) \cap Y$ is contained in
    $F \cap Y = E$.
    So $E$ is open in $Y$.
\end{proof}
\begin{corollary} \label{thm:relative:closed}
    Let $(X, d)$ be metric space and $E \subseteq Y \subseteq X$.
    Then $E$ is closed relative to $Y$ iff there exists a closed set
    $F \subseteq X$ such that $E = F \cap Y$.
\end{corollary}
\begin{proof}
    \Cref{thm:closed:open,thm:relative:open}.
\end{proof}
