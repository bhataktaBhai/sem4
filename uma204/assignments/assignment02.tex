\documentclass[12pt]{article}
\input{~/IISc/sem4/preamble}
\input{~/IISc/sem4/preamble/mvc}
\usepackage{pgfplots}
\pgfplotsset{compat=1.18}
\usepackage{tikz}
\usetikzlibrary{graphs, graphs.standard, quotes, positioning, arrows.meta}

\newcommand{\ind}[1]{\bm{1}_{#1}}
\newcommand{\given}{\mid}
\newcommand{\pms}{\{-1, 1\}}
\DeclareMathOperator*{\argmax}{argmax}
\DeclareMathOperator*{\argmin}{argmin}
\DeclareMathOperator*{\E}{\mathbb E}

\usepackage{amsmath}
\DeclareMathOperator\divv{div}
\DeclareMathOperator\hess{Hess}
% \DeclareMathOperator\Tr{Tr}

\usepackage{pgfplots}
\pgfplotsset{compat=1.18}
\usepackage{tikz}
\usetikzlibrary{graphs, graphs.standard, quotes, positioning, arrows.meta}

\newcommand{\ind}[1]{\bm{1}_{#1}}
\newcommand{\given}{\mid}
\newcommand{\pms}{\{-1, 1\}}
\DeclareMathOperator*{\argmax}{argmax}
\DeclareMathOperator*{\argmin}{argmin}
\DeclareMathOperator*{\E}{\mathbb E}

\usepackage{amsmath}
\DeclareMathOperator\divv{div}
\DeclareMathOperator\hess{Hess}
% \DeclareMathOperator\Tr{Tr}

\usepackage{pgfplots}
\pgfplotsset{compat=1.18}
\usepackage{tikz}
\usetikzlibrary{graphs, graphs.standard, quotes, positioning, arrows.meta}

\newcommand{\ind}[1]{\bm{1}_{#1}}
\newcommand{\given}{\mid}
\newcommand{\pms}{\{-1, 1\}}
\DeclareMathOperator*{\argmax}{argmax}
\DeclareMathOperator*{\argmin}{argmin}
\DeclareMathOperator*{\E}{\mathbb E}


\title{Assignment 2}
\setcounter{assignment}{2}
\author{Naman Mishra}
\date{07 January, 2024}

\begin{document}
\maketitle
\begin{problem}
    Let $F$ and $G$ be ordered fields with the LUB property.
    In Lecture 04, we defined $h\colon F \to G$ as \[
        h(z) = \sup_G\set{w \in \Q : w \le z}.
    \] Show that $h$ is a field isomorphism, \ie,
    \begin{enumerate}[label=(\arabic*)]
        \item $h$ is a bijection between $F$ and $G$,
        \item $h(x + y) = h(x) + h(y)$ for all $x, y \in F$,
        \item $h(x \cdot y) = h(x) \cdot h(y)$ for all $x, y \in F$.
    \end{enumerate}
\end{problem}
\begin{proof}
    \refifdef{thm:R:unique}{\Cref}{Lecture 4}.
\end{proof}

\begin{problem}
    In this problem, you may assume the well-definedness, commutativity and
    associativty of addition of Dedekind cuts (as defined in Lecture 04).
    Let $O = \set{z \in \Q : z < 0}$.
    Verify that $O$ is a Dedekind cut, and $A + O = A$ for all Dedekind cuts
    $A$.
    Let $A$ be a Dedekind cut.
    Define a Dedekind cut $B$ such that $A + B = O$.
    You must justify your answer.
\end{problem}
\begin{proof}
    \refifdef{thm:R:dedekind:negative}{\Cref}{Lecture 4}.
\end{proof}

\begin{problem}
    Let $a = (a_n)_{n \in \N}$ and $b = (b_n)_{n \in \N}$ be sequences of
    rational numbers such that $b_n \ne 0$ for all $n \in \N$.
    Suppose \[
        \lim_{n \to \infty} \frac{a_n}{b_n} = 1.
    \]
    \begin{enumerate}
        \item Are $a$ and $b$ equivalent?
        \item Are $a$ and $b$ equivalent if $a$ is a \Q-bounded sequence?
    \end{enumerate}
\end{problem}
\begin{solution} \leavevmode
    \begin{enumerate}
        \item No. $a_n = n + 1$ and $b_n = n$ gives a counterexample.
        \item Yes.
    \end{enumerate}
    Let $a$ be bounded by $M$.
    Let $n_0$ be such that for all $n \ge n_0$, $\frac12 < \frac{a_n}{b_n}$.
    Then, for all $n \ge n_0$, $\abs{b_n} < 2 \abs{a_n} \le 2M$.
    Thus $b$ is bounded.

    Let $\varepsilon > 0$.
    Let $N$ be such that for all $n \ge N$, \[
        \abs{\frac{a_n}{b_n} - 1} < \frac{\varepsilon}{2M}.
    \] Then for all $n \ge N$, \begin{align*}
        \abs{a_n - b_n} &= \abs{b_n} \abs{\frac{a_n}{b_n} - 1} \\
                  &< 2M \frac{\varepsilon}{2M} \\
                  &= \varepsilon.
    \end{align*}
\end{solution}

\begin{problem}
    You cannot use the existence (or the LUB property) of the ordered field
    of real numbers in this problem, so you must work ``within'' \Q.
    \begin{enumerate}
        \item Show that every monotone \Q-bounded sequence of rational
        numbers is \Q-Cauchy.
        \item Consider the following sequence: \[
            x_n = \begin{cases}
                2, & \text{if } n = 0, \\
                x_{n-1} - \frac{x_{n-1}^2 - 2}{2 x_{n-1}} & \text{if } n \ge 1
            \end{cases}
        \]
    \end{enumerate}
\end{problem}

\end{document}
